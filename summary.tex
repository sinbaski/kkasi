\documentclass{article}
\usepackage{amssymb,amsmath,amsthm,amsfonts}
\usepackage{mathrsfs}
\usepackage{dsfont}
\usepackage{enumerate}

%\newtheorem{mdef}{Definition}
%\newtheorem{theorem}{Theorem}
\newcommand{\eqsplit}[2]{
  \begin{equation}\label{#2}
    \begin{split}
      #1
    \end{split}
  \end{equation}}
\newcommand{\eqnsplit}[1]{
  \begin{eqnarray*}
    #1
  \end{eqnarray*}}
\newcommand{\tran}[1]{
  \tilde{#1}
}
\newcommand{\td}[2]{
  \frac{d #1}{d #2}
}
\newcommand{\pd}[2]{
  \frac{\partial #1}{\partial #2}
}
\newcommand{\ppd}[2]{
  \frac{\partial^2 #1}{\partial #2^2}
}
\newcommand{\pdd}[3]{
  \frac{\partial^2 #1}{\partial #2 \partial #3}
}
\newcommand{\otd}[1]{
  \frac{d}{d #1}
}
\newcommand{\opd}[1]{
  \frac{\partial}{\partial #1}
}
\newcommand{\oppd}[1]{
  \frac{\partial^2}{\partial #1^2}
}
\newcommand{\opdd}[2]{
  \frac{\partial^2}{\partial #1 \partial #2}
}
\newcommand{\ket}[1]{
  |#1\rangle
}
\newcommand{\bra}[1]{
  \langle#1|
}
\newcommand{\inn}[1]{
  \langle#1\rangle
}
\newcommand{\mean}[1]{
  \langle#1\rangle
}
\newcommand{\tr}{
  \text{tr}\,
}
\newcommand{\re}{
  \text{Re}\,
}
\newcommand\im{
  \text{Im}\,
}
\newcommand{\var}{
  \text{var}
}
\newcommand{\arcsinh}{
  \sinh^{-1}
}
\newcommand{\arccosh}{
  \cosh^{-1}
}
\newcommand{\erfc}{
  \text{erfc}
}
\newcommand{\E}{
  \mathbb{E}
}
\renewcommand{\P}{
  \mathbb{P}
}
\newcommand{\I}[1]{
  \mathbf{1}_{\{#1\}}
}
\newcommand{\1}[1]{
  \mathds{1}_{\{#1\}}
}
\newcommand{\diag}{
  \text{diag\,}
}
\newcommand{\M}{
  {\text{max}}
}
\newcommand{\m}{
  {\text{min}}
}
\newcommand{\ph}{
  {\text{arg}\,}
}
\newcommand\erf{
  \text{erf}
}
\renewcommand\vec[1]{
  \mathbf{#1}
}
\newcommand\mtx[1]{
  \mathbf{#1}
}
\newcommand\ed{
  \,{\buildrel d \over =}\,
}



\title{Mathematical Aspects of the Capital Market}
\author{Xie Xiaolei}
\date{\today}
\begin{document}
\begin{enumerate}
\item The Nordea 15min returns are well described by the model:
\begin{eqnarray*}
  r_t &=& \sigma_t a_t \\
  (1-B)(1-B^s)(\ln\sigma_t - \mean{\ln\sigma}) &=& (1-\theta B)(1-\Theta
  B^s) \alpha_t
\end{eqnarray*}
where $a_t \sim N(0, 1)$ and $\alpha_t$ is distributed as Johnson Su:
\[
\alpha_t = m + c\sinh{z - a \over b}
\]
Here $z \sim N(0, 1)$. $s$ is the number of seasons in the above
seasonal model. $s=33$ for this particular series. $34 \times
15\text{min} = 8.5\text{hr}$, a full trading day in Stockholm.

\begin{enumerate}
\item 
Andersen and Bollerslev et al reported that $\ln\sigma_t$ is Gaussian
distributed for the exchange rate they studied. However, the
unconditional distribution of $\ln\sigma_t$ in our case is evidently
non-Gaussian.

\item
The residuals $\alpha_t$ have no significant auto-correlation. Their
distribution is, however, not Gaussian but rather fat-tailed and
right-skewed.

Nevertheless, the distribution of $\sigma_t$ is indeed leptokurtic and
right-skewed as Andersen and Bollerslev et al reported for the
exchange rates.

\item
$\sigma_t$ can be approximated by the realized volatility,
which is the sum of squared returns that are sampled at a higher
frequency. The optimal frequency for this is debatable. I choose the
frequency with which $a_t$ is best fitted to a standard Gaussian, and
it turns out to be 30sec for 15min returns.

\item
Right now I am developing programs to estimate the above
seasonal model. Functions for ARIMA model estimation exist in Matlab,
but only for residuals distributed as Gaussian or Student's t. Both
fail to describe the residuals in our case.

\end{enumerate}
\item The Nordea 30min and 45min returns haven't been studied in
  detail but are expected to follow ARIMA seasonal models. Preliminary
  results suggest $\ln\sigma_t$ for 45min returns are Gaussian
  distributed. 
\end{enumerate}
\end{document}