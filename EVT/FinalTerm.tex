\documentclass{report}
\usepackage{graphicx}
\usepackage{subfigure}
\usepackage{multirow}
\usepackage{wrapfig}
\usepackage{amssymb}
\usepackage{amsmath}
\usepackage{mathrsfs}
\usepackage{enumerate}
% \usepackage{bookmark}

\usepackage{amssymb,amsmath,amsthm,amsfonts}
\usepackage{mathrsfs}
\usepackage{dsfont}
\usepackage{enumerate}

%\newtheorem{mdef}{Definition}
%\newtheorem{theorem}{Theorem}
\newcommand{\eqsplit}[2]{
  \begin{equation}\label{#2}
    \begin{split}
      #1
    \end{split}
  \end{equation}}
\newcommand{\eqnsplit}[1]{
  \begin{eqnarray*}
    #1
  \end{eqnarray*}}
\newcommand{\tran}[1]{
  \tilde{#1}
}
\newcommand{\td}[2]{
  \frac{d #1}{d #2}
}
\newcommand{\pd}[2]{
  \frac{\partial #1}{\partial #2}
}
\newcommand{\ppd}[2]{
  \frac{\partial^2 #1}{\partial #2^2}
}
\newcommand{\pdd}[3]{
  \frac{\partial^2 #1}{\partial #2 \partial #3}
}
\newcommand{\otd}[1]{
  \frac{d}{d #1}
}
\newcommand{\opd}[1]{
  \frac{\partial}{\partial #1}
}
\newcommand{\oppd}[1]{
  \frac{\partial^2}{\partial #1^2}
}
\newcommand{\opdd}[2]{
  \frac{\partial^2}{\partial #1 \partial #2}
}
\newcommand{\ket}[1]{
  |#1\rangle
}
\newcommand{\bra}[1]{
  \langle#1|
}
\newcommand{\inn}[1]{
  \langle#1\rangle
}
\newcommand{\mean}[1]{
  \langle#1\rangle
}
\newcommand{\tr}{
  \text{tr}\,
}
\newcommand{\re}{
  \text{Re}\,
}
\newcommand\im{
  \text{Im}\,
}
\newcommand{\var}{
  \text{var}
}
\newcommand{\arcsinh}{
  \sinh^{-1}
}
\newcommand{\arccosh}{
  \cosh^{-1}
}
\newcommand{\erfc}{
  \text{erfc}
}
\newcommand{\E}{
  \mathbb{E}
}
\renewcommand{\P}{
  \mathbb{P}
}
\newcommand{\I}[1]{
  \mathbf{1}_{\{#1\}}
}
\newcommand{\1}[1]{
  \mathds{1}_{\{#1\}}
}
\newcommand{\diag}{
  \text{diag\,}
}
\newcommand{\M}{
  {\text{max}}
}
\newcommand{\m}{
  {\text{min}}
}
\newcommand{\ph}{
  {\text{arg}\,}
}
\newcommand\erf{
  \text{erf}
}
\renewcommand\vec[1]{
  \mathbf{#1}
}
\newcommand\mtx[1]{
  \mathbf{#1}
}
\newcommand\ed{
  \,{\buildrel d \over =}\,
}



\author{Xie Xiaolei}
\date{\today}
\title{Solutions to the Final Term Test}
\begin{document}
\maketitle

\begin{enumerate}[1.]
\item
  \begin{enumerate}[(a)]
  \item In the following we write $M_{m, n}=\max(Y_m, Y_{m+1}, \cdots,
    Y_n)$ and $F_i(x) = P(Y_i \le x, \cdots, Y_{i+k-1} \le x)$ so that
    $P(X_i \le x) = F_{i+1}(x)$. Obviously $F_1 = F$. Now we consider,
    for $i \ge 1$
    \begin{eqnarray}
      && F_{i+1}(x) \nonumber \\
      &=& P(M_{i+1, i+k} \le x) \nonumber \\
      &=& P(M_{i+1, i+k} \le x | M_{i, i+k-1} \le x) F_i(x)
      \nonumber \\
      && +P(M_{i+1, i+k} \le x | M_{i, i+k-1} > x) (1- F_i(x)) \label{eq:1a}
    \end{eqnarray}
    Meanwhile
    \begin{eqnarray*}
      && P(M_{i+1, i+k} \le x | M_{i, i+k-1} \le x) \\
      &=& P(Y_{i+k} \le x) \\
      &=& F^{1/k}(x)
    \end{eqnarray*}
    and
    \begin{eqnarray*}
      && P(M_{i+1, i+k} \le x | M_{i, i+k-1} > x) \\
      &=& P(M_{i+1, i+k} \le x | Y_i > x, M_{i+1, i+k-1} \le x) \\
      && +P(M_{i+1, i+k} \le x | M_{i+1, i+k-1} > x) \\
      &=& P(Y_{i+k} \le x) + 0 \\
      &=& F^{1/k}(x)
    \end{eqnarray*}
    Substituting the two conditional probabilities into \eqref{eq:1a},
    we get
    \[
    F_{i+1} = F^{1/k}(x)
    \]
    The distribution function of $X_i$ is $F^{1/k}(x)$.

  \item The df. of $\{X_1, X_2, \cdots, X_i, \cdots\}$ are $\{F_2(x), F_3(x),
    \cdots, F_{i+1}(x), \cdots\}$, none of which depends on its
    respective index. Hence $X_i$ is a strictly stationary sequence.

  \item Notice that $M_n=\max(X_1, \cdots, X_n) = \max(Y_2, \cdots,
    Y_{n+k+1})$. Because $n \bar{F}(u_n) \to \tau$, Poisson's
    approximation gives
    \[
    \lim_{n \to \infty} P(\max(Y_1, \cdots, Y_n) \le u_n) = e^{-\tau}
    \]
    But
    \begin{eqnarray*}
      && P(\max(Y_1, \cdots, Y_n) \le u_n) \\
      &=& P(Y_1 = \max(Y_1, \cdots, Y_n)) P(Y_1 \le u_n) \\
      && +P(Y_1 \ne \max(Y_1, \cdots, Y_n)) P(\max(Y_2, \cdots, Y_n) \le
      u_n) \\
      &=& {1 \over n} F^{1/k}(u_n) + {n-1 \over n} P(\max(Y_2, \cdots,
      Y_n) \le u_n)
    \end{eqnarray*}
    Since ${1 \over n} F^{1/k}(u_n) \to 0$ and ${n-1 \over n} \to 1$
    as $n \to \infty$, we have $P(\max(Y_2, \cdots, Y_n) \le u_n)
    \to e^{-\tau}$, i.e.
    \begin{eqnarray*}
      \lim_{n \to \infty} P(\max(X_1, \cdots, X_{n-k-1}) \le u_n) &=&
      e^{-\tau}\\
      \lim_{n \to \infty} P(\max(X_1, \cdots, X_{n}) \le u_{n+k+1}) &=&
      e^{-\tau}\\
    \end{eqnarray*}

  \item It is straight-forward to calculate
    \begin{eqnarray*}
      N_n &=& \E \sum_{i=1}^{n-1} I_{\{X_i = X_{i+1}\}} \\
      &=& \sum_{i=1}^{n-1} P(X_i = X_{i+1}) \\
    \end{eqnarray*}
    where
    \begin{eqnarray*}
      && P(X_i = X_{i+1}) \\
      &=& P[Y_{i+1} \ne \max(Y_{i+1}, \cdots,
      Y_{i+k}), Y_{i+k+1} \ne \max(Y_{i+2}, \cdots, Y_{i+k+1})]
    \end{eqnarray*}
    Because $Y_i$ is an iid sequence, the two events of the above
    joint probability are independent. Thus we have
    \begin{eqnarray*}
      P(X_i = X_{i+1}) &=& \left({k-1 \over k}\right)^2
    \end{eqnarray*}
    and
    \begin{eqnarray*}
      N_n &=& (n-1) \left({k-1 \over k}\right)^2 \\
      &\ge& {n-1 \over 4}
    \end{eqnarray*}
    The expected number of equal pairs is at least 1/4 of the number
    of pairs.
  \end{enumerate}
\end{enumerate}
\end{document}