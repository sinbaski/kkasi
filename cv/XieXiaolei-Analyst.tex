% Jason R. Blevins - Curriculum Vitae
%
% Copyright (C) 2004-2010 Jason R. Blevins
% http://jblevins.org/projects/cv-template/
%
% You may use use this document as a template to create your own CV
% and you may redistribute the source code freely. No attribution is
% required in any resulting documents. I do ask that you please leave
% this notice and the above URL in the source code if you choose to
% redistribute this file.

\documentclass[10pt,letterpaper]{article}

\usepackage{hyperref}
\usepackage{geometry}
\usepackage{amsfonts, amsmath}
\usepackage[T1]{fontenc}
\usepackage{color}
\usepackage[usenames,dvipsnames,svgnames,table]{xcolor}
% Comment the following line to use the default Computer Modern font
% instead of the Palatino font provided by the mathpazo package.
% Remove the 'osf' bit if you don't like the old style figures.
%\usepackage[sc,osf]{mathpazo}

% In practice, I use the following font packages instead of mathpazo.
% beramono provides a nice fixed-width font. xagaramon uses the
% (commercial) Adobe Garamond font.
%\usepackage[scaled=0.75]{beramono}
%\usepackage[osf]{xagaramon}

% Set your name here
\def\name{Xie Xiaolei}

% The following metadata will show up in the PDF properties
\hypersetup{
  colorlinks = true,
  urlcolor = black,
  pdfauthor = {\name},
  pdfkeywords = {computer, mobile, information security},
  pdftitle = {\name: Curriculum Vitae},
  pdfsubject = {Curriculum Vitae},
  pdfpagemode = UseNone
}

\geometry{
  body={6.5in, 9.0in},
  left=1.0in,
  top=1.0in
}

% Customize page headers
\pagestyle{myheadings}
\markright{\name}
\thispagestyle{empty}

% Custom section fonts
\usepackage{sectsty}
\sectionfont{\ttfamily\mdseries\Large}
\subsectionfont{\ttfamily\mdseries\itshape\large}

% Other possible font commands include:
% \ttfamily for teletype,
% \sffamily for sans serif,
% \bfseries for bold,
% \scshape for small caps,
% \normalsize, \large, \Large, \LARGE sizes.

% Don't indent paragraphs.
\setlength\parindent{0em}

% Make lists without bullets and compact spacing
%\renewenvironment{itemize}{
\newenvironment{aperiod}{
  \begin{list}{}{
    \setlength{\leftmargin}{1.5em}
    \setlength{\itemsep}{1em}
    \setlength{\parskip}{0pt}
    \setlength{\parsep}{0.25em}
  }
}{
  \end{list}
}

\begin{document}

% Place name at left
{\huge Xie Xiaolei}

% Alternatively, print name centered and bold:
%\centerline{\huge \bf \name}

\vspace{0.25in}

\begin{minipage}[t]{0.5\textwidth}
  Citizenship: Sweden \\
  Phone: +46 70 489 0472 \\
  Email: \href{mailto:xie.xiaolei@gmail.com}{\tt xie.xiaolei@gmail.com} \\
  Address: Kiviksgatan 7C, 21440 Malm\"{o}, Sweden
\end{minipage}

\section*{Education}
\begin{itemize}
\item Ph.D of Financial Mathematics, October 2014 - present
  \begin{itemize}
    \item {\it Reseach Areas:} I am mainly interested in the
      determination of the largest eigenvalues of a covariance matrix
      built from samples with heavy-tails. The purpose is dimension
      reduction, by which the behaviour of a diverse portfolio is
      replicated by one with much few components and hence analysis of
      the portfolio becomes more simplified and focused.

      I am also interested in efficient simulation of multivariate
      GARCH structures, which, among others, helps the determination of
      model parameters when fitting such a model to a large
      multi-dimensional data set.
  \end{itemize}  

\item Master of Physics, Lund University, January 2011 - June 2014
  \begin{itemize}
  \item {\it Major Subject:} Econophysics. In my thesis I studied the
    average eigenvalue distribution of sample covariance matrices,
    with a focus on the impact of autocorrelations and heavy tails.

    % The returns' models are fitted to such Swedish stocks as Volvo B,
    % Ericsson B, and Nordea Bank, using both publicly available daily
    % statistics and intra-day transaction data that I have been
    % collecting using home-developed software.

    % The studies are carried out analytically wherever feasible, and
    % otherwise by approximation and simulation. The application of
    % probability theories, random matrices, and stochastic processes
    % are the most involved subjects.
  \end{itemize}

\item M.Sc.(Tech) Computer Science and Engineering, Helsinki
  University of Technology, 2005 - 2008.
  \begin{itemize}
  \item {\it Major Subject:} Mobile Computing - Services and Security
  \item {\it Average grade:} 4.17/5.
  \end{itemize}

\item B.Sc. Computer Science \& Technology, Beijing Institute of
  Technology, 2001- 2005.
  \begin{itemize}
  \item {\it Average grade:} 85.5/100.
  \item {\it Honors:} \textit{Distinguished Graduates of 2005}.
  \end{itemize}
\end{itemize}

\section*{Employment}
\begin{aperiod}
\item {\it \href{http://www.stericsson.com}{\tt
      ST-Ericsson}, Audio, Multimedia, Development Engineer, Sep 2008
    - Jan 2011}

  My main responsibility is software development in C, and sometimes
  C++ and Java, in the Linux operating system.

\item {\it \href{http://www.hiit.fi}{\tt Helsinki
      Institute of Information Technology}, Thesis Worker, Oct 2007 -
    May 2008}

  Funded by HIIT, I wrote my thesis ``A Mutual authentication and Key
  Agreement Protocol for the UMTS Network''. The algorithm of my
  protocol proposal was based on the Diffie-Hellman Exchange, but had
  significant modifications and extensions so that the mutual
  authentication between the mobile user and the network was delegated
  to the serving instead of the home network. The protocol design
  incorporated puzzle schemes that demanded unbalanced computing
  resources on the two sides to mitigate Denial-of-Service attacks.

\item {\it May 2007 - April 2004}
  I worked as software engineer in several companies. The programming
  languages used included C, C++, Java and python.
\end{aperiod}

\section*{Private Project}
{\it Trade data collector and analyzer}
With this home-developed software I have been collecting intra-day
transaction data and analyzing them in real-time with Matlab. I make
heavy use of the Matlab economitrics toolbox with its built-in
routines for estimating ARIMA and GARCH models and then forecasting
using the obtained model. The idea is to pre-estimate a model and thus
fix its structure, such as the distribution of the residuals, the
order of the auto-regressive and the moving-average components, the
order of differencing/fractional differencing, etc., and then to 
evolve the model parameters in real-time as new data are fed into
the system.
\end{document}
