%% start of file `template.tex'.
%% Copyright 2006-2013 Xavier Danaux (xdanaux@gmail.com).
%
% This work may be distributed and/or modified under the
% conditions of the LaTeX Project Public License version 1.3c,
% available at http://www.latex-project.org/lppl/.


\documentclass[10pt,a4paper, gentium]{moderncv}        % possible options include font size ('10pt', '11pt' and '12pt'), paper size ('a4paper', 'letterpaper', 'a5paper', 'legalpaper', 'executivepaper' and 'landscape') and font family ('sans' and 'roman')

% moderncv themes
\moderncvstyle{banking}                            % style options are 'casual' (default), 'classic', 'oldstyle' and 'banking'
\moderncvcolor{red}                                % color options 'blue' (default), 'orange', 'green', 'red', 'purple', 'grey' and 'black'
\renewcommand{\familydefault}{\sfdefault}         % to set the default font; use '\sfdefault' for the default sans serif font, '\rmdefault' for the default roman one, or any tex font name
%\nopagenumbers{}                                  % uncomment to suppress automatic page numbering for CVs longer than one page

% character encoding
\usepackage[utf8]{inputenc}                       % if you are not using xelatex ou lualatex, replace by the encoding you are using
%\usepackage{CJKutf8}                              % if you need to use CJK to typeset your resume in Chinese, Japanese or Korean

% adjust the page margins
\usepackage[scale=0.75]{geometry}
%\setlength{\hintscolumnwidth}{3cm}                % if you want to change the width of the column with the dates
%\setlength{\makecvtitlenamewidth}{10cm}           % for the 'classic' style, if you want to force the width allocated to your name and avoid line breaks. be careful though, the length is normally calculated to avoid any overlap with your personal info; use this at your own typographical risks...

% personal data
\name{Xie}{Xiaolei}
%\title{Resumé title}                               % optional, remove / comment the line if not wanted
\address{Kiviksgatan 7C}{21440 Malm\"o}{Sweden}% optional, remove / comment the line if not wanted; the "postcode city" and and "country" arguments can be omitted or provided empty
\phone[mobile]{+46~76~060~6638}                   % optional, remove / comment the line if not wanted
%\phone[fixed]{+2~(345)~678~901}                    % optional, remove / comment the line if not wanted
%\phone[fax]{+3~(456)~789~012}                      % optional, remove / comment the line if not wanted
\email{xie.xiaolei@gmail.com}                               % optional, remove / comment the line if not wanted
%\homepage{www.johndoe.com}                         % optional, remove / comment the line if not wanted
%\extrainfo{additional information}                 % optional, remove / comment the line if not wanted
%\photo[64pt][0.4pt]{picture}                       % optional, remove / comment the line if not wanted; '64pt' is the height the picture must be resized to, 0.4pt is the thickness of the frame around it (put it to 0pt for no frame) and 'picture' is the name of the picture file
%\quote{Some quote}                                 % optional, remove / comment the line if not wanted

% to show numerical labels in the bibliography (default is to show no labels); only useful if you make citations in your resume
%\makeatletter
%\renewcommand*{\bibliographyitemlabel}{\@biblabel{\arabic{enumiv}}}
%\makeatother
%\renewcommand*{\bibliographyitemlabel}{[\arabic{enumiv}]}% CONSIDER REPLACING THE ABOVE BY THIS

% bibliography with mutiple entries
%\usepackage{multibib}
%\newcites{book,misc}{{Books},{Others}}
%----------------------------------------------------------------------------------
%            content
%----------------------------------------------------------------------------------
\begin{document}
%-----       letter       ---------------------------------------------------------
% recipient data
\recipient{Cross Asset Quantitative Analysis team}{Citi}{London}
\date{\today}
\opening{Dear Sir/Madam,}
\closing{Best Regards,}
%\enclosure[Attached]{curriculum vit\ae{}}          % use an optional argument to use a string other than "Enclosure", or redefine \enclname
% \enclosure[Attached]{curriculum vit\ae{}, Master's thesis in
%   Econophysics}          % use an optional argument to use a string other than "Enclosure", or redefine \enclname
\makelettertitle

I see on Citi's website that you are recruiting a quantitative
analyst to join your cross asset quantitative analysis team. In the
description of the job, you emphasise knowledge and skills in time
series analysis, C++ programming and understanding of the financial
market, which are indeed areas that I know well and am passionate
about. So I would like to ask you for your consideration.

I am currently a PhD student at Copenhagen University, researching in
Mathematical statistics and particularly in time series analysis,
extreme value theory and random matrix theory. I will be available for
full-time work on 1st July 2017. I expect to receive my degree on 30th
September, 2017. So far I have co-authored and published two papers on
eigen systems of sample covariance matrices, with a focus on
heavy-tailed (e.g. power-law tail) time series. These results are
applicable to principle component analysis, for example.

I am also finalising two other papers. One of them is about how an
investor's preference (as quantified by {\em Generalised
  Disappointment Aversion}) of an equity varies with the tail
parameters of the stationary distribution of the equity. This
result can be applied to, for example, portfolio construction and
assessment of the overall market risk. As part of the paper, I have
also done statistical analysis on the tail parameters of equities in
the S\&P 500 index, specifically the ``Energy'', ``Consumer Staples''
and ``Information Technology'' sectors. R and C++ programming are
indispensable part of the work.

The fourth paper of mine is about estimation of the probability of very
large losses posed by portfolios whose volatility dynamics are
time series modelled by univariate GARCH(p,q) or multivariate
GARCH processes. Because the stationary distribution of such a process
is not known exactly, the aforementioned probability as well as the
related risk measures {\em Value at Risk} and {\em Expected
  Shortfall} cannot be computed precisely. In the paper I propose 
an estimator based on importance sampling and Markov chain theory and
naturally, would be more than happy to see it applied in practical
risk management. I have implemented the estimator in C++, and in
particular, have used the OpenMP directives to achieve high
performance via parallel computing.

Citi is a great investment bank that I have known since my
childhood. I know taking a role in this esteemed organisation presents
many challenges and opportunities. Through constant endeavour towards
ever greater excellence, I hope to develop a productive career that I
will be proud of by contributing my part to the success of Citi.

I look forward to discussing the job with you in more
details. You can always call me at +46 76 060 6638 or email me at
xie.xiaolei@gmail.com. It would be my pleasure to hear from you.

\makeletterclosing

\end{document}
