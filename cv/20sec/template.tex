%%%%%%%%%%%%%%%%%%%%%%%%%%%%%%%%%%%%%%%%%
% Twenty Seconds Resume/CV
% LaTeX Template
% Version 1.1 (8/1/17)
%
% This template has been downloaded from:
% http://www.LaTeXTemplates.com
%
% Original author:
% Carmine Spagnuolo (cspagnuolo@unisa.it) with major modifications by 
% Vel (vel@LaTeXTemplates.com)
%
% License:
% The MIT License (see included LICENSE file)
%
%%%%%%%%%%%%%%%%%%%%%%%%%%%%%%%%%%%%%%%%%

%----------------------------------------------------------------------------------------
%	PACKAGES AND OTHER DOCUMENT CONFIGURATIONS
%----------------------------------------------------------------------------------------

%% \documentclass[letterpaper]{twentysecondcv} % a4paper for A4
\documentclass[a4paper]{twentysecondcv} % a4paper for A4

%----------------------------------------------------------------------------------------
%	 PERSONAL INFORMATION
%----------------------------------------------------------------------------------------

% If you don't need one or more of the below, just remove the content leaving the command, e.g. \cvnumberphone{}

\profilepic{selfie1.jpg} % Profile picture

\cvname{Xie Xiaolei} % Your name
\cvjobtitle{Quantitative Analyst} % Job title/career

\cvdate{09 January 1983} % Date of birth
\cvaddress{Sweden} % Short address/location, use \newline if more than 1 line is required
\cvnumberphone{+46 760606638} % Phone number
\cvsite{https://www.linkedin.com/in/xie-xiaolei-bb0788b/} % Personal website
\cvmail{xie.xiaolei@gmail.com} % Email address

%----------------------------------------------------------------------------------------

\begin{document}

%----------------------------------------------------------------------------------------
%	 ABOUT ME
%----------------------------------------------------------------------------------------

\aboutme{
  \begin{itemize}
  \item Ph.D of Mathematical Statistics at Copenhagen University
  \item M.Sc. of Mathematical Physics
  \item M.Sc. of Computer Science
  \item 7 years work experience in programming 
  \item strong in statistical analysis \& numerical methods
  \end{itemize}
} % To have no About Me section, just remove all the text and leave \aboutme{}

%----------------------------------------------------------------------------------------
%	 SKILLS
%----------------------------------------------------------------------------------------

% Skill bar section, each skill must have a value between 0 an 6 (float)
\skills {{Python/3},{SQL/4},{GNU Scientific Library/5},{Matlab/5.2},{R/5.8},{C/5},{C++/5}}

%------------------------------------------------

% Skill text section, each skill must have a value between 0 an 6
%% \skillstext{{lovely/4},{narcissistic/3}}

%----------------------------------------------------------------------------------------

\makeprofile % Print the sidebar

%----------------------------------------------------------------------------------------
%	 INTERESTS
%----------------------------------------------------------------------------------------

%% \section{Summary}
 
%----------------------------------------------------------------------------------------
%	 EDUCATION
%----------------------------------------------------------------------------------------

\section{Education}

\begin{minipage}{0.15\linewidth}
  Oct. 2014 - Sep. 2017
\end{minipage} \hfill
\begin{minipage}{0.85\linewidth}
  \begin{itemize}
  \item Ph.D. in Mathematical Statistics
  \item Copenhagen University
  \item Areas of research:
    \begin{itemize}
      \item {\bf Time series analysis}: ARIMA, GARCH, stochastic volatility
        models
      \item {\bf Asset allocation \& risk management}: Principle component
        analysis; optimization of investor utility functions; extreme
        value theory
      \item {\bf Numerical methods}: Importance sampling, numerical
        optimization, estimation of GARCH(p,q) tail indices
    \end{itemize}
  \end{itemize}
\end{minipage}

\vspace{3mm}

\begin{minipage}{0.15\linewidth}
  Sep. 2011 - June 2014
\end{minipage} \hfill
\begin{minipage}{0.85\linewidth}
  \begin{itemize}
  \item M.Sc. in Mathematical Physics
  \item Lund University
  \item Thesis project: Distribution of eigenvalues of large
    sample covariance matrices in the infinite dimensional limit with
    a focus on the influence of autocorrelations and heavy tails. The
    theoretical derivation is based on the {\it free probability
      theory} and is checked against {\it intraday tick-by-tick
      prices} of high-volume Swedish stocks.
  \item Course work:    
    \begin{itemize}
      \item Mathematical methods of Physics and Engineering
      \item Statistical mechanics, Quantum mechanics, General
        relativity
    \end{itemize}
  \end{itemize}
\end{minipage}



%% \begin{minipage}[b]{0.2\linewidth}
%%   Oct. 2014 - Sep. 2017
%% \end{minipage} \hfill
%% \begin{minipage}{0.8\linewidth}
%%   \begin{itemize}
%%   \item Ph.D. in Mathematical Statistics
%%   \item Copenhagen University
%%   \item \emph{portfolio \& risk management} Balbla. Oh my god, I have
%%     to write a lot more about my research
%%   \end{itemize}
%% \end{minipage}


\section{Publications}

\begin{itemize}
\item Davis, R.A., Heiny, J., Mikosch, T. and {\bf Xie, X.},
  2016. {\it Extreme value analysis for the sample autocovariance
    matrices of heavy-tailed multivariate time series}.
  \underline{Extremes}, 19(3), pp.517-547.
  
  {\small
  We provide, among others, a limit theory for the largest eigenvalues
  of a sample covariance matrix of heavy-tailed, possibly linearly
  dependent data with infinite dimensions. We prove weak convergence of
  the point process of the largest eigenvalues to Poisson or cluster
  Poisson processes and, based on this convergence result, derive limit
  laws of functions of the ordered eigenvalues.}

\item Jan\ss en, A., Mikosch, T., Rezapour, M., {\bf Xie, X},
  2016. {\it The eigenvalues of the sample covariance matrix of a
    multivariate heavy-tailed stochastic volatility model}.
  To appear in the \underline{Bernoulli Journal}.
  
  {\small
  We derive multivariate $\alpha$-stable limit distributions of the
  sample covariance matrix when the sample size was large. We
  consider two possibilities of heavy-tails: 1. the innovation
  sequence is heavy tailed; and 2. the volatility sequence is
  heavy-tailed. While the eigen system of the first case is shown to
  resemble that of i.i.d data, the eigen system in the second case is
  found to allow dependencies in their limiting distributions, which
  are determined by the structure of the volatility sequence.}
\end{itemize}

\section{Papers in Finalization}
\begin{itemize}
\item Mikosch, T., de Vries, C., {\bf Xie, X.}, 2017,
  {\it Generalized Disappointment Aversion and heavy-tail property of
    equity return distributions}

  {\small
    We study the heavy tail properties of real-world equity return
    distributions and, based on the framework of Generalized
    Disappointment Aversion, derive how an investor's preference varies
    with the tail distribution characteristics, namely the tail index and
    the scale parameter. We find, when the equity return is t-distributed,
    the GDA preference actually decreases with the degree of freedom of
    the distribution, i.e. the tail index; we interprete this as a
    consequence of the symmetry of the distribution. When the equity
    return is assumed to have a shifted Pareto distribution with
    independent parameters of the left and the right tails, we find the
    GDA preference increasing with both the lower tail index and the scale
    parameter. We attribute this to the risk avert nature of rational
    investors. By numerical methods, optimal asset allocation is also
    solved in both cases.
  }
\end{itemize}
%----------------------------------------------------------------------------------------
%	 AWARDS
%----------------------------------------------------------------------------------------

%% \section{Awards}

%% \begin{twentyshort} % Environment for a short list with no descriptions
%% 	\twentyitemshort{1987}{All-Time Best Fantasy Novel.}
%% 	\twentyitemshort{1998}{All-Time Best Fantasy Novel before 1990.}
%% 	%\twentyitemshort{<dates>}{<title/description>}
%% \end{twentyshort}

%----------------------------------------------------------------------------------------
%	 EXPERIENCE
%----------------------------------------------------------------------------------------

%----------------------------------------------------------------------------------------
%	 OTHER INFORMATION
%----------------------------------------------------------------------------------------


%----------------------------------------------------------------------------------------
%	 SECOND PAGE EXAMPLE
%----------------------------------------------------------------------------------------

\newpage % Start a new page

\makeprofile % Print the sidebar

\section{Work Experience}

\begin{twenty} % Environment for a list with descriptions
	\twentyitem{1900}{Alice in Wonderland-The Circra (1900's) Silent Film.}{Film}{The first Alice on film was over a hundred years ago.}
	\twentyitem{1933}{Alice in Wonderland 1933 version.}{Film}{This film stars Ethel griffies and Charlotte Henry. It was a box office flop when it was released.}
	\twentyitem{1951}{Disney Film.}{Film}{Walt Disney brings Lewis Carroll's fantasy story to life in this well done animated classic. Even though many elements from the book were dropped, such as the duchess with the baby pig and mock turtle, this version is without a doubt the most famous Alice adaption made.}
	%\twentyitem{<dates>}{<title>}{<location>}{<description>}
\end{twenty}


\section{Other information}

\subsection{Review}

Alice approaches Wonderland as an anthropologist, but maintains a
strong sense of noblesse oblige that comes with her class status. She
has confidence in her social position, education, and the Victorian
virtue of good manners. Alice has a feeling of entitlement,
particularly when comparing herself to Mabel, whom she declares has a
``poky little house," and no toys. Additionally, she flaunts her
limited information base with anyone who will listen and becomes
increasingly obsessed with the importance of good manners as she deals
with the rude creatures of Wonderland. Alice maintains a superior
attitude and behaves with solicitous indulgence toward those she
believes are less privileged.

%\section{Other information}

%\subsection{Review}

%Alice approaches Wonderland as an anthropologist, but maintains a strong sense of noblesse oblige that comes with her class status. She has confidence in her social position, education, and the Victorian virtue of good manners. Alice has a feeling of entitlement, particularly when comparing herself to Mabel, whom she declares has a ``poky little house," and no toys. Additionally, she flaunts her limited information base with anyone who will listen and becomes increasingly obsessed with the importance of good manners as she deals with the rude creatures of Wonderland. Alice maintains a superior attitude and behaves with solicitous indulgence toward those she believes are less privileged.

%----------------------------------------------------------------------------------------

\end{document} 
