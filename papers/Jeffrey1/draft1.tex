\documentclass{article}
\usepackage{amsmath}
\usepackage{amsthm}
\usepackage{enumerate}
\usepackage[bookmarks=true]{hyperref}
\usepackage{bookmark}
\usepackage{graphicx}

\usepackage{amssymb,amsmath,amsthm,amsfonts}
\usepackage{mathrsfs}
\usepackage{dsfont}
\usepackage{enumerate}

%\newtheorem{mdef}{Definition}
%\newtheorem{theorem}{Theorem}
\newcommand{\eqsplit}[2]{
  \begin{equation}\label{#2}
    \begin{split}
      #1
    \end{split}
  \end{equation}}
\newcommand{\eqnsplit}[1]{
  \begin{eqnarray*}
    #1
  \end{eqnarray*}}
\newcommand{\tran}[1]{
  \tilde{#1}
}
\newcommand{\td}[2]{
  \frac{d #1}{d #2}
}
\newcommand{\pd}[2]{
  \frac{\partial #1}{\partial #2}
}
\newcommand{\ppd}[2]{
  \frac{\partial^2 #1}{\partial #2^2}
}
\newcommand{\pdd}[3]{
  \frac{\partial^2 #1}{\partial #2 \partial #3}
}
\newcommand{\otd}[1]{
  \frac{d}{d #1}
}
\newcommand{\opd}[1]{
  \frac{\partial}{\partial #1}
}
\newcommand{\oppd}[1]{
  \frac{\partial^2}{\partial #1^2}
}
\newcommand{\opdd}[2]{
  \frac{\partial^2}{\partial #1 \partial #2}
}
\newcommand{\ket}[1]{
  |#1\rangle
}
\newcommand{\bra}[1]{
  \langle#1|
}
\newcommand{\inn}[1]{
  \langle#1\rangle
}
\newcommand{\mean}[1]{
  \langle#1\rangle
}
\newcommand{\tr}{
  \text{tr}\,
}
\newcommand{\re}{
  \text{Re}\,
}
\newcommand\im{
  \text{Im}\,
}
\newcommand{\var}{
  \text{var}
}
\newcommand{\arcsinh}{
  \sinh^{-1}
}
\newcommand{\arccosh}{
  \cosh^{-1}
}
\newcommand{\erfc}{
  \text{erfc}
}
\newcommand{\E}{
  \mathbb{E}
}
\renewcommand{\P}{
  \mathbb{P}
}
\newcommand{\I}[1]{
  \mathbf{1}_{\{#1\}}
}
\newcommand{\1}[1]{
  \mathds{1}_{\{#1\}}
}
\newcommand{\diag}{
  \text{diag\,}
}
\newcommand{\M}{
  {\text{max}}
}
\newcommand{\m}{
  {\text{min}}
}
\newcommand{\ph}{
  {\text{arg}\,}
}
\newcommand\erf{
  \text{erf}
}
\renewcommand\vec[1]{
  \mathbf{#1}
}
\newcommand\mtx[1]{
  \mathbf{#1}
}
\newcommand\ed{
  \,{\buildrel d \over =}\,
}



\title{Importance Sampling}
\author{Xie Xiaolei}
\date{\today}

\begin{document}
\maketitle
\section{Introduction}
We consider the model
\begin{eqnarray*}
V_n &=& A_n V_{n-1} \\
V_0 &=& x_0 \in \mathbb S^{d-1}\\
\end{eqnarray*}
Use the following notations
\begin{eqnarray*}
X_n &=& \frac{V_n}{\|V_n\|} \\
S_n &=& \log \|A_n \cdots A_1 x_0\| \\
\xi_n &=& S_n - S_{n-1} = \log\frac{\|A_n \cdots A_1 x_0\|}{\|A_{n-1}
  \cdots A_1 x_0\|} \\
&=& \log\left\| A_n \frac{A_{n-1} \cdots A_1 x_0}{\|A_{n-1} \cdots A_1
    x_0\|} \right\|\\
&=& \log \|A_n X_{n-1}\|
\end{eqnarray*}
The pair $(X_n, S_n)$ is a Markov additive process with transition
kernel $P$. Assume conditions (M) and (A) of Kesten \cite{Kesten1973}:
\begin{enumerate}
\item The top Lyapunov exponent is negative, i.e.
  \begin{equation}
    \label{eq:neg_top_Lyapunov}
    \inf_{n \geq 1} \E \log \|A_n \cdots A_1\| < 0    
  \end{equation}
\item $\exists \xi > 0$ such that $\lambda(\xi) = 1$, where
  $$
  \lambda(\xi) := \inf_{n \geq 1} (\E \|A_n \cdots A_1\|^\xi)^{1/n}
  $$
\item $\exists \epsilon > 1$ such that $\epsilon \xi > 1/2$ and
  $\E[\|A\|^{\xi\epsilon}] < \infty$, $\E |B|^{\xi \epsilon} <
  \infty$, $\lambda(\epsilon \xi) < \infty$. Moreover, in case $\xi <
  1/(\epsilon + 1)$,
  \begin{enumerate}
  \item $\lambda(\epsilon \xi) < \lambda(-\xi)$
  \item and the deepness of the function $\log \lambda(\cdot)$ is
    assumed to satisfy
    \begin{equation}
      \label{eq:deepness}
      \inf_{\beta \in R} \log \lambda(\beta) < - {1 \over 2\xi} \log
      \lambda(-\epsilon \xi)
    \end{equation}
  \end{enumerate}
\end{enumerate}

\section{Consistency}\label{sec:consistency}
By the law of large numbers
\begin{eqnarray*}
  && \P(|V| > u) \\
  &=& \lim_{n \to \infty} {1 \over n} \sum_{i=0}^n \1{|V_i| > u}
\end{eqnarray*}
Define
\[
R_n := \inf\{0 \leq i \leq n: V_i \in \mathcal C\}
\]
and
\[
K_i := \inf\{k \geq 1: k > K_{i-1}, V_k \in \mathcal C, K_0 = 0\}
\]
Then one can write
\begin{eqnarray*}
  && \lim_{n \to \infty} {1 \over n} \sum_{i=1}^n \1{|V_i| > u} \\
  &=& \lim_{n \to \infty} {1 \over n} \left[
    \sum_{i=0}^{K_{R_n}-1} \1{|V_i| > u} + \sum_{i=K_{R_n}}^n \1{|V_i| > u}
\right]
\end{eqnarray*}
For the 2nd term, by a Borel-Cantelli argument, it maybe shown
\[
\lim_{n \to \infty} {1 \over n}\sum_{i=K_{R_n}}^n \1{|V_i| > u} = 0
\]
For the 1st term,
\begin{eqnarray*}
&& \lim_{n \to \infty} {1 \over n} \sum_{i=0}^{K_{R_n}-1} \1{|V_i| >
  u}  \\
&=& \lim_{n \to \infty} {R_n \over n} {1 \over R_n} \sum_{i=1}^{R_n}
\sum_{j=K_{i-1}}^{K_i-1}\1{|V_i| > u} \\
&=& \pi(\mathcal C) \E_\gamma N_u
\end{eqnarray*}
where the law of large numbers of Markov chains has been used to reach
the last line. In addition, it is assumed
\begin{eqnarray*}
  V_{K_i} &\sim& \gamma \; \forall i \geq 0 \\
  \gamma(E) &=& \pi(E)/\pi(\mathcal C)\; \forall E \in \mathcal
  B(\mathcal C)
\end{eqnarray*}
Define
\begin{eqnarray*}
  T_u &=& \inf\{n \geq 1: |V_n| > u\} \\
  \tau &\overset{d}{=}& K_i - K_{i-1} \\
  N_u &:=& \sum_{i=0}^{\tau-1} \1{|V_i| > u}  \\
\end{eqnarray*}
Then $\E_\gamma N_u$ may be evaluated as
\begin{eqnarray*}
  && \E_\gamma N_u \\
  &=& \E_\gamma N_u \1{T_u < \tau} \\
  &=& \int_{\mathds S^{d-1}} \int_{\mathds R} \cdots \int_{\mathds
    S^{d-1}} \int_{\mathds R} N_u \1{T_u < \tau} \times \\
  && \prod_{i=1}^{T_u} e^{-\xi(s_i - s_{i-1}) + \lambda(\xi)}
  {r(x_{i-1}; \xi) \over r(x_{i}; \xi)}P_\xi(x_{i-1}, dx_i \times ds_i) \times \\
  && \prod_{i=T_u+1}^{\tau-1} P(x_{i-1}, dx_i \times d s_i) \\
  &=& \E_{\mathcal D} \left[
    N_u \1{T_u < \tau} e^{-\xi S_{T_u}} {r(x_0; \xi)
      \over r(x_{T_u}; \xi)}
  \right]
\end{eqnarray*}
where
\[
P_\xi(x_{i-1}, dx_i \times ds_i) = e^{\xi(s_i - s_{i-1}) -
  \lambda(\xi)} {r(x_i; \xi) \over r(x_{i-1}; \xi)} P(x_{i-1}, dx_i
\times ds_i)
\]
is the $\xi$-shifted transition kernel of the {\it Markov Additive
  process} $(X_n, S_n)$. $\xi$ is chosen such that 
\[
\lambda(\xi) = \lim_{n \to \infty} \log \left(
\E \|A_n \cdots A_1\|^\xi
\right)^{1/n} = 0
\]
$\E_{\mathcal D}$ denotes expectation taken with respect to the dual
measure defined as
\[
P_{\mathcal D} (x_i, dx_{i+1} \times ds) = \left\{
  \begin{array}{ll}
    P_\xi (x_i, dx_{i+1} \times ds) & \text{ if } i < T_u \\
    P(x_i, dx_{i+1} \times ds) & \text{ if } i \geq T_u \\
  \end{array}
\right.
\]
Because the top Lyapunov exponent is negative, it follows from a lemma
in Kesten \cite{Kesten1973} that there is a $s > 0$ such that
$\lambda(s) < 0$.

Thus we have obtained a consistent estimator
$\pi(\mathcal C)\mathcal E_u$ for $\P(|V| > u)$:
\begin{eqnarray*}
\P(|V| > u) &=& \pi(\mathcal C) \E_{\mathcal D} \mathcal E_u \\
&=& \pi(\mathcal C) \E_{\mathcal D} \left[
  N_u \1{T_u < \tau} e^{-\xi S_{T_u}} {r(x_0; \xi)
    \over r(x_{T_u}; \xi)}
\right]
\end{eqnarray*}

\section{Efficiency}\label{sec:efficiency}
\begin{lemma}
  Let $\beta \in \mathbb R$ satisfy
  \begin{enumerate}
  \item
    \begin{equation}
      \label{eq:drift_cond1}
      \E \|A\|^\beta < \infty      
    \end{equation}
  \item 
    \begin{equation}
      \label{eq:drift_cond2}
    \inf_{\alpha > 0} {\E \|A\|^{\alpha + \beta}
      \over 
      \E \|A\|^{\beta}
    } < 1
    \end{equation}
  \end{enumerate}
  then $\forall \alpha > 0$ such that $\frac{\E\|A\|^{\beta +
      \alpha}}{\E\|A\|^\beta} < 1$ , we have
  \begin{equation}
    \label{eq:drift}
    \E_\beta \left[\left.
        |V_n|^\alpha r_\beta(|\tilde V_n|; \alpha) \I{C^\complement}(V_{n-1}) \right|
      \mathcal F_{n-1} \right] \leq |V_{n-1}|^\alpha r_\beta(|\tilde
    V_{n-1}|; \alpha) \I{C^\complement}(V_{n-1})
  \end{equation}
  where $r_\beta(\cdot; \alpha)$ is the unique right eigen function of the
  operator
  \[
  P_{\alpha, \beta} f(x) = \int |\hat a x|^\alpha f\left(
    {\hat a x \over |\hat a x|}
  \right) d\mu_A^\beta(\hat a)
  \]
  where the $\beta$-shifted measure $\mu_A^\beta$ satisfies
  \[
  d\mu_A^\beta(\hat a) = {\|\hat a\|^\beta d\mu_A(\hat a) \over \E\|A\|^\beta}
  \]
\end{lemma}

\begin{proof}
  Conditions \eqref{eq:drift_cond1} and \eqref{eq:drift_cond2} imply $\E_\beta
  \|A\|^\alpha < \infty$. Then, using the results of Buraczewski and
  Collamore et al \cite{BCDZ2014}, one may conclude that an eigen
  function $r_\beta(\cdot; \alpha)$ as well as an eigen measure
  $l_\beta(\cdot; \alpha)$ exists under the $\beta$-shifted
  measure. Thus, by defining the adjoint operator under  the shifted
  measure, the right eigen function $r_\beta(x; \alpha)$ can be represented as
  \[
  r_\beta(x; \alpha) = c \int_{S_+^{d-1}} \inn{x, y}^\alpha
  dl^*_{\beta}(x; \alpha)
  \]
Now, to prove \eqref{eq:drift}, consider
  \begin{eqnarray*}
    && \E_\beta \left[ |V_n|^\alpha r_\beta(\tilde V_n; \alpha) | \mathcal F_{n-1} \right]
    \\
    &=& \E_\beta\left[\int_{S_+^{d-1}} \inn{V_n, y}^\alpha dl^*_{\beta}(y; \alpha)|
      \mathcal F_{n-1} \right]
    \\
    &=& \E_\beta\left[\int_{S_+^{d-1}} (\inn{A_n V_{n-1}, y} + \inn{B_n,
        y})^\alpha dl^*_{\beta}(y; \alpha) | \mathcal F_{n-1} \right]
  \end{eqnarray*}
  \begin{enumerate}
  \item if $\alpha < 1$, by subadditivity we have
    \begin{eqnarray*}
      &&\E_\beta\left[\int_{S_+^{d-1}} (\inn{A_n V_{n-1}, y} + \inn{B_n,
          y})^\alpha dl^*_{\beta}(y; \alpha) | \mathcal F_{n-1} \right]\\
      &\leq& \E_\beta\left[\int_{S_+^{d-1}} \inn{A_n V_{n-1}, y}^\alpha
        dl^*_{\beta}(y; \alpha) | \mathcal F_{n-1} \right]
      + \E_\beta\left[\int_{S_+^{d-1}} \inn{B_n, y}^\alpha dl^*_{\beta}(y; \alpha) |
        \mathcal F_{n-1} \right] \\
      &=& \E_\beta\left[|V_{n-1}|^\alpha |A_n \tilde V_{n-1}|^\alpha\int_{S_+^{d-1}}
        \inn{{A_n V_{n-1} \over |A_n V_{n-1}|}, y}^\alpha
        dl^*_{\beta}(y; \alpha) | \mathcal F_{n-1} \right] \\
      && + \E_\beta\left[\int_{S_+^{d-1}} \inn{B_n, y}^\alpha dl^*_{\beta}(y; \alpha) |
        \mathcal F_{n-1} \right] \\
      &=& |V_{n-1}|^\alpha \left\{
        (P_{\alpha, \beta} r_{\beta})(\tilde V_{n-1}; \alpha) +
        {1 \over |V_{n-1}|^\alpha} \E_\beta |B_n|^\alpha r_\beta(\tilde
        B_n; \alpha) \right\} \\
      &=& |V_{n-1}|^\alpha r_\beta(\tilde V_{n-1}; \alpha)\left\{
        \lambda_\beta(\alpha) +
        {1 \over |V_{n-1}|^\alpha r_\beta(\tilde V_{n-1}; \alpha)} \E_\beta
        |B_n|^\alpha r_\beta(\tilde B_n; \alpha) \right\} \\
    \end{eqnarray*}
    By assumption, $\E_\beta|B_n|^\alpha < \infty$; and
    $r_\beta(\tilde B_n; \alpha) < \infty$, $r_\beta(\tilde V_{n-1};
    \alpha) < \infty$ according to Buraczewski and Collamore \cite{BCDZ2014}.
    Hence the infimum of the expression in the curly bracket over all $n \geq 1$ is
    $\lambda_\beta(\alpha) < 1$. Therefore, when $|V_{n-1}|$ is
    sufficiently large, $\rho_{\beta}(\alpha)$ an be chosen such that
    \begin{eqnarray*}
      \rho_{\beta}(\alpha) &<& 1 \\
      \E_\beta \left[ |V_n|^\alpha r_\beta(\tilde V_n; \alpha)  \I{C^\complement}(V_{n-1})
        | \mathcal F_{n-1} \right] &\leq&
      \rho_\beta(\alpha) |V_{n-1}|^\alpha r_\beta(\tilde V_{n-1}; \alpha) \I{C^\complement}(V_{n-1})
    \end{eqnarray*}
    where $C^\complement$ denotes the complement of the set $C$:
    \[
    C = \left\{
      v: |v| \leq M, M = \left[
        \E_\beta (|B|^\alpha r_\beta(\tilde B; \alpha)) 
        \over
        [1 - \lambda(\alpha)] r_\beta(\tilde v; \alpha)
      \right]^{1/\alpha}
    \right\}
    \]
  \end{enumerate}    
% that there exist a unique
%   probability measure $l_\alpha$ on $\mathbb S_+^{d-1}$ and a unique
%   and strictly positive function $r_\alpha \in \mathscr C_b(\mathbb
%   S_+^{d-1})$ such that
%   \begin{eqnarray*}
%     P_\alpha r(x; \alpha) &=& \lambda(\alpha) r(x; \alpha) \\
%     l_\alpha P_\alpha (S) &=& \lambda(\alpha) l_\alpha(S) \\
%     \int_{S_+^{d-1}} r(x; \alpha) l_\alpha(dx) &=& 1
%   \end{eqnarray*}
\end{proof}
\begin{remark}
  Iterating \eqref{eq:drift} yields
  \[
  \E_\beta \left[
    |V_n|^\alpha r_\beta(|\tilde V_n|; \alpha) \prod_{i=1}^{n-1}\I{C^\complement}(V_i)\right]
  \leq \rho_\beta(\alpha)^{n-1} |V_1|^\alpha r_\beta(|\tilde V_1|; \alpha) \I{C^\complement}(V_1)
  \]
  Then it follows
  \[
  \E_\beta \left[
    |V_n|^\alpha r_\beta(|\tilde V_n|; \alpha) \prod_{i=1}^{n}\I{C^\complement}(V_i)\right]
  \leq \rho_\beta(\alpha)^{n-1} |V_1|^\alpha r_\beta(|\tilde V_1|; \alpha) \I{C^\complement}(V_1)
  \]
  But $\prod_{i=1}^{n}\I{C^\complement}(V_i)$ implies $\tau > n$, and in this
  case $|V_n| > M$, where
  \[
  M = \left[
    \E_\beta (|B|^\alpha r_\beta(\tilde B; \alpha)) 
    \over
    (1 - \lambda(\alpha)) r_\beta(\tilde v; \alpha)
  \right]^{1/\alpha}
  \]
  Hence
  \begin{eqnarray}
    \E_\beta \left[
      M^\alpha \inf_{|\tilde V_n|} r_\beta(|\tilde V_n|; \alpha) \1{\tau > n}\right]
    &\leq& \rho_\beta(\alpha)^{n-1} |V_1|^\alpha r_\beta(|\tilde V_1|; \alpha) \I{C^\complement}(V_1) \nonumber \\
    \P_\beta(\tau > n) &\leq& K \rho_\beta(\alpha)^{n-1} \label{eq:ret_time}
  \end{eqnarray}
  for some constant $K$.
\end{remark}

\begin{theorem}
  The estimator $\mathcal E_u$ has bounded relative error, i.e.
  \begin{equation*}
    \limsup_{u \to \infty} {\var(\mathcal E_u) \over [\P(|V| > u)]^2} < \infty
  \end{equation*}
\end{theorem}
\begin{proof}
  The claim is equivalent to
  \[
  \limsup_{u \to \infty} {\E_D \mathcal E_u^2 \over [\P(|V| > u)]^2} < \infty
  \]
  By Kesten's theorem \cite{Kesten1973}, $\P(|V| > u) \sim C
  u^{-\xi}$. Hence, to prove the claim, one needs to check
  $u^{2\xi}\E_D \mathcal E_u^2 < \infty$, i.e.
  \[
  f(\xi) = \limsup_{u \to \infty} \E_D  \left[u^{2\xi}
    N_u^2 \1{T_u < \tau} e^{-2\xi S_{T_u}} {r^2(x_0; \xi)
      \over r^2(x_{T_u}; \xi)}\right]
  < \infty
  \]
  In the rest of the proof, we write $c, c_1, c_2$ for constants whose values
  have no importance and depend on the context.
  \begin{enumerate}
  \item We first consider the case $\lambda(-\xi\epsilon) <
    \infty$. Using the fact $|V_{T_u}| > u$, and that $r(\cdot; \xi)$
    is bounded from above and below, we obtain
    \begin{eqnarray*}
      f(\xi) &\leq& c \limsup_{u \to \infty} \E_D \left[
        \left|
          \sum_{n=0}^{T_u}
          \frac{
            N_u^{1/\xi} A_{T_u} \cdots A_{n+1} B_n 
          }{
            |A_{T_u} \cdots A_1 V_0|
          }
          \1{T_u < \tau}
        \right|^{2 \xi}
      \right] \\
    \end{eqnarray*}
    \begin{enumerate}
    \item If $\xi \geq 1/2$, Minkowski's inequality gives
      \begin{eqnarray}
        f(\xi) &\leq& c \limsup_{u \to \infty}
        \left\{
          \sum_{n=0}^{\infty}
          \left[
            \E_D \left|
              N_u^{1/\xi}
              \frac{
                A_{T_u} \cdots A_{n+1} B_n 
              }{
                |A_{T_u} \cdots A_1 V_0|
              }
              \1{n \leq T_u < \tau}
            \right|^{2 \xi}
          \right]^{1/2\xi}
        \right\}^{2\xi} \nonumber \\
        &\leq& c \limsup_{u \to \infty}
        \left\{
          \sum_{n=0}^{\infty}
          \left[
            \E_{-\xi} N_u^2 
            |A_{T_u} \cdots A_{n+1} B_n|^{2\xi}
            \1{n \leq T_u < \tau}
          \right]^{1/2\xi}
        \right\}^{2\xi} \label{eq:xi_above_half_prel}
      \end{eqnarray}
      By H\"older's inequality,
      \begin{eqnarray}
        && \left[ \E_{-\xi} N_u^2 
          |A_{T_u} \cdots A_{n+1} B_n|^{2\xi}
          \1{n \leq T_u < \tau} \right]^{1/2\xi} \nonumber \\
        &\leq& (\E_{-\xi} N_u^{2r})^{1/2r\xi}
        (\E_{-\xi}|B_n|^{2s\xi})^{1/2s\xi}  \nonumber \\
        && (\E_{-\xi} \|A_{T_u} \cdots A_{n+1}\|^{2s\xi}
        \1{n \leq T_u < \tau})^{1/2s\xi}
        \label{eq:xi_above_half}
      \end{eqnarray}
      As shown by Buraczewski, Damek
      and Mikosch \cite{BuraczewskiDamekMikosch2015},
      \[
      \lim_{k \to \infty}
      \left( \E_{-\xi}
        \|A_k \cdots A_{n+1}\|^{2s\xi}
      \right)^{1/(k-n)} = \lambda_{-\xi}(2s\xi)
      \]
      Hence $\forall a > 1$, $\exists N \geq 1$ such that $\forall k
      \geq n + N$,
      \[
      (\E_{-\xi}\|A_{k} \cdots A_{n+1}\|^{2s\xi})^{1/(k-n)} <
      a \lambda_{-\xi}(2s\xi)
      \]
      Meanwhile, for $k \in \{n+1, \dots, n+N-1\}$, due to the
      independence of $A_{n+1}, A_{n+2}, \dots, A_k$ we have
      \begin{eqnarray*}
        && \E_{-\xi} \|A_k \dots A_{n+1}\|^{2s\xi} \\
        &\leq& \left[\E_{-\xi} \|A_1\|^{2s\xi}\right]^{k-n} \\
        &=& \left[{\E \|A_1\|^{(2s - 1)\xi} \over \E \|A_1\|^{-\xi}}\right]^{k-n} \\
      \end{eqnarray*}
      Choose $s \in (1, {1 + \epsilon \over 2})$ such that $\E
      \|A_1\|^{(2s - 1)\xi} < \infty$; Furthermore, choose $s$ and $a$
      as close to 1 as is necessary to make $a \lambda(\xi(2s - 1)) <
      \lambda(-\xi)$ and hence $a \lambda_{-\xi}(2s\xi) < 1$. Such
      choices of $s$ and $a$ are possible since $\lambda(-\xi) >
      \lambda(\xi)=1$. This way,
      \begin{eqnarray*}
        && \E_{-\xi} \|A_{T_u} \cdots A_{n+1}\|^{2s\xi}
        \1{n \leq T_u < \tau} \\
        &=& \sum_{k=n}^\infty \E_{-\xi} \left(
          \|A_k \cdots A_{n+1}\|^{2s\xi}|T_u=k\right)
        \1{T_u < \tau} \P_{-\xi}(T_u = k) \\
        &\leq& c_1 \sum_{k=n}^{n+N-1} \1{T_u < \tau}
        P_{-\xi}(T_u = k) \\
        && + \sum_{k=n+N}^\infty
        \left[a\lambda_{-\xi}(2s\xi)\right]^{k-n}
        \1{T_u < \tau} \P_{-\xi}(T_u = k) \\
        % &\leq& c_1 \sum_{k=n}^{n+N-1} P_{-\xi}(\tau > k-1) \\
        % && + 
        % &\leq& c_1 \left[
        %   \frac{
        %   \lambda(\alpha - \xi)
        % }{
        %   \lambda(-\xi)
        % }
        % \right]^{n}
      \end{eqnarray*}
      Applying \eqref{eq:ret_time} with $\beta = -\xi$ gives
      \begin{eqnarray}
        && \E_{-\xi} \|A_{T_u} \cdots A_{n+1}\|^{2s\xi}
        \1{n \leq T_u < \tau} \nonumber \\
        &\leq& c_1 \sum_{k=n}^{n+N-1} {\lambda_{-\xi}(\alpha)}^k +
        c_2 a^{-n} \lambda_{-\xi}(2s\xi)^{-n}
        \sum_{k=n+N}^\infty \left[
          a \lambda_{-\xi}(2s\xi)
          \lambda_{-\xi}(\alpha)
        \right]^k \nonumber \\
        &\leq& c \lambda_{-\xi}(\alpha)^n \label{eq:norm_term}
      \end{eqnarray}
      for some $\alpha \in (0, 2\xi]$. Now we turn to $(\E_{-\xi}
      N_u^{2r})^{1/2r\xi}$:
      \begin{eqnarray}
        && \E_{-\xi} N_u^{2r} \nonumber \\
        &\leq& \E_{-\xi} \tau^{2r} \nonumber \\
        &\leq& \sum_{k=1}^{\infty} k^{2r} \P_{-\xi}(\tau > k-1)
        \nonumber \\
        &\leq& c_1 + c_2 \sum_{k=2}^{\infty} k^{2r}
        \lambda_{-\xi}(\alpha)^{k-2} < \infty \label{eq:Nu}
      \end{eqnarray}
      By assumption $\E \|B\|^{\epsilon \xi} < \infty$. Hence
      $\E|B|^{-\xi} > 1/\E|B|^\xi > 0$. It follows
      \begin{eqnarray}
        \E_{-\xi}|B_n|^{2s\xi} &=& {
          \E |B_n|^{\xi(2s - 1)}
          \over
          \E |B_n|^{-\xi}
        } \nonumber \\
        &\leq& \E |B|^{\epsilon \xi} \E |B|^{\xi} < \infty \label{eq:Bn}
      \end{eqnarray}
      Combining \eqref{eq:norm_term}, \eqref{eq:Nu}, \eqref{eq:Bn} and
      \eqref{eq:xi_above_half} we get
      \begin{eqnarray}
        && \left[ \E_{-\xi} N_u^2 
          |A_{T_u} \cdots A_{n+1} B_n|^{2\xi}
          \1{n \leq T_u < \tau} \right]^{1/2\xi} \nonumber \\
        &\leq& c \lambda_{-\xi}(\alpha)^n \label{eq:xi_above_half_final}
      \end{eqnarray}
      Inserting RHS of the above into \eqref{eq:xi_above_half_prel}
      gives
      \begin{eqnarray*}
        && \sum_{n=0}^{\infty}
        \left[
          \E_{-\xi} N_u^2 
          |A_{T_u} \cdots A_{n+1} B_n|^{2\xi}
          \1{n \leq T_u < \tau}
        \right]^{1/2\xi} \\
        &\leq& c \sum_{n=0}^{\infty} \lambda_{-\xi}(\alpha)^n \\
        &=& {c \over 1 - \lambda_{-\xi}(\alpha)}
      \end{eqnarray*}
      From this it is clear $f(\xi) < \infty$.

      % where $r, s$ are chosen such that $1/r + 1/s = 1$ and $s \in (1,
      % \frac{\epsilon + 1}{2})$. With such a choice $\E \|A\|^{\xi(2s -
      % 1)} < \E \|A\|^{\xi \epsilon} < \infty$.


      % Then H\"older's
      % inequality gives
      % \begin{eqnarray*}
      %   f(\xi) &\leq& \limsup_{u \to \infty}
      %   \left(\E_D N_u^{2r} \1{T_u < \tau}\right)^{1/r}
      %   \left(\E_D \left|
      %       \frac{
      %       \sum_{n=0}^{T_u} A_{T_u} \cdots A_{n+1} B_n 
      %     }{
      %       |A_{T_u} \cdots A_1 V_0|
      %     }
      %     \right|^{2 s \xi} \1{T_u < \tau}
      %   \right)^{1/s} \\
      %   &\leq& c \limsup_{u \to \infty} f_1(u) f_2(u, \xi)
      % \end{eqnarray*}
      % \begin{enumerate}
      % \item First of all, we prove $f_2(u,\xi) < \infty$. Minkowski's
      %   inequality yields
      %   \begin{eqnarray*}
      %     f_2(u, \xi)^{1/2\xi} &\leq& \sum_{n=0}^\infty \left(\E_D \left[
      %         \frac{
      %         |A_{T_u} \cdots A_{n+1} B_n |^{2s\xi}
      %       }{
      %         |A_{T_u} \cdots A_1 V_0|^{2s\xi}
      %       } \1{n \leq T_u < \tau}
      %       \right]\right)^{1/2s\xi} \\
      %     &=& \sum_{n=0}^\infty \left(\E \left[
      %         \frac{
      %         |A_{T_u} \cdots A_{n+1} B_n |^{2s\xi}
      %       }{
      %         |A_{T_u} \cdots A_1 X_0|^{\xi(2s - 1)}
      %       }
      %         \frac{
      %         r_{\xi}(X_{T_u})
      %       }{
      %         r_{\xi}(X_{0})
      %       }
      %         {1 \over |V_0|^{2s\xi}}\1{n \leq T_u < \tau}
      %       \right]\right)^{1/2s\xi} \\
      %     &\leq& c \sum_{n=0}^\infty
      %     \left(\E_{-\xi(2s - 1)} \left[
      %         |A_{T_u} \cdots A_{n+1} B_n |^{2s\xi}
      %         \lambda(-\xi(2s - 1))^{T_u}
      %         \1{n \leq T_u < \tau}
      %       \right]\right)^{1/2s\xi} \\
      %     &=& c \sum_{n=0}^\infty [\E_{-\xi(2s - 1)}
      %     f_{2,n}(u,\xi)]^{1/2s\xi}
      %   \end{eqnarray*}
      %   \begin{enumerate}
      %   \item If $\xi > 1/(1 + \epsilon)$, choose $s \in [1/2\xi,
      %     (\epsilon + 1)/2)$. With such a choice $(\cdot)^{1/2s\xi}$ is
      %     concave and subadditive. In this case we have
      %     \begin{eqnarray*}
      %       && [\E_{-\xi(2s - 1)} f_{2,n}(u,\xi)]^{1/2s\xi} \\
      %       &=& \left[
      %         \sum_{k=n}^\infty
      %         \E_{-\xi(2s - 1)}\left(
      %           \|A_k \cdots A_{n+1}\|^{2s\xi}
      %         \right)
      %         \E_{-\xi(2s - 1)} |B_n|^{2s\xi} \times \right.\\
      %       && \left. \lambda(-\xi(2s - 1))^k \1{\tau > T_u}
      %         \P_{-\xi(2s - 1)}(T_u \geq k) \right]^{1/2s\xi}\\
      %       &\leq& \sum_{k=n}^\infty
      %       \left[\E_{-\xi(2s - 1)}
      %         \|A_k \cdots A_{n+1}\|^{2s\xi}
      %       \right]^{1/2s\xi}
      %       \left[\E_{-\xi(2s - 1)} |B_n|^{2s\xi}\right]^{1/2s\xi} \times \\
      %       && \lambda(-\xi(2s - 1))^{k/2s\xi} \1{\tau > T_u}
      %       \left[ \P_{-\xi(2s - 1)}(T_u \geq k) \right]^{1/2s\xi}
      %     \end{eqnarray*}
      %     Obviously $\1{\tau > T_u} \P_{-\xi(2s - 1)}(T_u \geq k) \leq
      %     \P_{-\xi(2s - 1)}(\tau > k-1)$. Moreover
      %     \begin{eqnarray*}
      %       \E_{-\xi(2s - 1)} |B_n|^{2s\xi} &=& \frac{
      %       \E |B_n|^{\xi}
      %     } {
      %       \E |B_n|^{-\xi(2s - 1)}
      %     } \\
      %       &\leq& \E |B_n|^{\xi} \E |B_n|^{\xi(2s - 1)} \\
      %       &<& \E |B_n|^{\xi} \E |B|^{\xi \epsilon} < \infty
      %     \end{eqnarray*}
      %     Furthermore, as shown by Buraczewski, Damek
      %     and Mikosch \cite{BuraczewskiDamekMikosch2015},
      %     \[
      %     \lim_{k \to \infty}
      %     \left[ \E_{-\xi(2s - 1)}\left(
      %         \|A_k \cdots A_{n+1}\|^{2s\xi}|T_u = k
      %       \right) \right]^{1/(k-n)} = \lambda_{-\xi(2s - 1)}(2s\xi)
      %     \]
      %     Hence $\forall a > 1$, $\exists N \geq 1$ such that $\forall k
      %     \geq n + N$,
      %     \[
      %     (\E_{-\xi(2s - 1)}\|A_{k} \cdots A_{n+1}\|^{2s\xi})^{1/(k-n)} <
      %     a \lambda_{-\xi(2s - 1)}(2s\xi)
      %     \]
      %     Meanwhile, for $k \in \{n+1, \dots, n+N-1\}$, due to the
      %     independence of $A_{n+1}, A_{n+2}, \dots, A_k$ we have
      %     \begin{eqnarray*}
      %       && \E_{-\xi(2s - 1)} \|A_k \dots A_{n+1}\|^{2s\xi} \\
      %       &\leq& [\E_{-\xi(2s - 1)} \|A_1\|^{2s\xi}]^{k-n} \\
      %       &=& \left[{1 \over \E \|A_1\|^{-\xi(2s - 1)}}\right]^{k-n} \\
      %     \end{eqnarray*}
      %     By Jensen's inequality,
      %     \[
      %     \E \|A_1\|^{-\xi(2s - 1)}
      %     \geq {1 \over \E\|A_1\|^{\xi(2s - 1)}}
      %     \geq {1 \over \E\|A_1\|^{\xi\epsilon}}
      %     \]
      %     Hence
      %     \[
      %     \E_{-\xi(2s - 1)} \|A_k \dots A_{n+1}\|^{2s\xi} \leq
      %     \left[
      %       \E \|A_1\|^{\xi \epsilon}
      %     \right]^{k-n} < \infty      
      %     \]
      %     for $k=n, n+1, \dots, n+N-1$. Now we can write
      %     \begin{eqnarray*}
      %       && c \sum_{n=0}^\infty [\E_{-\xi(2s - 1)} f_{2,n}(u,\xi)]^{1/2s\xi} \\
      %       &\leq& c \sum_{n=0}^\infty \left\{
      %         c_1 \sum_{k=n}^{n+N-1} \lambda(-\xi(2s - 1))^{k/2s\xi} [\P_{-\xi(2s -
      %         1)}(\tau > k-1)]^{1/2s\xi} \right.\\
      %       && + \sum_{k=n+N}^\infty [a \lambda_{-\xi(2s - 1)}(2s\xi)]^{(k-n)/2s\xi}
      %       \lambda(-\xi(2s - 1))^{k/2s\xi} \times \\
      %       && \left.[\P_{-\xi(2s - 1)}(\tau > k-1)]^{1/2s\xi}
      %       \right\}
      %     \end{eqnarray*}
      %     Applying \eqref{eq:ret_time} with $\beta = -\xi(2s - 1)$ gives
      %     \[
      %     \P_{-\xi(2s - 1)}(\tau > k-1) \leq K \lambda_{-\xi(2s - 1)}(\alpha)^{k-2}
      %     \]
      %     where $\alpha > 0$ is chosen such that $\lambda_{-\xi(2s -
      %     1)}(\alpha) < 1$.
      %     %     This is always possible because
      %     %     $\lambda_{-\xi(2s - 1)}(\alpha) = \lambda(-\xi(2s - 1) + \alpha) /
      %     %     \lambda(-\xi(2s - 1))$, and ${d \lambda(-\xi(2s - 1) +
      %     %     \alpha) \over d\alpha} < 0$.
      %     Let $p = \lambda(\alpha - \xi(2s - 1))^{1/2s\xi} < 1$, and $q =
      %     \lambda(-\xi(2s - 1))^{1/2s\xi} > 1$. We have
      %     \begin{eqnarray*}
      %       && c \sum_{n=0}^\infty \left[
      %         \E_{-\xi(2s - 1)} f_{2,n}(u,\xi)
      %       \right]^{1/2s\xi} \\
      %       &\leq& c_1 \sum_{n=0}^\infty \sum_{k=n}^{n+N-1} p^{k-2} q^{2} +
      %       c \sum_{n=0}^{\infty} \sum_{k=n+N}^\infty a^{k-n} p^{k-2}
      %       q^{n-k+2} \\
      %       &=& c_1 p^{-2} q^2 {1 - p^N \over (1-p)^2 }
      %       + \sum_{n=0}^{\infty} a^{-n} p^{-2} q^{n+2} \sum_{k=n+N}^{\infty}
      %       \left({ap \over q}\right)^k
      %     \end{eqnarray*}
      %     Choose $a > 1$ such that $ap < q$. Then
      %     \begin{eqnarray*}
      %       && \sum_{n=0}^{\infty} a^{-n} p^{-2} q^{n+2} \sum_{k=n+N}^{\infty}
      %       \left({ap \over q}\right)^k \\
      %       &=& \frac{
      %       a^N p^{N-2} q^{2-N}
      %     }{
      %       1 - ap/q
      %     } {1 \over 1 - p}
      %     \end{eqnarray*}
      %     Thus
      %     \[
      %     \sum_{n=0}^\infty \left[
      %       \E_{-\xi(2s - 1)} f_{2,n}(u,\xi)
      %     \right]^{1/2s\xi} < \infty  
      %     \]

      %   \item If $\xi < 1/(1 + \epsilon)$, choose $s \in (\frac{1}{2\xi}
      %     - \frac{\epsilon - 1}{2}, \frac{\epsilon + 1}{2})$. Note the
      %     assumption $\epsilon\xi > 1/2$ imples $\frac{1}{2\xi} -
      %     \frac{\epsilon - 1}{2} < \frac{\epsilon + 1}{2}$. With such a
      %     choice, $(\cdot)^{1/2s\xi}$ is a convex functon. By Jensen's inequality 
      %     \begin{eqnarray*}
      %       && \E_{-\xi(2s - 1)} \|A\| \\
      %       &=& \frac{\E \|A\|^{1-\xi(2s - 1)}}{\E \|A\|^{-\xi(2s - 1)}}
      %       \\
      %       &\leq& \E \|A\|^{\epsilon \xi} \E \|A\|^{\xi(2s - 1)} \\
      %       &\leq& (\E \|A\|^{\epsilon \xi})^2 < \infty
      %     \end{eqnarray*}
      %     The same argument yields $\E_{-\xi(2s - 1)} \|A\| < \infty$.
      %     So, again using Jensen's inequality, we have
      %     \begin{eqnarray*}
      %       && [\E_{-\xi(2s - 1)} f_{2,n}(u,\xi)]^{1/2s\xi} \\
      %       &\leq& \E_{-\xi(2s - 1)} \left[ f_{2,n}(u,\xi)^{1/2s\xi} \right]\\
      %       &\leq& \sum_{k=n}^\infty
      %       \E_{-\xi(2s - 1)}\|A_k \cdots A_{n+1}\|
      %       \E_{-\xi(2s - 1)}|B_n| \times \\
      %       && \lambda(-\xi(2s - 1))^{k/2s\xi} \1{\tau > T_u}
      %       \P_{-\xi(2s - 1)}(T_u \geq k) \\
      %       &=& (\sum_{k=n}^{n+N-1} + \sum_{k=n+N}^\infty)
      %       \E_{-\xi(2s - 1)}\|A_k \cdots A_{n+1}\|
      %       \E_{-\xi(2s - 1)}|B_n| \times \\
      %       && \lambda(-\xi(2s - 1))^{k/2s\xi} \1{\tau > T_u}
      %       \P_{-\xi(2s - 1)}(T_u \geq k)
      %     \end{eqnarray*}
      %     where $N$ is such that $\forall k \geq n+N$,
      %     $\left[\E_{-\xi(2s - 1)} \|A_k \cdots A_{n+1}\|\right]^{k-n}
      %     \leq a \lambda_{-\xi(2s - 1)}(1)$ for any $a > 1$. The
      %     existence of such $N$ and $a$ is ensured by
      %     \begin{eqnarray*}
      %       \lim_{l \to \infty} \left(
      %         \E_{-\xi(2s - 1)}\|A_l \cdots A_1\|
      %       \right)^{1/l} &=& \lambda_{-\xi(2s - 1)}(1) \\
      %       &=& \frac{\lambda(1 - \xi(2s - 1))}{\lambda(-\xi(2s - 1))} < \infty
      %     \end{eqnarray*}
      %     It follows
      %     \begin{eqnarray*}
      %       && \sum_{k=n}^{n+N-1} \E_{-\xi(2s - 1)}\|A_k \cdots A_{n+1}\|
      %       \E_{-\xi(2s - 1)}|B_n| \times \\
      %       && \lambda(-\xi(2s - 1))^{k/2s\xi} \1{\tau > T_u}
      %       \P_{-\xi(2s - 1)}(T_u \geq k) \\
      %       &\leq& c \sum_{k=n}^{n+N-1} \P_{-\xi(2s - 1)}(\tau \geq k-1)
      %       \\
      %       &=& c \left[ \lambda_{-\xi(2s - 1)}(\alpha - \xi(2s - 1)) \right]^n
      %     \end{eqnarray*}
      %     where $\alpha > 0$ is chosen such that
      %     \[
      %     \lambda_{-\xi(2s - 1)}(\alpha - \xi(2s - 1)) = \frac{
      %     \lambda(\alpha - \xi(2s - 1))
      %   }{
      %     \lambda(-\xi(2s - 1))
      %   } < 1
      %     \]
      %     Such a choice is possible due to \eqref{eq:ret_time}. Meanwhile
      %     \begin{eqnarray*}
      %       && \sum_{k=n+N}^{\infty} \E_{-\xi(2s - 1)}\|A_k \cdots A_{n+1}\|
      %       \E_{-\xi(2s - 1)}|B_n| \times \\
      %       && \lambda(-\xi(2s - 1))^{k/2s\xi} \1{\tau > T_u}
      %       \P_{-\xi(2s - 1)}(T_u \geq k) \\
      %       &\leq& c \sum_{k=n+N}^{\infty} \left[
      %         a \lambda_{-\xi(2s - 1)}(1) \right]^{k-n} \lambda(-\xi(2s -
      %       1))^{k/2s\xi} \times \\
      %       && \P_{-\xi(2s - 1)}(\tau > k-1) \\
      %       &\leq& c \sum_{k=n+N}^\infty \left[
      %         \frac{
      %         a\lambda(1-\xi(2s - 1))
      %       }{
      %         \lambda(-\xi(2s - 1))
      %       }
      %       \right]^{k-n} \times \\
      %       && \lambda(-\xi(2s - 1))^{k/2s\xi}
      %       \lambda(\alpha - \xi(2s - 1))^k \times \\
      %       && \lambda(-\xi(2s - 1))^k
      %     \end{eqnarray*}
      %     Observe $0 < 1 - \xi(2s - 1) < \epsilon \xi$, which imples
      %     $\lambda(1 - \xi(2s - 1)) < \lambda(\epsilon \xi)$. By
      %     assumption $\lambda(\epsilon \xi) < \lambda(-\xi) <
      %     \lambda(-\xi(2s - 1))$, so
      %     \[
      %     \lambda(1 - \xi(2s - 1)) < \lambda(-\xi(2s - 1))
      %     \]
      %     Choose $a > 1$ so close to 1 that even
      %     \[
      %     a \lambda(1 - \xi(2s - 1)) < \lambda(-\xi(2s - 1))
      %     \]
      %     Furthermore,
      %     \begin{eqnarray*}
      %       {1 \over 2s\xi} \log \lambda(-\xi(2s - 1)) &<& {1 \over 2\xi}
      %       \log \lambda(-\epsilon \xi) < - \inf_{\beta \in R} \log
      %       \lambda(\beta)
      %     \end{eqnarray*}
      %     where the 2nd inequality follows from the deepness condition
      %     \eqref{eq:deepness}. Hence $\alpha > 0$ can be chosen such that
      %     \begin{eqnarray*}
      %       \log \lambda(\alpha - \xi(2s - 1)) &<& -{1 \over 2s\xi} \log
      %       \lambda(-\xi(2s - 1)) \\
      %       \lambda(\alpha - \xi(2s - 1)) &<& \lambda(-\xi(2s - 1))^{-1/2s\xi}
      %     \end{eqnarray*}
      %   \end{enumerate}
      % \item Secondly, we show $f_1(u) < \infty$.
      %   \begin{eqnarray*}
      %     && \E_D N_u^{2r} \1{T_u < \tau} \\
      %     &=& \E \left[
      %       N_u^{2r} |A_{T_u} \cdots A_1 X_0|^{\xi}
      %       \frac{r(X_{T_u}; \xi)}{r(X_0; \xi)}
      %       \1{T_u < \tau}
      %     \right]
      %   \end{eqnarray*}
      %   H\"older's inequality gives
      %   \begin{eqnarray*}
      %     && \E_D N_u^{2r} \1{T_u < \tau} \\
      %     &\leq& (\E N_u^{2rt})^{1/t}
      %     (\E |A_{T_u} \cdots A_1 X_0|^{s\xi} \1{T_u < \tau})^{1/s}
      %   \end{eqnarray*}
      %   where $1/s + 1/t = 1$.
      %   \begin{eqnarray*}
      %     && \E N_u^{2rt} \leq \E \tau^{2rt} \leq 1 + \sum_{n=2}^\infty
      %     n^{2rt} \P(\tau > n-1)
      %   \end{eqnarray*}
      %   Applying \eqref{eq:ret_time} with $\beta=0$ yields
      %   \[
      %   \P(\tau > n-1) \leq K \lambda(\alpha)^{n-2}
      %   \]
      %   where $\alpha > 0$ is chosen such that $\lambda(\alpha) < 1$. Thus
      %   \begin{eqnarray*}
      %     && \sum_{n=2}^\infty n^{2rt} \P(\tau > n-1) \leq K \sum_{n=2}^\infty n^{2rt}
      %     \lambda(\alpha)^{n-2} < \infty
      %   \end{eqnarray*}
      
      %   As for the term $\E(|A_{T_u} \cdots A_1 X_0|^{s\xi} \1{T_u <
      %   \tau})$, we have
      %   \begin{eqnarray*}
      %     && \E(|A_{T_u} \cdots A_1 X_0|^{s\xi} \1{T_u < \tau}) \\
      %     &\leq& \sum_{n=1}^{\infty} \E(\|A_n \cdots A_1 \|^{s\xi})
      %     \1{T_u < \tau} \P(T_u = n)
      %   \end{eqnarray*}
      %   As has been argued earlier, $\forall a > 1$, $\exists N \geq 1$
      %   such that $\forall n \geq N$, $\E\|A_n \cdots A_1\|^{s\xi} \leq
      %   [a \lambda(s\xi)]^{n}$. Therefore
      %   \begin{eqnarray*}
      %     && \sum_{n=1}^{\infty} \E(\|A_n \cdots A_1 \|^{s\xi})
      %     \1{T_u < \tau} \P(T_u = n) \\
      %     &\leq& c + \sum_{n=N}^\infty [a \lambda(s\xi)]^{n}
      %     \P(\tau > n-1)
      %   \end{eqnarray*}
      %   Now apply \eqref{eq:ret_time} with $\beta = 0$ and an $\alpha$
      %   that satisfies $\lambda(\alpha) < 1/a\lambda(s\xi)$. Such a choice
      %   of $\alpha$ is possible because $s > 1$ can be chosen
      %   as close to 1 as is necessary to make $\lambda(s\xi) > 1$ as close
      %   to 1 as is desired, while $\inf_{\alpha > 0} \lambda(\alpha) <
      %   1$.

      %   With $a, s, \alpha$ appropriately chosen,
      %   \begin{eqnarray*}
      %     && \sum_{n=1}^{\infty} \E(\|A_n \cdots A_1 \|^{s\xi})
      %     \1{T_u < \tau} \P(\tau > n-1) \\
      %     &\leq& c + K \lambda(\alpha)^{-2}\sum_{n=N}^\infty
      %     [a\lambda(s\xi) \lambda(\alpha)]^{n} < \infty
      %   \end{eqnarray*}
      % \end{enumerate}
      % Now that $f_1(u) < \infty$, $f_2(u, \xi) < \infty$,
      % \[
      % f(\xi) \leq c \limsup_{u \to \infty} f_1(u) f_2(u, \xi) < \infty.
      % \]
      \item If $\xi < 1/2$, the function $(\cdot)^{2\xi}$ is
        subadditive. Hence
        \begin{eqnarray*}
          f(\xi) &\leq& c \limsup_{u \to \infty}
          \sum_{n=0}^\infty \E_D N_u^2
          {|A_{T_u} \cdots A_{n+1}B_n|^{2\xi}
            \over
            |A_{T_u} \cdots A_1 V_0|^{2\xi}
          }
          \1{n \leq T_u < \tau} \\
          &\leq& c \limsup_{u \to \infty}
          \sum_{n=0}^{\infty}
            \E_{-\xi} N_u^2 
            |A_{T_u} \cdots A_{n+1} B_n|^{2\xi}
            \1{n \leq T_u < \tau}
        \end{eqnarray*}
        The same arguments that lead to \eqref{eq:xi_above_half_final}
        show $f(\xi) < \infty$.
    \end{enumerate}
  \item 
  \end{enumerate}
\end{proof}


\bibliographystyle{unsrt}
\bibliography{../../thesis/econophysics}
\end{document}
