\documentclass{article}
\usepackage{amsmath}
\usepackage{amsthm}
\usepackage{enumerate}
\usepackage[bookmarks=true]{hyperref}
\usepackage{bookmark}
\usepackage{graphicx}

\usepackage{amssymb,amsmath,amsthm,amsfonts}
\usepackage{mathrsfs}
\usepackage{dsfont}
\usepackage{enumerate}

%\newtheorem{mdef}{Definition}
%\newtheorem{theorem}{Theorem}
\newcommand{\eqsplit}[2]{
  \begin{equation}\label{#2}
    \begin{split}
      #1
    \end{split}
  \end{equation}}
\newcommand{\eqnsplit}[1]{
  \begin{eqnarray*}
    #1
  \end{eqnarray*}}
\newcommand{\tran}[1]{
  \tilde{#1}
}
\newcommand{\td}[2]{
  \frac{d #1}{d #2}
}
\newcommand{\pd}[2]{
  \frac{\partial #1}{\partial #2}
}
\newcommand{\ppd}[2]{
  \frac{\partial^2 #1}{\partial #2^2}
}
\newcommand{\pdd}[3]{
  \frac{\partial^2 #1}{\partial #2 \partial #3}
}
\newcommand{\otd}[1]{
  \frac{d}{d #1}
}
\newcommand{\opd}[1]{
  \frac{\partial}{\partial #1}
}
\newcommand{\oppd}[1]{
  \frac{\partial^2}{\partial #1^2}
}
\newcommand{\opdd}[2]{
  \frac{\partial^2}{\partial #1 \partial #2}
}
\newcommand{\ket}[1]{
  |#1\rangle
}
\newcommand{\bra}[1]{
  \langle#1|
}
\newcommand{\inn}[1]{
  \langle#1\rangle
}
\newcommand{\mean}[1]{
  \langle#1\rangle
}
\newcommand{\tr}{
  \text{tr}\,
}
\newcommand{\re}{
  \text{Re}\,
}
\newcommand\im{
  \text{Im}\,
}
\newcommand{\var}{
  \text{var}
}
\newcommand{\arcsinh}{
  \sinh^{-1}
}
\newcommand{\arccosh}{
  \cosh^{-1}
}
\newcommand{\erfc}{
  \text{erfc}
}
\newcommand{\E}{
  \mathbb{E}
}
\renewcommand{\P}{
  \mathbb{P}
}
\newcommand{\I}[1]{
  \mathbf{1}_{\{#1\}}
}
\newcommand{\1}[1]{
  \mathds{1}_{\{#1\}}
}
\newcommand{\diag}{
  \text{diag\,}
}
\newcommand{\M}{
  {\text{max}}
}
\newcommand{\m}{
  {\text{min}}
}
\newcommand{\ph}{
  {\text{arg}\,}
}
\newcommand\erf{
  \text{erf}
}
\renewcommand\vec[1]{
  \mathbf{#1}
}
\newcommand\mtx[1]{
  \mathbf{#1}
}
\newcommand\ed{
  \,{\buildrel d \over =}\,
}



\title{Importance Sampling}
% \author{Xie Xiaolei}
% \date{\today}
\begin{document}
\maketitle
\section{Consistency}\label{sec:consistency}
By the law of large numbers
\begin{eqnarray*}
  && \P(|V| > u) \\
  &=& \lim_{n \to \infty} {1 \over n} \sum_{i=0}^n \1{|V_i| > u}
\end{eqnarray*}
Define
\[
R_n := \inf\{0 \leq i \leq n: V_i \in \mathcal C\}
\]
and
\[
K_i := \inf\{k \geq 1: k > K_{i-1}, V_k \in \mathcal C, K_0 = 0\}
\]
Then one can write
\begin{eqnarray*}
  && \lim_{n \to \infty} {1 \over n} \sum_{i=1}^n \1{|V_i| > u} \\
  &=& \lim_{n \to \infty} {1 \over n} \left[
    \sum_{i=0}^{K_{R_n}-1} \1{|V_i| > u} + \sum_{i=K_{R_n}}^n \1{|V_i| > u}
\right]
\end{eqnarray*}
For the 2nd term, by a Borel-Cantelli argument, it maybe shown
\[
\lim_{n \to \infty} {1 \over n}\sum_{i=K_{R_n}}^n \1{|V_i| > u} = 0
\]
For the 1st term,
\begin{eqnarray*}
&& \lim_{n \to \infty} {1 \over n} \sum_{i=0}^{K_{R_n}-1} \1{|V_i| >
  u}  \\
&=& \lim_{n \to \infty} {R_n \over n} {1 \over R_n} \sum_{i=1}^{R_n}
\sum_{j=K_{i-1}}^{K_i-1}\1{|V_i| > u} \\
&=& \pi(\mathcal C) \E_\gamma N_u
\end{eqnarray*}
where the law of large numbers of Markov chains has been used to reach
the last line. In addition, it is assumed
\begin{eqnarray*}
  V_{K_i} &\sim& \gamma \; \forall i \geq 0 \\
  \gamma(E) &=& \pi(E)/\pi(\mathcal C)\; \forall E \in \mathcal
  B(\mathcal C)
\end{eqnarray*}
Define
\begin{eqnarray*}
  T_u &=& \inf\{n \geq 1: |V_n| > u\} \\
  \tau &\overset{d}{=}& K_i - K_{i-1} \\
  N_u &:=& \sum_{i=0}^{\tau-1} \1{|V_i| > u}  \\
\end{eqnarray*}
Then $\E_\gamma N_u$ may be evaluated as
\begin{eqnarray*}
  && \E_\gamma N_u \\
  &=& \E_\gamma N_u \1{T_u < \tau} \\
  &=& \int_{\mathds S^{d-1}} \int_{\mathds R} \cdots \int_{\mathds
    S^{d-1}} \int_{\mathds R} N_u \1{T_u < \tau} \times \\
  && \prod_{i=1}^{T_u} e^{-\xi(s_i - s_{i-1}) + \Lambda(\xi)}
  {r(x_{i-1}; \xi) \over r(x_{i}; \xi)}P_\xi(x_{i-1}, dx_i \times ds_i) \times \\
  && \prod_{i=T_u+1}^{\tau-1} P(x_{i-1}, dx_i \times d s_i)
\end{eqnarray*}
where
\[
P_\xi(x_{i-1}, dx_i \times ds_i) = e^{\xi(s_i - s_{i-1}) -
  \Lambda(\xi)} {r(x_i; \xi) \over r(x_{i-1}; \xi)} P(x_{i-1}, dx_i
\times ds_i)
\]
is the $\xi$-shifted transition kernel of the {\it Markov Additive
  process} $(X_n, S_n)$.
% $\xi$ is chosen such that 
% \[
% \lim_{n \to \infty} \left(
% \E \|A_n \cdots A_1\|^\xi
% \right)^{1/n} = 1
% \]
Thus we have obtained a consistent estimator $\mathcal E_u$ for
$\P(|V| > u)$:
\[
\mathcal E_u := \pi(\mathcal C) \E_\gamma (N_u)
\]

\section{Efficiency}\label{sec:efficiency}
\begin{theorem}
  The estimator $\mathcal E_u$ has bounded relative error, i.e.
  \begin{equation*}
    {\var(\mathcal E_u) \over (\E_D \mathcal E_u)^2} < \infty
  \end{equation*}
\end{theorem}
\bibliographystyle{unsrt}
\bibliography{../thesis/econophysics}
\end{document}
