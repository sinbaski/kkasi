\documentclass{article}

\title{Authorship Issues in Applied Probability Theory and Statistics}
\author{Xie Xiaolei}
\date{\today}

\begin{document}
\maketitle

\section{Introduction}
It is not unusual in the field of applied mathematics that several
researchers collaborate and produce a paper. However, their roles in
producing the paper vary. Quite often one or two people have an idea,
or in mathematical terminology, a conjecture; then, if feasible, they
try to verify the conjecture by computational simulations, i.e. they
develop computer software to create a large number of
realizations of the stochastic process of their interest and then
investigate the statistics of these realizations to find out whether
their initial conjecture is true. If the conjecture is not entirely
consistent with the result of simulation, they may adjust the
conjecture and run the simulation again. Not unexpectedly, these steps
are often iterated.

If the simulations eventually confirm the conjecture, the researchers
try to prove the conjecture with mathematical rigor. Again, this is a
complicated task that involves a considerable number of steps each
producing an intermediate result that is typically termed a lemma.

If the theoretical proof of the conjecture is successfully completed
with a number of supporting lemmas, the researchers are often
interested in real-world examples of their conjecture, which, now with
its rigorous proof, can be called a theorem. Once again, this often
imples programming: Computer software needs to be developed in order
to examine statistical data from sources of their interest, such as
stock trading, meteorological measurements, designed questionare,
etc. and look for signatures of the theorem, often in terms of
distributions of some random variables.

In summary, almost all significant research projects in applied
probability theory and statistics involve computer programming to some
extent. The holders of the initial idea, albeit clever and skilled,
may be not experts in programming and wish to have someone else do the work.
Moreover, even the theoretical development is often so
complicated that it has to be divided and ``outsourced'' to people who
may be unfamiliar with the central ideas of the initial conjecture.

Thus issues arise as to who should be listed as co-authors and in what
order. The rest of this essay discusses such potential issues in
details.


\end{document}