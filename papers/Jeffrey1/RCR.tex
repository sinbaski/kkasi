\documentclass{article}

\title{Authorship Issues in Applied Probability Theory and Statistics}
\author{Xie Xiaolei}

\begin{document}
\maketitle

\section{Introduction}
It is not unusual in the field of applied mathematics that several
researchers collaborate and produce a paper. However, their roles in
producing the paper vary. Quite often one or two people have an idea,
or in mathematical terminology, a conjecture; then, if feasible, they
try to verify the conjecture by computational simulations, i.e. they
develop computer software to create a large number of
realizations of the stochastic process of their interest and then
investigate the statistics of these realizations to find out whether
their initial conjecture is true. If the conjecture is not entirely
consistent with the result of simulation, they may adjust the
conjecture and run the simulation again. Not unexpectedly, these steps
are often iterated.

If the simulations eventually confirm the conjecture, the researchers
try to prove the conjecture with mathematical rigor. Again, this is a
complicated task that involves a considerable number of steps each
producing an intermediate result that is typically termed a lemma.

If the theoretical proof of the conjecture is successfully completed
with a number of supporting lemmas, the researchers are often
interested in real-world examples of their conjecture, which, now with
its rigorous proof, can be called a theorem. Once again, this often
implies programming: Computer software needs to be developed in order
to examine statistical data from sources of their interest, such as
stock trading, meteorological measurements, designed question are,
etc. and look for signatures of the theorem, often in terms of
distributions of some random variables.

In summary, almost all significant research projects in applied
probability theory and statistics involve computer programming to some
extent. The holders of the initial idea, albeit clever and skilled,
may be not experts in programming and wish to have someone else do the work.
Moreover, even the theoretical development is often so
complicated that it has to be divided and ``outsourced'' to people who
may be unfamiliar with the central ideas of the initial conjecture.

Thus issues arise as to who should be listed as co-authors and in what
order. The rest of this essay discusses such potential issues in
details.

\section{Software Developers' Roles \& Authorships}
Development of software in this context often involves two aspects,
firstly acquisition and preparation of data and secondly algorithmic
processing of data. Obviously, only the second part of the development
is related to the mathematical ideas of the work, and hence
intuitively, only the undertakers of the second part should be counted
as co-authors.

However, the first part of the software development, i.e. acquisition
and preparation of data can be creative by its own merit and according
to the Vancouver Recommendations \cite{RCR} (p.36) ``may'' qualify its
undertakers as co-authors. For example, while acquiring data, one may
invent unconventional methods to collect data from sources that are
otherwise inaccessible, or develop database routines that hugely
improve efficiency. More often than not, these software solutions are
not only laborious but also creative. That being said, these solutions
most likely will not be mentioned in the text of the paper since they
do not relate to the central mathematical ideas.


On the other hand, the software developers in question are often
master's students and won't be paid. Therefore, the only plausible way
to acknowledge their efforts is authorship. This is also seen as a
common and acceptable practice.

Nevertheless, it is my opinion that this type of software development
should not lead to authorship -- or more precisely, should not
lead to authorship of a paper on mathematics. It is more appropriate
to describe the innovations in a paper pertaining to computer
science. A possible approach is to have a joint project between
mathematics and computer science, in which researchers from both areas
work together but produce separate papers each describing their own
parts of the project.

\section{Conclusion}
Research in applied probability theory and statistics is often coupled
with software development that may be innovative by its own merit. It
is recommendable to have researchers from both areas work together but
write separate papers so that different creative work can be described
and published with the right authorships and in the right journals.



\begin{thebibliography}{9}

\bibitem{RCR}
  Karsten Klint Jensen, Louise Whiteley and Peter Sand\o e.
  University of Copenhagen,
  1st edition,
  2015.

\end{thebibliography}
\end{document}
