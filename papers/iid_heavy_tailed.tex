\documentclass{article}
\usepackage{amsmath}
\usepackage{enumerate}

\usepackage{amssymb,amsmath,amsthm}
%\newtheorem{mdef}{Definition}
%\newtheorem{theorem}{Theorem}
\newcommand{\eqsplit}[2]{
  \begin{equation}\label{#2}
    \begin{split}
      #1
    \end{split}
  \end{equation}}
\newcommand{\eqnsplit}[1]{
  \begin{eqnarray*}
    #1
  \end{eqnarray*}}
\newcommand{\tran}[1]{
  \tilde{#1}
}
\newcommand{\td}[2]{
  \frac{d #1}{d #2}
}
\newcommand{\pd}[2]{
  \frac{\partial #1}{\partial #2}
}
\newcommand{\ppd}[2]{
  \frac{\partial^2 #1}{\partial #2^2}
}
\newcommand{\pdd}[3]{
  \frac{\partial^2 #1}{\partial #2 \partial #3}
}
\newcommand{\otd}[1]{
  \frac{d}{d #1}
}
\newcommand{\opd}[1]{
  \frac{\partial}{\partial #1}
}
\newcommand{\oppd}[1]{
  \frac{\partial^2}{\partial #1^2}
}
\newcommand{\opdd}[2]{
  \frac{\partial^2}{\partial #1 \partial #2}
}
\newcommand{\ket}[1]{
  |#1\rangle
}
\newcommand{\bra}[1]{
  \langle#1|
}
\newcommand{\inn}[1]{
  \langle#1\rangle
}
\newcommand{\mean}[1]{
  \langle#1\rangle
}
\newcommand{\tr}{
  \text{tr}\,
}
\newcommand{\re}{
  \text{Re}\,
}
\newcommand\im{
  \text{Im}\,
}
\newcommand{\var}{
  \text{var}
}
\newcommand{\arcsinh}{
  \sinh^{-1}
}
\newcommand{\arccosh}{
  \cosh^{-1}
}
\newcommand{\erfc}{
  \text{erfc}
}
\newcommand{\E}{
  \mathbb{E}
}
\renewcommand{\P}{
  \mathbb{P}
}
\newcommand{\I}[1]{
  \mathbf{1}_{\{#1\}}
}
\newcommand{\1}[1]{
  \mathds{1}_{\{#1\}}
}
\newcommand{\diag}{
  \text{diag\,}
}
\newcommand{\M}{
  {\text{max}}
}
\newcommand{\m}{
  {\text{min}}
}
\newcommand{\ph}{
  {\text{arg}\,}
}
\newcommand\erf{
  \text{erf}
}
\renewcommand\vec[1]{
  \mathbf{#1}
}
\newcommand\mtx[1]{
  \mathbf{#1}
}
\newcommand\ed{
  \,{\buildrel d \over =}\,
}



\begin{document}
\section{iid sequences}
We want to prove
\[
a_n^{-2}\|\mtx C - \diag(\mtx C)\|_2 \xrightarrow{P} 0
\]
where $\mtx C$ has elements
\[
c_{ij} = \sum_{t=1}^n X_{it} X_{jt}
\]
where each $X_{it}$ are iid with regularly varying tails such that
$\pr(X_{it} > x) = p_+ L(x) x^{-\alpha}$, $\pr(X_{it} < -x) = p_- L(x) x^{-\alpha}$
for a slowly varying function $L(x)$, and $p_+ + p_- = 1$. The
normalizing sequence $a_n$ is such that $\pr(|x_{it}| > a_n) \sim {1
  \over n}$, implying $a_n \sim n^{1/\alpha}$.
\begin{proof}
  Using the inequality
  \[
  \|\mtx A\|_2 \leq \|\mtx A\|_\infty
  \]
  for an arbitrary matrix $\mtx A$, we have
  \begin{align*}
  \pr(\|\mtx C - \diag(\mtx C)\|_2 \geq a_n^2 \epsilon) & \leq
  \pr(\max_{1\leq i \leq p} \sum_{j=1, j\neq i}^p |\sum_{t=1}^n
  X_{it} X_{jt}| > a_n^2 \epsilon) \\
  &\leq \sum_{i=1}^p \sum_{j=1, j\neq i}^p \pr(|\sum_{t=1}^n
  X_{it} X_{jt}| > a_n^2 \epsilon) \\
  &= \sum_{i=1}^p \sum_{j=1, j\neq i}^p \left[\pr(\sum_{t=1}^n
    X_{it} X_{jt} > a_n^2 \epsilon) + \pr(\sum_{t=1}^n
    X_{it} X_{jt} < -a_n^2 \epsilon) \right]\\
  \end{align*}
  Here we note that $X_{it}$ and $X_{jt}$ are iid.
  \begin{enumerate}[i)]
  \item The case $\alpha < 1$ or $\alpha = 1$ \& $\E|X_{it}| =
    \infty$. Because $a_n^2 \epsilon / a_n \to \infty$, the large
    deviation result found in Cline and Hsing \cite{ClingHsing1998} is
    applicable. With it we obtain
    \begin{align*}
      \pr(\sum_{t=1}^n X_{it} X_{jt} > a_n^2 \epsilon) &\leq
      p_+ n \pr(|X_{it} X_{jt}| > a_n^2 \epsilon)
    \end{align*}
    Now that $|X_{it}|$ and $|X_{jt}|$ are iid and have regularly
    varying tail function $\pr(|X_{it}| > x) = L(x)x^{-\alpha}$,
    $|X_{it} X_{jt}|$ also has reguarly varying tail so that
    $\pr(|X_{it} X_{jt}| > x) = L_2(x) x^{-\alpha}$ according to
    A.H.Jessen and Mikosch \cite{JessenMikosch2006}. Thus we have
    \begin{align*}
      \pr(\sum_{t=1}^n X_{it} X_{jt} > a_n^2 \epsilon) &\leq
      p_+ n \pr(|X_{it} X_{jt}| > a_n^2 \epsilon) \\
      &\sim p_+ n  \epsilon^{-\alpha} \pr(|X_{it} X_{jt}| > a_n^2) \\
    \end{align*}
    Define sequence $b_n$ such that $\pr(|X_{it} X_{jt}| > b_n^2) \sim
    1/n^2$, then it follows $b_n \sim n^{1/\alpha} \sim a_n$. Thus
    \begin{align*}
      \pr(|X_{it} X_{jt}| > a_n^2) &\sim {1 \over n^2} \\
      \pr(\sum_{t=1}^n X_{it} X_{jt} > a_n^2 \epsilon) &\leq
      \epsilon^{-\alpha} {p_+ \over n} \to 0,\; n \to \infty
    \end{align*}

    \item The case $\alpha = 1$ \& $\E |X_{it}| < \infty$.
  \end{enumerate}
  
\end{proof}
\bibliographystyle{unsrt}
\bibliography{../thesis/econophysics}
\end{document}