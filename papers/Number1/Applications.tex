\subsection{Rank-transformed S\&P 500 Data}

For real financial data, it is unreasonable to expect all
the time series under consideration to have the same tail index. So a
preocedure of standardization has to be adopted. In this section we
use the rank transform.

Suppose we have $p$ time series each of length $n$ and have arranged
them in a $p\times n$ matrix $X_t^{(i)}$, where $i = 1, 2, \dots, p$ and
$t=1,2, \dots, n$. For each row of $X$, we transform it as
\begin{eqnarray*}
  R_t^{(i)} &=& -\left[
    \log \left({1 \over n+1} \sum_{\tau=1}^n \1_{X_\tau^{(i)} \leq X_{t}^{(i)}} \right)
  \right]^{-1}
\end{eqnarray*}
When $X_t^{(i)}$ are iid, the argument to the $\log$ function is
uniformly distributed, and hence $R_t^{(i)}$ is a standard Fr\'echet r.v.

Figure \ref{fig:EigenRatio} shows the ratio of successive eigenvalues
of the sample covariance matrix of a number of selected time series
included in the S\&P 500 index.
\begin{figure}[htb!]
  \centering
  \includegraphics[scale=0.6]{EigenRatio.pdf}
  \caption{$\lambda_{(i+1)} / \lambda_{(i)}$ versus $i$}
  \label{fig:EigenRatio}
\end{figure}
