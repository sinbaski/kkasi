%last edited 21.07.2015
\documentclass[11pt,reqno]{amsart}
\usepackage{amsmath,epsfig,graphicx,color}
%\usepackage[english,ngerman]{babel}
%\usepackage[latin1]{inputenc}
\usepackage{ifthen}
\usepackage{dsfont}
\usepackage{shadethm}
\usepackage{framed}
%\usepackage{pstricks}
%\usepackage{graphics}
%\usepackage{tocbibind}
%\usepackage{showkeys}
%\usepackage{units}
\usepackage{mathrsfs}
\usepackage{enumerate}
\usepackage{subfigure}
\usepackage{amssymb,latexsym}
%\usepackage{amsfonts}
%\usepackage{makeidx,showidx}
%\usepackage[sc]{mathpazo}
%\linespread{1.05}

%\usepackage[backref=page]{hyperref}
\usepackage{hyperref}
\hypersetup{
%linktoc=page,
%bookmarks=true, % show bookmarks bar?
%unicode=false, % non-Latin characters in Acrobat? bookmarks
%pdftoolbar=true, % show Acrobat? toolbar?
%pdfmenubar=true, % show Acrobat? menu?
%pdffitwindow=true, % window fit to page when opened
%pdfstartview={FitH}, % fits the width of the page to the window
pdftitle={Paper}, % title
pdfauthor={Johannes Heiny}, % author
pdfsubject={Title}, % subject of the document
%pdfcreator={Creator}, % creator of the document
%pvdfproducer={Producer}, % producer of the document
pdfkeywords={multivariate regular variation} , % list of keywords
%pdfnewwindow=true, % links in new window
%colorlinks=true, % false: boxed links; true: colored links
%linkcolor=blue, % color of internal links
%citecolor=blue, % color of links to bibliography
%filecolor=blue, % color of file links
%urlcolor=blue, % color of external links
urlcolor=black, 
  menucolor=black, 
  citecolor=black, 
  anchorcolor=black, 
  filecolor=black, 
  linkcolor=black, 
  colorlinks=true,
}

\textwidth 6.50in
\topmargin -0.50in
\oddsidemargin 0in
\evensidemargin 0in
\textheight 9.00in
%\pagestyle{plain}

\newcommand{\E}{\mathbb{E}}
\renewcommand{\P}{\mathbb{P}}
\newcommand{\1}{\mathds{1}}
\newcommand{\R}{\mathbb{R}}
\newcommand{\N}{\mathbb{N}}
\newcommand{\C}{\mathbb{C}}
\newcommand{\Z}{\mathbb{Z}}
\newcommand{\Var}{\operatorname{Var}}
\newcommand{\Fo}{\bar{F}}
\renewcommand{\b}[1]{\boldsymbol{#1}}
\newcommand{\Rq}{\mkern 1.5mu\overline{\mkern-1.5mu\R\mkern-3.0mu}\mkern 1.5mu}
\newcommand{\Frechet}{Fr\'{e}chet }
\renewcommand{\Finv}{F^{\gets}}
\DeclareMathOperator{\e}{e}
\newcommand{\inv}[1]{#1^{\gets}}
\newcommand{\x}{\boldsymbol{x}}
\newcommand{\y}{\boldsymbol{y}}
\newcommand{\X}{\boldsymbol{X}}
\newcommand{\Y}{\boldsymbol{Y}}
\newcommand{\bfZ}{\boldsymbol{Z}}
\newcommand{\0}{\boldsymbol{0}}
\newcommand{\dint}{\,\mathrm{d}}
\newcommand{\mB}{\mathcal{B}}
\newcommand{\norm}[1]{\|#1\|}
\newcommand{\twonorm}[1]{\|#1\|_2}
\newcommand{\inftynorm}[1]{\|#1\|_\infty}
\newcommand{\pp}[1]{\varepsilon_{#1}}
\newcommand{\vep}{\varepsilon}
\newcommand{\nto}{n \to \infty}
\newcommand{\xto}{x \to \infty}
\newcommand{\lhs}{left-hand side}
\newcommand{\rhs}{right-hand side}
\newcommand{\fidi}{finite-dimensional distribution}
\newcommand{\rv}{random variable}
\newcommand{\tr}{\operatorname{tr}}
\newcommand{\anp}{a_{np}^{-2}}
\newcommand{\Zbar}{\overline{Z}}
\newcommand{\M}{\text{max}}
\newcommand{\m}{\text{min}}
\newcommand{\floor}[1]{\lfloor #1 \rfloor}

%\renewcommand*{\backref}[1]{}
%\renewcommand*{\backrefalt}[4]{%
%    \ifcase #1 (Not cited.)%
%    \or        (Cited on page~#2.)%
%    \else      (Cited on pages~#2.)%
%    \fi}

%von Schmock
\newcommand{\4}{\mathchoice{\mskip1.5mu}{\mskip1.5mu}{}{}}
\newcommand{\5}{\mathchoice{\mskip-1.5mu}{\mskip-1.5mu}{}{}}
\newcommand{\2}{\penalty250\mskip\thickmuskip\mskip-\thinmuskip} % after comma

\newcommand{\levy}{L\'evy}
\newcommand{\slln}{strong law of large numbers}
\newcommand{\clt}{central limit theorem}
\newcommand{\sde}{stochastic differential equation}
\newcommand{\It}{It\^o}
\newcommand{\sta}{St\u aric\u a}
\newcommand{\ex}{{\rm e}\,}

\def\theequation{\thesection.\arabic{equation}}
\def\tag{\refstepcounter{equation}\leqno }
\def\neqno{\refstepcounter{equation}\leqno(\thesection.\arabic{equation})}

\newtheorem{lemma}{Lemma}[section]
\newtheorem{figur}[lemma]{Figure}
\newtheorem{theorem}[lemma]{Theorem}
\newtheorem{proposition}[lemma]{Proposition}
\newtheorem{definition}[lemma]{Definition}
\newtheorem{corollary}[lemma]{Corollary}
\newtheorem{example}[lemma]{Example}
\newtheorem{exercise}[lemma]{Exercise}
\newtheorem{remark}[lemma]{Remark}
\newtheorem{fig}[lemma]{Figure}
\newtheorem{tab}[lemma]{Table}
\newtheorem{conjecture}[lemma]{Conjecture}

\newcommand{\cid}{\stackrel{d}{\rightarrow}}
\newcommand{\cip}{\stackrel{\P}{\rightarrow}}

\newcommand{\var}{{\rm var}}
\newcommand{\med}{{\rm med}}
\newcommand{\cov}{{\rm cov}}
\newcommand{\corr}{{\rm corr}}
\newcommand{\as}{{\rm a.s.}}
\newcommand{\io}{{\rm i.o.}}
\newcommand{\Holder}{H\"older}

\newcommand{\cadlag}{c\`adl\`ag}

%--------------------------------------------------------------------------------------------
\begin{document}
\today
\bibliographystyle{plain}
\title{Asymptotic Distribution of Upper Order Statistics}

\begin{abstract}

\end{abstract}

\maketitle
% \tableofcontents

\section{Distribution function of upper order statistics in closed form}
\setcounter{equation}{0}
%------------------------------------------------------------------------
Here we consider the probability
\[
\P(Z_{k, n}^2 a_n^{-2} > u)
\]
We assume $\P(|Z| > x) = x^{-\alpha} L(x)$ and that there exists a
monotonically increasing sequence $a_n$ such that $\P(|Z| >
a_n) \sim {1 \over n}$. So we may write
\begin{eqnarray}
  && \P(Z_{k, n}^2 a_n^{-2} > u) \nonumber\\
  &=& \sum_{r=0}^{n-k} {n \choose r} \P(Z^2 \leq ua_n^2)^r \P(Z^2 >
  u a_n^2)^{n-r} \label{eq:A1}
\end{eqnarray}
Denote $S_k(u) = \lim_{n \to \infty}\P(Z_{k, n}^2 a_n^{-2} > u)$. We have
\begin{eqnarray*}
  && S_{k-1}(u) - S_k(u) \\
  &=& \lim_{n \to \infty}{n \choose n-k+1} \P(Z^2 \leq ua_n^2)^{n-k+1} \P(Z^2 > ua_n^2)^{k-1}
\end{eqnarray*}
Recall Stirling's formula
\[
\lim_{n \to \infty} {n! \over (n/e)^n \sqrt{2\pi n}} = 1
\]
and $\P(Z^2 > ua_n^2) = u^{-\alpha/2}/n$. It follows
% \begin{eqnarray*}
%   && {n \choose n-k+1} \P(Z^2 \leq ua_n^2)^{n-k+1} \P(Z^2 > ua_n^2)^{k-1} \\
%   &\leq& {n \choose \floor{n/2}} \exp(-\gamma_k u^{-\alpha/2}) \left(
%     {u^{-\alpha/2} \over n}
%   \right)^{n(1 - \gamma_k)-1} \\
%   &\sim& {4^{\floor{n/2}} \over \floor{n/2} \sqrt{\pi
%       \floor{n/2}}}(\floor{n/2}+1) \exp(-\gamma_k u^{-\alpha/2})
%   \left({u^{-\alpha/2} \over n}\right)^{n(1 - \gamma_k)-1} \\
%   &\to& 0
% \end{eqnarray*}
% Therefore we only need to consider such $k$ that $k/n \to 0$ as $n \to
% \infty$. In this case we have
\begin{eqnarray*}
  && S_{k-1}(u) - S_k(u) \\  
  &=& \lim_{n \to \infty} {n \choose n-k+1} \P(Z^2 \leq ua_n^2)^{n-k+1} \P(Z^2 > u)^{k-1} \\
  &=& \lim_{n \to \infty} {n^{n+1/2} \over (n-k+1)^{n-k+3/2}
    (k-1)^{k-1/2} \sqrt{2 \pi}} \left(1 -
    {u^{-\alpha/2} \over n}\right)^{n-k+1} \left({u^{-\alpha/2} \over
      n}\right)^{k-1} \\
  &=& \lim_{n \to \infty} \left(
    1 + {k - 1 \over n - k + 1}
  \right)^{n-k+3/2} (k-1)^{-k+1/2}
  \exp(-u^{-\alpha/2})u^{-(k-1)\alpha/2} {1 \over \sqrt{2 \pi}}\\
  &=& \exp(k-1 - u^{-\alpha/2})(k-1)^{-(k-1)-1/2}
  u^{-(k-1)\alpha/2} {1 \over \sqrt{2 \pi}}
\end{eqnarray*}
Let $j = k-1$ and write
\begin{eqnarray}
  && S_{j}(u) - S_{j+1}(u) \nonumber \\
  &=& e^{j - u^{-\alpha/2}} j^{-j-1/2} u^{-j\alpha/2} {1 \over
    \sqrt{2 \pi}} \label{eq:A}
\end{eqnarray}
On the other hand, the Markov property of order statistics gives
\[
  S_j(u) = -\lim_{n \to \infty}\int_{0}^u \left[
    {\P(Z^2 > ua_n^2) \over \P(Z^2 > xa_n^2)}
    \right]^j d\P(Z_{j+1, n}^2 > a_n^2 x) 
\]
Before we continue, we prove the existance of the above limit in lemma \ref{lem:A}.
\begin{lemma}\label{lem:A}
  The limit
\[
S_j(u) = -\lim_{n \to \infty}\int_{0}^u \left[
  {\P(Z^2 > ua_n^2) \over \P(Z^2 > xa_n^2)}
\right]^j d\P(Z_{j+1, n}^2 > a_n^2 x)
\]
exists.
\end{lemma}
\begin{proof}
The existence of the limit follows from the principle of dominated
convergence. The arguments are as follows: Let the density function of
$Z_{j+1, n}^2 a_n^{-2}$ be $f_{j+1,n}(x)$. So we may write
\begin{equation}
  \label{eq:A2}
  S_j(u) = \lim_{n \to \infty}\int_{0}^u \left[
    {\P(Z^2 > ua_n^2) \over \P(Z^2 > xa_n^2)}
  \right]^j f_{j+1,n}(x)dx
\end{equation}
First note
\[
\left[
    {\P(Z^2 > ua_n^2) \over \P(Z^2 > xa_n^2)}
  \right]^j = \P(Z_{j,n}^2 > ua_n^2 | Z_{j+1, n}^2 = xa_n^2) \leq 1
\]
Secondly, because the distribution of $Z$ is non-degenerate, its
density function is bounded. So the density function of
$Z_{j,n}^2a_n^{-2}$, which is the negated derivative of
\eqref{eq:A1}, is also bounded, i.e. $\exists b > 0$ such that
$f_{j, n}(x) < b$, for all $j \geq 1$ and $x > 0$. So the
boundedness of the integrand sequence in \eqref{eq:A2} is
established. Next we show the pointwise convergence of the
sequence. By regular variation, it is clear
\[
\lim_{n \to \infty} \left[
    {\P(Z^2 > ua_n^2) \over \P(Z^2 > xa_n^2)}
  \right]^j = \left(
    x \over u
    \right)^{j\alpha/2}
\]
It remains to show that the sequence $[f_{j,n}(x)]_{n=1,2,\dots}$
converges pointwise. For this purpose it suffices to show that
$\P(Z_{j, n}^2 a_n^{-2} > u)$ converges pointwise. Here the argument
is monotone convergence. It is clear $\P(Z_{j, n}^2 a_n^{-2} > u) \leq
1$. In the following we show $\P(Z_{j, n+1}^2 a_{n+1}^{-2} > u) \geq
\P(Z_{j, n}^2 a_n^{-2} > u)$ when $n$ is sufficiently large. Let
\[
q_n(x) = \P(Z^2 > x a_n^2)
\]
and note $q_{n+1}(x) \leq q_n(x)$, due to monotonicity of $a_n$.

\end{proof}

\begin{eqnarray}
   &&-\int_{0}^u\left({x \over u}\right)^{j\alpha/2} dS_{j+1}(x)
    \nonumber \\
    &=& -S_{j+1}(u) + {1 \over u} \int_{0}^u S_{j+1}(x) {j\alpha \over
    2} \left({x \over u}\right)^{j\alpha/2-1} dx \label{eq:B}
\end{eqnarray}
Combining \eqref{eq:A} and \eqref{eq:B} gives
\begin{eqnarray}
  g(u) &=& {j\alpha \over 4}\int_0^u g(x) dx - C_j
  \exp(-u^{-\alpha/2}) \nonumber \\
  {d \over du}\left[
    e^{-ju\alpha/4} g(u)
  \right] &=& -{C_j \alpha \over 2}u^{-\alpha/2 - 1}\exp\left(
    -u^{-\alpha/2} - uj\alpha/4
  \right) \label{eq:C}
\end{eqnarray}
where
\begin{eqnarray*}
  g(u) &=& u^{j\alpha/2} S_{j+1}(u) \\
  C_j &=& {e^j \over 2\sqrt{2\pi}j^{j + 1/2}}
\end{eqnarray*}
Denote the RHS of \eqref{eq:C} as $h(u)$. Since we are only interested
in the behavior of $S_{j}(\epsilon)$ for very small $\epsilon$, we can
taylor expand $e^{-ju\alpha/4}$ as
\[
e^{-ju\alpha/4} = 1 - {ju\alpha \over 4} + o(u)
\]
Therefore
\begin{eqnarray*}
  \int_0^\epsilon h(u) du &=& I_1 + I_2
\end{eqnarray*}
where
\begin{eqnarray*}
  I_1 &=& [1 + o(\epsilon)]\int_0^\epsilon C_j e^{-u^{-\alpha/2}}
  du^{-\alpha/2} \\
  &=& -[1 + o(\epsilon)]C_je^{-\epsilon^{-\alpha/2}}
\end{eqnarray*}
and
\begin{eqnarray*}
  I_2 &=& C_j {j\alpha^2 \over 8} \int_0^\epsilon u^{-\alpha/2}
  e^{-u^{-\alpha/2}}du \\
  &=& C_j {j\alpha \over 4} \Gamma(1 - 2/\alpha, \epsilon^{-\alpha/2})
\end{eqnarray*}
where $\Gamma(\alpha, x)$ is the upper incomplete $\Gamma$
function. So we have obtained
\begin{equation}
  \label{eq:result}
  S_{j+1}(\epsilon) = C_j e^{j\epsilon \alpha/4} \epsilon^{-j\alpha/2}
  \left\{
    {j\alpha \over 4}\Gamma(1- 2/\alpha, \epsilon^{-\alpha/2})
    - [1 + o(\epsilon)]\exp(-\epsilon^{-\alpha/2})
  \right\}
\end{equation}

\end{document}
