\documentclass{article}
\usepackage{amsmath}
\usepackage{enumerate}
\usepackage[bookmarks=true]{hyperref}
\usepackage{bookmark}

\usepackage{amssymb,amsmath,amsthm}
%\newtheorem{mdef}{Definition}
%\newtheorem{theorem}{Theorem}
\newcommand{\eqsplit}[2]{
  \begin{equation}\label{#2}
    \begin{split}
      #1
    \end{split}
  \end{equation}}
\newcommand{\eqnsplit}[1]{
  \begin{eqnarray*}
    #1
  \end{eqnarray*}}
\newcommand{\tran}[1]{
  \tilde{#1}
}
\newcommand{\td}[2]{
  \frac{d #1}{d #2}
}
\newcommand{\pd}[2]{
  \frac{\partial #1}{\partial #2}
}
\newcommand{\ppd}[2]{
  \frac{\partial^2 #1}{\partial #2^2}
}
\newcommand{\pdd}[3]{
  \frac{\partial^2 #1}{\partial #2 \partial #3}
}
\newcommand{\otd}[1]{
  \frac{d}{d #1}
}
\newcommand{\opd}[1]{
  \frac{\partial}{\partial #1}
}
\newcommand{\oppd}[1]{
  \frac{\partial^2}{\partial #1^2}
}
\newcommand{\opdd}[2]{
  \frac{\partial^2}{\partial #1 \partial #2}
}
\newcommand{\ket}[1]{
  |#1\rangle
}
\newcommand{\bra}[1]{
  \langle#1|
}
\newcommand{\inn}[1]{
  \langle#1\rangle
}
\newcommand{\mean}[1]{
  \langle#1\rangle
}
\newcommand{\tr}{
  \text{tr}\,
}
\newcommand{\re}{
  \text{Re}\,
}
\newcommand\im{
  \text{Im}\,
}
\newcommand{\var}{
  \text{var}
}
\newcommand{\arcsinh}{
  \sinh^{-1}
}
\newcommand{\arccosh}{
  \cosh^{-1}
}
\newcommand{\erfc}{
  \text{erfc}
}
\newcommand{\E}{
  \mathbb{E}
}
\renewcommand{\P}{
  \mathbb{P}
}
\newcommand{\I}[1]{
  \mathbf{1}_{\{#1\}}
}
\newcommand{\1}[1]{
  \mathds{1}_{\{#1\}}
}
\newcommand{\diag}{
  \text{diag\,}
}
\newcommand{\M}{
  {\text{max}}
}
\newcommand{\m}{
  {\text{min}}
}
\newcommand{\ph}{
  {\text{arg}\,}
}
\newcommand\erf{
  \text{erf}
}
\renewcommand\vec[1]{
  \mathbf{#1}
}
\newcommand\mtx[1]{
  \mathbf{#1}
}
\newcommand\ed{
  \,{\buildrel d \over =}\,
}



\begin{document}
\section{Proof of the Lemma}
Here we prove that, for a symmetric, positive semidefinite $p \times
p$ matrix $\mtx M = (m_{ij})$, whose $k$th principle sub-matrix is denoted
$M_k$, satisfies
\[
\lambda_1(\mtx M) \leq \lambda_1(M_{k}) + \sum_{i=k+1}^p m_{ii}
\]
\begin{proof}
  Because $M_k$ is positive semidefinite for each $k$ , we only need
  to prove the claim for $k=p-1$. For notational convenience, write
  $A$ for $M_{p-1}$ and $d$ for $m_{pp}$. Next we prove the claim when
  $A$ is positive definite.
  \begin{eqnarray*}
    \mtx M &=&
    \begin{pmatrix}
      A & v \\
      v' & d
    \end{pmatrix} \\
    &=&
    \begin{pmatrix}
      A & 0 \\
      v' & d - v' A^{-1} v
    \end{pmatrix}\left(
      \begin{array}{cc}
        I & -A^{-1} v \\
        0 & 1
      \end{array}\right)^{-1} \\
    &=& U V
  \end{eqnarray*}
  From $\det(M) \geq 0$ it follows $\det(A) (d - v' A^{-1} v) \geq
  0$. Since $\det(A) > 0$, $d - v' A^{-1} v \geq 0$. Moreover, $A$ is
  assumed positive definite, so $v' A^{-1} v \geq 0$. According
  to Bai and Silverstein\cite{BaiSilverstein2010} we have
  $\lambda_1(M) \leq s_1(U)$, where $s_1(U)$ denotes the largest
  singular value of $U$.
  \begin{eqnarray*}
    \lambda_1(M) &\leq& s_1(U) \\
    &=& \max\left\{\lambda_1(A), d - v' A^{-1} v \right\} \\
    &\leq& \lambda_1(A) + d
  \end{eqnarray*}
  Now we consider the case where $ d > 0$ and $A$ is not positive
  definite, i.e. it contains zero eigenvalues. When $A$ has only zero
  eigenvalues, our claim is clearly true; when $A$ has at least 1
  non-zero eigenvalue, we may apply elementary transforms to $M$,
  or in other words, multiply $M$ by an upper triangular matrix $T$
  and its transpose $T'$
  \begin{eqnarray*}
    T &=& 
    \begin{pmatrix}
      1 & T_{12} & \cdots & T_{1,p} \\
      0 & 1 & \cdots & T_{2, p} \\
      \vdots & \vdots & \ddots & \vdots \\
      0 & 0 & \cdots & 1 \\
    \end{pmatrix}
  \end{eqnarray*}
  so that
  \begin{eqnarray*}
    T M T'&=&
    \begin{pmatrix}
      0 & 0 & u_1 \\
      0 & A_2 & u_2 \\
      u'_1 & u'_2 & d
    \end{pmatrix} \\
    M  &=& T^{-1} M_1 T'^{-1}
  \end{eqnarray*}
  where $A_2$ is positive definite and has the same rank as
  A. Obviously $A_2$ has the same non-zero eigenvalues as $A$.
  Because $M$ is positive semi-definite, so is $M_1$. Then it follows
  $|M_{1;ij}| \leq \sqrt{M_{1;i,i} M_{1;j,j}}$ for all $1 \leq i,j
  \leq p$, implying $u_1 = 0$. Now we have
  \begin{eqnarray*}
    M &=& T^{-1} M_1 T'^{-1} \\
    &=& T^{-1}
    \begin{pmatrix}
      0 & 0 & 0 \\
      0 & A_2 & u_2 \\
      0 & u'_2 & d
    \end{pmatrix} T'^{-1} \\
  \end{eqnarray*}
  By the same inequality found in \cite{BaiSilverstein2010},
  \begin{eqnarray*}
    \lambda_1(M) &\leq& s_1(T^{-1}) \lambda_1(M_1) s_1(T'^{-1}) \\
    &=& \lambda_1(M_1) \\
    &=& \lambda_1 \left[\left(
      \begin{array}{cc}
        A_2 & u_2 \\
        u'_2 & d
      \end{array} \right)\right] \\
  &\leq& \lambda_1(A_2) + d
  \end{eqnarray*}
  where in the last inequality we apply the claim proven in the case
  of positive definite principle sub-matrix.
\end{proof}
\bibliographystyle{unsrt}
\bibliography{../thesis/econophysics}
\end{document}
