\documentclass{article}
\usepackage{amsmath}
\usepackage{amsthm}
\usepackage{enumerate}
\usepackage{subfigure}
\usepackage[bookmarks=true]{hyperref}
\usepackage{bookmark}
\usepackage{graphicx}

\usepackage{amssymb,amsmath,amsthm,amsfonts}
\usepackage{mathrsfs}
\usepackage{dsfont}
\usepackage{enumerate}

%\newtheorem{mdef}{Definition}
%\newtheorem{theorem}{Theorem}
\newcommand{\eqsplit}[2]{
  \begin{equation}\label{#2}
    \begin{split}
      #1
    \end{split}
  \end{equation}}
\newcommand{\eqnsplit}[1]{
  \begin{eqnarray*}
    #1
  \end{eqnarray*}}
\newcommand{\tran}[1]{
  \tilde{#1}
}
\newcommand{\td}[2]{
  \frac{d #1}{d #2}
}
\newcommand{\pd}[2]{
  \frac{\partial #1}{\partial #2}
}
\newcommand{\ppd}[2]{
  \frac{\partial^2 #1}{\partial #2^2}
}
\newcommand{\pdd}[3]{
  \frac{\partial^2 #1}{\partial #2 \partial #3}
}
\newcommand{\otd}[1]{
  \frac{d}{d #1}
}
\newcommand{\opd}[1]{
  \frac{\partial}{\partial #1}
}
\newcommand{\oppd}[1]{
  \frac{\partial^2}{\partial #1^2}
}
\newcommand{\opdd}[2]{
  \frac{\partial^2}{\partial #1 \partial #2}
}
\newcommand{\ket}[1]{
  |#1\rangle
}
\newcommand{\bra}[1]{
  \langle#1|
}
\newcommand{\inn}[1]{
  \langle#1\rangle
}
\newcommand{\mean}[1]{
  \langle#1\rangle
}
\newcommand{\tr}{
  \text{tr}\,
}
\newcommand{\re}{
  \text{Re}\,
}
\newcommand\im{
  \text{Im}\,
}
\newcommand{\var}{
  \text{var}
}
\newcommand{\arcsinh}{
  \sinh^{-1}
}
\newcommand{\arccosh}{
  \cosh^{-1}
}
\newcommand{\erfc}{
  \text{erfc}
}
\newcommand{\E}{
  \mathbb{E}
}
\renewcommand{\P}{
  \mathbb{P}
}
\newcommand{\I}[1]{
  \mathbf{1}_{\{#1\}}
}
\newcommand{\1}[1]{
  \mathds{1}_{\{#1\}}
}
\newcommand{\diag}{
  \text{diag\,}
}
\newcommand{\M}{
  {\text{max}}
}
\newcommand{\m}{
  {\text{min}}
}
\newcommand{\ph}{
  {\text{arg}\,}
}
\newcommand\erf{
  \text{erf}
}
\renewcommand\vec[1]{
  \mathbf{#1}
}
\newcommand\mtx[1]{
  \mathbf{#1}
}
\newcommand\ed{
  \,{\buildrel d \over =}\,
}



\title{Eigenvalues and Eigenvectors of Correlation/Covariance Matrices}
\author{Xie Xiaolei}
\date{\today}
\begin{document}
\maketitle
\section{Eigenvalues}
% \begin{figure}[htb!]
%   \centering
%   \subfigure[]{
%     \includegraphics[scale=0.3]{Eigenvalues_of_CovarianceMatrix_SP500.pdf}
%   }
%   \subfigure[]{
%     \includegraphics[scale=0.3]{Eigenvalues_of_CorrelationMatrix_SP500.pdf}
%   }
%   \caption{$\log_{10}(\lambda_{(i+1)} / \lambda_{(i)})$ versus $i$}
%   \label{fig:EigenRatio}
% \end{figure}
\begin{figure}[htb!]
  \centering
  \includegraphics[scale=0.5]{Eigenvalues_Cor_n_Cov_Matrix.pdf}  
  \caption{$\log_{10}(\lambda_{(i)})$}
  \label{fig:EigenRatio}
\end{figure}

\section{Eigenvectors}
% \begin{figure}[htb!]
%   \centering
%   \subfigure[]{
%     \includegraphics[scale=0.3]{LargestComponent_of_Eigenvectors_of_CovarianceMatrix_SP500.pdf}
%   }
%   \subfigure[]{
%     \includegraphics[scale=0.3]{LargestComponent_of_Eigenvectors_of_CorrelationMatrix_SP500.pdf}
%   }
%   \caption{$\max_{1 \leq k \leq p} |V_{ki}|$, where $V_{ki}$ is the
%     $k$-th component of the eigenvector associated with the $i$-th
%     largest eigenvalue.}
%   \label{fig:Eigenvector_largest_comp}
% \end{figure}
\begin{figure}[htb!]
  \centering
  \subfigure[Covariance Matrix]{
    \includegraphics[scale=0.3]{V1_CovarianceMatrix.pdf}
  }
  \subfigure[Correlation Matrix]{
    \includegraphics[scale=0.3]{V1_CorrelationMatrix.pdf}
  }
  \caption{Eigenvector corresponding to the largest eigenvalue}
  \label{fig:Eigenvector_largest_comp}
\end{figure}
\bibliographystyle{unsrt}
\bibliography{../thesis/econophysics}
\end{document}
