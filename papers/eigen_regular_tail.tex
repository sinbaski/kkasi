\documentclass{article}
\usepackage{amsmath}
\usepackage{enumerate}
\usepackage[bookmarks=true]{hyperref}
\usepackage{bookmark}

\usepackage{amssymb,amsmath,amsthm,amsfonts}
\usepackage{mathrsfs}
\usepackage{dsfont}
\usepackage{enumerate}

%\newtheorem{mdef}{Definition}
%\newtheorem{theorem}{Theorem}
\newcommand{\eqsplit}[2]{
  \begin{equation}\label{#2}
    \begin{split}
      #1
    \end{split}
  \end{equation}}
\newcommand{\eqnsplit}[1]{
  \begin{eqnarray*}
    #1
  \end{eqnarray*}}
\newcommand{\tran}[1]{
  \tilde{#1}
}
\newcommand{\td}[2]{
  \frac{d #1}{d #2}
}
\newcommand{\pd}[2]{
  \frac{\partial #1}{\partial #2}
}
\newcommand{\ppd}[2]{
  \frac{\partial^2 #1}{\partial #2^2}
}
\newcommand{\pdd}[3]{
  \frac{\partial^2 #1}{\partial #2 \partial #3}
}
\newcommand{\otd}[1]{
  \frac{d}{d #1}
}
\newcommand{\opd}[1]{
  \frac{\partial}{\partial #1}
}
\newcommand{\oppd}[1]{
  \frac{\partial^2}{\partial #1^2}
}
\newcommand{\opdd}[2]{
  \frac{\partial^2}{\partial #1 \partial #2}
}
\newcommand{\ket}[1]{
  |#1\rangle
}
\newcommand{\bra}[1]{
  \langle#1|
}
\newcommand{\inn}[1]{
  \langle#1\rangle
}
\newcommand{\mean}[1]{
  \langle#1\rangle
}
\newcommand{\tr}{
  \text{tr}\,
}
\newcommand{\re}{
  \text{Re}\,
}
\newcommand\im{
  \text{Im}\,
}
\newcommand{\var}{
  \text{var}
}
\newcommand{\arcsinh}{
  \sinh^{-1}
}
\newcommand{\arccosh}{
  \cosh^{-1}
}
\newcommand{\erfc}{
  \text{erfc}
}
\newcommand{\E}{
  \mathbb{E}
}
\renewcommand{\P}{
  \mathbb{P}
}
\newcommand{\I}[1]{
  \mathbf{1}_{\{#1\}}
}
\newcommand{\1}[1]{
  \mathds{1}_{\{#1\}}
}
\newcommand{\diag}{
  \text{diag\,}
}
\newcommand{\M}{
  {\text{max}}
}
\newcommand{\m}{
  {\text{min}}
}
\newcommand{\ph}{
  {\text{arg}\,}
}
\newcommand\erf{
  \text{erf}
}
\renewcommand\vec[1]{
  \mathbf{#1}
}
\newcommand\mtx[1]{
  \mathbf{#1}
}
\newcommand\ed{
  \,{\buildrel d \over =}\,
}



\begin{document}
% \section{iid sequences}\label{sec:iid}
% Consider iid sequences $\{X_{it}\}$, $1 \leq i \leq p$, $1 \leq t \leq
% n$. $p$ is fixed while $n \to \infty$. $\pr(X_{it} > x) = p_+ L(x)
% x^{-\alpha}$, $\pr(X_{it} < -x) = p_- L(x) x^{-\alpha}$ for a slowly
% varying function $L(x)$, and $p_+ + p_- = 1$. The normalizing sequence
% $a_n$ is such that $\pr(|x_{it}| > a_n) \sim {1 \over n}$, implying
% $a_n \sim n^{1/\alpha} l(n)$, where $l(n)$ is a slowing varying
% function. The sample covariance matrix is
% \[
% \mtx C = \sum_{t=1}^n \mtx X_t \mtx{X'_t}
% \]
% where $\mtx X_t = (X_{1t}, \cdots, X_{pt})'$.
% For convenience, we let $c$ stand for any constant whose value is not
% of importance in the rest of this report.

% \subsection[E(X) infinite, alpha in {(0, 1]}]{\texorpdfstring{The case
%     $\E|X_{it}| = \infty$ with $\alpha \in (0, 1]$}}
% In this case we want to prove
% \[
% a_n^{-2}\|\mtx C - \diag(\mtx C)\|_2 \xrightarrow{P} 0
% \]

% \begin{proof}
%   Using the inequality
%   \[
%   \|\mtx A\|_2 \leq \|\mtx A\|_\infty
%   \]
%   for an arbitrary matrix $\mtx A$, we have
%   \begin{align*}
%     \pr(\|\mtx C - \diag(\mtx C)\|_2 \geq a_n^2 \epsilon) & \leq
%     \pr(\max_{1\leq i \leq p} \sum_{j=1, j\neq i}^p |\sum_{t=1}^n
%     X_{it} X_{jt}| > a_n^2 \epsilon) \\
%     &\leq \sum_{i=1}^p \sum_{j=1, j\neq i}^p \pr(|\sum_{t=1}^n
%     X_{it} X_{jt}| > a_n^2 \epsilon) \\
%     &= \sum_{i=1}^p \sum_{j=1, j\neq i}^p \left[\pr(\sum_{t=1}^n
%       X_{it} X_{jt} > a_n^2 \epsilon) + \pr(\sum_{t=1}^n
%       X_{it} X_{jt} < -a_n^2 \epsilon) \right]\\
%   \end{align*}
%   Here we note that $X_{it}$ and $X_{jt}$ are iid, and, assuming $\alpha
%   < 1$, $\E |X_{it} X_{jt}| = \infty$. Furthermore, $a_n^2 \epsilon /
%   a_n \to \infty$. Hence the large deviation result found in 
%   Cline and Hsing \cite{ClingHsing1998} is applicable. With it we
%   obtain
%   \begin{align*}
%     \pr(\sum_{t=1}^n X_{it} X_{jt} > a_n^2 \epsilon) &\leq
%     c n \pr(|X_{it} X_{jt}| > a_n^2 \epsilon)
%   \end{align*}
%   Now that $|X_{it}|$ and $|X_{jt}|$ are iid and have regularly
%   varying tail function $\pr(|X_{it}| > x) = L(x)x^{-\alpha}$,
%   $|X_{it} X_{jt}|$ also has reguarly varying tail so that
%   $\pr(|X_{it} X_{jt}| > x) = L_2(x) x^{-\alpha}$ according to
%   A.H.Jessen and Mikosch \cite{JessenMikosch2006}. Thus we have
%   \begin{align*}
%     \pr(\sum_{t=1}^n X_{it} X_{jt} > a_n^2 \epsilon) &\leq
%     c n \pr(|X_{it} X_{jt}| > a_n^2 \epsilon) \\
%     &\sim c n a_n^{-2\alpha} \\
%     &= c l(n)^{-2\alpha} n^{-1} \to 0
%   \end{align*}
%   Similarly
%   \begin{eqnarray*}
%     \pr(\sum_{t=1}^n X_{it} X_{jt} < -a_n^2 \epsilon) &\leq& c n
%     a_n^{-2\alpha} \to 0
%   \end{eqnarray*}
% \end{proof}

% \subsection[E(X) finite, alpha in {[1,2]}]{\texorpdfstring{The case
%     $\E|X_{it}| < \infty$ and $\alpha \in [1,2]$}} \label{sec:B}
% In this case we prove a similar result on
% \[
% \mtx A = \mtx C - \E \mtx C
% \]
% In the following we prove
% \[
% a_n^{-2} \|\mtx A - \diag(\mtx A)\|_2 \xrightarrow{P} 0
% \]
% \begin{proof}
%   \begin{eqnarray}
%     && \pr(a_n^{-2} \|\mtx A - \diag(\mtx A)\|_2 > \epsilon) \nonumber \\
%     &\leq& \pr(a_n^{-2} \|\mtx A - \diag(\mtx A)\|_\infty > \epsilon) \nonumber \\
%     &=& \pr(\max_{1\leq i \leq p} \sum_{j=1, j\neq i}^p |\sum_{t=1}^n
%     X_{it} X_{jt} - \E X_{it} X_{jt} | > a_n^2 \epsilon) \nonumber \\
%     &\leq& \sum_{i=1}^p \sum_{j=1, j\neq i}^p \pr(|\sum_{t=1}^n
%     X_{it} X_{jt} - \E X_{it} X_{jt} | > a_n^2 \epsilon) \nonumber \\
%     &=& \sum_{i=1}^p \sum_{j=1, j\neq i}^p \left[\pr(\sum_{t=1}^n
%       X_{it} X_{jt} - \E X_{it} X_{jt} > a_n^2 \epsilon) + \right.\nonumber \\
%     && \left. \pr(\sum_{t=1}^n X_{it} X_{jt} - \E X_{it} X_{jt} < -a_n^2
%     \epsilon) \right] \label{eq:A}
%   \end{eqnarray}
%   To use Cline and Hsing's result \cite{ClingHsing1998} of large
%   deviations, we first need to verify $\E |X_{it} X_{jt} - \E X_{it} X_{jt}|
%   < \infty$. It is obvious $\E (X_{it} X_{jt} - \E X_{it} X_{jt}) = 0$.
%   \begin{align*}
%     \E |X_{it} X_{jt} - \E X_{it} X_{jt}| &\leq \E |X_{it} X_{jt}| +
%     \E |\E X_{it} X_{jt}|
%   \end{align*}
%   By Jensen's inequality
%   \[
%   |\E X_{it} X_{jt}| \leq \E |X_{it} X_{jt}|
%   \]
%   Therefore
%   \begin{align*}
%     \E |X_{it} X_{jt} - \E X_{it} X_{jt}| &\leq 2 \E |X_{it} X_{jt}|
%     \\
%     &= 2 \E|X_{it}| \E| X_{jt}| < \infty
%   \end{align*}
%   Secondly, we need to establish that, for any sequence $(b_n)$ satisfying
%   \begin{equation}
%     \label{eq:B2}
%     \pr\left(
%       |X_{it} X_{jt} - \E X_{it} X_{jt}| > b_n
%     \right) \sim 1/n
%   \end{equation}
%   we have $a_n^2 \epsilon / b_n \to \infty$, which is required by
%   Cline and Hsing \cite{ClingHsing1998}. For this reason, we note
%   \begin{eqnarray*}
%     && \pr\left(
%     |X_{it} X_{jt} - \E X_{it} X_{jt}| > b_n
%   \right) \\
%   &=& \pr(X_{it} X_{jt} > b_n + \E X_{it} X_{jt}) + \pr(X_{it} X_{jt} <
%   -b_n + \E X_{it} X_{jt})
%   \end{eqnarray*}
%   Because $X_{it} X_{jt}$ has regularly varying tail with index
%   $-\alpha$ according to A.H.Jessen and Mikosch
%   \cite{JessenMikosch2006}, we have
%   \begin{eqnarray*}
%     && \pr(X_{it} X_{jt} > b_n + \E X_{it} X_{jt}) + \pr(X_{it} X_{jt} <
%     -b_n + \E X_{it} X_{jt}) \\
%     &\sim& L_1(b_n + \E X_{it} X_{jt}) (b_n + \E X_{it}
%     X_{jt})^{-\alpha} \\
%     && +L_2(b_n - \E X_{it} X_{jt}) (b_n - \E X_{it}
%     X_{jt})^{-\alpha} \\
%     &\sim& [L_1(b_n) + L_2(b_n)] b_n^{-\alpha}
%   \end{eqnarray*}
%   where $L_1$ and $L_2$ are slowing varying functions. Then it follows
%   $b_n = n^{1/\alpha} l_1(n)$ for a slowing varying function
%   $l_1(n)$. Apparently
%   \begin{eqnarray*}
%     {a_n^2 \over b_n} &=& {l^2(n) \over l_1(n)} n^{1/\alpha} \to \infty
%   \end{eqnarray*}
%   Therefore Cline and Hsing's large deviation result
%   \cite{ClingHsing1998} gives
%   \begin{eqnarray}
%     && \pr(\sum_{t=1}^n X_{it} X_{jt} - \E X_{it} X_{jt} > a_n^2
%     \epsilon) \nonumber \\
%     &\sim& c n \pr(|X_{it} X_{jt} - \E X_{it} X_{jt}| > a_n^2
%     \epsilon) \nonumber \\
%     &\sim& cn [L_1(a_n^2) + L_2(a_n^2)] a_n^{-2\alpha} \nonumber \\
%     &\sim& c {L_1(a_n^2) + L_2(a_n^2) \over l^{2\alpha}(n)} n^{-1}
%     \to 0 \text{ , }n \to \infty \label{eq:B3}
%   \end{eqnarray}
%   Analogously,
%   \begin{eqnarray}
%     && \pr(\sum_{t=1}^n X_{it} X_{jt} - \E X_{it} X_{jt} < -a_n^2
%     \epsilon) \nonumber \\
%     &\sim& c n \pr(|X_{it} X_{jt} - \E X_{it} X_{jt}| > a_n^2 \to 0
%     \text{ , }n \to \infty \label{eq:B4}
%   \end{eqnarray}

% \end{proof}

% \subsection[alpha in (2,4)]{\texorpdfstring{The case $\alpha \in (2, 4)$}}
% As in the case of \S\ref{sec:B}, we shall prove
% \[
% a_n^{-2} \|\mtx A - \diag(\mtx A)\|_2 \xrightarrow{P} 0
% \]
% \begin{proof}
%   The arguments are the same as those leading to \eqref{eq:A}. Now
%   that $\alpha > 2$, we need to verify $a_n^2\epsilon > \sqrt{(\alpha
%     - 2)n \ln n}$ in order to use the large deviation result of Nagaev
%   \cite{nagaev1979}.
%   \begin{eqnarray*}
%     {a_n^2 \over \sqrt{(\alpha - 2)n \ln n}} &=&
%     {l^2(n) \over \sqrt{(\alpha - 2)\ln n}} n^{2/\alpha - 1/2}
%   \end{eqnarray*}
%   Since $2 < \alpha < 4$, the RHS of the last expression tends to
%   $\infty$. So the condition $a_n^2\epsilon > \sqrt{(\alpha
%     - 2)n \ln n}$ is satisfied. Therefore, by the same arguments as in
%   \S\ref{sec:B}, we reach the conclusions
%   \begin{eqnarray*}
%     \pr(\sum_{t=1}^n X_{it} X_{jt} - \E X_{it} X_{jt} > a_n^2
%     \epsilon) &\to& 0 \\
%     \pr(\sum_{t=1}^n X_{it} X_{jt} - \E X_{it} X_{jt} < -a_n^2
%     \epsilon) &\to& 0
%   \end{eqnarray*}
% \end{proof}
% \section{Correlated Sequences}
% Now we consider the situation where the sequences $X_{it}$ is
% constructed as
% \[
% X_{it} = \sum_{k=0}^{\infty} \sum_{l=0}^{\infty} h_{kl} Z_{i-k, t-l}
% \]
% where $Z_{a,b}$ are iid and $a,b \in \mathbb{Z}$ while $i=1,\cdots,p$,
% $t=1,\cdots,n$ with $p$ fixed and $n \to \infty$. $Z_{a,b}$ has
% regularly varying tails as specified in \S\ref{sec:iid}.
% Define matrix $\mtx M = \mtx {H H'}$ and matrix $\mtx M^{(m)}_q$ with
% elements
% \[
% M^{(m)}_{q; i,j} = \left\{
%   \begin{array}{ll}
%     M_{i-q, j-q} & \text{if } i,j = q, q+1, \cdots, q+m \\
%     0 & \text{otherwise}
%   \end{array}
% \right.
% i,j = 1,2,\cdots,p
% \]
% In addition define $\mtx X^{(m)}$ to have elements
% \[
% X^{(m)}_{ij} = \sum_{k=0}^{m} \sum_{l=0}^{\infty} h_{kl} Z_{i-k, t-l}
% \]
% and
% \[
% D_i = \sum_{j=1}^p Z_{ij}^2
% \]
% Following the arguments of Davis, Mikosch and
% Pfaffel\cite{Mikosch2014} and replacing $a_{np}$ with $a_n$, it can be
% proven
% \begin{enumerate}[i)]
% \item if $\alpha \in (0, 1]$ and $\E|Z| = \infty$ or if $\alpha \in
%   [1, 2)$, $\E|Z| < \infty$ and $\E Z = 0$.
%   \[
%   \lim_{m\to \infty} \limsup_{n\to\infty} \pr\left(
%     \|\mtx{X X'} - \mtx X^{(m)} \mtx X'^{(m)}\|_2 \geq a_n^2 \epsilon
%   \right) = 0 \text{, } \forall \epsilon > 0  
%   \]
% \item if $\alpha \in [2, 4)$ and $\E Z = 0$
%   \begin{align*}
%     & \lim_{m\to \infty} \limsup_{n\to\infty} \pr\left(
%       \|(\mtx{XX'} - \E\mtx {XX'}) - (\mtx{X^{(m)} X'^{(m)}} - \E
%       \mtx{X^{(m)} X'^{(m)}})\|_2 \geq a_n^2 \epsilon \right) \\
%     &= 0 \text{, } \forall \epsilon > 0
%   \end{align*}
% \end{enumerate}

% % \subsection[Approximating XX' when alpha in (0,2)]{Approximating $XX'$
% %   when $\alpha \in (0,2)$}
% % \begin{eqnarray*}
% %   \|XX' - X^{(m)}(X^{(m)})'\|_2 &\leq& \|XX' -
% %   X^{(m)}(X^{(m)})'\|_\infty \\
% %   &=& \max_{i = 1,\dots,p} \sum_{j=1}^p \left|
% %     \sum_{t=1}^n \sum_{k \vee k' > m} \sum_{l,l'=0}^\infty
% %     h_{k,l}h_{k',l'} Z_{i-k, t-l} Z_{j-k', t-l'} \right| \\
% %   &=& \max_{i = 1,\dots,p} \sum_{j=1}^p \left|
% %     \sum_{t=1}^n \sum_{k \vee (k+j-i) > m} \sum_{l=0}^\infty
% %     h_{k,l}h_{j-i+k,l} Z_{i-k, t-l}^2 \right| \\
% %   && + \max_{i = 1,\dots,p} \sum_{j=1}^p \left|
% %     \sum_{t=1}^n \sum_{k \vee k'>m} \sum_{l \neq l'}
% %     h_{k,l} h_{k',l'} Z_{i-k, t-l} Z_{j-k',t-l} \right| \\
% %   && + \max_{i = 1,\dots,p} \sum_{j=1}^p \left|
% %     \sum_{t=1}^n \sum_{\substack{k \vee k'>m \\i-k \neq j-k'}}
% %     \sum_{l=0}^\infty h_{k,l}h_{k',l'} Z_{i-k, t-l} Z_{j-k',t-l}
% %   \right| \\
% %   &=& I_n^{(1)} + I_n^{(2)} + I_n^{(3)}
% % \end{eqnarray*}

% \subsection[Approximation by DM when alpha in (0,2)]{Approximating
%   $X^{(m)}(X^{(m)})'$ when $\alpha \in (0, 2)$}
% We have in this case
% \[
% a_n^{-2} \|\mtx X^{(m)} (\mtx X^{(m)})' - \sum_{q=-m}^p D_q \mtx M_q^{(m)}\|_2
% \xrightarrow{P} 0
% \]
% \begin{proof}
%   The proof is essentially the same as its counterpart in Davis,
%   Mikosch and Pfaffel \cite{Mikosch2014}. The $(i,j)$-th element of
%   $\mtx{X^{(m)} X'^{(m)}}$ is
%   \begin{eqnarray*}
%     (\mtx{X^{(m)} X'^{(m)}})_{ij} &=&
%     \sum_{t=1}^n \sum_{k=0}^m \sum_{l=0}^\infty \sum_{k'=0}^m
%     \sum_{l'=0}^\infty h_{kl} h_{k'l'} Z_{i-k, t-l} Z_{j-k', t-l'} \\
%     &=& \sum_{t=1}^n \sum_{k=0}^m \sum_{l=0}^\infty h_{kl}h_{j-i+k,l}
%     Z_{i-k, t-l}^2 \\
%     && + \sum_{t=1}^n \sum_{k, k'=0}^m \sum_{\substack{l,l'=0\\l\neq l'}}^\infty
%     h_{kl} h_{k'l'} Z_{i-k, t-l} Z_{j-k', t-l'} \\
%     && + \sum_{t=1}^n \sum_{\substack{k,k'=0\\ i-k \neq j-k'}}^m \sum_{l=0}^{\infty}
%     h_{kl} h_{k'l'} Z_{i-k, t-l} Z_{j-k', t-l} \\
%     &=& I^{(1)}_{ij} + I^{(2)}_{ij} + I^{(3)}_{ij}
%   \end{eqnarray*}
%   Next we prove
%   \begin{eqnarray*}
%     \pr\left(
%       a_n^{-2} \|\mtx I^{(1)} - \sum_{q=-m}^p D_q \mtx M_{q}^{(m)}\|_2 > \epsilon
%     \right) \to 0 \text{, as } n \to 0
%   \end{eqnarray*}
%   From $\|\cdot\|_2 \leq \|\cdot\|_\infty$ it follows
%   \begin{eqnarray*}
%     && \pr\left(
%       a_n^{-2} \|\mtx I^{(1)} - \sum_{q=-m}^p D_q \mtx M_{q}^{(m)}\|_2 > \epsilon
%     \right) \\
%     &\leq& \sum_{i=1}^p \sum_{j=1}^p \pr\left(
%       \left| I^{(1)}_{ij} - \sum_{q=-m}^p D_q M_{q; ij}^{(m)} \right| > a_n^2
%       \epsilon \right)
%   \end{eqnarray*}
%   Observe
%   \begin{eqnarray*}
%     && I^{(1)}_{ij} - \sum_{q=-m}^p D_q M_{q; ij}^{(m)} \\
%     &=& \sum_{t=1}^n \sum_{k=0}^m \sum_{l=0}^\infty h_{kl}h_{j-i+k,l}
%     Z_{i-k, t-l}^2 - \sum_{k=i-p}^{i+m} D_{i-k} M_{i-k; ij}^{(m)}
%   \end{eqnarray*}
%   Note that $M_{i-k; ij}^{(m)}$ is non-zero only if $i-k \leq i \leq
%   i-k+m$, that is, if $0 \leq k \leq m$. Hence we can write
%   \begin{eqnarray*}
%     && \sum_{t=1}^n \sum_{k=0}^m \sum_{l=0}^\infty h_{kl}h_{j-i+k,l}
%     Z_{i-k, t-l}^2 - \sum_{k=i-p}^{i+m} D_{i-k} M_{i-k; ij}^{(m)} \\
%     &=& \sum_{t=1}^n \sum_{k=0}^m \left(
%       \sum_{l=0}^\infty h_{kl}h_{j-i+k,l} Z_{i-k, t-l}^2 - Z_{i-k, t}^2 M_{k,
%         j-i+k} \right) \\
%     &=& \sum_{t=1}^n \sum_{k=0}^m \left(
%       \sum_{l=0}^\infty h_{kl}h_{j-i+k,l} Z_{i-k, t-l}^2 - \sum_{l=0}^{\infty}
%       h_{kl} h_{j-i+k, l} Z_{i-k, t}^2  \right) \\
%     &=& \sum_{k=0}^m \sum_{l=0}^\infty h_{kl} h_{j-i+k, l}\left(
%       \sum_{t=1}^l Z_{i-k, t-l}^2 - \sum_{t=n-l+1}^n Z_{i-k, t}^2
%     \right)
%   \end{eqnarray*}
%   Next we prove $\pr\left[\sum_{k=0}^m \sum_{l=0}^\infty h_{kl} h_{j-i+k, l}
%     \sum_{t=1}^l Z_{i-k, t-l}^2 > a_n^2 \epsilon \right] \to 0$ as
%   $n \to \infty$.
%   \begin{eqnarray*}
%     && \pr\left[\sum_{k=0}^m \sum_{l=0}^\infty h_{kl} h_{j-i+k, l}
%       \sum_{t=1}^l Z_{i-k, t-l}^2 > a_n^2 \epsilon \right] \\
%     &\leq& \pr\left[\sum_{k=0}^m \sum_{l=0}^\infty | h_{kl} h_{j-i+k,
%         l} | \sum_{t=1}^l Z_{i-k, t-l}^2 > a_n^2 \epsilon \right] \\
%     &\leq& \pr\left[c \sum_{k=0}^m \sum_{l=0}^\infty | h_{kl} |
%       \sum_{t=1}^l Z_{i-k, t-l}^2 > a_n^2 \epsilon \right] \\
%     &\leq& \sum_{k=0}^m \pr\left[c \sum_{l=0}^\infty | h_{kl} |
%       \sum_{t=1}^l Z_{i-k, t-l}^2 > a_n^2 \epsilon \right] \\
%     &\leq& \sum_{k=0}^m \pr\left[c \sum_{l=0}^\infty | h_{kl} |
%       \sum_{q=0}^{l-1} Z_{i-k, -q}^2 > a_n^2 \epsilon \right] \\
%   \end{eqnarray*}
%   Using the fact that the sequence of $Z$ is iid and re-arranging the
%   indices, we get
%   \begin{eqnarray*}
%     && \sum_{k=0}^m \pr\left[c \sum_{l=0}^\infty | h_{kl} |
%       \sum_{q=0}^{l-1} Z_{i-k, -q}^2 > a_n^2 \epsilon \right] \\
%     &=& \sum_{k=0}^m \pr\left[c \sum_{l=0}^\infty | h_{kl} |
%       \sum_{q=0}^{l-1} Z_{i-k, q}^2 > a_n^2 \epsilon \right] \\
%     &=& \sum_{k=0}^m \pr\left[c \sum_{q=0}^{\infty} Z_{i-k, q}^2
%       \sum_{l=q+1}^\infty |h_{kl}| > a_n^2 \epsilon \right] \\
%   \end{eqnarray*}
%   Then by applying Jessen and Mikosch \cite{JessenMikosch2006} to the
%   iid sequence of $(Z_{i-k, q}^2)_{q=0}^{\infty}$ we get
%   \begin{eqnarray*}
%     && \sum_{k=0}^m \pr\left[c \sum_{q=0}^{\infty} Z_{i-k, q}^2
%       \sum_{l=q+1}^\infty |h_{kl}| > a_n^2 \epsilon \right] \\
%     &\sim& \sum_{k=0}^m \pr(Z^2 > a_n^2 \epsilon/c) c_1
%     \sum_{q=0}^{\infty} \left( \sum_{l=q+1}^\infty |h_{kl}|
%     \right)^{\alpha/2} \\
%     &\sim& c {L(n) \over n} \sum_{k=0}^m \sum_{q=0}^{\infty} \left(
%       \sum_{l=q+1}^\infty |h_{kl}| \right)^{\alpha/2} \to 0
%   \end{eqnarray*}
%   where in the last step the assumption $\sum_{k=0}^{\infty}
%   \left(\sum_{l=k}^{\infty} |h_{kl}|\right)^{\alpha/2-\epsilon} <
%   \infty$, $\forall \epsilon > 0$ has been used.

%   Because $p$ is assumed constant, the same arguments employed in
%   \cite{Mikosch2014} can be applied here with $a_{np}$ replaced by
%   $a_n$ to show
%   \begin{eqnarray*}
%     & \pr\left(
%       a_n^{-2} \|\mtx I^{(2)}\|_2 > \epsilon
%     \right) \to 0 \text{, } \forall \epsilon > 0 \text{ as } n \to
%     \infty \\
%     & \pr\left(
%       a_n^{-2} \|\mtx I^{(3)}\|_2 > \epsilon
%     \right) \to 0 \text{, } \forall \epsilon > 0 \text{ as } n \to \infty
%   \end{eqnarray*}
% \end{proof}

% \subsection[alpha in (2,4)]{The case $\alpha \in (2,4)$}
% In this case we have
% \[
% a_n^{-2} \|\mtx{X^{(m)} (X^{(m)})'} - \E(\mtx{X^{(m)} (X^{(m)})'}) -
% \sum_{q=-m}^p (D_q - \E D) \mtx M_q^{(m)}\|_2
% \xrightarrow{P} 0
% \]
% The proof is the same as its counterpart in \cite{Mikosch2014} except
% that the approximating sequence in this case is $\sum_{q=-m}^p (D_q -
% \E D) \mtx M_q^{(m)}$ and, because $p$ is fixed, this sequence doesn't
% need further approximation.

% \section{Lognormal Volatility Model}
% We consider the model
% \[
% X_{it} = \sigma_{it} Z_{it}
% \]
% where $\pr(Z_{it} > x) = p_+ L(x) x^{-\alpha}$, $\pr(Z_{it} < -x) =
% p_- L(x) x^{-\alpha}$ for a slowly varying function $L(x)$. $p_+ +
% p_- = 1$ and $0 < \alpha < 4$.
% \begin{eqnarray*}
%   \ln \sigma_{it} = \sum_{k=0}^\infty \sum_{l=0}^\infty h_{k,l}
%   \eta_{i-k,t-l}
% \end{eqnarray*}
% where $\eta_{i,t} \sim N(0, 1)$.

\section{Proof of the Lemma}
Here we prove that, for a symmetric, positive semidefinite $p \times
p$ matrix $\mtx M = (m_{ij})$, whose $k$th principle sub-matrix is denoted
$M_k$, satisfies
\[
\lambda_1(\mtx M) \leq \lambda_1(M_{k}) + \sum_{i=k+1}^p m_{ii}
\]
\begin{proof}
  Because $M_k$ is positive semidefinite for each $k$ , we only need
  to prove the claim for $k=p-1$. For notational convenience, write
  $A$ for $M_{p-1}$ and $d$ for $m_{pp}$. Next we prove the claim when
  $A$ is positive definite.
  \begin{eqnarray*}
    \mtx M &=&
    \begin{pmatrix}
      A & v \\
      v' & d
    \end{pmatrix} \\
    &=&
    \begin{pmatrix}
      A & 0 \\
      v' & d - v' A^{-1} v
    \end{pmatrix}\left(
      \begin{array}{cc}
        I & -A^{-1} v \\
        0 & 1
      \end{array}\right)^{-1} \\
    &=& U V
  \end{eqnarray*}
  From $\det(M) \geq 0$ it follows $\det(A) (d - v' A^{-1} v) \geq
  0$. Since $\det(A) > 0$, $d - v' A^{-1} v \geq 0$. Moreover, $A$ is
  assumed positive definite, so $v' A^{-1} v \geq 0$. According
  to Bai and Silverstein\cite{BaiSilverstein2010} we have
  $\lambda_1(M) \leq s_1(U)$, where $s_1(U)$ denotes the largest
  singular value of $U$.
  \begin{eqnarray*}
    \lambda_1(M) &\leq& s_1(U) \\
    &=& \max\left\{\lambda_1(A), d - v' A^{-1} v \right\} \\
    &\leq& \lambda_1(A) + d
  \end{eqnarray*}
  When $A$ is not positive definite, i.e. $\det(A)=0$,
  semi-positiveness implies $\det(A) d - v' v \geq 0$, hence $v =
  0$. It immediately follows
  \begin{eqnarray*}
    M &=&
    \begin{pmatrix}
      A & 0 \\
      0 & d
    \end{pmatrix} \\
    \lambda_1(M) &=& \max\{\lambda_1(A), d\} \\
    &\leq& \lambda_1(A) + d
  \end{eqnarray*}


  % Now we consider the case where $ d > 0$ and $A$ is not positive
  % definite, i.e. it has zero eigenvalues and $\det(A)=0$. First notice
  % $\det(A) = 0$ implies $v = 0$ because semi-positiveness requires $0 =
  % d \det(A) \geq v' v$.

  % When $A$ has only zero eigenvalues, $A$ and $v$ contain only zeros
  % and hence $\lambda_1(M) = d$, our claim is true; when $A$
  % has at least 1 non-zero eigenvalue, we may re-arrange the rows and
  % columns of $M$, or in other words, multiply $M$ by an orthogonal
  % matrix $Q$ and its transpose $Q'$ so that
  % \begin{eqnarray*}
  %   M = Q
  %   \begin{pmatrix}
  %     A_1 & B_1 & 0 \\
  %     B'_1 & A_2 & 0 \\
  %     0 & 0 & d
  %   \end{pmatrix} Q'
  % \end{eqnarray*}
  % where $A_2$ is positive definite and has the same rank as $A$. Thus
  % the rows of $(A_1, B_1, 0)$ are linear combinations of those of
  % $(B'_1, A_2, 0)$ and hence can be eliminated. Then it follows
  % that $A$ has the same eigenvalues as the matrix
  % \[
  % \begin{pmatrix}
  %     A_1 & B_1 \\
  %     B'_1 & A_2 \\
  % \end{pmatrix}
  % \]
  % and the same non-zero eigenvalues as $A_2$; $M$ has the same
  % non-zero eigenvalues as the matrix
  % \[
  % M_1 = 
  % \begin{pmatrix}
  %     A_2 & 0 \\
  %     0 & d
  % \end{pmatrix}
  % \]
  % So we only need to prove the claim for $M_1$. Since $A_2$ is
  % positive definite, this is clear from our earlier argument.
\end{proof}
\bibliographystyle{unsrt}
\bibliography{../thesis/econophysics}
\end{document}
