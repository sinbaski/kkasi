\documentclass{article}
\usepackage{amsmath}
\usepackage{amsthm}
\usepackage{enumerate}
\usepackage[bookmarks=true]{hyperref}
\usepackage{bookmark}

\usepackage{amssymb,amsmath,amsthm,amsfonts}
\usepackage{mathrsfs}
\usepackage{dsfont}
\usepackage{enumerate}

%\newtheorem{mdef}{Definition}
%\newtheorem{theorem}{Theorem}
\newcommand{\eqsplit}[2]{
  \begin{equation}\label{#2}
    \begin{split}
      #1
    \end{split}
  \end{equation}}
\newcommand{\eqnsplit}[1]{
  \begin{eqnarray*}
    #1
  \end{eqnarray*}}
\newcommand{\tran}[1]{
  \tilde{#1}
}
\newcommand{\td}[2]{
  \frac{d #1}{d #2}
}
\newcommand{\pd}[2]{
  \frac{\partial #1}{\partial #2}
}
\newcommand{\ppd}[2]{
  \frac{\partial^2 #1}{\partial #2^2}
}
\newcommand{\pdd}[3]{
  \frac{\partial^2 #1}{\partial #2 \partial #3}
}
\newcommand{\otd}[1]{
  \frac{d}{d #1}
}
\newcommand{\opd}[1]{
  \frac{\partial}{\partial #1}
}
\newcommand{\oppd}[1]{
  \frac{\partial^2}{\partial #1^2}
}
\newcommand{\opdd}[2]{
  \frac{\partial^2}{\partial #1 \partial #2}
}
\newcommand{\ket}[1]{
  |#1\rangle
}
\newcommand{\bra}[1]{
  \langle#1|
}
\newcommand{\inn}[1]{
  \langle#1\rangle
}
\newcommand{\mean}[1]{
  \langle#1\rangle
}
\newcommand{\tr}{
  \text{tr}\,
}
\newcommand{\re}{
  \text{Re}\,
}
\newcommand\im{
  \text{Im}\,
}
\newcommand{\var}{
  \text{var}
}
\newcommand{\arcsinh}{
  \sinh^{-1}
}
\newcommand{\arccosh}{
  \cosh^{-1}
}
\newcommand{\erfc}{
  \text{erfc}
}
\newcommand{\E}{
  \mathbb{E}
}
\renewcommand{\P}{
  \mathbb{P}
}
\newcommand{\I}[1]{
  \mathbf{1}_{\{#1\}}
}
\newcommand{\1}[1]{
  \mathds{1}_{\{#1\}}
}
\newcommand{\diag}{
  \text{diag\,}
}
\newcommand{\M}{
  {\text{max}}
}
\newcommand{\m}{
  {\text{min}}
}
\newcommand{\ph}{
  {\text{arg}\,}
}
\newcommand\erf{
  \text{erf}
}
\renewcommand\vec[1]{
  \mathbf{#1}
}
\newcommand\mtx[1]{
  \mathbf{#1}
}
\newcommand\ed{
  \,{\buildrel d \over =}\,
}



\begin{document}
\section{iid sequences}\label{sec:iid}
Consider iid sequences $\{X_{it}\}$, $1 \leq i \leq p$, $1 \leq t \leq
n$. $p$ is fixed while $n \to \infty$. $\pr(X_{it} > x) = p_+ L(x)
x^{-\alpha}$, $\pr(X_{it} < -x) = p_- L(x) x^{-\alpha}$ for a slowly
varying function $L(x)$, and $p_+ + p_- = 1$. The normalizing sequence
$a_n$ is such that $\pr(|x_{it}| > a_n) \sim {1 \over n}$, implying
$a_n \sim n^{1/\alpha} l(n)$, where $l(n)$ is a slowing varying
function. Moreover, we assume $\E X_{11} = 0$ if $\E |X_{11}| <
\infty$. The sample covariance matrix is $\mtx {XX'}$.

For convenience, we let $c$ stand for any constant whose value is not
of importance in the rest of this report. We shall prove
\[
a_n^{-2}\|\mtx {XX'} - \diag(\mtx {XX'})\|_2 \xrightarrow{P} 0
\]

\begin{proof}
  Using the inequality
  \[
  \|\mtx A\|_2 \leq \|\mtx A\|_\infty
  \]
  for an arbitrary matrix $\mtx A$, we have
  \begin{align*}
    \pr(\|\mtx {XX'} - \diag(\mtx {XX'})\|_2 \geq a_n^2 \epsilon) & \leq
    \pr\left(\max_{1\leq i \leq p} \sum_{j=1, j\neq i}^p |\sum_{t=1}^n
    X_{it} X_{jt}| > a_n^2 \epsilon\right) \\
    &\leq \sum_{i=1}^p \sum_{j=1, j\neq i}^p \pr\left(|\sum_{t=1}^n
    X_{1t} X_{2t}| > {a_n^2 \epsilon \over p-1}\right) \\
  \end{align*}
  Since $\E X_{1t} X_{2t} = (\E X_{11})^2$, which is either 0 or
  $\infty$, the large deviation result found in Cline and Hsing
  \cite{ClingHsing1998} and Nagaev \cite{nagaev1979} is
  applicable. With it we obtain
  \begin{align*}
    \pr\left(|\sum_{t=1}^n X_{1t} X_{2t}| > {a_n^2 \epsilon \over p-1}\right) &\sim
    c n \pr\left(|X_{11} X_{21}| > {a_n^2 \epsilon \over p-1 }\right)
  \end{align*}
  Now that $|X_{11}|$ and $|X_{21}|$ are iid and regularly varying
  with index $\alpha$, so is $|X_{11} X_{21}|$, according to Jessen
  and Mikosch \cite{JessenMikosch2006}. Thus we have
  \[
  c n \pr\left(|X_{11} X_{21}| > {a_n^2 \epsilon \over p-1 }\right)
  \to 0
  \]
\end{proof}
Now, using the result just proved and applying Weyl's perturbation
theorem we obtain
\[
a_n^{-2} \max_{i=1,\dots,p} |\lambda^C_{(i)} -
\lambda^{\text{diag}}_{(i)}| \leq  a_n^{-2}\|C - \diag{C}\| \to 0
\]
Therefore the eigenvalues of $C$ can be approximated by those of
$\diag(C)$:
\begin{eqnarray*}
a_n^{-2}\lambda^{\text{diag}}_{i}  &=& a_n^{-2} \sum_{t=1}^n X_{it}^2
\end{eqnarray*}

\subsection{Upper Order Statistics of the eigenvalues}
Because $\pr(X_{11}^2 > a_n^2) \sim 1/n$, the central limit theorem
gives, according to Mikosch et al\cite{Embrechts1997},
\[
a_n^{-2} \sum_{t=1}^n X_{it}^2 \xrightarrow{d} \xi_i \sim S_{\alpha/2}
\]
where $S_{\alpha/2}$ denotes an $\alpha/2$-stable distribution. Since
$(\xi_i)_{i=1}^p$ are iid, the joint density function of the $k$ upper
order statistics of $\lambda_i$, the eigenvalues of $a_n^{-2} X X'$
follows as
\begin{eqnarray*}
  f_{\lambda_{(1)}, \dots, \lambda_{(k)}}(x_1, \dots, x_k) &=&
  p(p-1)\cdots(p-k+1) S_{\alpha/2}^{p-k}(x_k) \prod_{i=1}^k
  f_{\alpha/2}(x_i)
\end{eqnarray*}
where $f_{\alpha/2}(x) = {d S_{\alpha/2}(x) \over dx}$.


% \input{Mikosch_correlated_sequences.tex}
\section{Correlated Sequences}
\subsection{MA(1) process}
First consider the model
\[
X_{it} = Z_{it} + \theta Z_{i,t-1}
\]
We have
\begin{eqnarray*}
  a_n^{-2}(XX')_{ij} &=& a_n^{-2} \sum_{t=1}^n (Z_{it} + \theta
  Z_{i,t-1}) (Z_{jt} + \theta Z_{j,t-1}) \\
  &=& a_n^{-2} \sum_{t=1}^n Z_{it} Z_{jt} + a_n^{-2} \theta \sum_{t=1}^n
  Z_{it} Z_{j,t-1} \\
  && +\theta a_n^{-2} \sum_{t=1}^n Z_{i, t-1}
  Z_{j,t} + \theta^2 a_n^{-2} \sum_{t=1}^n Z_{i,t-1} Z_{j,t-1}
\end{eqnarray*}
Consider $\pr(a_n^{-2} \sum_{t=1}^n Z_{1t} Z_{2t} > \epsilon)$. Since
$Z_1 Z_2 \in \mathcal R_{-\alpha}$ (See Mikosch and Jessen
\cite{JessenMikosch2006}) and $\E Z_1 Z_2$ is either 0 or $\infty$, by
the large deviation result of Cline and Hsing \cite{ClingHsing1998}
and Nagaev \cite{nagaev1979}
\begin{eqnarray*}
  && \pr(a_n^{-2} \sum_{t=1}^n Z_{1t} Z_{2t} > \epsilon)  \\
  &\sim& c n \pr(|Z_1 Z_2| > a_n^{2} \epsilon)  \\
\end{eqnarray*}
Because $a_n \sim n^{1/\alpha} l(n)$, we have
\begin{eqnarray*}
  \pr(|Z_1 Z_2| > a_n^{2} \epsilon) &\sim& n^{-2} L_{12}(n)
\end{eqnarray*}
where $L_{12}(n)$ is a slowly varying function. Thus
\[
\pr(a_n^{-2} \sum_{t=1}^n Z_{1t} Z_{2t} > \epsilon) \to 0
\]
% \begin{eqnarray*}
%   C &=& a_n^{-2} \left[ XX' - (1 + \theta^2)
%     \begin{pmatrix}
%       \sum_{t=1}^n Z_{1t}^2 & & \\
%       & \ddots & \\
%       & & \sum_{t=1}^n Z_{pt}^2
%     \end{pmatrix}
%   \right] \\
%   C_{ij} &=& a_n^{-2} \left[\sum_{t=1}^n (Z_{it} + \theta Z_{i, t-1})
%     (Z_{j,t} + \theta Z_{j, t-1}) - \I{i=j} (1 + \theta^2) \sum_{t=1}^n
%     Z_{it}^2 \right] \\
%   &=& a_n^{-2} \left[
%     \sum_{t=1}^\infty 
%   \right]
% \end{eqnarray*}

On the other hand $Z^2 \in \mathcal R_{-\alpha/2}$, hence
\begin{eqnarray*}
  \pr(a_n^{-2} \sum_{t=1}^n Z_{1t}^2 > \epsilon) &\sim& c n
  \pr(Z_{1t}^2 > a_n^2 \epsilon) \\
  &\sim& c n \pr(|Z_{1t}| > a_n) \\
  &\sim& c L(n)
\end{eqnarray*}
Therefore
\begin{eqnarray*}
  \pr(a_n^{-2} \sum_{t=1}^n Z_{it} Z_{jt}\I{i \neq j} > \epsilon) \to
  0 \\
  \pr(a_n^{-2} \sum_{t=1}^n Z_{i,t-1} Z_{j,t-1} \I{i \neq j} >
  \epsilon) \to 0 \\
  \pr(a_n^{-2} \sum_{t=1}^n Z_{it} Z_{j,t-1} > \epsilon) \to 0
  \\
  \pr(a_n^{-2} \sum_{t=1}^n Z_{i, t-1} Z_{j,t} > \epsilon) \to 0
\end{eqnarray*}
Then it follows, for $i \neq j$, $\pr(a_n^{-2} (XX')_{ij} >
\epsilon) \to 0$; and for $i=j$
\begin{eqnarray*}
  && \pr\left[ a_n^{-2} (XX')_{ij} - a_n^{-2} (1 + \theta^2)\sum_{t=1}^\infty
      Z_{it}Z_{jt} \I{i=j} > \epsilon \right] \\
  &\sim& \pr\left[ a_n^{-2} Z_{in}^2 + \theta^2 a_n^{-2} Z_{i0}^2 >
    \epsilon \right] \\
  &\leq& \pr(Z_{in}^2 > a_n^2 \epsilon/2) + \pr(Z_{i0}^2 > \theta^{-2}
  a_n^2 \epsilon/2) \to 0 \text{, } n \to 0\\
\end{eqnarray*}
By Weyl's perturbation theorem, the ordered eigenvalues of
$a_n^{-2}XX'$, $\lambda_{(1)} > \lambda_{(2)} > \dots >
\lambda_{(p)}$ converge in probability:
\[
(\lambda_{(1)}, \lambda_{(2)}, \dots, \lambda_{(p)}) \xrightarrow{P}
(1+\theta^2)(\xi_{(1)}, \xi_{(2)}, \dots, \xi_{(p)})
\]
where $\xi_i$ are iid $\alpha/2$-stable rvs.
\subsection{Cross-correlation limited to order 1}
Now we consider the model
\[
X_{it} = Z_{it} + \theta Z_{i-1,t}
\]


% where $Z_{i,t}$ are iid sequences as detailed in \S\ref{sec:iid}. If
% $h_{kl}$ can be written as $h_{kl} = \psi_k \varphi_l$, then we have
% \begin{eqnarray*}
%   X_{i,t} &=& \sum_{k=0}^\infty \psi_k \sum_{l=0}^\infty \varphi_l
%   Z_{i-k, t-l}
% \end{eqnarray*}
% Because $Z \in \mathcal{R}_{-\alpha}$, according to Mikosch et al
% \cite{Embrechts1997}, lemma A3.26, we have for $x \to \infty$
% \begin{eqnarray*}
% \pr(\sum_{l=0}^\infty \varphi_l Z_{i-k, t-l} > x) &\sim& \pr(|Z| > x)
% \sum_{l=0}^\infty |\varphi_l|^{\alpha} (p_+ \I{\varphi_l > 0} + p_-
% \I{\varphi_l < 0})\\
% \pr(\sum_{l=0}^\infty \varphi_l Z_{i-k, t-l} < -x) &\sim& \pr(|Z| > x)
% \sum_{l=0}^\infty |\varphi_l|^{\alpha} (p_+ \I{\varphi_l < 0} + p_-
% \I{\varphi_l > 0})
% \end{eqnarray*}
% Define 
% \begin{eqnarray*}
%   Y_{it} &=& \sum_{l=0}^\infty \varphi_l Z_{i, t-l}
% \end{eqnarray*}
% Clearly $\E Y_{i,t_1} Y_{j,t_2} = 0$ for $i \neq j$.
% \[
% X_{it} = \sum_{k=0}^\infty \psi_k Y_{i-k, t}
% \]

% \subsection[alpha larger than 2]{$\alpha > 2$}
% When $\sum_{j=0}^{\infty}|\psi_j| < \infty$,
% $\sum_{j=0}^{\infty}|\psi_j|^2 j < \infty$ and $\alpha > 2$,
% define for $g > 0$
% \[
% \tilde\rho_t(g) = \left({1 \over p}\sum_{k=1}^{p-g}
%   X_{kt}X_{k+g,t}\right)
% \left({1 \over p}\sum_{k=1}^{p} X_{kt}^2\right)^{-1}
% \]
% Then according to Brockwell et al \cite{Brockwell1991}, theorem 7.2.1,
% we have
% \begin{eqnarray*}
%   \begin{pmatrix}
%     \tilde\rho(1) \\
%     \tilde\rho(2) \\
%     \vdots \\
%     \tilde\rho(g)
%   \end{pmatrix} &\sim&
%   N((\rho(1), \rho(2), \dots, \rho(g))', p^{-1}\mtx W)
% \end{eqnarray*}
% where by Bartlett's formula the matrix $\mtx W$ has elements $w_{ij}$
% \begin{eqnarray*}
%   w_{ij} &=& \sum_{k=1}^{\infty} \left[
%   \rho(k + i) + \rho(k-i) -2\rho(k)\rho(i)
%   \right] \times \left[
%     \rho(k+j) + \rho(k-j) -2\rho(k)\rho(j)
%   \right] \\
%   \rho(g) &=& \left(
%     \sum_{k=0}^\infty \psi_k^2
%   \right)^{-1} \left(\sum_{k=0}^\infty \psi_k \psi_{k+g}\right)
% \end{eqnarray*}

\section{Lognormal Volatility Model}
We consider the model
\[
X_{it} = \sigma_{it} Z_{it}
\]
where $\pr(Z_{it} > x) = p_+ L(x) x^{-\alpha}$, $\pr(Z_{it} < -x) =
p_- L(x) x^{-\alpha}$ for a slowly varying function $L(x)$. $p_+ +
p_- = 1$ and $0 < \alpha < 4$.
\begin{eqnarray*}
  \ln \sigma_{it} = \sum_{k=0}^\infty \sum_{l=0}^\infty h_{k,l}
  \eta_{i-k,t-l}
\end{eqnarray*}
where $\eta_{i,t} \sim N(0, 1)$.

\bibliographystyle{unsrt}
\bibliography{../thesis/econophysics}
\end{document}
