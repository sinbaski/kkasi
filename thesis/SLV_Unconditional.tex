\chapter{Unconditional Distribution Functions of Stochastic
  Log-Volatility Models}
\label{chp:SLV_unconditional}
In this chapter we study the unconditional distribution functions of
Stochastic Log-Volatility models specified as equation
\ref{eq:SLV_spec}. As is discussed in section \ref{sec:SLV_model},
$\ln\sigma_t$ can be described by an ARMA or ARIMA \footnote{ARIMA ---
  Integrated Autoregressive Moving Average} model, possibly with
seasonal components. Here we note that all these models can be
re-written as a moving average model, which is infinite in extent if
autoregressive components are present:
\begin{eqnarray*}
  \ln \sigma_t &=& y_t + \sum_{n=1}^\infty c_n y_{t-n} + \text{Const.}
\end{eqnarray*}
Since the $y_t$ are independent and identically distributed,
\begin{eqnarray*}
  y_t + \sum_{n=1}^\infty c_n y_{t-n}  
\end{eqnarray*}
has Gaussian distribution by the central limit theorem, on condition
that $y_t$ for all $t$ have finite second moment --- this is what we
assume in the rest of this section. It follows from the above equation
that the unconditional distribution of $\ln \sigma_t$ is the same as the
distribution of $\bar{v} + v$ where $v \sim N(0, \sigma)$ and
$\bar{v}$, $\sigma$ are constants. Now we can state that the
unconditional distribution of the returns $r_t$
\begin{eqnarray*}
  r_t &=& \mu + \sigma_t b_t\\
  &=& \mu + \exp\left(
    y_t + \sum_{n=1}^\infty c_n y_{t-n} + \text{Const.}
  \right)
\end{eqnarray*}
 is the same as
\begin{equation}  \label{eq:UnconditionalPdf}
  \begin{aligned}
    r &= \mu + e^{\bar{v} + v} b \\
  \end{aligned}
\end{equation}
where $b \sim N(0, 1)$. For convenience, let $r' = e^v b$

In section \ref{sec:SLV_Symmetric} we first study the model in the
relatively simple case when $v$ and $b$ are uncorrelated and $\mu
= 0$. If this simplified version proves inadequate, one may resort to
the general model studied in section \ref{sec:SLV_Asymmetric}.

\section{The Simplified model}\label{sec:SLV_Symmetric}
In the following we derive the unconditional PDF of $r'$, denoted
$f_{r'}(x)$. Then the PDF of $r$ is $e^{-\bar{v}}f_{r'}(e^{-\bar{v}}x)$.
\begin{eqnarray*}
  P(r' < x) &=& P(b < xe^{-v}) \\
  f_{r'}(x) &=& f_b(xe^{-v}) e^{-v}
\end{eqnarray*}
Averaging over all $v$, we get
\begin{equation}\label{eq:UncondPDFSymmetric}
  \begin{aligned}
    f_{r'}(x) =& \int_{-\infty}^{\infty} dv (2\pi\sigma^2)^{-1/2}
    e^{-v^2/2\sigma^2}(2\pi)^{-1/2} \exp(-x^2e^{-2v}/2) e^{-v} \\
    =& {1 \over 2\pi\sigma} \int_{-\infty}^{\infty} dv
    \exp\left(
      -{1 \over 2\sigma^2} v^2 - v -{1 \over 2} x^2 e^{-2v}
    \right)
  \end{aligned}
  \end{equation}
The last part of the integrand, $e^{-x^2 e^{-2v} / 2}$, is plotted in
figure \ref{fig:DoubleExp}.
\begin{figure}[htb!]
  \centering
  \includegraphics[scale=0.5, clip=true, trim=85 252 100
  231]{../pics/DoubleExp.pdf}
  \caption{\small \it Plot of $\exp(-{1 \over 2} x^2 e^{-2v})$}
  \label{fig:DoubleExp}
\end{figure}
Therefore we make the following approximation:
\[
\exp\left(-{1 \over 2} x^2 e^{-2v}\right) \approx \left\{
  \begin{array}{lr}
    0 & \text{if } v < \ln|x| -{1 \over 2} \ln(2\ln 2) \\
    1 & \text{otherwise}
  \end{array}
\right.
\]
Here we note that $\exp\left(-{1 \over 2} x^2 e^{-2v}\right) = 1/2$ at
$v = \ln|x| -{1 \over 2} \ln(2\ln 2)$.

With this approximation we have
\begin{eqnarray*}
  f_{r'}(x) &=& {1\over C}{1 \over 2\pi\sigma} \int_{\ln(|x|/\sqrt{\ln
      4})}^{\infty} dv
  \exp\left(-{1 \over 2\sigma^2} v^2 - v\right) \\
  &=& {1\over C}{e^{\sigma^2 / 2} \over \sqrt{8\pi}} \text{erfc} \left(
    {1 \over \sqrt{2}\sigma} \ln{|x| \over \sqrt{\ln 4}} + {\sigma
      \over \sqrt{2}}
  \right)
\end{eqnarray*}
where $1/C$ has been added for the purpose of normalization.

At large $x$, we may use the asymptotic expansion of $\mathrm{erfc}$
to write
\begin{eqnarray*}
  Cf_{r'}(x) &=& {e^{\sigma^2 / 2} \over \sqrt{8\pi}} \mathrm{erfc}(\xi) \\
  &=& {e^{\sigma^2 / 2} \over \sqrt{8 \pi}}
  \frac{e^{-\xi^2}}{\xi\sqrt{\pi}}\left[
    1 +
    \sum_{n=1}^N (-1)^n \frac{(2n-1)!!}{(2\xi^2)^n} \right] +
  O(\xi^{-2N-1} e^{-\xi^2})
\end{eqnarray*}
where $\xi = {1 \over \sqrt{2}\sigma} \ln{|x| \over \sqrt{\ln 4}} +
{\sigma \over \sqrt{2}}$. The slowest-decaying term is
\[
f_{r,0}(x) = {e^{\sigma^2 / 2} \over \sqrt{8\pi}}
\frac{e^{-\xi^2}}{\xi\sqrt{\pi}}
\]
Let $\zeta = {|x| \over \sqrt{\ln 4}}$. With a bit manipulation one
obtains
\begin{equation*}
  f_{r,0}(x) = {1 \over \pi \sqrt{8}}{1 \over
    \left(\ln\zeta/\sigma\sqrt{2} +
      \sigma/\sqrt{2}\right)\zeta\zeta^{\ln \zeta / 2\sigma^2}
  }
\end{equation*}
From the last equation one can see that, at any neighborhood of large
$x$, $f_{r'}(x)$ may be approximated by $C/|x|^\alpha$, i.e. a power law.

In the following we work out the normalization constant C.
\begin{eqnarray*}
  \int_{-\infty}^{\infty} f_{r'}(x) dx &=& {1 \over C}{e^{\sigma^2 / 2} \over
    \sqrt{8\pi}} \int_{-\infty}^{\infty} \text{erfc} \left({1 \over
      \sqrt{2}\sigma} \ln{|x| \over \sqrt{\ln 4}} + {\sigma \over
      \sqrt{2}} \right) dx\\
  &=& {2 \over C}{e^{\sigma^2 / 2} \over
    \sqrt{8\pi}} \int_{0}^{\infty} \text{erfc} \left({1 \over
      \sqrt{2}\sigma} \ln{x \over \sqrt{\ln 4}} + {\sigma \over
      \sqrt{2}} \right) dx\\
\end{eqnarray*}
Let
\begin{eqnarray*}
  a &=& {1 \over \sigma \sqrt 2} \\
  b &=& -{1 \over 2 \sigma \sqrt 2}\ln\ln 4 + {\sigma \over \sqrt 2} \\
  y &=& a\ln|x|+b \\
\end{eqnarray*}
Then
\begin{eqnarray*}
  \int_{-\infty}^{\infty} f_{r'}(x) dx &=& {2 \over C}{e^{\sigma^2 / 2} \over
    \sqrt{8\pi}} \int_{-\infty}^{\infty} dy {e^{(y-b)/a} \over a}
  \text{erfc}(y) \\
  &=& {2 \over C}{e^{\sigma^2 / 2} \over
    \sqrt{8\pi}} \left.e^{(y-b)/a}
    \text{erfc}(y)\right|_{y=-\infty}^{\infty}
  + {2 \over C}{e^{\sigma^2 / 2} \over
    \sqrt{8\pi}} \int_{-\infty}^{\infty} dy {2 \over \sqrt \pi}
  e^{(y-b)/a} e^{-y^2} \\
  &=& {2 \over C}{e^{\sigma^2 / 2} \over \sqrt{8\pi}}  2 e^{\ln\ln 4/2
    - \sigma^2/2}\\
  C &=& \sqrt{2 \ln 4\over \pi} \\
  &\approx& 0.9394
\end{eqnarray*}
Thus we can write in summary:
\begin{eqnarray*}
  f_r(x) &=& e^{-\bar{v}} f_{r'}(e^{-\bar{v}} x) \\
    &=& {e^{-\bar{v}} \over C}{e^{\sigma^2 / 2} \over \sqrt{8\pi}}
    \text{erfc} \left({1 \over \sqrt{2}\sigma} \ln{|e^{-\bar{v}}x| \over \sqrt{\ln
          4}} + {\sigma \over \sqrt{2}}
    \right) \\
\end{eqnarray*}
Using the same technique for integration as for normalization, the
cummulative distribution function of $r$ is found to be $F(x)$,
which is the following:
\begin{enumerate}
\item if $x < 0$
  \begin{eqnarray*}
    F(x) &=& {\sqrt{\ln 4} \over C}{e^{\sigma^2 / 2} \over \sqrt{8\pi}}
    \left[
      {e^{-\bar{v}}x \over \sqrt{\ln 4}}\text{erfc}\left(
        {1 \over \sigma \sqrt 2} \ln{-e^{-\bar{v}}x \over \sqrt{\ln
            4}} + {\sigma \over \sqrt 2}
      \right) \right.\\
      && \left. + e^{-\sigma^2 / 2} \text{erfc}\left(
        {1 \over \sigma \sqrt 2} \ln{-e^{-\bar{v}}x \over \sqrt{\ln 4}}
      \right)
    \right]
  \end{eqnarray*}
\item if $x \geq 0$
  \begin{eqnarray*}
    F(x) &=& \frac{1}{2} + {\sqrt{\ln 4} \over C}{e^{\sigma^2 / 2} \over
      \sqrt{8\pi}} \left[
      {e^{-\bar{v}}x \over \sqrt{\ln 4}} \text{erfc}\left(
        {1 \over \sigma \sqrt 2} \ln{e^{-\bar{v}}x \over \sqrt{\ln
            4}} + {\sigma \over \sqrt 2}
      \right) \right. \\
      && \left. + e^{-\sigma^2/2} \text{erfc}\left(
        -{1 \over \sigma \sqrt 2} \ln{e^{-\bar{v}}x \over \sqrt{\ln 4}}
      \right)
    \right]
  \end{eqnarray*}
\end{enumerate}

To verify the validity of the model, we fit the above probability
density function to the de-meaned 30min returns of Volvo B
\footnote{By ``de-meaned returns'' we mean the quantity $r_t - 
  \mean{r_t}$, where $r_t$ are the measured returns and $\mean{r_t}$
  is the sample mean. The data set covers the transaction records of
  Volvo B on the OMX market (Stockholm) between 2013-10-10 and
  2014-03-12. The returns are computed using 1-minute mean prices.}
by means of maximum likelihood estimate using the MATLAB function
``mle''. Then for the parameters $\sigma$ and $\bar{v}$ we get
\begin{eqnarray*}
  \sigma &=& 0.6355 \\
  \bar{v} &=& -6.0308
\end{eqnarray*}
Then we plot $P(r' > x)$ of the model against its empirical
counterpart on a log-log scale, as shown in figure
\ref{fig:volvo_30min_ret}.
\begin{figure}[htb!]
  % \vspace{-15mm}
  \begin{center}
    \includegraphics[scale=0.4, clip=true, trim=98 231 116
    126]{../pics/volvo_30min_ret.pdf}
  \end{center}
  %\vspace{-5mm}
  \caption{\small \it{$P(r' > x)$ for $x > 0$. Blue: empirical
      probabilities. Red: probabilities predicted by the model.}}
  \label{fig:volvo_30min_ret}
\end{figure}

In general figure \ref{fig:volvo_30min_ret} shows a good fit, but a
closer look reveals that deviations are significant in the regions $r
\in (0, \sigma_r)$ and $r \in (2\sigma_r, 3\sigma_r)$, where
$\sigma_r$ stands for the empirical standard deviation of the
de-meaned returns. These observations are
shown in greater details in figure \ref{fig:volvo_30min_ret2}.
\begin{figure}[htb!]
  \centering
  \subfigure[$\ln(P(r > x))$ with $x \in (0, \sigma_r)$]{
    \includegraphics[scale=0.4, clip=true, trim=93 224 115
    125]{../pics/volvo_30min_ret_0-sigma.pdf}
  }
  \subfigure[$\ln(P(r > x))$ with $x \in (2\sigma_r, 3\sigma_r)$]{
    \includegraphics[scale=0.4, clip=true, trim=93 224 115
    125]{../pics/volvo_30min_ret_2sigma-3sigma.pdf}
  }
  \caption{\small \it Surviving probabilities. Blue: empirical values of $\ln(P(r
    > x))$. Red: Predicted values of $\ln(P(r > x))$.}
  \label{fig:volvo_30min_ret2}
\end{figure}

Moreover, the inefficiency of the model is also manifest in the
skewness of the data. For the Volvo 30min returns, the data has a
skewness of 0.2419, but our model is strictly symmetric, since
$x$ appears in equation \ref{eq:UncondPDFSymmetric} only as $x^2$.
Hence the above model needs to be improved to accommodate the non-zero
skewness as well as to account for the discrepancies shown in figure
\ref{fig:volvo_30min_ret2}. This is the subject of the next section.

\section{The General model}\label{sec:SLV_Asymmetric}
It has long been hypothesized in the liturature that skewness is the
result of price-volatility correlation (for example
\cite{Potters2003}). Therefore, the most apparent modification is to
allow $v$ and $b$ to be correlated. For convenience we decompose
$v \sim N(0, \sigma)$ as $v = \sigma a$ and assume
\begin{eqnarray*}
  \begin{pmatrix}
    a \\
    b
  \end{pmatrix} \sim N(0, \Sigma)
\end{eqnarray*}
with a covariance matrix
\begin{eqnarray*}
    \Sigma &=&
  \begin{pmatrix}
    1 & \psi \\
    \psi & 1
  \end{pmatrix}
\end{eqnarray*}
where $|\psi| < 1$. Then we rewrite equation \ref{eq:UnconditionalPdf}
as
\begin{equation}
  \label{eq:r_t}
  \begin{aligned}
    r &= \mu + e^{\bar{v}} r' \\
    r' &= e^{\sigma a} b \\
  \end{aligned}
\end{equation}
Then
\begin{eqnarray*}
  && P(r' < x)\\
  &=& P(b < e^{-a\sigma}x) \\
  &=& {1 \over 2\pi |\det(\Sigma)|^{1/2}}
  \int_{-\infty}^{\infty} da \int_{-\infty}^{e^{-a\sigma}x} db
  \exp\left[
    -{1 \over 2} (a, b) \Sigma^{-1}
    \begin{pmatrix}
      a \\
      b
    \end{pmatrix}
  \right] \\
\end{eqnarray*}
After some manipulations we get
\begin{equation}\label{eq:UncondCDFAsymmetric}
  \begin{aligned}
    & P(r' < x) \\  
    &= {1 \over 2\pi \sqrt{1 - \psi^2}} \int_{-\infty}^{\infty} da
    e^{-a^2/2} \int_{-\infty}^{e^{-a\sigma}x} db
    \exp\left[
      - {(b - a\psi)^2 \over 2(1 - \psi^2)}
    \right] \\
    &= {1 \over 2\sqrt{2\pi}} \int_{-\infty}^{\infty} e^{-a^2/2}
    \text{erfc}{a\psi - e^{-a\sigma} x \over \sqrt{2(1-\psi^2)}} da
  \end{aligned}
\end{equation}
Differentiating with respect to $x$ yields
\begin{eqnarray}
  f_{r'}(x) &=& {1 \over 2\pi \sqrt{1 - \psi^2}} \times \nonumber \\
  && \int_{-\infty}^\infty da \exp
  \left[- {
      a^2 + 2\sigma(1 - \psi^2)a - 2a\psi e^{-a\sigma} x + e^{-2a\sigma} x^2
      \over
      2(1 - \psi^2)
  }
  \right] \label{eq:UncondPDFAsymmetric} \\
  f_r(x) &=&  \td{r'}{r} f_{r'}[e^{-\bar{v}} (x - \mu)] \nonumber \\
  &=& e^{-\bar{v}} f_{r'}[e^{-\bar{v}}
  (x-\mu)] \label{eq:UncondPDFAsymmetric1}
\end{eqnarray}
Unlike \ref{eq:UncondPDFSymmetric} where a relatively simple
approximation can be found and thus leads to an analytic result of the
integral, no such approximation has been found by the
author for the integral \ref{eq:UncondPDFAsymmetric}. However, the
moment generating functions (MGF) of PDF \ref{eq:UncondPDFAsymmetric}
and \ref{eq:UncondPDFAsymmetric1} are easy enough to find and give the
moments about the origin in closed form. Then by matching the analytic
expressions of the moments with their statistical values from the
sample, the parameters $\sigma$, $\psi$ and $\bar{v}$ corresponding to the
sample can be obtained.

The MGF of \ref{eq:UncondPDFAsymmetric1} can be found as follows:
\begin{eqnarray*}
  && M_{r}(t) \\
  &=& \E(e^{tr}) \\
  &=& {1 \over 2\pi \sqrt{1 - \psi^2}} \int_{-\infty}^{\infty} da \exp \left(
    -{a^2 \over 2} - \sigma a
  \right) \int_{-\infty}^{\infty} dx \exp\left[
    t(e^{\bar{v}}x + \mu) - {
      (a\psi - e^{-a \sigma} x)^2 \over 2 (1 - \psi^2)
    }
  \right] \\
  &=& {1 \over \sqrt{2 \pi}} \int_{-\infty}^{\infty} da \exp\left[
    -{a^2 \over 2} + {1 \over 2} e^{2 a \sigma + 2\bar{v}} (1 - \psi^2) t^2 +
    a \psi e^{a \sigma + \bar{v}} t + \mu t
  \right]
\end{eqnarray*}
With the MGF, the first 4 moments of $r$ can be computed:
\begin{equation}
  \label{eq:AsymmetricMoments}
  \begin{aligned}
    \E(r) &= \mu +e^{\bar{v}+\frac{\sigma ^2}{2}} \sigma  \psi\\
    \E(r^2) &= \mu ^2+2 e^{\bar{v}+\frac{\sigma ^2}{2}} \mu  \sigma  \psi
    +e^{2 \left(\bar{v}+\sigma ^2\right)} \left(1+4 \sigma ^2 \psi^2\right)\\
    \E(r^3) &= \mu ^3+3 e^{\bar{v}+\frac{\sigma ^2}{2}} \mu ^2 \sigma \psi
    +9 e^{3 \bar{v}+\frac{9 \sigma ^2}{2}} \sigma \psi \left(1+3 \sigma ^2
      \psi ^2\right)
    +3e^{2 \left(\bar{v}+\sigma ^2\right)} \mu  \left(1+4 \sigma ^2 \psi
      ^2\right) \\
    \E(r^4) &= \mu ^4+4 e^{\bar{v}+\frac{\sigma ^2}{2}} \mu ^3 \sigma  \psi
    +36 e^{3 \bar{v}+\frac{9 \sigma ^2}{2}} \mu  \sigma  \psi  \left(1+3
      \sigma ^2 \psi ^2\right)\\ 
    & +6 e^{2 \left(\bar{v}+\sigma ^2\right)} \mu ^2 \left(1+4 \sigma ^2 \psi
      ^2\right)
    +e^{4 \bar{v}+8 \sigma ^2} \left(3+96 \sigma ^2 \psi ^2+256 \sigma ^4
      \psi ^4\right)
  \end{aligned}
\end{equation}
These equations are rather complicated and directly solving them to
obtain the parameters $\mu$, $\sigma$, $\psi$ and $\bar{v}$ is
infeasible. However, in practice, $\E(r)$ is often very small --- so, to
get a rough estimate of the parameters, we may set $\mu = 0$ in the
above equations and, with a bit of manipulation, find the following
equation for $\psi\sigma$:
\begin{equation}
  \label{eq:moment1}
  {\E(r^3) \E(r^4)^{-3/8} \over \E(r^2)^{3/4}} =
  {9\sigma \psi (1 + 3\sigma^2\psi^2) \over (1 +
    4\sigma^2\psi^2)^{3/4}}
  {1
    \over
    (3 + 96\sigma^2\psi^2 + 256\sigma^4 \psi^4)^{3/8}
  }
\end{equation}
This equation can be solved numerically for a given sample to yield an
estimate for $\sigma\psi$, which in turn can be substituted in
equations \ref{eq:AsymmetricMoments}, where $\mu$ has been set to 0,
to give estimates for all the 4 parameters. These estimates then serve
as initial values in a numerical solution to
\ref{eq:AsymmetricMoments} where $\mu$ is kept as a free
variable. This full solution can now be used as the initial estimate
in an MLE procedure.

It is particularly important in our situation to have a good initial
estimate, because, as has been shown earlier, the PDF and CDF cannot
be obtained in closed forms and consequently have to be evaluated by
numerical integration. This is a rather costly procedure especially in
the context of MLE. So the computation of the moments under the
assumption of $\mu = 0$ is worthwhile.

Following the aforementioned procedure, i.e. computing the initial
estimate by matching the moments and then refining the estimate by
MLE, we obtain the parameter values for a number of return series,
including the Volvo B 30-minute returns described in
section \ref{sec:SLV_Symmetric}. In the following, we list the parameter
values and show the comparison between the empirical distribution
functions and their counterparts from the model.

Clearly these figures show a fairly good match of the empirical
distribution functions to their model counterparts. So we are once
again assured of the validity of ARIMA log-volatility models. However,
we also see that the skewness of the returns of Volvo B and Ericsson B
have rather different values when computed from the sample and from
the model. This could be the consequence of the limited sample size or
deficiencies in the estimation procedure described above. We leave
these issues to future studies.

%  This indicates the parameter $\psi$, which is the
% correlation between the log-volatility and the return, may be more
% complicated than the simple constant that has been assumed so far. We
% leave this to later studies.

\begin{table}[htb!]
  \centering
  \begin{tabular}{|c|c|c|c|c|}
    \hline
    & $\psi$ & $\sigma$ & $\bar{v}$ & $\mu$ \\
    \hline
    Volvo B & -1.5747e-02 & 4.6929e-01 & -6.2304e+00 &
    -5.3040e-05 \\
    Nordea Bank & 1.7234e-02 & 4.2780e-01 & -6.3758e+00 & 7.5230e-05
    \\
    Ericsson B & 1.8486e-02 & 4.4905e-01 & -6.2834e+00 & -9.2817e-05 \\
    \hline
  \end{tabular}
  \caption{\small \it Parameter Values of Selected Assets'
    returns. The time series are 30-minute returns and run from
    2013/10/10 to 2014/03/12.}
  \label{tab:assets_params}
\end{table}

\begin{table}[htb!]
  \centering
  \begin{tabular}{|c|c|c|c|c|}
    \hline
    & mean & std & skewness & kurtosis \\
    \hline
    \multirow{2}{*}{Volvo B} & -8.6766e-05 & 2.4500e-03 &
    -2.7217e-02 & 7.3394e+00 \\
    & -6.9281e-05 & 2.4538e-03 & -7.2677e-02 & 7.2472e+00 \\
    \hline
    \multirow{2}{*}{Nordea Bank} & 7.4821e-05 & 2.0500e-03 &
    6.9433e-02 & 6.7427e+00 \\
    & 8.8983e-05 & 2.0443e-03 & 6.7125e-02 & 6.2446e+00 \\
    \hline
    \multirow{2}{*}{Ericsson B} & -8.2606e-05 & 2.2939e-03 &
    2.6586e-01 & 7.6716e+00 \\
    & -7.5675e-05 & 2.2844e-03 & 7.8573e-02 & 6.7299e+00 \\
    \hline
  \end{tabular}
  \caption{\small \it Moments of Selected Assets' returns. For each return
    series, the 1st row contains the sample moments, while the 2nd
    constains those computed with MLE parameters. The time series are
    30-minute returns and run from 2013/10/10 to 2014/03/12.}
  \label{tab:assets_moments}
\end{table}

\begin{figure}[htb!]
  \centering
  \includegraphics[scale=0.5, clip=true, trim=50 233 63
  136]{../pics/Volvo_B_30m_returns.pdf}
  \caption{\small \it Volvo B 30min returns' unconditional distribution fit to
    the Stochastic Volatility model. The time series runs from
    2013/10/10 to 2014/03/12. Left: $P(r < x)$ with $x < 0$; Right:
    $P(r > x)$ with $x > 0$. Both plots are on log-log scale.}
  \label{fig:Volvo_B_30m_returns}
\end{figure}

\begin{figure}[htb!]
  \centering
  \includegraphics[scale=0.5, clip=true, trim=65 241 72
  143]{../pics/Nordea_30m_returns.pdf}
  \caption{\small \it Nordea Bank 30min returns' unconditional
    distribution fit to the Stochastic Volatility model. The time
    series runs from 2013/10/10 to 2014/03/12. Left: $P(r < x)$ with
    $x < 0$; Right: $P(r > x)$ with $x > 0$. Both plots are on log-log scale.}
  \label{fig:Nordea_30m_returns}
\end{figure}

\begin{figure}[htb!]
  \centering
  \includegraphics[scale=0.5, clip=true, trim=21 241 28
  146]{../pics/Ericsson_30m_returns.pdf}
  \caption{\small \it Ericsson B 30min returns' unconditional
    distribution fit to the Stochastic Volatility model. The time
    series runs from 2013/10/10 to 2014/03/12. Left: $P(r < x)$ with
    $x < 0$; Right: $P(r > x)$ with $x > 0$. Both plots are on log-log scale.}
  \label{fig:Ericsson_30m_returns}
\end{figure}

\section{Relation to Conditional Distribution Functions}
In the last section we have shown that the unconditional distribution of
the returns are skewed, which implies the variate in the
log-volatility $a$ is correlated to the variate in the return
$b$.

In the context of conditional distributions and forecast, this
correlation translates to the correlation between $b_t$ and the
residual of the log-volatility, denoted $y_t$ in the previous
chapters. Now let us consider the forecast function of an ARIMA model:
\begin{eqnarray*}
  \ln \sigma_t &=& y_t + \sum_{i=1}^P \phi_i \ln \sigma_{t-i} -
  \sum_{i=1}^Q \theta_i y_{t-i}
\end{eqnarray*}
where $\phi_i \leq 1$ if the model involves integration, and $\phi_i <
1$ otherwise. Comparing this equation with equation \ref{eq:r_t}, one
immediately realizes the PDF of $r_t$
\begin{eqnarray*}
  r_t &=& \mu + \sigma_t b_t
\end{eqnarray*}
is given by equation \ref{eq:UncondPDFAsymmetric1} and its moments
given by the equations \ref{eq:AsymmetricMoments} if one makes the
following substitutions:
\begin{eqnarray*}
  \sigma &\to& \text{std}(y_t) \\
  \bar{v} &\to& \sum_{i=1}^P \phi_i \ln \sigma_{t-i} - \sum_{i=1}^Q
  \theta_i y_{t-i} \\
  \psi &\to& \text{corr}(y_t, b_t)
\end{eqnarray*}

For the Nordea 15-minute series considered in section \ref{sec:nordea2_15min},
$\text{corr}(y_t, b_t)$ is found to be $3.56 \times 10^{-2}$; while
for the Volvo series considered in section \ref{sec:volvo},
$\text{corr}(y_t, b_t)$ is found to be $-7.3 \times 10^{-3}$. These
values are comparable to those in table \ref{tab:assets_params}, where
$\psi$ for the Nordea series is $1.7234 \times 10^{-2}$ and $-1.5747
\times 10^{-2}$ for the Volvo series. The similarity in these values
provides some evidence about the validity of the model.

It is certainly impossible to compare a predicted conditional
distribution with observations, but conditional distributions are of
great interest in the context of risk management and derivative
pricing, so we point out how they may be calculated for ARIMA
log-volatility models.
