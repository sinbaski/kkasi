\documentclass{report}
\usepackage{graphicx}
\usepackage{subfigure}
\usepackage{multirow}
\usepackage{wrapfig}
\usepackage{amssymb}
\usepackage{amsfonts, amsmath}
\usepackage{mathrsfs}
\usepackage{enumerate}
\usepackage[bookmarks=true]{hyperref}
\usepackage{bookmark}

\usepackage{amssymb,amsmath,amsthm,amsfonts}
\usepackage{mathrsfs}
\usepackage{dsfont}
\usepackage{enumerate}

%\newtheorem{mdef}{Definition}
%\newtheorem{theorem}{Theorem}
\newcommand{\eqsplit}[2]{
  \begin{equation}\label{#2}
    \begin{split}
      #1
    \end{split}
  \end{equation}}
\newcommand{\eqnsplit}[1]{
  \begin{eqnarray*}
    #1
  \end{eqnarray*}}
\newcommand{\tran}[1]{
  \tilde{#1}
}
\newcommand{\td}[2]{
  \frac{d #1}{d #2}
}
\newcommand{\pd}[2]{
  \frac{\partial #1}{\partial #2}
}
\newcommand{\ppd}[2]{
  \frac{\partial^2 #1}{\partial #2^2}
}
\newcommand{\pdd}[3]{
  \frac{\partial^2 #1}{\partial #2 \partial #3}
}
\newcommand{\otd}[1]{
  \frac{d}{d #1}
}
\newcommand{\opd}[1]{
  \frac{\partial}{\partial #1}
}
\newcommand{\oppd}[1]{
  \frac{\partial^2}{\partial #1^2}
}
\newcommand{\opdd}[2]{
  \frac{\partial^2}{\partial #1 \partial #2}
}
\newcommand{\ket}[1]{
  |#1\rangle
}
\newcommand{\bra}[1]{
  \langle#1|
}
\newcommand{\inn}[1]{
  \langle#1\rangle
}
\newcommand{\mean}[1]{
  \langle#1\rangle
}
\newcommand{\tr}{
  \text{tr}\,
}
\newcommand{\re}{
  \text{Re}\,
}
\newcommand\im{
  \text{Im}\,
}
\newcommand{\var}{
  \text{var}
}
\newcommand{\arcsinh}{
  \sinh^{-1}
}
\newcommand{\arccosh}{
  \cosh^{-1}
}
\newcommand{\erfc}{
  \text{erfc}
}
\newcommand{\E}{
  \mathbb{E}
}
\renewcommand{\P}{
  \mathbb{P}
}
\newcommand{\I}[1]{
  \mathbf{1}_{\{#1\}}
}
\newcommand{\1}[1]{
  \mathds{1}_{\{#1\}}
}
\newcommand{\diag}{
  \text{diag\,}
}
\newcommand{\M}{
  {\text{max}}
}
\newcommand{\m}{
  {\text{min}}
}
\newcommand{\ph}{
  {\text{arg}\,}
}
\newcommand\erf{
  \text{erf}
}
\renewcommand\vec[1]{
  \mathbf{#1}
}
\newcommand\mtx[1]{
  \mathbf{#1}
}
\newcommand\ed{
  \,{\buildrel d \over =}\,
}



\DeclareGraphicsExtensions{.pdf,.png,.jpg}

% \title{Stocks' Cross-Correlations}
% \subtitle{}

\author{Xie Xiaolei}
\date{\today}
\begin{document}

\begin{titlepage}
\begin{center}

% Upper part of the page. The '~' is needed because \\
% only works if a paragraph has started.
\includegraphics[width=0.5\textwidth]{../pics/lund_uni-logo_s}~\\[1cm]

% \textsc{\LARGE Lund University}\\[1.5cm]

\textsc{\Large Master's Thesis Project}\\[0.5cm]

% Title
%\HRule \\[0.4cm]
{ \huge \bfseries Returns Models and Their Implications
  on Cross-Correlation Matrix \\[0.4cm] }

%\HRule \\[1.5cm]

% Author and supervisor
\begin{minipage}{0.4\textwidth}
\begin{flushleft} \large
\emph{Author:}\\
\textsc{Xie} Xiaolei
\end{flushleft}
\end{minipage}
\begin{minipage}{0.4\textwidth}
\begin{flushright} \large
\emph{Supervisor:} \\
Prof. Sven \textsc{\AA berg}
\end{flushright}
\end{minipage}

\vfill

% Bottom of the page
{\large \today}

\end{center}
\end{titlepage}
% \maketitle
\tableofcontents

\chapter{Introduction}
The mathematics of the financial market has always been a topic that
arouses interest and imagination, and with no doubt, has been
studied from many aspects and in many different ways.

Central to all these studies are the concepts of probability
distributions and correlations. The values of stocks, futures,
options, etc. are stochastic in nature and are governed by the laws of
stochastic processes, which are expressed in terms of probability
distributions and correlations.

Meanwhile, the observables of the market are prices, volumes,
turn-over (the amount of money paid in a trade), names of the brokers,
and time of the trades. These quantities don't make much sense
by themselves but do reveal the probabilistic dynamics of the market
when put together and turned into statistics.

Here by dynamics we mean how the value of an asset is affected by its
own history as well as by the histories and current values of other
assets in the market. The influence from the asset's own history is
termed autocorrelations, i.e. correlations in time, while
the influence from other assets are termed cross correlations.

Mathematical models that describe the aforementioned observables in
terms of probability distributions, autocorrelations and cross
correlations give predictive power and help determine a fair price of
a given asset. Hence they are a central topic of the mathematics of
the financial market.

Once a model has been tentatively constructed, it must be
verified. The verification again relies on the statistics of the
observables. Over the years, a number of phenomena have been
consistently reported and become generally accepted. These are referred
to as stylized facts, which are detailed in \S
\ref{sec:StylizedFacts} and constitute a standardized test suit for
model verification.

% Judged by the stylized facts, each of the models proposed in the
% literature has its own advantages and disadvantages. The most
% prominent and influential among them are arguably the GARCH
% models. Their first and simplest variant was proposed by Bollerslev in
% 1990 \cite{Bollerslev86} and has been developed tremendously since
% then.

% An apparent advantage of the GARCH models is their simplicity. They
% are linear models, where the conditional variance is modeled as a
% linear combination of past variances and past innovations squared. As
% a result, it is relatively easy to fit a model to a given set of data,
% i.e. to estimate the model parameters with respect to the
% data. Because of the linearity, the estimation of model parameters
% often give consistent results, regardless of the choice of initial
% parameter values or the choice of pre-sample data.

% In contrast, many of the other types of models are often specified in
% a complicated, although more advanced way, for example, by stochastic
% differential equations. This kind of specifications certainly have the
% advantage of directly linking to the theory of stochastic processes,
% but nonetheless are difficult to fit to data, one of the most tricky
% problems being the high sensitivity to the choice of initial parameter
% values.

% Another advantage of GARCH models is that their tail behavior is
% better understood than that of their competitors. The simplest variant
% of GARCH, namely the GARCH(1,1) models are proven to have regularly
% varying tails, or in common terms, power-law tails
% \cite{mikosch2000}. In particular, if the $\alpha_1$ and $\beta_1$
% parameters of the model sum up to 1, the return series generated by
% the model has a power-law tail with power exponent 2, and consequently
% the squared returns have a tail with power exponent 1 -- by the
% central limit theorem the sum of the squared returns will converge to
% the Cauchy distribution, a stable distribution with L\'evy index 1. We
% explore this fact in more details in chapter
% \ref{chp:CrossCorrelationFat} where the cross-correlation matrix is
% studied.

% The GARCH model with $\alpha_1 + \beta_1 = 1$ is more often referred
% to as an IGARCH model, aka. an integrated GARCH model. It is of
% particular interest because, very often, when fitting a GARCH(1,1)
% model to real financial data, one finds $\alpha_1 + \beta_1$ extremely
% close to 1, i.e. at values like 0.99, 0.995, etc.

% A third advantage of GARCH models is their flexibility. Namely one can
% choose any appropriate distribution to model the innovations. While
% the Gaussian and the student's t distributions are the most common,
% the use of other distributions such as asymmetric student's t and
% Johnson Su has also been studied in the literature (see
% \cite{Simonato2012}).

% However, GARCH models have their disadvantages too. The original
% GARCH model does not account for skewness in the distribution of
% the returns. As a result, the skewness, often attributed to
% price-volatility correlations, have to be added ``by hand''. This kind
% of treatment fixes the model on the one hand, but introduces some
% arbitrariness on the other. Put in another way, skewness does not
% follow naturally from the structure of the model but rather is added
% explicitly.

% Moreover, fitting a GARCH model often results in a large number of
% parameters; 16 to 30 are not uncommon when intra-day data are being
% dealt with. This raises the question of true fitness: the model
% specification may indeed fit the selected data set but very well fall
% short of an accurate description of the time series in question. In
% other words, such a model is unstable and hence unreliable: if one
% removes a small fraction of data from the beginning of the set and
% appends some to the end -- a necessary operation for the evolution of
% the model -- the model parameters change drastically and so do the
% predictions.

% As intra-day transaction data become more accessible, the volatility,
% which has been considered an unobservable quantity by models like
% GARCH, can be estimated by the so called realized volatility, which is
% the square root of the sum of squared returns that have been sampled
% at a higher frequency than are those that one intends to model. For
% example, it has been shown that the volatility of daily returns is
% best estimated using 17-minute intra-day returns \cite{Sahalia05}.

% Now that the volatility can be estimated, models that take advantage
% of this are expected to achieve significantly higher accuracy than
% those that do not. The model presented in this thesis is built indeed
% in this spirit.

% Like the GARCH models, our proposed model has the simple structure
% $r_t = \sigma_t z_t$, where $z_t$ has unit variance and is
% drawn from any appropriate distribution; $\sigma_t$ is the
% volatility. Unlike GARCH, which treats $\sigma_t$ as deterministic at
% time $t$, we treat $\sigma_t$ as random and model $\ln\sigma_t$ using
% time series analysis. This is possible because past values of
% $\ln\sigma_t$ can be estimated using past values of the returns at a
% higher frequency, as has just been discussed.

% Our model has a number of advantages too. Like the original
% GARCH model, it is simple. It has two random variables $z_t$ and
% $\ln\sigma_t$, the latter being the subject of time series
% modeling. The use of conventional time series analysis means the
% model can be not only more accurate but also more economic -- long
% memory of the volatility, which is the cause of a large number of
% parameters in a GARCH model, can be modeled by differencing or
% fractional differencing. As a result, much fewer parameters are needed
% for the model.

% Secondly, unlike GARCH models, skewness follows directly from the
% correlation between the two random variables $z_t$ and
% $\ln\sigma_t$. As shown in chapter \ref{chp:PriceModels},
% positive/negative correlation leads to positive/negative skewness,
% and skewness increases with the strength of correlation.

% Thirdly, our model has sub-exponential tails, as shown in chapter
% \ref{chp:PriceModels}. This type of tails are not as fat as regularly
% varying tails, but in our specific case, the tails do behave
% approximately as a power-law in the regions relevant to stock and
% index returns. After all, power-law behavior is an empirical result
% obtained in the returns data, and such data are not available to a
% statistically significant amount at extremely large values. In plain
% words, situations where one makes a 20\%+ profit on a single stock
% trade in a single day happen too rarely to allow any statistical
% statement.

Another important part of the thesis is about the cross-correlation
matrix, for which we study the distribution of its elements as well
as its eigenvalues. When the cross-correlation matrix is constructed
from returns with simple Gaussian distribution, the matrix is called a
Wishart Matrix and has been studied extensively in the literature. If
the returns have L\'evy distributions, the matrix is termed
Wishart-L\'evy and has been studied to some extent, particularly
regarding its eigenvalue distribution \cite{politi2010}.

However, it is understood that real stock/index returns are much more
complicated than a straight-forward Gaussian or L\'evy distribution can
describe - instead, one needs structured models for their description.

Therefore in this thesis we study how the elements and the eigenvalues
of the cross-correlation matrix are distributed when the matrix is
built from returns described by different returns' models, specifically
GARCH(1,1) and ARIMA-log-volatility models, whose the tail behavior is
understood. Moreover, we also study how auto-correlations in the
returns influence the aforementioned distributions -- such
auto-correlations, known as second-order auto-correlations decay
exponentially but may still leave footprints in the cross-correlation
matrix.

The rest of this report is organized as follows: in
\S\ref{sec:FundamentalConcepts} we present a few very important
concepts and notations that we will often refer to in later
chapters. Then chapter \ref{chp:PriceModels} reviews some of the most
influential returns' models, calibrates them to a few intra-day return
series and compares their performance. The unconditional distribution
functions of ARIMA-log-volatility models, which is not seen to be
given a full discussion in the literature, are also calculated and
discussed. Chapter \ref{chp:Gaussian} investigates the elements' and
eigenvalues' distributions of the Wishart matrix, where the influence
of auto-correlations in the constituent returns are studied in
detail. Chapter \ref{chp:CrossCorrelationFat} looks at the elements'
and eigenvalues' distributions 


\section{Fundamental Concepts \&
  Notations}\label{sec:FundamentalConcepts}
This section is a list of a few concepts that may be unfamiliar to the
reader and that we will often refer to later in the thesis:
\begin{itemize}
\item Return. Given a fixed time interval $[t - \Delta t, t]$, for example, a
  day, a week, a month, etc, and a particular asset, for example, a
  share in company ABC, the return of this asset over the time
  interval is defined as
  \begin{eqnarray*}
    r_{t, \Delta t} &=& \ln p_{t} - \ln p_{t - \Delta t} \\
    &=& \ln \left(1 + {p_{t} - p_{t - \Delta t} \over p_{t - \Delta t}}\right) \\
    &\approx& {p_{t} - p_{t - \Delta t} \over p_{t - \Delta t}}
  \end{eqnarray*}
  where $p_t$ is the price of the asset at time $t$. $\Delta t$ is
  sometimes called the time-lag of the return. Quite often, where
  confusion is not possible, we will just write $r_t$ to mean $r_{t,
    \Delta t}$, the time lag either does not matter or is clear from
  the context.

\item Autocorrelation. By autocorrelation, denoted $\rho_k$ here, we
  mean the correlation between two temporally separated observations
  of the same time series:
  \begin{eqnarray*}
    \rho_k &=& {
      \E\left[(a_t -\E(a_t))(a_{t-k} - \E(a_{t-k})\right]
      \over
      \sqrt{\text{var}(a_t)}\sqrt{\text{var}(a_{t-k})}
    }
  \end{eqnarray*}
  Here $\E(x)$ stands for the expectation value of $x$, and
  $\text{var}(x)$ stands for the variance of $x$. $k$ is called the
  time-lag and is the temporal seperation of the two observations
  measured by the number of observations in between. For example, the
  time-lag between the 1st and 3rd observation is 2.

\item Cross-Correlation Matrix. Correlations between the returns of a
  group of assets are described by the cross-correlation matrix. When
  the asset returns are described by a stable distribution law with
  L\'evy index $\alpha$, the empirical cross-correlation matrix is
  constructed as
  \begin{equation}
    \label{eq:cross-correlation}
    \begin{aligned}
      C_{ij} &= {1 \over T^{2/\alpha}} \sum_{t=1}^T [r_{i,t}-\E(r_i)]
      [r_{j,t}-\E(r_j)] \\
      &= {1 \over T^{2/\alpha}} RR'
    \end{aligned}
  \end{equation}
  where $r_{i,t}$ is the return of the i-th asset at time t and is
  placed at the entry (i, t) of matrix R; $R'$ denotes the
  transpose of R. In most practical situations, one has abundant data
  for each and every asset. Thus $T \geq N$ is assumed throughout this
  thesis.

\item Auto-regressive processes. A time series $r_t$ is called an
  auto-regressive process of order $p$ and denoted AR(p), if it can be
  written in the following form:
  \begin{eqnarray*}
    r_t &=& \sum_{i=1}^p \phi_i r_{t-i} + a_t
  \end{eqnarray*}
  where, for all $i$, $\phi_i \in (-1, 1)$, and the $a_t$'s are
  independent and identically distributed (iid.) random variables with
  zero mean. Obviously this implies the mean of $r_t$ is zero
  too. Apart from this, their distribution of $a_t$ is not restricted
  to any particular form.
  
  Of particular interest to this thesis is the AR(1) process:
  \begin{eqnarray*}
    r_t = \phi r_{t-1} + a_t
  \end{eqnarray*}
  Its autocorrelation function $\rho_k = \text{corr}(r_t, r_{t-k})$
  ($k = 0, 1, 2, \cdots$) can be easily shown to fall off
  exponentially:
  \begin{eqnarray*}
    \rho_k &=& {\E(r_tr_{t-k}) - \E(r_{t-k})\E(r_{t}) \over
      \sqrt{\var(r_t) \var(r_{t-k})}} \\
    &=& {\phi \E(r_{t-1}r_{t-k}) + \E(a_t r_{t-k})
      \over
      \sqrt{\var(r_t) \var(r_{t-k})}
    } \\
    &=& \phi \rho_{k-1}
  \end{eqnarray*}
  where $\E(a_t r_{t-k}) = 0$ follows from the fact that any return
  $r_{t1}$ must not depend on disturbances $a_{t2}$ that occur later
  in time. The last equation means $\rho_k$ is a geometric series. Since
  $\rho_0 = 1$, we have
  \begin{eqnarray*}
    \rho_k &=& \phi^k \\
    |\rho_k| &=& e^{k\ln|\phi|} \\
  \end{eqnarray*}
  Note that $|\phi| < 1$ and hence $\ln|\phi| < 0$.

  Although $k$ can only take integer values, it is still useful to
  define a correlation time $\tau$ such that $\phi^\tau = 1/2$. Such a
  quantity is more intuitive and constitutes a measure of
  autocorrelations that is universal and comparable among different
  time series' models. From the definition of $\tau$ we get
  \begin{equation}
    \label{eq:tau_def}
    \begin{aligned}
      \phi^\tau &= 1/2 \\
      \tau &= -{\ln 2 \over \ln\phi} \\
      \phi &= 2^{-1/\tau}
    \end{aligned}
  \end{equation}
  Figure \ref{fig:AR1-autocorrelation} shows the autocorrelation
  function of the AR(1) model. As proven above, this function decays
  exponentially.
  \begin{figure}[htb!]
    %\vspace{-15mm}
    \centering
    \includegraphics[scale=0.5, clip=true, trim=113 229 115
    139]{../pics/AR1-autocorrelation.pdf}
    \caption{\small \it Autocorrelation function of the AR(1) model
      with $\phi=1/\sqrt{2}$. Red circles: autocorrelations at $k=0, 1,
      2, \cdots$. Blue line: $e^{t\ln\phi}$.}
    \label{fig:AR1-autocorrelation}
  \end{figure}
  For more details about conventional time series' models, see
  \cite{BoxJenkins94}.
\end{itemize}

For purposes of later reference, we also list some notations that may
cause confusion to the reader:
\begin{itemize}
\item $\E(x)$ or $\mean{x}$: The expectation value of $x$.
\item $\var(x)$: the variance of $x$.
\item $\text{cov}(x,y)$: the covariance of $x$ and $y$.
\item $\text{corr}(x,y)$: the correlation between $x$ and $y$, i.e.
  \begin{equation*}
    \text{corr}(x,y) = {\text{cov}(x,y) \over \sqrt{\var(x)\var(y)}}
  \end{equation*}
\end{itemize}

\section{Stylized facts}\label{sec:StylizedFacts}
This section presents and explains a few ``stylized facts'',
i.e. phenomena that are widely observed and accepted as true.
\begin{itemize}
\item Fat tails. It has been consistently reported by various studies
  - for example \cite{Potters2003} and \cite{Mantegna2000} - that the
  probability density function (PDF) of stock/index returns are not
  Gaussian. Unlike the Gaussian PDF, which is symmetric and falls off
  very quickly as its argument moves from the center to the outskirts
  (tails), the PDF of stock/index returns are higher on the tails,
  i.e. the probability of large fluctuations is higher than is dictated
  by a Gaussian distribution - in fact, even higher than dictated by
  an exponential function. Empirical studies suggest the distribution
  function of the returns follows a power-law on the tails --- the
  exponent of the power depends on the specific stock/index.

  Figure \ref{fig:FatTail} illustrates this feature.
  \begin{figure}[htb!]
    \centering
    \includegraphics[scale=0.6, clip=true, trim=92 229 110
    140]{../pics/FatTail.pdf}
    \caption{\small \it Fat tails of {\it Nordea Bank} 15-minute
      returns during the period 2013-10-10 and 2014-01-29. The returns
      are computed using minute-by-minute average prices.}
    \label{fig:FatTail}
  \end{figure}

\item Non-zero skewness. Apart from fat tails, the PDF of stock/index
  returns are often also skewed. If the skewness is positive
  (negative), the probability of very large positive (negative)
  returns is higher than that of very large negative (positive)
  returns, even though the mean of the returns is 0 or extremely
  close to 0.

  Table \ref{tab:EmpiricalSkewness} lists the skewness of a few
  Swedish stocks traded on the Stockholm OMX market.
  \begin{table}[htb!]
    \footnotesize
    \centering
    \begin{tabular}{|c|c|c|c|c|c|c|}
      \hline
      Nordea Bank & Volvo B & Boliden & ABB Ltd & H\&M & Scania
      B & Ericsson B \\
      \hline
      0.1362 & 0.0471 & 0.0567 & -0.0364 & -0.0168 &
      -1.1554 & 0.2086 \\
      \hline
    \end{tabular}
    \caption{\small \it Skewness of Stock Returns. All the returns
      have time-lag of 15 minutes and are computed using paid prices
      between 2013-10-10 and 2014-01-29. }
    \label{tab:EmpiricalSkewness}
  \end{table}

\item Higher-order autocorrelation. To the lowest order, return series
  are not auto-correlated --- if they are, the auto-correlations would
  present an obvious opportunity of making easy profit and be
  exploited and vanish as soon as they appear. However, the squared
  returns do have significant auto-correlations, as figure
  \ref{fig:nordea_15min_acf} illustrates. These auto-correlations
  suggest the variances of the returns at subsequent time steps are
  correlated. These are referred to as higher-order auto-correlations
  and make a subjet of returns' models.
  
\end{itemize}

\chapter{Return Models}\label{chp:PriceModels}
In this chapter we review some of the discrete-time return models and
fit them to intraday returns. The intention is to compare these models
in terms of forecast accuracy and to understand their statistical
properties. In the following we first describe these models briefly,
then in section \ref{chp:nordea_15min}, section \ref{sec:volvo_30} and appendix
\ref{chp:appendix2} we describe how the \gls{garch} and the \gls{sv} models
are fitted to intraday returns and compare their forecasts. In section
\ref{sec:XieCalc} we calculate the unconditional distribution
functions of the \gls{sv} model.

% \gls{frt} is good.
% section \ref{sec:Garch_model} reviews the \gls{garch} model using the Nordea
% 15-minute returns as an example. Section \ref{sec:SLV_model} examines the
% stochastic log-volatility model and discusses the possibility of
% describing the residuals of the log-volotility series using the
% Johnson Su distribution. The same Nordea return series is fitted as an
% example. Section \ref{sec:XieCalc} derives the unconditional distribution
% function of the stochastic log-volotility model and compares the
% analytic results to the 30-minute returns of a number of Swedish
% stocks.

\begin{enumerate}
\item{\bf Gaussian Distribution}

The justification of modeling return series as independent,
identically distributed Gaussian variates comes from imagining the
price process $S(t)$ as a geometric Brownian motion whose increment
$\sigma dw_t$ at each time step is independent and scales as $\sigma
\sqrt{dt}$. Then, by adding a drift term $\mu dt$ that represents some
deterministic trend in the price process, one can express the
price $S(t)$ as a stochastic differential equation:
\begin{eqnarray*}
  dS &=& S\mu dt + S\sigma dw_t \\
\end{eqnarray*}
Then by It\^o's lemma, a stochastic differential equation for the
logarithmic price $\ln S$ can be obtained
\begin{eqnarray*}
  d(\ln S) &=& (\mu - \frac{1}{2} \sigma^2)dt + \sigma dw_t \\
\end{eqnarray*}
It follows from this equation that $\ln S(t) - \ln S_0$, where $S_0$
is the price at time 0, has Gaussian distribution with mean $(\mu -
\sigma^2/2)t$ and variance $\sigma^2 t$ \cite{Bernt2000}. Therefore,
in a discrete-time model, where the length of each time step $\Delta
t$ is fixed, the return $r_t = \ln S(t) - \ln S(t - \Delta t)$ is
assumed to be Gaussian distributed and have mean and variance that are
functions of $\Delta t$.

The picture depicted above is of course overly simplified, and the
distribution of returns is not really Gaussian. Nevertheless, the
assumption of Gaussian distributed returns underlies such important
theories as Black and Scholes theory of option pricing. Thus, albeit
inaccurate in describing very large returns, the Gaussian distribution
as a return model shall not be forgotten. In the next few sections,
we discuss some more realistic return models.

\item{\bf \gls{garch} models}

``GARCH'' is the acronym for ``Generalized Autoregressive Conditional
Heteroscedasticity''. A \gls{garch}(p, q) model is defined by the following
equation system \cite{Bollerslev86}:
\begin{equation}
  \label{eq:garch_def}
  \begin{aligned}
    r_t &= \mu_t + \epsilon_t \\
  \end{aligned}
\end{equation}
where $r_t$ is the return as mentioned earlier; $p$ and $q$ are
constant integers; $\mu_t$ is the mean process and considered a small
constant for intraday returns. $\epsilon_t$ is called the innovation
of the return and is a variate whose conditional distribution
\footnotemark is Gaussian and has variance $\sigma^2_t$:
\begin{eqnarray*}
  \sigma_t^2 &=& \alpha_0 + \sum_{i=1}^q \alpha_i \epsilon_{t-i}^2 +
  \sum_{i=1}^p \beta_i \sigma_{t-i}^2
\end{eqnarray*}
\footnotetext{
By conditional distribution we mean the distribution conditional on
the variate's history: The conditional distribution at time $t$ means
the distribution conditional on the history up to time $t-1$
in the discrete-time setting.
}

It can be shown that the autocorrelation function $\varrho_n =
\text{corr}(\epsilon_t, \epsilon_{t-n})$ for $n > \text{max}(p,q)$ of
$\epsilon_t^2$ is given by \cite{Bollerslev87}:
\begin{eqnarray*}
\varrho_n &=& \sum_{i=1} ^{p \vee q}
(\alpha_i + \beta_i) \varrho_{n-i} \text{ for $n > p$}
\end{eqnarray*}
where $\alpha_i$ with $i > q$ and $\beta_i$ with $i > p$ are taken as
zeros. $p \vee q$ denotes the maximum of p and q. From these
equations, it is clear that the partial autocorrelation function cuts
off at $\max(p, q)$.

Here we note that, in the \gls{garch} model, the volatility $\sigma_t$ is
$\mathcal{F}_{t-1}$ measurable: given the history up to time $t-1$,
$\sigma_t$ is a deterministic quantity. In contrast, \gls{sv} models,
which we discuss shortly, treat $\sigma_t$ as a random variable even
given the same history.

\item {\bf Stochastic Volatility Models}\label{sec:SLV_model}

For the purpose of intraday returns, we specify the \gls{sv} model as
\begin{eqnarray}
   r_{t, t-h} &=& \ln S_{t} - \ln S_{t-h} \nonumber \\
   r_{t, t-h} &=& \mu + \sigma_{t, t-h} b_t \label{eq:SLV_spec}
\end{eqnarray}
where $S_t$ is the price of the asset at time $t$; $b_t \sim N(0,
1)$; $r_{t, t-h}$ is the return over the time interval $[t-h, t]$.
For simplicity, in the rest of this chapter we shall just write
$t$ for the subscript ``t, t-h'', since the time interval $h$ is fixed
for each time series and is a known constant.

Andersen et al proved the following in \cite{Andersen03} (theorem 2):
\begin{equation}
  \label{eq:normal_r}
  r_t|\mathcal{F}_{t-h} \sim N(\int_{0}^h \mu_{t-h+s} ds, \int_{0}^h
  \sigma_{t-h+s}^2 ds)
\end{equation}
In plain words, conditional on the information up to time $t-h$, the
distribution of $r_t$ is Gaussian with integrated mean and variance.
The variance of this conditional distribution, $\int_{0}^h
\sigma_{t-h+s}^2 ds$, can be approximated by \cite{Protter05,
  Andersen03}:
\begin{equation}
  \label{eq:rv_def}
  \int_{0}^h \sigma_{t-h+s}^2 ds = \sum_{k=1}^n \left(
    \ln S_{t-h+kh/n} - \ln S_{t-h+(k-1)h/n} \right)^2
\end{equation}
where $n$ is a chosen constant. The square root of the right hand side
of equation \ref{eq:rv_def} is the realized volatility, call it
$\hat{\sigma}_t$.

With the availability of transaction data, estimating conditional
volatility using returns sampled at a higher frequency (realized
volatility) gives superior accuracy and reliability. However, at which
frequency the time series should be sampled (the choice of $n$) in
order to give an unbiased and consistent estimate of the volatility is
not a trivial question. Naively one would believe that the often the
series is sampled, the better the estimate, but in fact, due to noise
introduced by market micro-structure, there is an optimal sampling
frequency, depending on $h$. While the method to determine this
optimal frequency is a subject of debate (see for example
\cite{Sahalia05}), it is not difficult to find a fairly satisfactory
frequency in practice:

Keeping equation \ref{eq:normal_r} in mind, one can simply try a few
frequencies and compare the distribution of $(r_t -
\E(r_t))/\hat{\sigma}_t$ with the standard Gaussian. If the two match,
$\hat{\sigma}_t$ is a good approximation of
\[
\sigma_t = \left(\int_{0}^h \sigma_{t-h+s}^2 ds \right)^{1/2}
\]

Once a good approximation of $\sigma_t$ has been found, it is
convenient to model $\ln\sigma_t$ so that the positivity of $\sigma_t$
is implied by construction \cite{Mikosch2009}. As is seen in later
sections, \gls{arima} models often serve well for this purpose. An
\gls{arima}(p, d, q) model, where $p,d,q$ are integers, is defined as
\cite{BoxJenkins94}
\begin{eqnarray*}
  (1 - \sum_{i=1}^p \phi_i B^i) (1 - B)^d \ln \sigma_t &=& (1 -
  \sum_{i=1}^q \theta_i B^i) y_t
\end{eqnarray*}
where $B$ is the back-shift operator such that $B x_t = x_{t-1}$ for
any time series $x_t$. $y_t$ are termed the residuals of the
model, $d$ is the order of integration, and $p, q$  are orders of
autoregression and moving average, respectively. $\phi_i$ and
$\theta_i$ are constant parameters.

Quite often, the auto-correlation function of the time series in
question manifests periodic patterns, thus seasonal components are
added to the above model to account for the seasonality:
\begin{eqnarray*}
&& (1 - \sum_{i=1}^P \Phi_i B^{is}) (1 - \sum_{i=1}^p \phi_i B^i) (1 - B^s)^D (1 -
B)^d\ln \sigma_t \\
&& = (1 - \sum_{i=1}^Q \Theta_i B^{is}) (1 -
\sum_{i=1}^q \theta_i B^i)y_t
\end{eqnarray*}
where $D$, analogous to $d$ in the non-seasonal model, is the order of
seasonal integration; $P, Q$ are orders of seasonal autoregression and
moving average, respectively. $\Phi_i$ and $\Theta_i$ are constant
parameters. In the most general situations, the seasonal and the
non-seasonal components do not necessarily combine in the above
multiplicative fashion, so the following model is also of interest:
\begin{eqnarray*}
(1 - \sum_{i=1}^{p+P} \phi_i B^i) (1 - B^s)^D(1 - B)^d \ln
\sigma_t &=& (1 - \sum_{i=1}^{q+Q} \theta_i B^i) y_t
\end{eqnarray*}
In all the above cases, if $d = 0$ and $D = 0$, the model does not
involve integration and hence is called an \gls{arma}(p,q) model. An
even simpler case is where $q = 0$ and $Q = 0$ in addition to $d = 0$
and $D = 0$. These conditions make the model a pure \gls{ar}
process, denoted AR(p).
\end{enumerate}

To compare \gls{garch} and \gls{sv} models at the face of intraday returns,
we study the {\it Nordea Bank} 15-minute returns during the period
2012/01/16 - 2012/04/20 in section \ref{chp:nordea_15min}, 
and {\it Volvo B} 30-minute returns during 2013/10/10 - 2014/04/04 in
section \ref{sec:volvo_30}. Another 4 intraday series are also studied and
collected in appendix \ref{chp:appendix2}. We have chosen these
particular stocks because they have the largest trading volumes in the
Swedish market and hence provide the largest amount of data for
analysis. The time intervals of 15 and 30 minutes are chosen because
they are relatively short and hence give a large amount of data and
yet they are not too short for noise of market
micro-structure\footnotemark to become a concern.
\footnotetext{
Noise of market micro-structure is due to, most importantly,
the difference between the bid and asked prices (bid-ask spread), and
the discreteness of price changes.
}

\section{Case Study: Nordea 15-minute Returns}
\label{chp:nordea_15min}
In this section we investigate the Nordea 15-minute returns sampled
during the period 2012/01/16 - 2012/04/20. In total, these amount to
2022 returns. We use the first 80\% (1617) for model estimation and  the
remaining 20\% (405) for comparing with model forcasts. In section
\ref{sec:nordea_15min_garch} we study the series with a \gls{garch} model 
and in section \ref{sec:nordea_15min_arima} we study it with a
\gls{sv} model.

\subsection{GARCH Model}\label{sec:nordea_15min_garch}
When volatilities are auto-correlated or squared returns and
volatilities are correlated, a \gls{garch}(p, q) model may be
appropriate for the return series under investigation. To find out
whehter this is true in our case, we plot the \gls{acf} of the squared
returns. This is shown in figure \ref{fig:nordea_15min_acf}. If the
aforementioned features are absent from the series, the
auto-correlations are expected to be Gaussian distributed with mean 0
and variance 1/T, and hence mostly reside within the confidence bounds
set by $\pm 2/\sqrt{T}$ \cite{Bollerslev86, Bollerslev87}.

Clearly this is not the case in figure \ref{fig:nordea_15min_acf} ---
the first 5 auto-correlations are rather significant. Moreover, figure
\ref{fig:nordea_15min_vlt_acf} shows even more clearly that the
conditional variances of the series are correlated. These observations
suggest a \gls{garch}(p, q) model can be appropriate.
\begin{figure}[htb!]
  \centering
  \subfigure[]{
    \includegraphics[scale=0.4, clip=true, trim=90 257 105
    220]{../pics/nordea_15min_acf.pdf}
    \label{fig:nordea_15min_acf}
  }
  \subfigure[]{
    \includegraphics[scale=0.4, clip=true, trim=90 259 104
    218]{../pics/nordea_15min_vlt_acf.pdf}
    \label{fig:nordea_15min_vlt_acf}
  }
  \caption{\small \it \ref{fig:nordea_15min_acf}: Auto-correlations
    (ACF) of the squared returns; \ref{fig:nordea_15min_vlt_acf}:
    Auto-correlations (ACF) of squared realized volatilities
    ($\hat{\sigma}^2_t$). The blue lines are confident bounds set at
    $\pm 2/\sqrt{T}$. $T$ is the length of the time series.}
\end{figure}

Starting with a \gls{garch}(1,1) model and taking advantage of the knowledge
that the log-volatility $\ln \sigma_t$ has seasonality $s=33$ (see
figure \ref{fig:nordea_15min_logvol_acf}), we fit to the return series
a \gls{garch}(33, 33) model, limiting to lags 1 and 33 for both ARCH and
\gls{garch} parameters.
\begin{eqnarray*}
  r_t &=& \mu + \epsilon_t \\
  \epsilon_t &=& \sigma_t z_t \\
  \sigma^2_t &=& \alpha_0 + \alpha_1 \epsilon^2_{t-1} + \alpha_s
  \epsilon^2_{t-s} + \beta_1 \sigma^2_{t-1} + \beta_s \sigma^2_{t-s}
\end{eqnarray*}
where the mean process of $r_t$, denoted $\mu_t$ earlier, is simplified
to a mere constant $\mu$ owing to its smallness. by \gls{mle},
parameter values listed in table \ref{tab:nordea_15min_garch} are obtained.
\begin{table}[htb!]
  \centering
  \begin{tabular}{|c|c|c|c|c|c|}
    \hline
    Parameter & $\alpha_0$ & $\alpha_1$ & $\alpha_s$ & $\beta_1$ &
    $\beta_s$ \\
    \hline
    Value & $4.7833 \times 10^{-7}$ & 0.1600 & 0.0667 & 0.6846 &
    0.0342 \\
    \hline
  \end{tabular}
  \caption{\small \it GARCH model parameters}
  \label{tab:nordea_15min_garch}
\end{table}

\subsection{Stochastic Volatility Model}\label{sec:nordea_15min_arima}
For the Nordea Bank 15-minute returns under consideration, it can be
verified that the square root of the sum of squared 30-second returns
makes a good proxy for the volatility. This can be seen from the
probability plot of $z_t = (r_t - \E(r_t))/\hat{\sigma}_t$ (figure
\ref{fig:nordea_bank_15min_z_prob}), i.e. the quotient of the 
15-minute returns over the volatility proxy.
\begin{figure}[htb!]
  \centering
    \includegraphics[scale=0.4, clip=true, trim=80 258 104
    220]{../pics/nordea_bank_15min_z_prob.pdf}
  \caption{\small \it Probability plot of $z_t =
    (r_t-\E(r_t))/\sigma_t$. $\epsilon_t$ are derived 
      from Nordea Bank 15min returns while $\sigma_t$ are realized
      volatilities calculated using 30s returns within each 15min
      interval. Horizontal axis: $z_t$; Vertical axis: CDF of
    $z_t$, arranged on such a scale that the CDF of the standard
    Gaussian is a straight line.}
  \label{fig:nordea_bank_15min_z_prob}
\end{figure}
% In addition, one can see
% from figure \ref{fig:nordea_15min_quotient_acf} and
% \ref{fig:nordea_15min_quotient_squared_acf} that there is essentially
% no auto-correlation in the $z_t$ or the $z_t^2$ series.
% \begin{figure}[htb!]
%   \centering
%   \subfigure[ACF of $z_t$]{
%     \includegraphics[scale=0.4, clip=true, trim=95 236 118
%     200]{../pics/nordea_15min_quotient_acf.pdf}
%     \label{fig:nordea_15min_quotient_acf}
%   }
%   \subfigure[ACF of $z_t^2$]{
%     \includegraphics[scale=0.4, clip=true, trim=95 236 118
%     200]{../pics/nordea_15min_quotient_squared_acf.pdf}
%     \label{fig:nordea_15min_quotient_squared_acf}
%   }
%   \caption{\small \it Nordea 15min $z_t$ and $z_t^2$ ACF.}
% \end{figure}

Andersen and Bollerslev et al reported that, for the exchange
rates between Deutch mark, yen and dollar, $\ln \sigma_t$ is
Gaussian distributed \cite{Andersen03}. This is, however, not the case
for our modestly sized series. In fact, in our case, $\ln \sigma_t$ is
right skewed (skewness 0.3342) and leptokurtic (kurtosis 6.1006). See
figure \ref{fig:nordea_15min_logvol_prob}
\begin{figure}[htb!]
  \centering
  %\vspace{-15mm}
  \includegraphics[scale=0.4, clip=true, trim=80 223 107
  4]{../pics/nordea_15min_logvol_prob.pdf}
  \caption{\small \it Probability plot of Nordea 15min log-volatility
    $\ln\sigma_t$ unconditional distribution}
  \label{fig:nordea_15min_logvol_prob}
\end{figure}
Moreover, the series of $\ln\sigma_t$ shows long-lasting and
periodic autocorrelations with an apparent period of 33 (see figure
\ref{fig:nordea_15min_logvol_acf}). This suggests the series may be
described by a seasonal \gls{arima} model. Thus we first simplify the
series by differencing \cite{BoxJenkins94}:
\begin{equation}
  \label{eq:differenced_lv}
  w_t = (1-B)(1-B^s)\ln\sigma_t  
\end{equation}
where $B$ is the back-shift operator\footnote{For example, $B\,x_t =
  x_{t-1}$} and $s=33$ is the seasonality.

The autocorrelation function of the differenced process $w_t$, as
shown in figure \ref{fig:nordea_15min_w_acf}, clearly points to a seasonal
moving-average model: There are only 4 non-zero autocorrelations in
the plot, located at lags 1, 32, 33, 34, respectively; furthermore,
the two at 32 and 34 are approximately equal. Thus we can write down
the model as
\begin{eqnarray}
  w_t &=& (1 - \theta B)(1 - \Theta B^s) y_t \label{eq:nordea_w}
\end{eqnarray}
where $\theta$ and $\Theta$ are parameters to be determined and $y_t$
is a noise process with constant variance $\sigma_y^2$ and mean
0. $y_t$ is often refered to as the residuals.
\begin{figure}[htb!]
  \centering
  \subfigure[ACF of log-volatility ($\ln\sigma_t$)]{
    \includegraphics[scale=0.4, clip=true, trim=95 230 112
    235]{../pics/nordea_15min_logvol_acf.pdf}
    \label{fig:nordea_15min_logvol_acf}
  }
  \subfigure[ACF of differenced log-volatility ($w_t$)]{
    \includegraphics[scale=0.4, clip=true, trim=95 230 112
    235]{../pics/nordea_15min_w_acf.pdf}
    \label{fig:nordea_15min_w_acf}
  }
  \caption{\small \it Auto-correlations of Nordea 15min log-volatility
    ($\ln\sigma_t$) and differenced log-volatility ($w_t$).}
  \label{fig:nordea1_15min_acf}
\end{figure}

The above seasonal moving average model has the following
autocovariance structure \cite{BoxJenkins94}:
\begin{eqnarray*}
  \gamma_0 &=& \sigma_y^2 (1 + \theta^2)(1 + \Theta^2) \\
  \gamma_1 &=& -\sigma_y^2\theta(1 + \Theta^2) \\
  \gamma_s &=& -\sigma_y^2\Theta(1 + \theta^2) \\
  \gamma_{s+1} &=& \gamma_{s-1}\;=\;\sigma_y^2\theta\Theta
\end{eqnarray*}
These equations together with the measured autocorrelations make
possible an initial estimate of the parameters $\theta$ and $\Theta$:
\begin{eqnarray*}
  {\varrho_{s+1}/\varrho_s} &=& {\gamma_{s+1}/\gamma_s} \;=\; -{\theta \over
    1 + \theta^2} \\
  {\varrho_{s+1}/\varrho_1} &=& {\gamma_{s+1}/\gamma_1} \;=\; -{\Theta \over
    1 + \Theta^2} \\
\end{eqnarray*}
Substituting in the measured values shown in table
\ref{tab:nordea_15min_w_acf},
\begin{table}[htb!]
  \centering
  \begin{tabular}{|c|c|c|c|}
    \hline
    $\varrho_1$ & $\varrho_{s-1}$ & $\varrho_s$ & $\varrho_{s+1}$ \\
    \hline
    -0.4703 &  0.2053 & -0.4564 &  0.2212 \\
    \hline
  \end{tabular}
  \caption{\small \it autocorrelations of differenced log-volatility
    ($w_t$)}
  \label{tab:nordea_15min_w_acf}
\end{table}
we get
\begin{eqnarray*}
  \theta &=& 0.6890 \\
  \Theta &=& 0.6378
\end{eqnarray*}
Among the two roots of each of the 2nd order equations in the above,
we have chosen the one in the range $(-1, 1)$ so as to ensure
invertibility of the model \cite{BoxJenkins94}.

With an estimate of $\theta$ and $\Theta$, one can then infer the
noise process i.e. the residuals $y_t$:
\begin{equation}
  \label{eq:infer_y}
  y_t = w_t + \theta y_{t-1} + \Theta y_{t-s} - \theta \Theta y_{t-s-1}
\end{equation}
where we substitute $y_t\;(t \leq 0)$ with their unconditional
expectation 0.

In order to forecast the $w_t$ process, and hence the return process
itself, we must also know the distribution of $y_t$. Moreover, to
properly estimate the parameters of the model in the sense of maximum
likelihood, we are also in need of the distribution of $y_t$.

Figure \ref{fig:nordea_15min_y_qq} shows the normal probability plot of
$y_t$. It is evident from this figure that $y_t$ has fat tails.
\begin{figure}[htb!]
  \centering
  \includegraphics[scale=0.4, clip=true, trim=78 255 109
  123]{../pics/nordea2_y_normplot.pdf}
  \caption{\small \it Normal probability plot of residuals of
    log-volatility ($y_t$)}
  \label{fig:nordea_15min_y_qq}
\end{figure}
In addition, a simple calculation reveals that the distribution of
$y_t$ has skewness 0.2988 (shown in table
\ref{tab:nordea_15min_y_moments}).
\begin{table}[htb!]
  \centering
  \begin{tabular}{|c|c|c|c|}
    \hline
    mean & variance & skewness & kurtosis \\
    \hline
    0.0012 & 0.0935 & 0.2988 & 6.8691 \\
    \hline
  \end{tabular}
  \caption{\small \it Moments of log-volatility residuals ($y_t$)}
  \label{tab:nordea_15min_y_moments}
\end{table}
Based on this information, we find that $y_t$ can be well
described by a Johnson Su distribution \cite{Shang2004}:
\[
  y_t = \xi + \lambda\sinh{z_t - \gamma \over \delta}
\]
where $\gamma, \delta, \lambda, \xi$ are parameters to be determined
and $z_t \sim N(0, 1)$. The goodness of fitting is demonstrated in
figure \ref{fig:nordea_15min_y_js_fit} by the
empirical cummulative distribution function in comparison to the
theoretical one.
\begin{figure}[htb!]
  %\vspace{-18mm}
  \centering
    \includegraphics[scale=0.4, clip=true, trim=92 229 116
    133]{../pics/nordea_15min_y_js_fit.pdf}
    \caption{\small \it Log-volatility residuals $y_t$ fitted to a
      Johnson Su distribution. Horizontal: values of $y_t$, denoted x;
      Vertical: $\ln\left(P(y_t < x)\right)$.}
    \label{fig:nordea_15min_y_js_fit}
\end{figure}

The first 4 moments of the Johnson Su distribution are expressible
in closed form in $\gamma, \delta, \lambda, \xi$ \cite{Shang2004}.
% \begin{eqnarray*}
%   w &=& \exp{1 \over \delta^2} \\
%   \Omega &=& {\gamma \over \delta} \\
%   \text{E}(y) &=& -w^{1/2} \lambda \sinh\Omega + \xi\\
%   \text{std}(y) &=& \lambda \left[{1 \over 2}(w-1)(w\cosh 2\Omega +
%     1)\right]^{1/2} \\
%   \text{skewness}(y) &=& {
%     \sqrt{(1/2)w(w-1)} [w(w+2)\sinh 3\Omega + 3\sinh\Omega]
%     \over
%     (w\cosh 2\Omega + 1)^{3/2}} \\
%   \text{kurtosis}(y) &=& {
%     w^2(w^4 + 2w^3 + 3w^2 - 3)\cosh 4\Omega + 4w^2 (w+2) \cosh 2\Omega
%     + 3(2w+1) \over
%     2(w\cosh 2\Omega + 1)^2 }
% \end{eqnarray*}
By matching the theoretical expressions of the moments with their
measured values, and taking help from published tables
\cite{Johnson1965}, one can solve for the parameters $\gamma, \delta,
\lambda, \xi$.

Under the assumption of i.i.d Johnson Su distributed residuals, the
log-likelihood function of the parameters $\theta, \Theta$
conditional on the sample $w_t$ can be written as
\[
L(\theta, \Theta) = -{1 \over 2}\sum_{t=1}^n z_t^2 + n \ln{\delta
  \over \lambda \sqrt{2\pi}} - {1 \over 2}\sum_{t=1}^n \ln\left[
  1 + \left({y_t - \xi \over \lambda}\right)^2
\right]
\]
where $y_t$ are inferred from $w_t$ using eq.\ref{eq:infer_y}
and $z_t$ from $y_t$ using
\[
z_t = \delta \sinh^{-1}{y - \xi \over \lambda} + \gamma
\]
Note that $\gamma, \delta, \lambda, \xi$ are not really free
parameters but rather are implied by $\theta$ and $\Theta$: Once the
latter have been chosen and the corresponding $y_t$ inferred, the
former are determined by the moments of $y_t$.

% The eventual MLE is done in Matlab with the ``active set''
% algorithm. The initial as well as the final estimation results are
% listed in table \ref{tab:nordea_15min_js_param}:
% \begin{table}[htb!]
%   \centering
%   \begin{tabular}{|c|c|c|c|c|c|c|}
%     \hline
%     & $\gamma$ & $\delta$ & $\lambda$ & $\xi$ & $\theta$ & $\Theta$ \\
%     \hline
%     initial estimate & 0.1476 & 1.5121 & 0.3633 & 0.0454 & 0.6890 &
%     0.6378 \\
%     \hline
%     MLE estimate & 0.1319 & 1.5266 & 0.3735 & 0.0410 & 0.6639 & 0.6025
%     \\
%     \hline
%   \end{tabular}
%   \caption{\small \it Nordea 15min estimation results}
%   \label{tab:nordea_15min_js_param}
% \end{table}

\subsection{Comparison of the Forecasts}
In this section we compare the one-step-ahead forecasts from the \gls{garch}
model and from the \gls{sv} model. For this purpose, we
compute the difference between a forecast $\ln \sigma^F_t$ and its measured
counterpart, i.e. the realized volatility of the same period $\ln
\hat{\sigma}_t$. As a reference, we also consider the results obtained
by taking the mean of the realized volatilities of the first 80\% of
the data set as forecast for the volatilities of the remaining 20\%. We
call this naive forecast the ``sample mean''.

First of all, we look at the means and standard deviations of $\ln \sigma^F_t -
\ln \hat{\sigma}_t$, which are listed in table \ref{tab:nordea_2012}.
\begin{table}[htb!]
  \centering
  \begin{tabular}{|c|c|c|c|}
    \hline
    & \gls{sv} & \gls{garch} & Sample mean \\
    \hline
    $\E(\ln \sigma^F_t - \ln \hat{\sigma}_t)$ & 0.0040 & -0.0008 &
    -0.2210 \\
    \hline
    $\text{std}(\ln \sigma^F_t - \ln \hat{\sigma}_t)$ & 0.2659 & 0.3011 &
    0.2893 \\
    \hline
  \end{tabular}
  \caption{\small \it Mean and standard Deviation of the forecasts'
    distribution}
  \label{tab:nordea_2012}
\end{table}
It is seen from table \ref{tab:nordea_2012} that, on average, the \gls{sv}
model over-estimates while \gls{garch} under-estimates. In terms of the
standard deviation of $\ln \sigma^F_t - \ln \hat{\sigma}_t$, the \gls{sv}
model wins with a small margin. In contrast, the sample mean forecast
clearly under-estimates the log-volatilities to a large extent --- the
efforts of building models has not been wasted.

Figure \ref{fig:nordea_2012} compares the 3 kinds of forecasts by
plotting the distribution function and the complementary distribution
function of $\ln \sigma^F_t - \ln \hat{\sigma}_t$. Here one can see
that the \gls{sv} model yields a better quality of forecasts than does \gls{garch} 
with respect to both under-estimates and over-estimates.

\begin{figure}[htb!]
  \centering
    \includegraphics[scale=0.55, clip=true, trim=40 296 16
    260]{../pics/nordea_2012.pdf}
  \caption{\small \it Blue: SV forecasts; Green: GARCH forecasts; Red:
    sample mean forecasts. Left: $P(\ln \sigma^F_t - \ln \hat{\sigma}_t < x)$;
    Right: $P(\ln \sigma^F_t - \ln \hat{\sigma}_t > x)$. Horizontal: $x$.}
  \label{fig:nordea_2012}
\end{figure}

Another measure of the forecasts' quality can be the percentage of good
forecasts, where the criterion of ``good'' is defined, respectively,
as the forecast lying within 1\%, 5\%, or 10\% of the corresponding
realized volatility. Table \ref{tab:nordea_2012_good} shows the
respective percentage of the 3 kinds of forecasts. Again in this table
it is seen that the \gls{sv} model gives more accurate forecasts than does
\gls{garch}. For example, defining a ``good'' forecast as one that lies
within 5\% of the realized volatility, the probability of obtaining
such a good forecast is 75\% using the \gls{sv} model, 71\% using the
\gls{garch} model, and only 53\% using the sample mean.
\begin{table}[htb!]
  \centering
  \begin{tabular}{|c|c|c|c|}
    \hline
    ${|\ln \sigma^F_t - \ln \hat{\sigma}_t| \over |\ln \hat{\sigma}_t|}$
    & \gls{sv} & \gls{garch} & sample mean \\
    \hline
    1\% & 22\% & 14\% & 11\% \\
    \hline
    5\% & 75\% & 71\% & 53\% \\
    \hline
    10\% & 97\% & 94\% & 89\% \\
    \hline
  \end{tabular}
  % \begin{tabular}{|c|c|c|c|}
  %   \hline
  %   ${\ln \sigma^F_t - \ln \hat{\sigma}_t \over |\ln \hat{\sigma}_t|}$
  %   & \gls{sv} & GARCH & sample mean \\
  %   \hline
  %   1\% & 21.73\% & 13.83\% & 11.36\% \\
  %   \hline
  %   5\% & 74.57\% & 70.62\% & 53.33\% \\
  %   \hline
  %   10\% & 97.28\% & 93.83\% & 89.14\% \\
  %   \hline
  % \end{tabular}
  \caption{\small \it The percentage of ``good'' forecasts when the
    criterion of being good is deviating no more than 1\%, 5\% or
    10\%.}
  \label{tab:nordea_2012_good}
\end{table}

\section{Case study: Volvo 30-minute Returns}
\label{sec:volvo_30}
In this section we model the log-volatility of {\it Volvo B} 30-minute
returns during the period 2013/10/10 - 2014/04/04. This series
contains 1884 log-volatilities computed using 2-minute returns in each
30-minute interval. Among them we use the first 1507 for model
estimation and the last 377 for forecast and model verification.

The left plot of figure \ref{fig:volvo_inno_and_lv_acf} shows the
distribution of $(r_t - \E(r_t))/\sigma_t$. We observe in the figure a
nice Gaussian variate, so we can be sure that the sum of squared
2-minute returns makes a good approximation to the variance of
30-minute returns in this particular case.
\begin{figure}[htb!]
  \centering
  \includegraphics[scale=0.4, clip=true, trim=0 256 0
  151]{../pics/volvo_inno_and_lv_acf.pdf}
  \caption{\small \it Left: probability plot of $(r_t -
    \E(r_t))/\sigma_t$; Right: auto-correlations of $\ln \sigma_t$}
  \label{fig:volvo_inno_and_lv_acf}
\end{figure}

Guided by the auto-correlations of $\ln\sigma_t$ shown in the right
plot of figure \ref{fig:volvo_inno_and_lv_acf} we find the following
model:
\begin{eqnarray}
  && (1 - B)(1 - B^s) \ln \sigma_t \nonumber \\
  &=& (1 - \theta_1 B - \theta_2B^2 -
  \theta_3B^3 - \theta_4B^4) (1 - \Theta B^s)
  y_t \label{eq:volvo_lv_model}
\end{eqnarray}
where $s = 16$ is the seasonality and is apparent from the
auto-correlations of $\ln\sigma_t$. Fitting this model to the
measured realized volatilities yields the parameter values listed in
table
\ref{tab:volvo_params}.
\begin{table}[htb!]
  \centering
  \begin{tabular}{|c|c|c|c|c|c|c|}
    \hline
    Parameter & $\theta_1$ & $\theta_2$ & $\theta_3$ & $\theta_4$ &
    $\Theta$ & residual variance \\
    \hline
    Value & 0.7305 & 0.0575 & 0.0574 & 0.0346 & 0.8324 & 0.1340\\
    \hline
  \end{tabular}
  \caption{\small \it Volvo B log-volatility parameters}
  \label{tab:volvo_params}
\end{table}
Forecasting using the estimated model parameters gives the forecast
series $\ln \sigma^{\text{SV}}_t$. To access the quality of the
forecast, we also estimate a \gls{garch} model using the returns. The result
is a \gls{garch}(1, 1) model, whose parameter values are listed in table
\ref{tab:volvo_garch}.
\begin{table}[htb!]
  \centering
  \begin{tabular}{|c|c|c|c|}
    \hline
    Parameter & $\alpha_0$ & $\alpha_1$  & $\beta_1$ \\
    \hline
    Value & $3.125 \times 10^{-7}$ & 0.05 & 0.90 \\
    \hline
  \end{tabular}
  \caption{\small \it GARCH(1, 1) model of Volvo B 30-minute returns}
  \label{tab:volvo_garch}
\end{table}

The forecasts from \gls{sv}, \gls{garch}, and the sample mean are compared
using the difference $\ln \sigma^F_t - \ln\hat{\sigma}_t$, where $\ln
\sigma^F_t$ stands for the forecast. The distributions of this
difference is plotted in figure \ref{fig:volvo_slv_garch_cmp}; the
mean and the standard deviation of the distributions are listed in
table \ref{tab:volvo_slv_garch_cmp}.
\begin{table}[htb!]
  \centering
  \begin{tabular}{|c|c|c|c|}
    \hline
    & \gls{sv} & \gls{garch} & Sample mean \\
    \hline
    $\E(\ln \sigma^F_t - \ln \hat{\sigma}_t)$ & -0.0123 &
    0.0242 & -0.1055 \\
    \hline
    $\text{std}(\ln \sigma^F_t - \ln \hat{\sigma}_t)$ & 0.3261 &
    0.4250 & 0.3708 \\
    \hline
  \end{tabular}
  \caption{\small \it Standard deviation of $\ln\sigma^F_t -
    \ln\hat{\sigma}_t$}
  \label{tab:volvo_slv_garch_cmp}
\end{table}

\begin{figure}[htb!]
  \centering
  \includegraphics[scale=0.5, clip=true, trim=10 282 0
  243]{../pics/volvo_slv_garch_cmp.pdf}
  \caption{\small \it Blue: SV forecasts; Green: GARCH forecasts; Red:
    sample mean forecasts. Left: $P(\ln \sigma^F_t - \ln
    \hat{\sigma}_t < x)$; Right: $P(\ln \sigma^F_t - \ln
    \hat{\sigma}_t > x)$. Horizontal: $x$.}
  \label{fig:volvo_slv_garch_cmp}
\end{figure}
Figure \ref{fig:volvo_slv_garch_cmp} shows, as in the previous case of
Nordea Bank 15-minute returns, the \gls{sv} model performs the best, \gls{garch}
the second, and the sample mean the worst. However, when it comes to
over-estimates, the sample mean appears to be the best estimator,
while \gls{sv} excels over \gls{garch}. But a check of $\E(\ln\sigma_t)$ over
the sample for estimation (1507 data points) and over the sample for
comparison (377 data points) reveals that the first sample has mean
-6.3127 while the second has -6.2072. This increment explains the low
probability of over-estimation when using the first sample mean as
forecast. 

Table \ref{tab:volvo_good_percentage} compares the fraction of
``good'' estimates as measured by ${|\ln\sigma^F_t -
  \ln\hat{\sigma}_t| \over |\ln \hat{\sigma}_t|}$ being less than 1\%,
5\% and 10\%.
\begin{table}[htb!]
  \centering
  \begin{tabular}{|c|c|c|c|}
    \hline
    ${|\ln \sigma^F_t - \ln \hat{\sigma}_t| \over |\ln
      \hat{\sigma}_t|}$ &
    \gls{sv} & \gls{garch} & Sample Mean \\
    \hline
    1\% & 22\% & 12\% & 14\% \\
    \hline
    5\% & 72\% & 62\% & 66\% \\
    \hline
    10\% & 92\% & 88\% & 90\% \\
    \hline
  \end{tabular}
  % \begin{tabular}{|c|c|c|c|}
  %   \hline
  %   ${|\ln \sigma^F_t - \ln \hat{\sigma}_t| \over |\ln
  %     \hat{\sigma}_t|}$ &
  %   \gls{sv} & \gls{garch} & Sample Mean \\
  %   \hline
  %   1\% & 21.75\% & 12.20\% & 14.06\% \\
  %   \hline
  %   5\% & 71.62\% & 61.54\% & 66.31\% \\
  %   \hline
  %   10\% & 92.31\% & 88.06\% & 89.92\% \\
  %   \hline
  % \end{tabular}
  \caption{\small \it Fraction of ``good'' forecasts as defined by
    ${|\ln \sigma^F_t - \ln \hat{\sigma}_t| \over |\ln
      \hat{\sigma}_t|}$ being less than 1\%, 5\% and 10\%.}
  \label{tab:volvo_good_percentage}
\end{table}
We see from the table that the \gls{sv} model consistently excels over
the other two alternatives. In addition, it is also noted that the
\gls{garch} forecast is even worse than the sample mean. This is surprising
but not totally unexpected --- with only 3 parameters, the \gls{garch}(1,1)
model can only describe the most prominent auto-correlations in the
volatility. When the volatility is influenced by relatively weak
auto-correlations at several different time lags, the \gls{garch} forecast
cannot be expected to have good accuracy.

In this particular case, we see that the \gls{arima} model has 3 relatively
small moving average coefficients, located at lags 2, 3, and 4 and
evaluated to 0.06, 0.06, 0.03, suggesting a scattered auto-correlation
structure, so the \gls{garch} model cannot be expected to perform very
well. In contrast, the Volvo 15-minute returns studied in
section \ref{sec:volvo} has more concentrated auto-correlations --- 0.12 and
0.06 at lags 2 and 3 --- thus the \gls{garch}(1,1) model is also found to
perform better and even marginally better than the \gls{sv} model for the
same series.

\section{Unconditional Distribution Functions of SV Models}
\label{sec:XieCalc}
In this section we study the unconditional distribution function of
the \gls{sv} model specified as equation
\ref{eq:SLV_spec}. As is discussed at the beginning of this chapter,
$\ln\sigma_t$ can be described by an \gls{arma} or \gls{arima} model,
possibly with seasonal components. Here we note that all these models
can be re-written as a moving average model, which is infinite in extent if
autoregressive components are present:
\begin{eqnarray*}
  \ln \sigma_t &=& y_t + \sum_{n=1}^\infty c_n y_{t-n} + \text{Const.}
\end{eqnarray*}
Since the $y_t$ are independent and identically distributed,
\begin{eqnarray*}
  y_t + \sum_{n=1}^\infty c_n y_{t-n}  
\end{eqnarray*}
has Gaussian distribution by the central limit theorem, on condition
that $y_t$ for all $t$ have finite second moment --- this is what we
assume in the rest of this section. It follows that the unconditional
distribution of $\ln \sigma_t$ is the same as the distribution of
$\bar{v} + v$ where $v \sim N(0, \sigma)$ and $\bar{v}$, $\sigma$ are
constants. Now we can state that the unconditional distribution of the
returns $r_t$
\begin{eqnarray*}
  r_t &=& \mu + \sigma_t b_t\\
  &=& \mu + b_t \exp\left(
    y_t + \sum_{n=1}^\infty c_n y_{t-n} + \text{Const.}
  \right)
\end{eqnarray*}
 is the same as
\begin{equation}  \label{eq:UnconditionalPdf}
  \begin{aligned}
    r &= \mu + e^{\bar{v} + v} b \\
  \end{aligned}
\end{equation}
where $b \sim N(0, 1)$. For convenience, let $r' = e^v b$

In section \ref{sec:SLV_Symmetric} we first study the model in the
relatively simple case when $v$ and $b$ are uncorrelated and $\mu
= 0$. If this simplified version proves inadequate, one may resort to
the general model studied in section \ref{sec:SLV_Asymmetric}.

\subsection{The Simplified model}\label{sec:SLV_Symmetric}
In the following we derive the unconditional \gls{pdf} of $r'$, the
de-meaned and rescaled return. Denote this \gls{pdf} $f_{r'}(x)$. Then
the \gls{pdf} of $r$ is $e^{-\bar{v}}f_{r'}(e^{-\bar{v}}x)$. First we
consider
\begin{eqnarray*}
  P(r' < x) &=& P(b < xe^{-v}) \\
  f_{r'}(x) &=& f_b(xe^{-v}) e^{-v}
\end{eqnarray*}
Averaging over all $v$, we get
\begin{equation}\label{eq:UncondPDFSymmetric}
  \begin{aligned}
    f_{r'}(x) =& \int_{-\infty}^{\infty} dv (2\pi\sigma^2)^{-1/2}
    e^{-v^2/2\sigma^2}(2\pi)^{-1/2} \exp(-x^2e^{-2v}/2) e^{-v} \\
    =& {1 \over 2\pi\sigma} \int_{-\infty}^{\infty} dv
    \exp\left(
      -{1 \over 2\sigma^2} v^2 - v -{1 \over 2} x^2 e^{-2v}
    \right)
  \end{aligned}
  \end{equation}
The last part of the integrand, $e^{-x^2 e^{-2v} / 2}$, is plotted in
figure \ref{fig:DoubleExp}.
\begin{figure}[htb!]
  \centering
  \includegraphics[scale=0.5, clip=true, trim=85 252 100
  231]{../pics/DoubleExp.pdf}
  \caption{\small \it Plot of $\exp(-{1 \over 2} x^2 e^{-2v})$}
  \label{fig:DoubleExp}
\end{figure}
Therefore we make the following approximation:
\[
\exp\left(-{1 \over 2} x^2 e^{-2v}\right) \approx \left\{
  \begin{array}{lr}
    0 & \text{if } v < \ln|x| -{1 \over 2} \ln(2\ln 2) \\
    1 & \text{otherwise}
  \end{array}
\right.
\]
Here we note that $\exp\left(-{1 \over 2} x^2 e^{-2v}\right) = 1/2$ at
$v = \ln|x| -{1 \over 2} \ln(2\ln 2)$.

With this approximation we have
\begin{eqnarray*}
  f_{r'}(x) &=& {1\over C}{1 \over 2\pi\sigma} \int_{\ln(|x|/\sqrt{\ln
      4})}^{\infty} dv
  \exp\left(-{1 \over 2\sigma^2} v^2 - v\right) \\
  &=& {1\over C}{e^{\sigma^2 / 2} \over \sqrt{8\pi}} \text{erfc} \left(
    {1 \over \sqrt{2}\sigma} \ln{|x| \over \sqrt{\ln 4}} + {\sigma
      \over \sqrt{2}}
  \right)
\end{eqnarray*}
where $1/C$ has been added for the purpose of normalization.

At large $x$, we may use the asymptotic expansion of $\mathrm{erfc}$
to write
\begin{eqnarray*}
  Cf_{r'}(x) &=& {e^{\sigma^2 / 2} \over \sqrt{8\pi}} \mathrm{erfc}(\xi) \\
  &=& {e^{\sigma^2 / 2} \over \sqrt{8 \pi}}
  \frac{e^{-\xi^2}}{\xi\sqrt{\pi}}\left[
    1 +
    \sum_{n=1}^N (-1)^n \frac{(2n-1)!!}{(2\xi^2)^n} \right] +
  O(\xi^{-2N-1} e^{-\xi^2})
\end{eqnarray*}
where $\xi = {1 \over \sqrt{2}\sigma} \ln{|x| \over \sqrt{\ln 4}} +
{\sigma \over \sqrt{2}}$. The slowest-decaying term is
\[
f_{r,0}(x) = {e^{\sigma^2 / 2} \over \sqrt{8\pi}}
\frac{e^{-\xi^2}}{\xi\sqrt{\pi}}
\]
Let $\zeta = {|x| \over \sqrt{\ln 4}}$. With a bit manipulation one
obtains
\begin{equation*}
  f_{r,0}(x) = {1 \over \pi \sqrt{8}}{1 \over
    \left(\ln\zeta/\sigma\sqrt{2} +
      \sigma/\sqrt{2}\right)\zeta\zeta^{\ln \zeta / 2\sigma^2}
  }
\end{equation*}
From the last equation one can see that, at any neighborhood of large
$x$, $f_{r'}(x)$ may be approximated by $C/|x|^\alpha$, i.e. a power law.
The normalization constant $C$ is calculated in appendix
\ref{chp:symmetric_SV_norm_const} to be $C = \sqrt{2 \ln 4\over
  \pi}$. So, in summary, we can write:
\begin{eqnarray*}
  f_r(x) &=& e^{-\bar{v}} f_{r'}(e^{-\bar{v}} x) \\
    &=& {e^{-\bar{v}} \over C}{e^{\sigma^2 / 2} \over \sqrt{8\pi}}
    \text{erfc} \left({1 \over \sqrt{2}\sigma} \ln{|e^{-\bar{v}}x| \over \sqrt{\ln
          4}} + {\sigma \over \sqrt{2}}
    \right) \\
\end{eqnarray*}
Using the same technique for integration as for normalization, the
cummulative distribution function of $r$ is found to be $F(x)$,
which is the following:
\begin{enumerate}
\item if $x < 0$
  \begin{eqnarray*}
    F(x) &=& {\sqrt{\ln 4} \over C}{e^{\sigma^2 / 2} \over \sqrt{8\pi}}
    \left[
      {e^{-\bar{v}}x \over \sqrt{\ln 4}}\text{erfc}\left(
        {1 \over \sigma \sqrt 2} \ln{-e^{-\bar{v}}x \over \sqrt{\ln
            4}} + {\sigma \over \sqrt 2}
      \right) \right.\\
      && \left. + e^{-\sigma^2 / 2} \text{erfc}\left(
        {1 \over \sigma \sqrt 2} \ln{-e^{-\bar{v}}x \over \sqrt{\ln 4}}
      \right)
    \right]
  \end{eqnarray*}
\item if $x \geq 0$
  \begin{eqnarray*}
    F(x) &=& \frac{1}{2} + {\sqrt{\ln 4} \over C}{e^{\sigma^2 / 2} \over
      \sqrt{8\pi}} \left[
      {e^{-\bar{v}}x \over \sqrt{\ln 4}} \text{erfc}\left(
        {1 \over \sigma \sqrt 2} \ln{e^{-\bar{v}}x \over \sqrt{\ln
            4}} + {\sigma \over \sqrt 2}
      \right) \right. \\
      && \left. + e^{-\sigma^2/2} \text{erfc}\left(
        -{1 \over \sigma \sqrt 2} \ln{e^{-\bar{v}}x \over \sqrt{\ln 4}}
      \right)
    \right]
  \end{eqnarray*}
\end{enumerate}

To verify the validity of the model, we fit the above probability
density function to the de-meaned 30min returns of Volvo B
\footnote{By ``de-meaned returns'' we mean the quantity $r_t - 
  \mean{r_t}$, where $r_t$ are the measured returns and $\mean{r_t}$
  is the sample mean. The data set covers the transaction records of
  Volvo B on the OMX market (Stockholm) between 2013-10-10 and
  2014-03-12. The returns are computed using 1-minute mean prices.}
by means of \gls{mle} using the MATLAB function ``mle''. Then for the
parameters $\sigma$ and $\bar{v}$ we get
\begin{eqnarray*}
  \sigma &=& 0.6355 \\
  \bar{v} &=& -6.0308
\end{eqnarray*}
Then we plot $P(r' > x)$ of the model against its empirical
counterpart on a log-log scale, as shown in figure
\ref{fig:volvo_30min_ret}.
\begin{figure}[htb!]
  % \vspace{-15mm}
  \begin{center}
    \includegraphics[scale=0.4, clip=true, trim=98 231 116
    126]{../pics/volvo_30min_ret.pdf}
  \end{center}
  %\vspace{-5mm}
  \caption{\small \it{$P(r' > x)$ for $x > 0$. Blue: empirical
      probabilities. Red: probabilities predicted by the model.}}
  \label{fig:volvo_30min_ret}
\end{figure}

In general figure \ref{fig:volvo_30min_ret} shows a good fit, but a
closer look reveals that deviations are significant in the regions $r
\in (0, \sigma_r)$ and $r \in (2\sigma_r, 3\sigma_r)$, where
$\sigma_r$ stands for the empirical standard deviation of the
de-meaned returns. These observations are
shown in greater details in figure \ref{fig:volvo_30min_ret2}.
\begin{figure}[htb!]
  \centering
  \subfigure[$\ln(P(r > x))$ with $x \in (0, \sigma_r)$]{
    \includegraphics[scale=0.4, clip=true, trim=93 224 115
    125]{../pics/volvo_30min_ret_0-sigma.pdf}
  }
  \subfigure[$\ln(P(r > x))$ with $x \in (2\sigma_r, 3\sigma_r)$]{
    \includegraphics[scale=0.4, clip=true, trim=93 224 115
    125]{../pics/volvo_30min_ret_2sigma-3sigma.pdf}
  }
  \caption{\small \it Surviving probabilities. Blue: empirical values of $\ln(P(r
    > x))$. Red: Predicted values of $\ln(P(r > x))$.}
  \label{fig:volvo_30min_ret2}
\end{figure}

Moreover, the inefficiency of the model is also manifest in the
skewness of the data. For the Volvo 30min returns, the data has a
skewness of 0.2419, but our model is strictly symmetric, since the
return $x$ appears in equation \ref{eq:UncondPDFSymmetric} only as $x^2$.
Hence the above model needs to be improved to accommodate the non-zero
skewness as well as to account for the discrepancies shown in figure
\ref{fig:volvo_30min_ret2}. This is the subject of the next section.

\subsection{The General model}\label{sec:SLV_Asymmetric}
It has long been hypothesized in the liturature that skewness is the
result of price-volatility correlation (for example
\cite{Potters2003}). Therefore, the most apparent modification is to
allow $v$, the zero-mean Gaussian variate in the log-volatility, and
$b$, the zero-mean Gaussian variate in the innovation of the return,
to be correlated. For convenience we decompose $v \sim N(0, \sigma)$
as $v = \sigma a$ and assume
\begin{eqnarray*}
  \begin{pmatrix}
    a \\
    b
  \end{pmatrix} \sim N(0, \Sigma)
\end{eqnarray*}
with a covariance matrix
\begin{eqnarray*}
    \Sigma &=&
  \begin{pmatrix}
    1 & \psi \\
    \psi & 1
  \end{pmatrix}
\end{eqnarray*}
where $|\psi| < 1$. Then we rewrite equation \ref{eq:UnconditionalPdf}
as
\begin{equation}
  \label{eq:r_t}
  \begin{aligned}
    r &= \mu + e^{\bar{v}} r' \\
    r' &= e^{\sigma a} b \\
  \end{aligned}
\end{equation}
We then get
\begin{eqnarray*}
  && P(r' < x)\\
  &=& P(b < e^{-a\sigma}x) \\
  &=& {1 \over 2\pi |\det(\Sigma)|^{1/2}}
  \int_{-\infty}^{\infty} da \int_{-\infty}^{e^{-a\sigma}x} db
  \exp\left[
    -{1 \over 2} (a, b) \Sigma^{-1}
    \begin{pmatrix}
      a \\
      b
    \end{pmatrix}
  \right] \\
\end{eqnarray*}
After some manipulations we get
\begin{equation}\label{eq:UncondCDFAsymmetric}
  \begin{aligned}
    & P(r' < x) \\  
    &= {1 \over 2\pi \sqrt{1 - \psi^2}} \int_{-\infty}^{\infty} da
    e^{-a^2/2} \int_{-\infty}^{e^{-a\sigma}x} db
    \exp\left[
      - {(b - a\psi)^2 \over 2(1 - \psi^2)}
    \right] \\
    &= {1 \over 2\sqrt{2\pi}} \int_{-\infty}^{\infty} e^{-a^2/2}
    \text{erfc}{a\psi - e^{-a\sigma} x \over \sqrt{2(1-\psi^2)}} da
  \end{aligned}
\end{equation}
Differentiating with respect to $x$ yields the \gls{pdf}
\begin{eqnarray}
  f_{r'}(x) &=& {1 \over 2\pi \sqrt{1 - \psi^2}} \times \nonumber \\
  && \int_{-\infty}^\infty da \exp
  \left[- {
      a^2 + 2\sigma(1 - \psi^2)a - 2a\psi e^{-a\sigma} x + e^{-2a\sigma} x^2
      \over
      2(1 - \psi^2)
  }
  \right] \label{eq:UncondPDFAsymmetric} \\
  f_r(x) &=&  \td{r'}{r} f_{r'}[e^{-\bar{v}} (x - \mu)] \nonumber \\
  &=& e^{-\bar{v}} f_{r'}[e^{-\bar{v}}
  (x-\mu)] \label{eq:UncondPDFAsymmetric1}
\end{eqnarray}
Unlike \ref{eq:UncondPDFSymmetric} where a relatively simple
approximation can be found and thus leads to an analytic result of the
integral, no such approximation has been found by the
author for the integral \ref{eq:UncondPDFAsymmetric}. However, the
\gls{mgf} of \gls{pdf} \ref{eq:UncondPDFAsymmetric}
and \ref{eq:UncondPDFAsymmetric1} are easy enough to find and give the
moments about the origin in closed form. Then by matching the analytic
expressions of the moments with their statistical values from the
sample, the parameters $\sigma$, $\psi$ and $\bar{v}$ corresponding to the
sample can be obtained.

The \gls{mgf} of \ref{eq:UncondPDFAsymmetric1} can be found as follows:
\begin{eqnarray*}
  && M_{r}(t) \\
  &=& \E(e^{tr}) \\
  &=& {1 \over 2\pi \sqrt{1 - \psi^2}} \int_{-\infty}^{\infty} da \exp \left(
    -{a^2 \over 2} - \sigma a
  \right) \int_{-\infty}^{\infty} dx \exp\left[
    t(e^{\bar{v}}x + \mu) - {
      (a\psi - e^{-a \sigma} x)^2 \over 2 (1 - \psi^2)
    }
  \right] \\
  &=& {1 \over \sqrt{2 \pi}} \int_{-\infty}^{\infty} da \exp\left[
    -{a^2 \over 2} + {1 \over 2} e^{2 a \sigma + 2\bar{v}} (1 - \psi^2) t^2 +
    a \psi e^{a \sigma + \bar{v}} t + \mu t
  \right]
\end{eqnarray*}
With the \gls{mgf}, the first 4 moments of $r$ can be computed:
\begin{equation}
  \label{eq:AsymmetricMoments}
  \begin{aligned}
    \E(r) &= \mu +e^{\bar{v}+\frac{\sigma ^2}{2}} \sigma  \psi\\
    \E(r^2) &= \mu ^2+2 e^{\bar{v}+\frac{\sigma ^2}{2}} \mu  \sigma  \psi
    +e^{2 \left(\bar{v}+\sigma ^2\right)} \left(1+4 \sigma ^2 \psi^2\right)\\
    \E(r^3) &= \mu ^3+3 e^{\bar{v}+\frac{\sigma ^2}{2}} \mu ^2 \sigma \psi
    +9 e^{3 \bar{v}+\frac{9 \sigma ^2}{2}} \sigma \psi \left(1+3 \sigma ^2
      \psi ^2\right)
    +3e^{2 \left(\bar{v}+\sigma ^2\right)} \mu  \left(1+4 \sigma ^2 \psi
      ^2\right) \\
    \E(r^4) &= \mu ^4+4 e^{\bar{v}+\frac{\sigma ^2}{2}} \mu ^3 \sigma  \psi
    +36 e^{3 \bar{v}+\frac{9 \sigma ^2}{2}} \mu  \sigma  \psi  \left(1+3
      \sigma ^2 \psi ^2\right)\\ 
    & +6 e^{2 \left(\bar{v}+\sigma ^2\right)} \mu ^2 \left(1+4 \sigma ^2 \psi
      ^2\right)
    +e^{4 \bar{v}+8 \sigma ^2} \left(3+96 \sigma ^2 \psi ^2+256 \sigma ^4
      \psi ^4\right)
  \end{aligned}
\end{equation}
These equations are rather complicated and directly solving them to
obtain the parameters $\mu$, $\sigma$, $\psi$ and $\bar{v}$ is
infeasible. However, in practice, $\E(r)$ is often very small --- so, to
get a rough estimate of the parameters, we may set $\mu = 0$ in the
above equations and, with a bit of manipulation, find the following
equation for $\psi\sigma$:
\begin{equation}
  \label{eq:moment1}
  {\E(r^3) \E(r^4)^{-3/8} \over \E(r^2)^{3/4}} =
  {9\sigma \psi (1 + 3\sigma^2\psi^2) \over (1 +
    4\sigma^2\psi^2)^{3/4}}
  {1
    \over
    (3 + 96\sigma^2\psi^2 + 256\sigma^4 \psi^4)^{3/8}
  }
\end{equation}
This equation can be solved numerically for a given sample to yield an
estimate for $\sigma\psi$, which in turn can be substituted in
equations \ref{eq:AsymmetricMoments}, where $\mu$ has been set to 0,
to give estimates for all the 4 parameters. These estimates then serve
as initial values in a numerical solution to
\ref{eq:AsymmetricMoments} where $\mu$ is kept as a free
variable. This full solution can now be used as the initial estimate
in an \gls{mle} procedure.

It is particularly important in our situation to have a good initial
estimate, because, as has been shown earlier, the \gls{pdf} and the
\gls{cdf} cannot be obtained in closed forms and consequently have to
be evaluated by numerical integration. This is a rather costly
procedure especially in the context of \gls{mle}. So the computation of the
moments under the assumption of $\mu = 0$ is worthwhile.

Following the aforementioned procedure, i.e. computing the initial
estimate by matching the moments and then refining the estimate by
\gls{mle}, we obtain the parameter values for a number of return series,
including the Volvo B 30-minute returns described in
section \ref{sec:SLV_Symmetric}. Table \ref{tab:assets_params} and
\ref{tab:assets_moments} present the obtained parameter values and the
resulting moments, respectively. Figure \ref{fig:Volvo_B_30m_returns},
\ref{fig:Nordea_30m_returns}, and \ref{fig:Ericsson_30m_returns}
compare the empirical distribution functions with the analytic
distribution functions evaluated at these parameters.
\begin{figure}[htb!]
  \centering
  \includegraphics[scale=0.5, clip=true, trim=50 233 63
  136]{../pics/Volvo_B_30m_returns.pdf}
  \caption{\small \it Volvo B 30min returns' unconditional distribution fit to
    the SV model. The time series runs from
    2013/10/10 to 2014/03/12. Left: $P(r < x)$ with $x < 0$; Right:
    $P(r > x)$ with $x > 0$. Blue: empirical CDF obtained from data;
    Green: model CDF computed using equation
    \ref{eq:UncondCDFAsymmetric} with $x$ replaced by
    $(x-\mu)e^{-\bar{v}}$. Both plots are on log-log scale.}
  \label{fig:Volvo_B_30m_returns}
\end{figure}
\begin{figure}[htb!]
  \centering
  \includegraphics[scale=0.5, clip=true, trim=65 241 72
  143]{../pics/Nordea_30m_returns.pdf}
  \caption{\small \it Nordea Bank 30min returns' unconditional
    distribution fit to the SV model. The time
    series runs from 2013/10/10 to 2014/03/12. Left: $P(r < x)$ with
    $x < 0$; Right: $P(r > x)$ with $x > 0$. Blue: empirical CDF
    obtained from data; Green: model CDF computed using equation
    \ref{eq:UncondCDFAsymmetric} with $x$ replaced by
    $(x-\mu)e^{-\bar{v}}$. Both plots are on log-log scale.}
  \label{fig:Nordea_30m_returns}
\end{figure}
\begin{figure}[htb!]
  \centering
  \includegraphics[scale=0.5, clip=true, trim=21 241 28
  146]{../pics/Ericsson_30m_returns.pdf}
  \caption{\small \it Ericsson B 30min returns' unconditional
    distribution fit to the SV model. The time
    series runs from 2013/10/10 to 2014/03/12. Left: $P(r < x)$ with
    $x < 0$; Right: $P(r > x)$ with $x > 0$. Blue: empirical CDF
    obtained from data; Green: model CDF computed using equation
    \ref{eq:UncondCDFAsymmetric} with $x$ replaced by
    $(x-\mu)e^{-\bar{v}}$. Both plots are on log-log scale.}
  \label{fig:Ericsson_30m_returns}
\end{figure}
\begin{table}[htb!]
  \centering
  \begin{tabular}{|c|c|c|c|c|}
    \hline
    & $\psi$ & $\sigma$ & $\bar{v}$ & $\mu$ \\
    \hline
    Volvo B & -1.6$\times 10^{-2}$ & 4.7$\times 10^{-1}$ & -6.2&
    -5.3$\times 10^{-5}$ \\
    Nordea Bank & 1.7$\times 10^{-2}$ & 4.3$\times 10^{-1}$ & -6.4 &
    7.5$\times 10^{-5}$ \\
    Ericsson B & 1.8$\times 10^{-2}$ & 4.5$\times 10^{-1}$ & -6.3 &
    -9.3$\times 10^{-5}$ \\
    \hline
  \end{tabular}
  % \begin{tabular}{|c|c|c|c|c|}
  %   \hline
  %   & $\psi$ & $\sigma$ & $\bar{v}$ & $\mu$ \\
  %   \hline
  %   Volvo B & -1.5747e-02 & 4.6929e-01 & -6.2304e+00 &
  %   -5.3040e-05 \\
  %   Nordea Bank & 1.7234e-02 & 4.2780e-01 & -6.3758e+00 & 7.5230e-05
  %   \\
  %   Ericsson B & 1.8486e-02 & 4.4905e-01 & -6.2834e+00 & -9.2817e-05 \\
  %   \hline
  % \end{tabular}
  \caption{\small \it Parameter Values of Selected Assets'
    returns. The time series are 30-minute returns and run from
    2013/10/10 to 2014/03/12.}
  \label{tab:assets_params}
\end{table}
\begin{table}[htb!]
  \centering
  \begin{tabular}{|c|c|c|c|c|}
    \hline
    & mean & std & skewness & kurtosis \\
    \hline
    \multirow{2}{*}{Volvo B} & -8.7$\times 10^{-5}$ & 2.5$\times 10^{-3}$ &
    -2.7$\times 10^{-2}$ & 7.3 \\
    & -6.9$\times 10^{-5}$ & 2.5$\times 10^{-3}$ & -7.3$\times
    10^{-2}$ & 7.2 \\
    \hline
    \multirow{2}{*}{Nordea Bank} & 7.5$\times 10^{-5}$ & 2.0$\times 10^{-3}$ &
    6.9$\times 10^{-2}$ & 6.7 \\
    & 8.9$\times 10^{-5}$ & 2.0$\times 10^{-3}$ &
    6.7$\times 10^{-2}$ & 6.2 \\
    \hline
    \multirow{2}{*}{Ericsson B} & -8.3$\times 10^{-5}$ & 2.3$\times 10^{-3}$ &
    2.7$\times 10^{-1}$ & 7.7 \\
    & -7.6$\times 10^{-5}$ & 2.3$\times 10^{-3}$ &
    7.9$\times 10^{-2}$ & 6.7 \\
    \hline
  \end{tabular}
  \caption{\small \it Moments of Selected Assets' returns. For each return
    series, the 1st row contains the sample moments, while the 2nd
    constains those computed with MLE parameters. The time series are
    30-minute returns and run from 2013/10/10 to 2014/03/12.}
  \label{tab:assets_moments}
\end{table}

Clearly these figures show a fairly good match of the empirical and
the analytic distribution functions. However, we also see that the
skewness of the returns of Volvo B and Ericsson B have rather
different values when computed from the sample and from the
model. This could be the consequence of the limited sample size or
deficiencies in the estimation procedure described above. We leave
these issues to future studies.

%  This indicates the parameter $\psi$, which is the
% correlation between the log-volatility and the return, may be more
% complicated than the simple constant that has been assumed so far. We
% leave this to later studies.

\subsection{Relation to Conditional Distribution Functions}
\label{sec:SV_Conditional}
In the last section we have shown that the unconditional distribution of
the returns are skewed, which implies the zero-mean Gaussian variate $a$ in the
log-volatility is correlated to the zero-mean Gaussian variate $b$ in
the return (c.f. equation \ref{eq:UnconditionalPdf}).

In the context of conditional distributions and forecast, this
correlation translates to the correlation between the residual of the
log-volatility, denoted $y_t$ in section \ref{chp:nordea_15min} and
section \ref{sec:volvo}, and $b_t$. Now let us consider the forecast
function of an \gls{arima} model:
\begin{eqnarray*}
  \ln \sigma_t &=& y_t + \sum_{i=1}^P \phi_i \ln \sigma_{t-i} -
  \sum_{i=1}^Q \theta_i y_{t-i}
\end{eqnarray*}
where $\phi_i < 2$ if the model involves integration, and $\phi_i <
1$ otherwise. Comparing this equation with equation \ref{eq:r_t}, one
immediately realizes that the conditional \gls{pdf} of $r_t$
\begin{eqnarray*}
  r_t &=& \mu + \sigma_t b_t \\
  &=& \mu + \exp\left(
    y_t + \sum_{i=1}^P \phi_i \ln \sigma_{t-i} -
    \sum_{i=1}^Q \theta_i y_{t-i}
  \right) b_t
\end{eqnarray*}
is given by equation \ref{eq:UncondPDFAsymmetric1} and its moments
given by the equations \ref{eq:AsymmetricMoments} if one makes the
following substitutions:
\begin{eqnarray*}
  \sigma &\to& \text{std}(y_t) \\
  \bar{v} &\to& \sum_{i=1}^P \phi_i \ln \sigma_{t-i} - \sum_{i=1}^Q
  \theta_i y_{t-i} \\
  \psi &\to& \text{corr}(y_t, b_t)
\end{eqnarray*}

For the Nordea series considered in section \ref{chp:nordea_15min},
$\text{corr}(y_t, b_t)$ is found to be $3.56 \times 10^{-2}$; while
for the Volvo series considered in section \ref{sec:volvo},
$\text{corr}(y_t, b_t)$ is found to be $-7.3 \times 10^{-3}$. These
values are comparable to those in table \ref{tab:assets_params}, where
$\psi$ is $1.7 \times 10^{-2}$ for the Nordea series and $-1.6
\times 10^{-2}$ for the Volvo series. The similarity in these values
provides some evidence about the validity of the model.

% It is certainly impossible to compare a predicted conditional
% distribution with observations, but conditional distributions are of
% great interest in the context of risk management and derivative
% pricing, so we point out how they may be calculated for ARIMA
% log-volatility models.

\chapter{Covariance Matrix of Gaussian Returns}
\label{chp:Gaussian}
In chapter \ref{chp:PriceModels} we have studied \gls{garch} and
\gls{sv} models and seen their power of forecasting future
volatilities. However, we have not considered the important fact that
a financial market comprises many assets and the volatilities of these
assets are correlated to each other in a complicated
manner. Practically useful volatility forecasts require good
understanding of these correlations.

In the literature, covariance matrices of Gaussian and L\'evy
distributed returns have been studied (see e.g. \cite{politi2010,
  Chiani2012, Lalley2013}). However, as is seen in chapter
\ref{chp:PriceModels}, real-world returns are not described by any
particular distribution but rather by stochastic processes that
account for auto-correlations in the returns and the volatilities.

Therefore, the focus of this and the next chapter is on covariance
matrices of realistic return series, and especially covariance matrices
in the case when the return series have considerable
auto-correlations. In particular, we study covariance matrices of
\gls{garch} (1,1) return series in chapter
\ref{chp:CrossCorrelationFat} and show the influence of
auto-correlations on these matrices. But before that, it is useful to
first understand the influence of auto-correlations on covariance
matrices of Gaussian return series. When auto-correlations are absent,
these matrices are called Wishart matrices and have been studied
extensively. The results we obtain in this chapter will provide a
reference to the studies in chapter \ref{chp:CrossCorrelationFat}.


% In this chapter we present some results about the covariance
% matrix of Gaussian returns. The assumption of Gaussian returns is of
% course an over-simplification of reallity, but nevertheless lends some
% insight into the problem of elements' and eigenvalues' distributions of
% a covariance matrix.
% Our primary interest is in the influence that auto-correlations in
% returns exert on the covariance matrix.

In section \ref{sec:GCC-analytical} we discuss how the distributions
of the matrix elements are affected by autocorrelations, and in
section \ref{sec:GCC-numerical} we investigate the distribution of the
eigenvalues.


\section{Distribution of the Matrix Elements}
\label{sec:GCC-analytical}
In this section we study the distribution of the elements of a
covariance matrix $C = \mtx{RR'}/T$, where
\begin{eqnarray*}
  \mtx R &=&
  \begin{pmatrix}
    r_{11} & r_{12} & \cdots & r_{1T} \\
    r_{21} & r_{22} & \cdots & r_{2T} \\
    \vdots & \vdots & \ddots & \vdots \\
    r_{N1} & r_{N22} & \cdots & r_{NT} \\
  \end{pmatrix}
\end{eqnarray*}
and $\mtx R'$ denotes the transpose of $\mtx R$.
In words, an element $r_{it}$ of $\mtx R$ is the return of asset $i$
at time $t$, with $i=1,\cdots, N$ and $t = 1, \cdots, T$. If each
column of $\mtx R$ follows a zero-mean Gaussian distribution, i.e.
\begin{eqnarray*}
  \begin{pmatrix}
    r_{1t} \\
    \vdots \\
    r_{Nt}
  \end{pmatrix} \sim N(0, \mtx \Sigma)
\end{eqnarray*}
for all $t = 1, \cdots, T$, and none of the return series is
auto-correlated, i.e. $\text{corr}(r_{it}, r_{i,t'}) = 0$ for all $i =
1, \cdots, N$ and $t \neq t'$, then $\mtx{RR'}$ is a Wishart matrix whose
probability density function is well known \cite{Anderson2003}.

When auto-correlations are indeed present in the returns, $\mtx{RR'}$
no longer follows the Wishart distribution. However, the joint distribution
function of its elements can be expressed in terms of the Wishart \gls{pdf}
and the auto-correlations. We show this in appendix
\ref{app:pdf_gaussian1}. Also, in appendix
\ref{chp:gaussian_elements_dist} we derive an approximate expression 
for the asymptotic distribution of these matrix elements, assuming
$r_{it}$ is an \gls{ar}(1) process \footnotemark, i.e. $r_{it} =
\phi r_{i, t-1} + a_{it}$, and
\footnotetext{
  It is straight forward to derive the auto-correlation function
  $\varrho_k$ of an \gls{ar}(1) process with autoregressive
  coefficient $\phi$:
  \begin{eqnarray*}
    \varrho_k &=& \text{corr}(r_{it}, r_{i, t-k}) \\
    &=& \phi \varrho_{k-1} \\
    &=& \phi^k
  \end{eqnarray*}
  Let $\tau$ denote the time lag at which $\varrho_{\tau} = 1/2$, then
  it follows $\tau = -\ln 2/\ln \phi$.
}
\begin{eqnarray*}
  \begin{pmatrix}
    a_{1t} \\
    \vdots \\
    a_{Nt}
  \end{pmatrix} \sim N(0, \mtx \Sigma)
\end{eqnarray*}
where
\begin{eqnarray*}
  \mtx \Sigma &=& \sigma^2
  \begin{pmatrix}
    1 & \rho & \cdots & \rho \\
    \rho & 1 & \cdots & \rho \\
    \vdots & \vdots & \ddots & \vdots \\
    \rho & \rho & \cdots & 1
  \end{pmatrix}
\end{eqnarray*}
where $-1 < \rho < 1$ is a constant parameter describing the correlation
between $a_{it}$ and $a_{jt}$ --- so, by construction, we assume such
correlations are constant across all asset pairs and over all
time.

The elements $C_{ij}$ ($i \neq j$) of the covariance matrix, $C
= \mtx{RR'}/T$, are found to be normally distributed with mean
\begin{eqnarray}
  \mu'_X &=& {\sigma^2 \over \sqrt{2\pi} (1 - \phi^2)(1 -
    \rho^2)^{1/4}} \left[ P^{-3/2}_{-1/2}(-\rho) -
    P^{-3/2}_{-1/2}(\rho)
  \right] \label{eq:gaussian_mean2}
\end{eqnarray}
and variance
\begin{eqnarray}
  \sigma'^2_X &=& {1 \over (1 - \phi^2)^2}\left[
    \sum_{t=1}^T \sum_{k=1}^{t-1} 2\left(
      \phi^k \over T
    \right)^2 \sigma^6 + \sum_{t=1}^T
    {\sigma^4 (1 - \rho^2)^2 \over T^2} v^2(\rho)
  \right] \nonumber \\
  &=& {2 \sigma^6 \over T (1 - \phi^2)^2} \left[
    {\phi^2 \over 1 - \phi^2} -
    {\phi^2 (1 - \phi^{2T}) \over
      T(1 - \phi^2)}
  \right] + {\sigma^4 (1 - \rho^2)^2 v^2(\rho) \over
    T (1 - \phi^2)^2} \nonumber \\
  &\approx& {2 \sigma^6 \phi^2 \over T (1 - \phi^2)^3}
  + {\sigma^4 (1 - \rho^2)^2 v^2(\rho) \over
    T (1 - \phi^2)^2} \label{eq:gaussian_variance2}
\end{eqnarray}
where $P^\mu_\nu(\cdot)$ is Ferrer's function of the first
kind. See appendix \ref{chp:gaussian_elements_dist}. For the
definition of Ferrer's function, see equation
\ref{eq:Ferrers_1st}.

Equation \ref{eq:gaussian_mean2} tells that, if
two return series $i$ and $j$ are not correlated, i.e. $\rho = 0$,
auto-correlation in the returns does not introduce a bias into the
estimation of their covariance, i.e. $\mu'_X$, since the difference
between the two Ferrer's functions evaluate to 0 in equation
\ref{eq:gaussian_mean2}; if, however, the return series are indeed
correlated, auto-correlation in the returns rescales the
covariance through a multiplicative factor $1/(1 - \phi^2)$.

In addition, equation \ref{eq:gaussian_variance2} tells that
auto-correlation in the returns always makes the covariance
estimation more noisy --- auto-correlation not only rescales the
variance of the no-autocorrelation estimation by $1/(1 - \phi^2)^2$
but even adds an extra term ${2 \sigma^6 \phi^2 \over T (1 -
  \phi^2)^3}$.

For the diagonal elements of the covariance matrix $C$ we have
\begin{eqnarray}
  \E(C_{ii}) &=& {1 \over T}\left[
    \sum_{k=0}^{t-1} \phi^{2k} \sigma^2
  \right] \nonumber \\
  &=& {\sigma^2 \over (1 - \phi^2) T} \left[
    T - {\phi^2 (1 - \phi^{2T}) \over 1 - \phi^2}
  \right] \nonumber \\
  & \approx & {\sigma^2 \over 1 - \phi^2} \left[
    1 - {\phi^2 \over T (1 - \phi^2)}
  \right] \label{eq:gaussian_cii_mean2}
\end{eqnarray}
and
\begin{eqnarray}
  \var(C_{ii}) &=& \sum_{t=1}^T \left[
    \sum_{k=0}^{t-1} {\phi^{4k} \sigma^4 \over T^2} 2 +
    \sum_{k,l=0}^{t-1} {\phi^{2(k+l)} \over T^2} \sigma^6
  \right] \nonumber \\
  &=& \sum_{t=1}^T \left[
    {2 \sigma^4 \over T^2} {1 - \phi^{4t} \over 1 - \phi^4} +
    {\sigma^6 \over T^2} \left(
      {1 - \phi^{2t} \over 1- \phi^2}
    \right)^2 \right] \nonumber \\
  &=& {2 \sigma^4 \over T (1 - \phi^4)} -
  {2 \sigma^4 \phi^4 (1 - \phi^{4T}) \over T^2(1 - \phi^4)^2} +
  \nonumber \\
  && {\sigma^6 \over T (1 -\phi^2)^2} -
  {2 \sigma^6 \phi^2 (1 - \phi^{2T}) \over T^2 (1 - \phi^2)^3} +
  {\sigma^6 \phi^4 (1 - \phi^{4T}) \over T^2 (1 - \phi^2)^2 (1 -
    \phi^4)} \nonumber \\
  &\approx& {2 \sigma^4 \over T (1 - \phi^4)} + {\sigma^6 \over T (1
    -\phi^2)^2} \label{eq:gaussian_cii_variance2}
\end{eqnarray}

From equation \ref{eq:gaussian_cii_mean2} we see that auto-correlation
in the returns increases the variance of the return series; and from
equation \ref{eq:gaussian_cii_variance2} we see that the variance of
that variance estimation is also increased by
auto-correlations. Moreover, we note that $\var(C_{ii})$ scales with T
approximately as $1/T$, similar to the behavior of
$\var(C_{ij})$. This is to be compared with the case of \gls{garch}
returns discussed in chapter \ref{chp:CrossCorrelationFat}.

\section{Distribution of the Eigenvalues}
\label{sec:GCC-numerical}
For a Wishart matrix $\mtx{RR'}$, theoretical results are available
for the eigenvalue distribution, and we summarize them in appendix
\ref{sec:wishart_eigen_dist}. In short, the joint probability density
function of the eigenvalues is given by equation
\ref{eq:wishart_eigen_pdf} when neither auto-correlation nor
cross-correlation is present in $\mtx R$. Moreover, the mariginal
distribution of the largest eigenvalue is approximated by a gamma
distribution \cite{Chiani2012}.

However, as detailed in the derivation leading to equation
\ref{eq:cross-corr-matrix-PDF}, the distribution of $\mtx{RR'}$ is not
Wishart when the columns of $\mtx{R}$ are correlated
(auto-correlation). Deriving the eigenvalue distribution analytically
in this case is beyond the scope of this thesis. Instead, we resort to
numerical methods.

As before we consider the \gls{ar}(1) process:
\begin{equation*}
  \vec{r}_t = \phi \vec{r}_{t-1} + \vec{a}_{t}
\end{equation*}
where $\vec{a}_t \sim N(0, \mtx{I})$, i.e. the elements of $\vec{a}_t$ are
independent Gaussian random variable with zero mean and unit
variance. Now we investigate how the eigenvalue distribution depends on
the auto-correlation strength parameter $\phi$. Figure
\ref{fig:GaussianMarkovSpectrumPDF} shows the results of the
simulation.
\begin{figure}[htb!]
  \begin{center}
    \includegraphics[scale=0.4, clip=true, trim=90 228 115
    226]{../pics/GaussianMarkovSpectrumPDF.pdf}
  \end{center}
  \vspace{-10mm}
  \caption{\small \it
      Eigenvalue distribution with correlation time $\tau$ ranging
      from 0 to 3. The 1st blue line, which is shown as stairs, is the
      theoretical eigenvalue distribution according to the
      Marcenko-Pastur law (see eq.\ref{eq:MP_pdf}). In the simulation
      we have chosen $q = N/T = 50/1000 = 0.05$ and the standard
      deviation of the returns $\sigma=1$. For each value of $\tau$ we
      generate 2000 instances of $N \times T$ random matrix $R$, and
      compute C as $C=RR'/T$. Hence each curve in the figure is
      constructed from 2000 sets of eigenvalues. The correspondance
      between the correlation time $\tau$ and the auto-regressive
      coefficient $\phi$ is $\tau = -\ln 2 / \ln \phi$.
    }
  \label{fig:GaussianMarkovSpectrumPDF}
\end{figure}
It is clear from the figure that the maximum eigenvalue moves
consistently to the right as the value of $\phi$ increases, and, as
shown in figure \ref{fig:Gaussian_mineig}, the minimum eigenvalue
also increases with $\phi$.
\begin{figure}[htb!]
  \centering
  \includegraphics[scale=0.4, clip=true, trim=100 226 116
  133]{../pics/Gaussian_mineig.pdf}
  \caption{\small \it The minimum eigenvalue versus
    auto-correlation strength $\phi$. For each value of $\phi$ 2000
    random matrices are generated and their eigenvalues are
    calculated. The minimum eigenvalue of each random matrix is noted
    and the mean of the 2000 such minimum eigenvalues are plotted
    against the chosen value of $\phi$. 20 values of $\phi$ are
    included in the plot, ranging from 0 to 0.95 with step size 0.05.}
  \label{fig:Gaussian_mineig}
\end{figure}

Chiani showed that the marginal distribution of the maximum eigenvalue
($\lambda_1$) is approximately gamma when neither cross-correlation
nor auto-correlation is present\cite{Chiani2012}. So we compare
in figure \ref{fig:GaussianMarkov005MaxEigCDF_loglog} the empirical
cumulative distribution function (CDF) of the maximum eigenvalue 
with the CDF of a gamma distribution. The cases where
auto-correlations are present ($\phi > 0$) have also been included.
\begin{figure}
  \begin{center}
    \includegraphics[scale=0.5, clip=true, trim=0 238 0
    197]{../pics/GaussianMarkov005MaxEigCDF_loglog.pdf}
  \end{center}
  \caption{\small \it Cummulative distribution function (CDF) of the
    maximum eigenvalue ($\lambda_1$) from numerical simulations (blue
    lines) are compared to the fitted gamma distribution (red
    lines). Each pair of CDFs correspond to a fixed autocorrelation
    strength ($\phi$). The parameters $k$ and $\theta$ of the gamma
    distribution are fit to data by matching the 2nd and the 3rd
    moments of the gamma distribution to the corresponding moments of
    the empirical distribution. Then the parameter $\alpha$ in
    equation \ref{eq:TracyWidom-Gamma} is chosen to be $\alpha =
    k\theta - \E\left({\lambda_1 - \mu_{NT} \over
        \sigma_{NT}}\right)$. The curves are plotted on log-log
    scale.}
  \label{fig:GaussianMarkov005MaxEigCDF_loglog}
  %\vspace{-10mm}
\end{figure}
It is seen in the figure that gamma distributions with different
parameters fit fairly well in all cases. So we conclude that a
gamma distribution not only approximates the maximum eigenvalue
distribution at the absence of autocorrelations but does so even at
the {\it presence} of autocorrelations.

Since the maximum eigenvalue distribution is approximated by a gamma
distribution characterized by parameters $k$, $\theta$, and $\alpha$
\footnotemark, the influence of the autocorrelations can be
characterized by the dependence of $k$, $\theta$, and  $\alpha$ on
$\phi$. While these dependences are rather intricate, good support can
be found in the data for the following approximate relation:
\footnotetext{
  The mean, variance and skewness of the gamma distribution are given
  by
  \begin{eqnarray*}
    \text{mean} &=& k\theta \\
    \text{variance} &=& k\theta^2 \\
    \text{skewness} &=& 2/\sqrt{k}
  \end{eqnarray*}
}
\begin{eqnarray}
  k\theta &=& a \tan^2{\pi \phi \over 2} + b\tan{\pi \phi \over 2} +
  c \label{eq:k_theta-phi}
\end{eqnarray}
Here we note that $k\theta$ is the mean of the gamma distribution.
To verify this relation, we first fit a 2nd order polynomial and
obtain the coefficients $a$, $b$, $c$; then for each data point
$k_n\theta_n$ we solve the quadratic equation
\begin{eqnarray}
  a \tan^2{\pi \phi'_n \over 2} + b\tan{\pi \phi'_n \over 2} + c -
  k_n\theta_n &=& 0\label{eq:k_theta-phi_2}
\end{eqnarray}
for $\tan{\pi \phi'_n \over 2}$. If relation \ref{eq:k_theta-phi} is a
good approximation, a close match between $\tan{\pi \phi'_n \over 2}$
and $\tan{\pi \phi_n \over 2}$ is expected. From figure 
\ref{fig:phi_and_roots} one can see this is indeed the case.

\begin{figure}[htb!]
  \centering
    \includegraphics[scale=0.5, clip=true, trim=37 217 39
    170]{../pics/phi_and_roots.pdf}
  \caption{\small \it Upper plot: $\tan{\pi \phi'_n \over 2}$ against
    $\tan{\pi \phi_n \over 2}$. The fitted line has equation $y_n =
    0.995 \tan{\pi \phi_n \over 2} + 0.0146$. Lower plot: Residuals of
    the linear fit, i.e. $\tan{\pi \phi'_n \over 2} - y_n$. 20 values
    of $\phi$ are included in the plot, ranging from 0 to 0.95 with
    step size 0.05.}
  \label{fig:phi_and_roots}
\end{figure}

% Figure \ref{fig:GaussianMarkovMaxEig_k-phi} shows how the mean of the
% fitted gamma distribution, i.e. $k\theta$, varies as the
% autocorrelation strengthens ($\phi$ increases). 
% \begin{figure}[htb!]
%   %\vspace{-10mm}
%   \centering
%     \includegraphics[scale=0.5, clip=true, trim=100 223 112
%     141]{../pics/GaussianMarkov05MaxEig_k-phi.pdf}
%     \caption{\small \it The parameter $k$ against autocorrelation strength
%       $\phi$. Blue crosses: Empirical values of $k$. Red: best fitting
%       line in terms of {\it Least Square Errors}.}
%   \label{fig:GaussianMarkovMaxEig_k-phi}
% \end{figure}
% From the equation of the fitting line we can directly read out
% \begin{eqnarray*}
%   k &=& a\phi + b \\
%   &=& a\left(1 \over 2\right)^{1/\tau} + b
% \end{eqnarray*}
% where $a = -26$ and $b = 47$.

% For the parameter $\theta$, its behavior is more conveniently
% described in terms of the correlation time $\tau$. Figure
% \ref{fig:GaussianMarkovMaxEig_theta-tau} plots the values of $\theta$
% against those of $\tau$ together with a fitting quadratic
% function. Higher order polynomials provide a slightly better
% fit, but coefficients of the 3rd order and above are less than 1/1000
% times the coefficients of the 2nd and the 1st order. Therefore a 2nd
% order polynomial has been chosen.
% \begin{figure}[htb!]
%   \vspace{-15mm}
%   \centering
%   \includegraphics[scale=0.5, clip=true, trim=99 230 114
%   139]{../pics/GaussianMarkov05MaxEig_theta-tau.pdf} 
%   \caption{\small \it Vertical axis: $\theta$; Horizontal axis:
%     correlation time $\tau$. Blue: empirical values of $\theta$. Cyan:
%     Fitting quadratic function.}
%   \label{fig:GaussianMarkovMaxEig_theta-tau}
% \end{figure}
% The equation of this polynomial is
% \begin{eqnarray*}
%   \theta &=& A\tau^2 + B\tau + C
% \end{eqnarray*}
% where $A = 2.1\times 10^{-4}$, $B = -1.1\times 10^{-4}$, $C =
% 3.2\times 10^{-4}$.

% The parameter $\alpha$ shifts the gamma distribution $\mathscr{G}(k,
% \theta)$ to match the mean of ${(\lambda_1 -
%   \mu_{NT})/\sigma_{NT}}$. The behavior of $\alpha$ with respect to
% changing autocorrelation is shown in figure
% \ref{fig:GaussianMarkovMaxEig_alpha-tau}.
% \begin{figure}[htb!]
%   \vspace{-15mm}
%   \centering
%   \includegraphics[scale=0.5, clip=true, trim=104 229 114
%   139]{../pics/GaussianMarkovMaxEig_alpha-tau.pdf}
%   \caption{\small \it The mean-shift parameter $\alpha$ against
%     correlation time $\tau$. Blue: empirical values of $\alpha$. Cyan:
%     quadratic fit.}
%   \label{fig:GaussianMarkovMaxEig_alpha-tau}
% \end{figure}
% The equation of $\alpha$ is then inferred from the fitting line:
% \begin{equation*}
%   \alpha = D\tau^2 + E\tau + F
% \end{equation*}
% where $D = -3.4\times 10^{-3}$, $E = -4.6\times 10^{-2}$ and $F = 69$.

% So combining the results of $\alpha$, $k$ and $\theta$ we can express
% the moments of the Tracy-Widom variate ${(\lambda_1 -
%   \mu_{NT})/\sigma_{NT}}$:
% \begin{equation}\label{eq:lambda1_MeanVariance}
%   \begin{aligned}
%     \E({\lambda_1 - \mu_{NT} \over \sigma_{NT}}) &= k\theta -
%     \alpha \\
%     &= \left[-a\left(1 \over 2\right)^{1/\tau} + b\right](A\tau^2 +
%     B\tau - C) - (D\tau^2 + E\tau + F) \\
%     \var({\lambda_1 - \mu_{NT} \over \sigma_{NT}}) &=
%     k\theta^2 \\
%     &= \left[-a\left(1 \over 2\right)^{1/\tau} + b\right](A\tau^2 +
%     B\tau - C)^2 \\
%   \end{aligned}
% \end{equation}
% Figure \ref{fig:GaussianMarkovMaxEig_tw_moments} shows the empirical
% moments of ${(\lambda_1 - \mu_{NT})/\sigma_{NT}}$ together with the
% corresponding values computed using the above formulas.
% \begin{figure}[htb!]
%   \vspace{-15mm}
%   \centering
%   \includegraphics[scale=0.46, clip=true, trim=48 243 6
%   134]{../pics/GaussianMarkovMaxEig_tw_moments.pdf}
%   \caption{\small \it empirical moments of ${(\lambda_1 -
%       \mu_{NT})/\sigma_{NT}}$ against their theoretical
%     counterparts. Left: Empirical/theoretical mean against $\tau$;
%     Right: Empirical/theoretical variance against $\tau$. Blue
%     crosses: empirical values; Red line: fitting curves. Horizontal
%     axis: correlation time $\tau$.}
% \label{fig:GaussianMarkovMaxEig_tw_moments}
% \end{figure}
% The good fitness shown in figure
% \ref{fig:GaussianMarkovMaxEig_tw_moments} allows us to conclude 
% that, within the range of the correlation time that we have studied,
% namely $\tau \in [0, 13.51]$, the $k$ parameter is a linear function
% of $\phi = 2^{-1/\tau}$ while $\theta$ and $\alpha$ are quadratic
% functions of $\tau$. The moments of the transformed maximum
% eigenvalue, namely ${(\lambda_1 - \mu_{NT})/\sigma_{NT}}$, are thus
% expressed as functions of the correlation time $\tau$ via the
% parameters $k$, $\theta$ and $\alpha$.

% Figure \ref{fig:GaussianMarkovMaxEigPDF_original} shows the PDF of the
% maximum eigenvalue ($\lambda_1$) for a range of values of the
% correlation time $\tau$. One can clearly see that the mean of the
% distribution moves to the right and the width of the distribution
% increases as $\tau$ takes on larger and larger values. However, one
% must not forget that the behavior of the mean and the width is random
% in nature. Eq. \ref{eq:lambda1_MeanVariance} describes such behavior in
% the asymptotic limit, i.e. if an infinite number of random matrices
% are generated and their respective maximum eigenvalues computed for
% each and every value of correlation time $\tau$.
% \begin{figure}[htb!]
%   \begin{center}
%     \includegraphics[scale=0.5, clip=true, trim=98 228 112
%     221]{../pics/GaussianMarkovMaxEigPDF_original.pdf}
%     \caption{\small \it Probability density function of the maximum eigenvalue
%       ($\lambda_1$).}
%   \end{center}
%   \label{fig:GaussianMarkovMaxEigPDF_original}
% \end{figure}



\chapter{Correlation Matrix of GARCH(1,1) Returns}
\label{chp:CrossCorrelationFat}

% In this chapter we study the cross-correlation matrix $C$ constructed from
% returns that have a fat-tailed distribution, i.e.
% \begin{eqnarray*}
%   C &=& {1 \over T^{2/\alpha}} RR'
% \end{eqnarray*}
% where $R$ is an $N\times T$ returns' matrix whose entries are
% \begin{eqnarray}\label{eq:CrossCorrelationMatrix}
%   R_{it} &=& \left\{
%   \begin{array}{ll}
%     a_{it} & \text{if $t=1$} \\
%     \phi R_{i, t-1} + a_{it} & \text{if $t>1$} \\
%   \end{array} \right.
% \end{eqnarray}
% Here $a_{it}$ are random variables drawn from a stable distribution
% with l\'evy index $\alpha$. In \S\ref{sec:LevyWishart} we look at
% the case where the stable distribution is chosen to be Cauchy; in
% \S\ref{sec:XieWishart} we look at the case where $a_{it}$ is
% described by an ARIMA log-volatility model, which is detailed in
% \S\ref{sec:SLV_model}; finally in \S\ref{sec:GarchWishart} we
% investigate the case where $a_{it}$ is a GARCH(1, 1) process, which is
% proven to have regularly varying tails by Mikosch et al
% \cite{mikosch2000}.

% \section{Wishart-L\'evy Matrices}\label{sec:LevyWishart}
% In this section we study the case where the distribution of $a_{it}$
% as appear in equation \ref{eq:CrossCorrelationMatrix} is chosen to be
% Cauchy, a stable distribution with L\'evy index $\alpha =
% 1$. Politi et al derived the eigenvalue distribution of Wishart-L\'evy
% matrices in \cite{politi2010} using free probability theory. Here we
% show, qualitatively, how the distribution changes when
% auto-correlations are present among the return series. To be precise,
% the PDF of $a_{it}$ is
% \begin{eqnarray*}
%   f_a(x) &=& {\gamma \over \pi} {1 \over \gamma^2 + x^2}
% \end{eqnarray*}

% By numerically simulating a large number of cross-correlation matrices
% and solving for the eigenvalues of each simulated matrix, one can
% obtain the eigenvalue distribution corresponding to each value of
% $\phi$. Figure \ref{fig:WishartLevySpectraPositive} shows the
% distribution for a range of positive $\phi$ values, and figure
% \ref{fig:WishartLevySpectraNegative} shows those for a range of
% negative $\phi$ values.
% \begin{figure}[htb!]
%   \centering
%   \subfigure[]{
%     \includegraphics[scale=0.35, clip=true, trim=94 223 109
%     134]{../pics/WishartLevyEigDist.pdf}
%     \label{fig:WishartLevySpectraPositive}
%   }
%   \subfigure[]{
%     \includegraphics[scale=0.35, clip=true, trim=94 223 109
%     134]{../pics/WishartLevyNegEigDist.pdf}
%     \label{fig:WishartLevySpectraNegative}
%   }
%   \caption{\small \it Eigenvalue distribution (PDF) of Wishart-L\'evy
%     Matrices. Horizontal axis: eigenvalues ($\lambda$) on log-scale;
%     Vertical axis: PDF ($f_\phi(\lambda)$) of the eigenvalues on
%     log-scale. \ref{fig:WishartLevySpectraPositive}: $\phi > 0$;
%     \ref{fig:WishartLevySpectraNegative}: $\phi < 0$;  The $|\phi|$
%     values 0, 0.5, 0.8, and 0.99 correspond to $\tau$ (correlation
%     time) values 0, 1.0, 3.10, and 68.97.  The correlation time is
%     independent of the sign of $\phi$.}
% \end{figure}

% It is seen in these figures
% \begin{itemize}
% \item the eigenvalue distribution is identical for $\phi$ and
%   $-\phi$;
% \item the distribution is not bounded within a finite interval but rather
%   extends to infinity, in constrast to the Marcenko-Pastur law that
%   bounds the maximum eigenvalue at $\sigma^2(1 + \sqrt{N/T})^2$;
% \item as auto-correlation increases, the distribution flatens,
%   implying probability being transferred to large eigenvalues.
% \end{itemize}

% \section{Cross-Correlation Matrix of ARIMA Log-volatility
%   returns}\label{sec:XieWishart}
% In this section we study the cross-correlation matrix of return series
% specified as follows:
% \begin{equation}
%   \label{eq:CrossCorrSLV1}
%   \begin{aligned}
%     R_{it} &= \phi R_{i, t-1} + a_{it} \\
%     a_{it} &= e^{v_{it}} b_{it}
%   \end{aligned}
% \end{equation}
% where $v_{it}$ is typically described by an ARIMA model as discussed
% in \S\ref{sec:SLV_model}. The unconditional distribution of
% such returns is calculated in \S\ref{sec:XieCalc}, where it is shown
% that all the moments of the distribution are finite. Based on this
% observation, we expect the Marcenko-Pastur law to emerge for the
% eigenvalue distribution of the cross-correlation matrix C, when
% $\phi=0$, i.e. when no auto-correlations are absent. In this case C is
% built as
% \begin{eqnarray*}
%   C &=& {1 \over T} \sum_{t=1}^T (DR)(DR)' \\
% \end{eqnarray*}
% where $R'$ denotes the transpose of $R$ and
% \begin{eqnarray*}
%   D &=&
%   \begin{pmatrix}
%     1/\sigma_1 & 0 & \cdots & 0 \\
%     0 & 1/\sigma_2 & \cdots & 0 \\ 
%     \vdots & \vdots & \ddots & 0 \\
%     0 & 0 & \cdots & 1/\sigma_N
%   \end{pmatrix}
% \end{eqnarray*}
% $\sigma_i$ is the standard deviation of return series $i$. By
% numerical simulation as described above, the eigenvalue distribution
% of C is obtained. We plot it in figure \ref{fig:FatWishartSpectra}.
% \begin{figure}[htb!]
%   \centering
%   \vspace{-20mm}
%   \includegraphics[scale=0.5, clip=true, trim=91 229 116
%   137]{../pics/FatWishartEigDist.pdf}
%   \caption{\small \it Eigenvalue distribution (PDF) of a
%     cross-correlation matrix with returns described by
%     \ref{eq:CrossCorrSLV1}. $\phi$ = 0, 0.5, 0.8, 0.95 correspond to
%     correlation time $\tau$ = 0.00, 1.00, 3.11, 13.51}
%   \label{fig:FatWishartSpectra}
% \end{figure}
% As expected, the case with zero auto-correlation has an eigenvalue
% distribution governed by the Marcenko-Pastur law. As auto-correlations
% become stronger, the distribution deforms, but in a rather different
% way if compared to the case of Gaussian returns. Figure
% \ref{fig:FatWishartSpectra} shows that the minimum eigenvalue
% decreases with increasing $\phi$ in the present case of ARIMA
% log-volatility returns, while figure \ref{fig:Gaussian_mineig} shows
% that, in the case of Gaussian returns, the minimum eigenvalue actually
% increases with $\phi$. We leave this interesting comparison to future
% investigations.

% \section{Cross-Correlation Matrix of GARCH(1,1) returns}
% \label{sec:GarchWishart}
In this chapter we consider the cross-correlation matrix of N
identically specified, possibly auto-correlated GARCH(1, 1)
processes:
\begin{eqnarray}
  r_{it} &=& \phi r_{i, t-1} + \epsilon_{it} \nonumber \\
  \epsilon_{it} &=& \sigma_{it} a_{it} \label{eq:garch_spec}
\end{eqnarray}
where $i=1,2,...,N$; $t=1,2,...,T$; $a_{it}$ is independent,
identically distributed, and
\begin{eqnarray*}
  \sigma_{it}^2 &=& \alpha_0 + \alpha_1 a_{i, t-1}^2 + \beta_1
  \sigma_{i,t-1}^2
\end{eqnarray*}
Mikosch et al showed in \cite{mikosch2000} that a GARCH(1,1) process
satisfying
\begin{eqnarray*}
  \alpha_0 &>& 0 \\
  \E(\alpha_1 Z^2 + \beta_1) &<& 0 \\
\end{eqnarray*}
and
\begin{eqnarray*}
  \E[(\alpha_1 z^2 + \beta_1)^{p/2}] &\geq& 1 \\
  \E|Z|^p \ln|Z| &\leq& \infty
\end{eqnarray*}
for some $p > 0$, is stationary and has regularly varying tails. The
tail exponent $\alpha$ is determined by:
\begin{equation}\label{eq:garch_alpha}
  \E[(\alpha_1 Z^2 + \beta_1)^{\alpha/2}] = 1
\end{equation}
Here $Z$ is a random variable that has the same distribution as
$a_{it}$. In our simulations described hereafter $a_{it}, Z \sim N(0,
1)$. In this section and the next, we first study situations where no
auto-correlations are present among the returns, i.e. $\phi = 0$; then
in \S\ref{sec:garch_nonzero_phi} we look at how auto-correlations
change the picture.

With regularly varying tails, the eigenvalue distribution of a
cross-correlation matrix built from GARCH(1,1) returns is expected to
differ from the Marcenko-Pastur law. In figure
\ref{fig:garch_ev_cii} we simulate N=50 independent GARCH(1,1)
returns series, each with identical parameters, namely $\alpha_0 =
2.3\times 10^{-6}$, $\alpha_1 = 0.15$, $\beta_1 = 0.84$, $\phi = 0$
and T=$8\times10^4$ time steps, then we build the cross-correlation
matrix as
\begin{eqnarray*}
  C &=& {1 \over T^{2/\alpha}} RR'
\end{eqnarray*}
where R is an $N\times T$ matrix, with elements $r_{it}$ specified by
\ref{eq:garch_spec}. The PDF of the eigenvalue distribution of C is
plotted in figure \ref{fig:garch_ev_cii_linear} and CDF of the
distribution is plotted on log-log scale in
\ref{fig:garch_ev_cii_loglog}.
\begin{figure}[htb!]
  \centering
  \subfigure[]{
    \includegraphics[scale=0.40, clip=true, trim=104 271 111
    216]{../pics/garch_ev_cii.pdf}
    \label{fig:garch_ev_cii_linear}
  }
  \subfigure[]{
    \includegraphics[scale=0.40, clip=true, trim=111 241 110
    134]{../pics/garch_ev_cii_loglog.pdf}
    \label{fig:garch_ev_cii_loglog}
  }
  \caption{\small \it \ref{fig:garch_ev_cii_linear}: Eigenvalues \&
    Diagonal elements' distribution of a cross-correlation matrix
    built from independent GARCH return series. Blue: PDF of
    eigenvalues; Green: PDF of diagonal elements; Red: PDF of a
    $\alpha$-stable distribution fitted to the diagonal
    elements. \ref{fig:garch_ev_cii_loglog}: CDF of the same
    quantities in 10-based log-log scale.}
  \label{fig:garch_ev_cii}
\end{figure}
Also plotted in the same figure is the distribution of the diagonal
elements of C. It is clear from the figure that the two PDFs coincide,
implying C is diagonal. This is further confirmed by figure
\ref{fig:garch_cij} which shows the distribution of the non-diagonal
elements of C. One can see the non-diagonal elements are distributed
symmetrically around 0 with a very small width in comparison to the
distribution of the diagonal elements - in fact, 1 order of magnitude
smaller ($2.14\times10^{-5}$ v.s. $7.31\times 10^{-4}$). Hence $C$ is
very close to a diagonal matrix.
\begin{figure}[htb!]
  \centering
    \includegraphics[scale=0.5, clip=true, trim=95 253 91
    223]{../pics/garch_cij.pdf}
  \caption{\small \it Distribution of the non-diagonal elements of
    the cross-correlation matrix. Blue: PDF of the non-diagonal
    elements; Green: $\alpha$-stable distribution fitted to the
    non-diagonal elements' PDF.}
  \label{fig:garch_cij}
\end{figure}

Figure \ref{fig:garch_ev_cii} also shows the two curves are well fitted
by an $\alpha$-stable distribution, with $\alpha \approx 1.38$. This
is really an expected result for the diagonal elements. Using
$\alpha_1 = 0.15$, $\beta_1 = 0.84$, which are the values used for
simulating the GARCH returns, one can obtain, by solving equation
\ref{eq:garch_alpha}, $\alpha=2.96$. Then according to Mikosch et al
\cite{mikosch2000}
\begin{eqnarray*}
  P(|r_t| > x) &\sim& {\E(z^\alpha) c_0 \over x^\alpha} \\
  P(|r_t|^2 > x) &\sim& {\E(z^\alpha) c_0 \over x^{\alpha/2}} \\
\end{eqnarray*}
for some constant $c_0$. Now that $P(|r_t|^2 > x)$ has power-law tail
behavior with power $\alpha/2 < 2$, one can deduce
\begin{equation}
  \label{eq:stable_CLT}
  \begin{aligned}
    \sum_{t=1}^T r_t^2 &\xrightarrow{d} S(\alpha/2,
    1, \gamma, \mu) \text{ as $T \to \infty$}
  \end{aligned}
\end{equation}
where $\xrightarrow{d}$ denotes convergence in distribution, and
$S(\alpha/2, 1, \gamma, \mu)$ denotes an $\alpha$-stable distribution
with parameters $(\alpha/2, 1, \gamma, \mu)$. Here $\alpha/2$ is the
L\'evy index, 1 is the asymmetry, $\gamma$ is the scale parameter
and $\mu$ is the mean value of the distribution. Asymmetry being 1
means a random variable so distributed only takes positive values
\cite{Bilik2008, Embrechts1997}.

The mean $\mu$ in equation \ref{eq:stable_CLT} is given by
\begin{eqnarray*}
  \mu &=& T \E(|r_t|^2) \\
  &=& T{\alpha_0 \over 1 - \alpha_1 - \beta_1}
\end{eqnarray*}
and the scale parameter $\gamma$ in \ref{eq:stable_CLT} is determined
by the limit \cite{Bilik2008}
\begin{eqnarray*}
  \lim_{T\to\infty} {T \E(z^{\alpha/2}) c_0 \over \gamma^{\alpha/2}}
  &=& C_{\alpha/2}
\end{eqnarray*}
where
\begin{eqnarray*}
  C_{\alpha/2} &=& \left( \int_0^\infty {\sin x \over x^{\alpha/2}} dx
  \right)^{-1} \\
  &\approx& {1 \over \sqrt{2 \pi}}
\end{eqnarray*}
Therefore
\begin{eqnarray*}
  \gamma^{\alpha/2} &=& \sqrt{2\pi} T \E\left(
    |z|^{\alpha/2}
  \right) c_0 \\
  \gamma &=& (2\pi)^{1/\alpha} T^{2/\alpha} \E\left(
    |z|^{\alpha/2}
  \right)^{2/\alpha} c_0^{2/\alpha}
\end{eqnarray*}
Here we note that an $\alpha$-stable distribution $S(\alpha, \beta,
\gamma, \mu)$ has characteristic function \cite{Guhr2007}
\begin{eqnarray*}
  \phi(k; \alpha, \beta, \gamma, \mu) &=& \exp\left[
    i\mu k - \gamma^\alpha |k|^\alpha \left(
      1 - i \beta {k \over |k|} \tan{\pi \alpha \over 2}
    \right) \right] \text{ for $\alpha \neq 1$}
\end{eqnarray*}
from which we see $\phi(ak; \alpha, \beta, \gamma, \mu) = \phi(k;
\alpha, \beta, a\gamma, a\mu)$, implying that, if $x \sim S(\alpha,
\beta, \gamma, \mu)$, then $ax \sim S(\alpha, \beta, a\gamma, a\mu)$.

Now that
\begin{eqnarray*}
  \sum_{t=1}^T r_t^2 &\xrightarrow{d} S(\alpha/2,
  1, \gamma, \mu) \text{ as $T \to \infty$}  
\end{eqnarray*}
we have
\begin{eqnarray*}
  C_{ii} &=& {1 \over T^{2/\alpha}}\sum_{t=1}^T r_{it}^2 \\
  &\xrightarrow{d}&
  S(\alpha/2, 1, \gamma_D, \mu_D) \text{ as $T \to \infty$}
\end{eqnarray*}
where
\begin{eqnarray}\label{eq:garch_wishart_cij_params}
  \gamma_D &=& (2\pi)^{1/\alpha} \E\left(|z|^{\alpha/2}
  \right)^{2/\alpha} c_0^{2/\alpha} \\
  \mu_D &=& {\alpha_0 \over 1 - \alpha_1 - \beta_1} T^{1 - 2/\alpha}
  \nonumber
\end{eqnarray}
So the diagonal elements of the cross-correlation matrix converge to
an $\alpha$-stable distribution with L\'evy index $\alpha/2 \approx
1.48$. This is consistent with the measured index 1.38 within
measurement errors.

Now we look at the distribution of the non-diagonal elements. At first
glance, one might be tempted to think that figure \ref{fig:garch_cij}
shows a Gaussian distribution, but in fact, it is not. This is
illustrated in the probability plot in figure
\ref{fig:garch_nondiag_probplot}. In comparison to the PDF of a
Gaussian, the PDF of the nondiagonal elements is significantly fatter
on the tails.
\begin{figure}[htb!]
  \centering
  \includegraphics[scale=0.5, clip=true, trim=81 229 114
    121]{../pics/garch_cij_prob.pdf}
    \caption{\small \it Probability plot of the non-diagonal
      elements. Blue: The accumulative probability function
      (CDF) of the non-diagonal elements. Black, dashed: CDF of the
      Gaussian distribution that has the same mean and variance as the
      sample. The graph is arranged on such a scale that the Gaussian
      CDF is a straight line.
    }
  \label{fig:garch_nondiag_probplot}
\end{figure}

The parameters of the non-diagonal elements' distribution $S(\alpha',
\beta', \gamma', \mu')$ can be obtained via fitting. The results have
been shown in the legend of figure \ref{fig:garch_cij}. In table
\ref{tab:garch_wishart_cij_params} we list them with a higher
precision.
\begin{table}[htb!]
  \centering
  \begin{tabular}{|c|c|c|c|}
    \hline
    $\alpha'$ & $\beta'$ & $\gamma'$ & $\mu'$ \\
    \hline
    1.9453 & 0.0018 & $2.1381 \times 10^{-5}$ & $2.0637\times 10^{-9}$ \\
    \hline
  \end{tabular}
  \caption{\small \it Parameters of the non-diagonal elements' distribution}
  \label{tab:garch_wishart_cij_params}
\end{table}
Since $r_{it}$ and $r_{jt}$ are independent of each other, $\beta'$
and $\mu'$ are expected to be 0 --- but with a finite T, some deviation
from 0 is not surprising.

Now that the parameters' values have been obtained, the way they scale
with T, i.e. the length of the return series, can be deduced. Consider
\begin{equation}\label{eq:garch_cij1}
  T^{2/\alpha} C_{ij} = \sum_{t=1}^T r_{it} r_{jt} \xrightarrow{d} S(\alpha',
  0, \gamma', 0)
\end{equation}
where the width $\gamma'$ is determined by \cite{Bilik2008},
\begin{eqnarray*}
  \lim_{T\to\infty} {T C \over \gamma'^{\alpha'}}
  &=& C_{\alpha'} \\
  C_{\alpha'} &=& \left( \int_0^\infty {\sin x \over x^{\alpha'}} dx
  \right)^{-1} \\
\end{eqnarray*}
Here the constant C is such that
\begin{equation*}
  P(|r_{it}r_{jt}| > x) \sim {C \over x^{\alpha'}} \text{ as $x \to \infty$}
\end{equation*}
So we have
\begin{eqnarray*}
  \gamma' &=& {\left( CT \over C_{\alpha'}\right)^{1/\alpha'}}
\end{eqnarray*}
Divide throughout equation \ref{eq:garch_cij1} by $T^{2/\alpha}$ then
gives
\begin{equation}\label{eq:garch_cij2}
  C_{ij} = {1 \over T^{2/\alpha}}\sum_{t=1}^T r_{it} r_{jt}
  \xrightarrow{d}
  S(\alpha', 0, \gamma_N, 0)
\end{equation}
where
\begin{eqnarray*}
  \gamma_N &=& {\left(C \over C_{\alpha'}\right)^{1/\alpha'}}
  T^{1/\alpha'-2/\alpha} \\
\end{eqnarray*}
So we see that distribution of the non-diagonal elements has a width
that scales with T as $T^{1/\alpha' - 2/\alpha} \approx T^{-1/6}$,
while the width of the diagonal elements' distribution, as shown in
equation \ref{eq:garch_wishart_cij_params}, does not scale with
T. This is to be compared with the Wishart case where the return
series have Gaussian distribution, and hence the asymptotic
distributions of both the diagonal and the non-diagonal elements of the
covariance matrix are Gaussian with a variance that scales as $1/T$
(c.f. \ref{sec:GCC-analytical}).

% We also notice that, given our particular choice of $\alpha_1$ and
% $\beta_1$, the covariance matrix's convergence to a diagonal matrix
% is slow, since the non-diagonal elements decay to zero only 

% In the Wishart case, the diagonal and the non-diagonal elements
% scale in the same way with T and thus have comparable sizes at large
% T --- so the matrix is not diagonal and the eigenvalue distribution has
% the Marcenko-Pastur law. In the GARCH case, the diagonal elements do
% not scale with T while the non-diagonal elements scale as $T^{-1/6}$,
% thus at large T, the non-diagonal elements are significantly smaller
% than the diagonal ones --- so a diagonal matrix arises.

\section{Implications of Finite Number of Observations}
In the last section we have discussed the limiting situation where the
return series of the covariance matrix have an infinite number of
observations ($T \to \infty$), and in the simulation studies we have
generated a large number of observations for each return series,
namely $T = 8 \times 10^4$. However, in practice, one often does not
have such a large number of observations available. Therefore, it is
useful to investigate situations where $T$ only has a modest size.

Figure \ref{fig:GarchCiiEig1} shows the diagonal elements' as well as
the eigenvalues' distributions when the number of return series (N) is
fixed at 250 and the number of observations in each series (T) is
increased from 3000 to 6000.
\begin{figure}[htb!]
  \centering
    \includegraphics[scale=0.5, clip=true, trim=81 248 75
    195]{../pics/GarchCiiEig1.pdf}
  \caption{\small \it The Diagonal elements' and the eigenvalues'
    distributions with modest T. Blue: eigenvalues' CDF;
    Green: Diagonal elements' CDF; Red: CDF of the $\alpha$-stable
    distribution fitted to the diagonal elements. All the curves are
    drawn on 10-based log-log scale.
  }
  \label{fig:GarchCiiEig1}
\end{figure}
It is seen from the figure that, compared to the earlier case where $T
= 8 \times 10^4$, the eigenvalue distribution and the diagonal
elements' distribution do not coincide as well but differ rather
siginificantly for small values. For large values, however, the two do
coincide and comply with the fitted $\alpha$-stable
distribution. Moreover, the difference between the eigenvalues'
distribution and the diagonal elements' distribution, as well as that
between the diagonal elements' distribution and the fitting
$\alpha$-stable distribution, is also seen to diminish as T
increases.

The convergence of the eigenvalues ($\lambda$) towards the diagonal
elements $C_{ii}$ as $\lambda \to \infty$ is really an anticipated
result. For convenience, we order the diagonal elements so that
\begin{eqnarray*}
  C_{11} < C_{22} < \cdots < C_{NN}
\end{eqnarray*}
where, as before, $N$ is the dimension of the covariance matrix
$C$. Then, as $x \to \infty$,
\begin{eqnarray*}
  P(C_{ii} > x) &\sim& {c \over x^{\alpha/2}} \\
  f(C_{ii}) &\sim& {c \over x^{\alpha/2 + 1}} \\
\end{eqnarray*}
where $f(\cdot)$ denotes the PDF of the diagonal elements and $c$ is
some constant. Thus, at the limit $C_{ii} \to \infty$, the distance
between two adjacent diagonal elements $C_{ii}$ and $C_{i+1, i+1}$ can
be expressed as
\begin{eqnarray*}
{1 \over N (C_{i+1, i+1} - C_{ii})} &=& f(C_{ii})\\
C_{i+1, i+1} - C_{ii} &\sim& {C_{ii}^{\alpha/2 + 1} \over c N}
\end{eqnarray*}

Thus as $C_{ii} \to \infty$, $C_{i+1, i+1} - C_{ii} \to \infty$ while
$C_{ij} \to 0$ ($i \neq j$). Now that the spacing between adjacent
diagonal elements become wider and the non-diagonal elements become
smaller, the eigenstates considered as a mixture of the basis states,
become more and more localized to a prominent basis state. This
localization can be measured by the size of the component in each
eigenvector that has the largest absolute value($|c|_\M$), provided
that the eigenvectors have been normalized.

Figure \ref{fig:garch_eigenvec_Cmax} shows how $|c|_\M$ changes
in response to increasing $\lambda$. It is seen that as T increases,
localization of the eigenstates proceeds from those with very
large eigenvalues towards those with relatively smaller
eigenvalues. At the same time, the minimum of $|c|_\M$ increases
and so does the minimum of the eigenvalues. The last point here is
further illustrated in figure \ref{fig:garch_eigenvec_Cmax_dist} where
the PDF of $|c|_\M$ is plotted. We see in this figure that an
increased value of T leads to advancement of $\min(|c|_\M)$ to larger
values as well as to an increased proportion of large $|c|_\M$. The
mean of $|c|_\M$ has apparently been increased too.
\begin{figure}[htb!]
  \centering
  \includegraphics[scale=0.5, clip=true, trim=14 199 31
  155]{../pics/garch_eigenvec_Cmax.pdf}
  \caption{\small \it largest component ($|c|_\M$) corresponding to
    the eigenvalue. N is fixed to 50.}
  \label{fig:garch_eigenvec_Cmax}
\end{figure}

\begin{figure}[htb!]
  \centering
  \includegraphics[scale=0.5, clip=true, trim=86 257 112
  136]{../pics/garch_eigenvec_Cmax_dist.pdf}
  \caption{\small \it PDF of the largest component ($|c|_\M$) of
    each eigenvector. N is fixed to 50.}
  \label{fig:garch_eigenvec_Cmax_dist}
\end{figure}

Another informative quantity that measures the localization is the
``Inverse Participation Ratio'' (IPR). For a given normalized eigenvector
$\vec{c}_i = (c_{1, i}, c_{2, i}, ..., c_{N, i})$, the IPR is defined
as \cite{Aberg2013}
\begin{eqnarray*}
  \text{IPR}(\vec{c}_i) &=& \sum_{k=1}^N c_{k,i}^4 \\
\end{eqnarray*}
Figure \ref{fig:garch_eigenvec_PR} shows ${1 \over
  N \cdot \text{IPR}(\vec{c}_i)}$ in correspondence to the
eigenvalues. This quantity is sometimes termed the
normalized participation ratio (PR) and measures the proportion of
basis vectors that contribute considerably to the eigenvector in
question. From this figure we see that, for all values of T, if an
eigenvalue is larger than $10^{-1.5} \approx 0.03$, its corresponding
participation ratio is less than $10^{-1.6} = 2.5\%$, meaning less
than $50 \times 0.025 = 1.26$ basis vectors contribute --- each of the
corresponding eigenvectors is localized to a single basis vector and
hence the distribution of such large eigenvalues is the same as the
diagonal elements' distribution.

Figure \ref{fig:garch_eigenvec_PR_dist} shows the PDF of the
normalized PR. We see that as T increases, the distribution of PR is
compressed towards 0, suggesting increased localization of the
eigenvectors.
\begin{figure}[htb!]
  \centering
  \includegraphics[scale=0.5, clip=true, trim=49 208 70
  162]{../pics/garch_eigenvec_PR.pdf}
  \caption{\small \it Normalized participation ratio (PR)
    versus eigenvalue.}.
  \label{fig:garch_eigenvec_PR}
\end{figure}

\begin{figure}[htb!]
  \centering
  \includegraphics[scale=0.5, clip=true, trim=97 259 113
  226]{../pics/garch_eigenvec_PR_dist.pdf}
  \caption{\small \it PDF of the normalized PR. N is fixed to
    50.}
  \label{fig:garch_eigenvec_PR_dist}
\end{figure}
In conclusion, localization of the eigenvectors, which implies 
coincident eigenvalue and diagonal elements' distributions, begins with
those associated to large eigenvalues. Increased observation points
lead to increased localization and hence increased coincident sections
of the eigenvalue and diagonal elements' distributions. However, this
increment with T is slow, because the diagonal elements mean $\mu_D$
increases only as a fractional power of $T$, namely $T^{1 -
  2/\alpha}$, and the non-diagonal elements' variance decreases only
as a fractional power too, namely $T^{1/\alpha' - 2/\alpha}$. These
have been detailed in equations \ref{eq:garch_wishart_cij_params} and
\ref{eq:garch_cij2}.

\section{Influence of auto-correlations}
\label{sec:garch_nonzero_phi}
In the previous two sections we have studied situations where
$\phi=0$ in the specification \ref{eq:garch_spec}, i.e. no
auto-correlation is in the returns. In this section we investigate how
auto-correlations change the picture.

% When auto-correlations are present among the returns, the
% cross-correlation matrix is expected to change. How exactly it changes
% is the subject of this section. Here we consider a model specified
% as follows:
% \begin{eqnarray*}
%   r_{it} &=& \phi r_{i, t-1} + \epsilon_{it} \\
%   \epsilon_{it} &=& \sigma_{it} z_{it} \\
%   \sigma_{it}^2 &=& \alpha_0 + \alpha_1 \epsilon_{i, t-1}^2 + \beta_1
%   \sigma_{i, t-1}^2 \\
%   z_{it} &\sim& N(0, 1)
% \end{eqnarray*}
% where $r_{it}$ is the element of the R matrix at the $i$-th row and the
% $t$-th column. In the simulations described below R has N=50 rows and
% $T = 8 \times 10^4$ columns. The parameters $\alpha_1$ and $\beta_1$
% are the same as in the previous case of zero auto-correlations, namely
% $\alpha_1 = 0.15$ and $\beta_1 = 0.84$. The cross-correlation matrix C
% is built from R using
% \begin{eqnarray*}
%   C &=& {1 \over T^{2/\alpha}} RR'
% \end{eqnarray*}
% where $\alpha$ is 2.96 as before.

Figure \ref{fig:GarchEigDiag1} shows the eigenvalue as well as the
diagonal elements' distribution when $\phi = 0.95$, i.e. $\tau =
13.51$. The values of N and T are 50 and $8\times 10^4$ as before. The
GARCH(1,1) parameters are also unchanged, namely $\alpha_0 = 2.3\times
10^{-6}$, $\alpha_1 = 0.15$, and $\beta_1 = 0.84$. Figure
\ref{fig:GarchNondiag1} shows the non-diagonal elements' distribution
in the same setup. Included in these plots are $50 \times 2000 =
1\times 10^5$ eigenvalues and diagonal elements, as well as ${50
  \choose 2} \times 2000 = 2,450,000$ non-diagonal elements. These
data come from 2000 simulated matrices.
\begin{figure}[htb!]
  \centering
  \subfigure[]{
    \includegraphics[scale=0.42, clip=true, trim=112 272 103
    217]{../pics/GarchEigDiag1.pdf}
    \label{fig:GarchEigDiag1}
  }
  \subfigure[]{
    \includegraphics[scale=0.42, clip=true, trim=105 272 101
    213]{../pics/GarchNondiag1.pdf}
    \label{fig:GarchNondiag1}
  }
  \caption{\small \it \ref{fig:GarchEigDiag1}: Eigenvalues' and
    diagonal elements' distribution when $\phi = 0.95$, i.e. $\tau$ =
    13.51; \ref{fig:GarchNondiag1}: Non-diagonal elements'
    distribution in the same situation.}
\end{figure}

From figure \ref{fig:GarchEigDiag1} we see that, as auto-correlations
become significant, the distribution of the eigenvalues no longer
coincides with the diagonal elements' distribution --- instead it
becomes wider and fatter on the tails. We also notice that the widths
of both the diagonal and the non-diagonal elements' PDF's have
increased.  In figure \ref{fig:garch_ev_cii} we see that, when no
auto-correlation is present, the PDF of the diagonal elements has
width ($\gamma$) $7.31 \times 10^{-4}$, while in figure
\ref{fig:GarchEigDiag1} we see that the width has become $7.91 \times
10^{-3}$ as $\tau$ becomes 13.51. Similarly the non-diagonal elements'
PDF has width $2.14 \times 10^{-5}$ when $\tau=0$, as shown in figure
\ref{fig:garch_cij}, and this width becomes $9.60 \times 10^{-4}$ when
$\tau = 13.51$, as shown in figure \ref{fig:GarchNondiag1}.

Figure \ref{fig:GarchSpectrumAutocorrelated} shows the eigenvalues'
distribution corresponding to a range of $\phi$ values. The number of
eigenvalues in each curve is the same as in figure
\ref{fig:GarchEigDiag1}.
\begin{figure}[htb!]
  \centering
  \includegraphics[scale=0.5, clip=true, trim=85 258 103
  229]{../pics/GarchSpectrumAutocorrelated.pdf}
  \caption{\small \it Eigenvalue distribution of cross-correlation
    matrix built from auto-correlated GARCH processes. $\phi$ = 0,
    0.5, 0.8, 0.955, 0.97 correspond to correlation time $\tau$ =
    0, 1.00, 3.11, 15.05, 22.76.}
  \label{fig:GarchSpectrumAutocorrelated}
\end{figure}
From this figure one can see that, as auto-correlation strengthens,
\begin{itemize}
\item the PDF of the eigenvalue distribution flattens and widens;
\item the minimum as well as the maximum eigenvalues increase.
\end{itemize}

To find out more about this series of deformation, we first look at
how the largest component and the normalized participation ratio of
the eigenvectors change as $\phi$ takes on larger values. Figure
\ref{fig:garch_eigenvec_Cmax_corr} shows the largest eigenvector
component $|c|_\M$ in correspondence to the eigenvalue. Apparently, as
auto-correlation strengthens, the eigenvectors' composition
fractures, leading to a reduced degree of localization and even
reduced certainty of localization --- for a fixed eigenvalue, $|c|_\M$
now varies in a larger range than it does with smaller $\phi$.
\begin{figure}[htb!]
  \centering
  \includegraphics[scale=0.5, clip=true, trim=0 197 0
  154]{../pics/garch_eigenvec_Cmax_corr.pdf}
  \caption{\small \it Eigenvectors' largest component in
    correspondence to the eigenvalues. N is fixed to 50.}
  \label{fig:garch_eigenvec_Cmax_corr}
\end{figure}


The same story of reduced localization is also evident from the plot
of the normalized participation ratio (PR), shown in figure
\ref{fig:garch_eigenvec_PR_dist_corr}, and from the PDF of $|c|_\m$
shown in figure \ref{fig:garch_eigenvec_Cmax_dist_corr}. It is seen
in \ref{fig:garch_eigenvec_PR_dist_corr} that the peak at the left 
of the plot, representing the group of localized eigenvectors, falls
with increased auto-correlation, and essentially disappears when
$\phi$ reaches the extreme value 0.99. Figure
\ref{fig:garch_eigenvec_Cmax_dist_corr} shows the proportion of large
$|c|_\M$ values is severely reduced and the mean of $|c|_\M$ is pushed
to smaller values by increased auto-correlation.
\begin{figure}[htb!]
  \centering
  \subfigure[]{
    \includegraphics[scale=0.4, clip=true, trim=95 260 111
    232]{../pics/garch_eigenvec_PR_dist_corr.pdf}
    \label{fig:garch_eigenvec_PR_dist_corr}
  }
  \subfigure[]{
    \includegraphics[scale=0.4, clip=true, trim=95 260 111
    232]{../pics/garch_eigenvec_Cmax_dist_corr.pdf}
    \label{fig:garch_eigenvec_Cmax_dist_corr}
  }
  \caption{\small \it \ref{fig:garch_eigenvec_PR_dist_corr}: PDF of
    the normalized participation ratio
    (PR). \ref{fig:garch_eigenvec_Cmax_dist_corr}: PDF of the largest
    component of the eigenvectors. N is fixed to 50. $\phi$ values of
    0, 0.6000, 0.9000, 0.9900 correspond to correlation time $\tau$ =
    0, 1.3569, 6.5788, 68.9676}
\end{figure}

It is also useful to look at how the fraction of localized eigenvectors
changes with the auto-correlation. For definiteness, we classify
an eigenvector as being localized when (1) its number of participating
basis vectors is less than 2, or (2) the largest of its components'
absolute values is larger than 0.9.

Figure \ref{fig:localization_ratio} shows how the ratio of localized
eigenvectors depends on the auto-correlation strength $\phi$. In
either way of classification, the ratio falls with $\phi$ in
accordance with a power law, the power exponent lying a bit below 2.
This is further confirmed in plot \ref{fig:localization_ratio2}, where
the ratios are plotted versus $\phi$ on log-log scale.
\begin{figure}[htb!]
  \centering
  \subfigure[]{
    \includegraphics[scale=0.38, clip=true, trim=93 229 115
    134]{../pics/localization_ratio.pdf}
    \label{fig:localization_ratio}
  }
  \subfigure[]{
    \includegraphics[scale=0.38, clip=true, trim=87 227 115
    133]{../pics/localization_ratio2.pdf}
    \label{fig:localization_ratio2}
  }
  \caption{\small \it \ref{fig:localization_ratio}: Ratio of localized
    eigenvectors versus the auto-correlation strength $\phi$. Black
    ``+'': ratio of localized eigenvectors as measured by the number
    of participating basis vectors being lower than 2; Black ``x'':
    ratio of localized eigenvectors as measured by the largest
    component being larger than 0.9. Blue curve: quadradic function
    fitted to ``+''. Green curve: quadratic function fitted to
    ``x''. There are 27 data points in the plot, corresponding to 27
    $\phi$ values: 0 to 0.8 with step size 0.05, and 0.9 to 0.99 with
    step size 0.01. The corresponding values of the correlation time
    $\tau$ range from 0 to 69. \ref{fig:localization_ratio2}: $\ln
    (f_\M - f)$ is plotted versus $\ln \phi$, where $f$ stands for the
    ratio of localized eigenvectors. The fitted curves are linear.}
\end{figure}

% \subsection{Correlation Matrix of a Factor Model}
%% Result of normalizing by 1/T
% In this section we consider the estimation of correlation matrix in
% the text of a factor model: the observed return series
% $\{r_{it}\}$ ($1 \leq i \leq N$, $1 \leq t \leq T$) are driven by a
% number of unobserved and uncorrelated random variables $z_{it}$
% (factors), each of which is an auto-correlated GARCH(1,1) process as
% specified by equation \ref{eq:garch_spec}. Factor GARCH models have
% been studied extensively in the literature. A good review is found in
% \cite{Mikosch2009}. Here we are interested in the impact of
% auto-correlations in $z_{it}$ on the cross-correlation matrix of
% $r_{it}$.

% To be specific we adopt the generalized orthogonal GARCH model by
% Weide \cite{Weide2002}:
% \begin{eqnarray*}
%   \vec{r_t} &=& \mtx{W}\vec{z_t}
% \end{eqnarray*}
% where $\mtx{W}$ is an $N \times N$ invertible matrix; $\vec{z_t}$ and
% $\vec{r_t}$ are $N \times 1$ column vectors. Moreover, to meet our
% interest, $\vec{z_t}$ are assumed to be auto-correlated:
% \begin{eqnarray*}
%   \vec{z}_t &=& \phi \vec{z}_{t-1} + \vec{y}_t \\
%   \vec{y}_t &=& \mtx{H}^y_t \vec{x}_t \\
% \end{eqnarray*}
% where $\vec{x_t} \sim N(\vec{0}, \mtx{I})$, and $\mtx{H}^y_t$,
% the conditional covariance matrix of $\vec{y}_t$, is given by
% \begin{eqnarray*}
%   \mtx{H}^y_t &=& (\mtx{I} - \mtx A - \mtx B) + \mtx A \odot
%   (\vec{r}_{t-1} \vec{r}'_{t-1}) + \mtx B
%   \mtx{H}^y_{t-1}
% \end{eqnarray*}
% Here $\mtx A$ and $\mtx B$ are diagonal $N \times N$ matrix and
% $\odot$ denotes element-wise multiplication. The form of the constant
% term, $\mtx{I} - \mtx A - \mtx B$, ensures that the unconditional
% variance of $\vec{y}_t$ is $\mtx I$. In the simulations described in
% the rest of this section, we use the parametrization $\mtx A =
% \alpha_1 \mtx I$, $\mtx B = \beta_1 \mtx I$, where $\alpha_1 = 0.15$,
% $\beta_1 = 0.84$. For tractability, we fix $N=4$ and consider one
% particular covariance matrix of $\vec{r}_{it}$:
% \begin{eqnarray*}
%   \mtx C &=& \text{cov}(\vec{r}_{it}) \\
%   &=& \begin{pmatrix}
%     1 & \rho & \rho & \rho \\
%     \rho & 1 & \rho & \rho \\
%     \rho & \rho & 1 & \rho \\
%     \rho & \rho & \rho & 1 \\
%   \end{pmatrix}
% \end{eqnarray*}
% where $\rho = 0.5$. The eigenvalue decomposition of $\mtx C$ is
% \begin{eqnarray*}
%   \mtx C &=& \mtx{U} \mtx{\Lambda} \mtx{U}' \\
%   \mtx U &=&
%   \begin{pmatrix}
%     -1/\sqrt{2} & -1/\sqrt{6} & -1/2\sqrt{3} & 1/2 \\
%     1/\sqrt{2} & -1/\sqrt{6} & -1/2\sqrt{3} & 1/2 \\
%     0 & \sqrt{2}/\sqrt{3} & -1/2\sqrt{3} & 1/2 \\
%     0 & 0 & \sqrt{3}/2 & 1/2 \\
%   \end{pmatrix} \\
%   \mtx \Lambda &=&
%   \begin{pmatrix}
%     1-\rho & 0 & 0 & 0 \\
%     0 & 1-\rho & 0 & 0 \\
%     0 & 0 & 1-\rho & 0 \\
%     0 & 0 & 0 & 1+ 3\rho \\
%   \end{pmatrix}
% \end{eqnarray*}

%% Abandoned.
% In the last section we have investigated what the measured
% covariance matrix looks like when the return series have
% auto-correlated GARCH(1,1) returns but are actually independent of
% each other. In this section we investigate the situation when the
% return series are truly correlated. In particular, we study the
% following case:
% \begin{equation}\label{eq:garch_correlated_returns}
%   \begin{aligned}
%     x_{1t} &= r_{1t} & \\
%     x_{it} &= \sqrt{1 - \rho^2} r_{it} + \rho r_{1t} & \text{i = 2, 3,
%       ..., N}\\
%   \end{aligned}
% \end{equation}
% where $x_{it}$ are the return series whose covariance matrix is the
% subject of the current study and $r_{it}$ are independent
% auto-correlated GARCH(1,1) processes as specified by equation
% \ref{eq:garch_spec}.

% The covariance matrix C of the returns $\{x_{it}\}_{t=1}^T$ can be trivially
% computed 
% \begin{eqnarray}
%   C_{ij} &=& {1 \over T^{2/\alpha}} \sum_{t=1}^T x_{it} x_{jt}
%   \nonumber \\
%   C &\xrightarrow{T \to \infty}&
%   \begin{pmatrix}
%     \sigma_1^2 & \rho \sigma_1^2 & \cdots & \rho \sigma_1^2 \\
%     \rho \sigma_1^2 & (1 - \rho^2) \sigma_2^2 + \rho^2 \sigma_1^2 & \cdots &
%     \rho^2 \sigma_1^2 \\
%     \vdots & \vdots & \ddots & \vdots \\
%     \rho \sigma_1^2 & \rho^2 \sigma_1^2 & \cdots & (1 - \rho^2)
%     \sigma_N^2 + \rho^2 \sigma_1^2 
%   \end{pmatrix} \label{eq:garch_cov}
% \end{eqnarray}
% where $\sigma_i$ is the standard deviation of the GARCH(1,1) series
% $\{r_{it}\}_{t=1}^T$:
% \begin{eqnarray*}
%   \sigma_i = {1 \over T^{2/\alpha}} \sum_{t=1}^T r_{it}^2
% \end{eqnarray*}

% We would like to find out, when the covariance matrix of the $x$
% series is estimated, how the auto-correlation in the $r$ series affects
% the result --- specifically, how the auto-correlation strength $\phi$
% (c.f. equation \ref{eq:garch_spec}) affects the distribution of
% $C_{ij}$ when T is not very large.

% Figure \ref{fig:garch_correlated_Cij} shows the distribution of
% $C_{1j} / C_{11}$ ($j \geq 2$) and $C_{i1} / C_{11}$ ($i \geq
% 2$). These rescaled elements, as seen from equation
% \ref{eq:garch_cov}, have mean $\rho$. This is exactly what one sees in
% figure \ref{fig:garch_correlated_Cij}. It is also clear from the
% figure that $\phi$, the strength of auto-correlation, does not
% introduce a bias but does increase the variance of the
% estimation. This is the same conclusion as drawn in
% \S\ref{sec:garch_nonzero_phi}.
% \begin{figure}[htb!]
%   \centering
%     \includegraphics[scale=0.5, clip=true, trim=96 260 112
%     227]{../pics/garch_correlated_Cij.pdf}
%   \caption{\small \it Probability density function (PDF) of $C_{1j} /
%     C_{11}$ ($j \geq 2$) and $C_{i1} / C_{11}$ ($i \geq 2$). The
%     covariance matrix is constructed from 50 return series
%     $\{x_{it}\}$, $i = 1, 2, ..., 50$, and $t=1, 2, ..., 600$, as
%     specified by equation \ref{eq:garch_correlated_returns}. $\rho =
%     0.5$. 4000 instances of the covariance matrix are generated for
%     each value of $\phi$.}
%   \label{fig:garch_correlated_Cij}
% \end{figure}
% One can also find support in the data that the aforementioned
% distribution has power-law tails. This is shown in figure
% \ref{fig:garch_mix_Cij_tail} where the power of the left
% and the right tail is found to be 3.74 and 3.40, respectively.
% \begin{figure}[htb!]
%   \centering
%     \includegraphics[scale=0.5, clip=true, trim=105 229 107
%     120]{../pics/garch_mix_Cij_tail.pdf}
%   \caption{\small \it Horizontal axis (x) : the elements $C_{1j} /
%     C_{11}$ ($j \geq 2$) and $C_{i1} / C_{11}$ ($i \geq 2$) shifted to
%     center around 0. The section with $0.15 < x < 0.35$ is
%     plotted. Vertical axis: See the legend. }
%   \label{fig:garch_mix_Cij_tail}
% \end{figure}

% It is also interesting to see how the eigenvalues and eigenvectors
% behave given the non-trivial cross-correlation and the
% auto-correlation in the returns. However, the cross-correlation matrix
% of equation \ref{eq:garch_cov} has rather complicated eigenvalues and
% eigenvectors. It has been useful in studying the non-diagonal matrix
% elements but becomes clumsy with regard to eigenvalues and
% eigenvectors. Therefore we look at the following return series
% instead:
% \begin{eqnarray}
%   \begin{pmatrix}
%     x_{1t} \\
%     x_{2t} \\
%   \end{pmatrix} &=&
%   \begin{pmatrix}
%     \cos\theta/\sigma_1 & -\sqrt{2} \sin\theta/\sigma_2  \\
%      \sin\theta/\sigma_1 & \sqrt{2} \cos\theta/\sigma_2 \\
%   \end{pmatrix}
%   \begin{pmatrix}
%     r_{1t} \\
%     r_{2t} \\
%   \end{pmatrix} \label{eq:x-r_transformation}
% \end{eqnarray}
% where $r_{it}$ are independent, auto-correlated GARCH(1,1) processes
% as specified by equation \ref{eq:garch_spec}. Each of them has
% standard deviation $\sigma_i$ ($i = 1, 2$). It is easy to
% find the covariance matrix C of the series $\{x_{it}\}_{t=1}^T$ ($i =
% 1,2$):
% \begin{eqnarray*}
%   C &=& 
%   \begin{pmatrix}
%     1 + \sin\theta^2 & -\sin\theta\cos\theta \\
%     -\sin\theta\cos\theta & 1 + \cos\theta^2 \\
%   \end{pmatrix}
% \end{eqnarray*}
% The eigenvalues and eigenvectors of $C$ are
% \begin{eqnarray*}
%   CX &=& X
%   \begin{pmatrix}
%     1 & \\
%     & 2 \\
%   \end{pmatrix} \\
%   X &=& 
%   \begin{pmatrix}
%     \cos\theta & -\sin\theta \\
%     \sin\theta & \cos\theta \\
%   \end{pmatrix}
% \end{eqnarray*}

% Figure \ref{fig:garch_rotated_Cii} shows the probability plot of the
% diagonal elements of $C$. Evidently, their distribution is very close
% to a Gaussian, although the Anderson-Darling test rejects Gaussianity
% at the 5\% significance level. Through equation
% \ref{eq:x-r_transformation} the differences in variance of the
% realizations of $\{r_{1t}\}_{t=1}^T$ and $\{r_{2t}\}_{t=1}^T$ are not
% carried over to $\{x_{1t}\}_{t=1}^T$ and $\{x_{2t}\}_{t=1}^T$, and
% therefore the differences in variance of the latter series are simply
% the results of numerical computation and hence expected to be small
% and Gaussian distributed. One can also see from figure
% \ref{fig:garch_rotated_Cii} that the different auto-correlations
% (different $\phi$ values) in the $\{r_{1t}\}_{t=1}^T$ and
% $\{r_{2t}\}_{t=1}^T$ series do not change the Gaussianity. The
% variances of these Gaussian distributions, however, are indeed
% increased by increased auto-correlation.
% \begin{figure}[htb!]
%   \centering
%     \includegraphics[scale=0.4, clip=true, trim=24 152 44
%     18]{../pics/garch_rotated_Cii.pdf}
%   \caption{\small \it Probability plot of $C_{11}$ corresponding to
%     different auto-correlation strength ($\phi$) in the
%     $r$-series. When $\phi = 0, 0.2, 0.4, 0.6$, the variance is $3.380
%     \times 10^{-4}$, $3.639 \times 10^{-4}$, $4.512 \times 10^{-4}$,
%     $7.506 \times 10^{-4}$, respectively.} 
%   \label{fig:garch_rotated_Cii}
% \end{figure}

% On the other hand, the non-diagonal elements of $C$ have a very
% different distribution from the diagonal ones, as shown in figure
% \ref{fig:garch_rotated_Cij}.
% \begin{figure}[htb!]
%   \centering
%     \includegraphics[scale=0.5, clip=true, trim=24 303 30
%     274]{../pics/garch_rotated_Cij.pdf}
%   \caption{\small \it Probability density function (PDF) of the
%     non-diagonal elements of C. Corresponding to $\phi = 0, 0.2, 0.4,
%     0.6, 0.8$, the variance of the distribution is
%     $6.77\times10^{-8}$, $1.529\times10^{-7}$, $3.896\times10^{-7}$,
%     $1.124\times10^{-6}$, $5.445\times10^{-6}$.}
%   \label{fig:garch_rotated_Cij}
% \end{figure}
% One can see that the distribution of the non-diagonal elements is
% left-skewed, regardless of the auto-correlation $\phi$. This is to be
% compared with the distributions shown in figure
% \ref{fig:garch_correlated_Cij}, where the skewness is small and
% changes sign for different $\phi$ values, whereas in figure
% \ref{fig:garch_rotated_Cij}, the skewness is consistently negative and
% takes comparable values as $\phi$ changes from 0 to 0.9. We list these
% skewness values in table \ref{tab:garch_rotated_var_skw}.
% \begin{table}[htb!]
%   \centering
%   \begin{tabular}{|c|c|c|c|c|c|}
%     \hline
%     $\phi$ & 0 & 0.1 & 0.2 & 0.3 & 0.4 \\
%     \hline 
%     variance& 6.770e-08 & 1.008e-07 & 1.529e-07 & 2.339e-07 & 3.896e-07 \\
%     \hline
%     skewness & -1.62 & -1.65 & -1.96 & -1.86 & -1.67 \\
%     \hline
%   \end{tabular}
%   \begin{tabular}{|c|c|c|c|c|c|}
%     \hline
%     $\phi$ & 0.5 & 0.6 & 0.7 & 0.8 & 0.9 \\
%     \hline
%     variance& 6.022e-07 & 1.124e-06 & 2.156e-06 & 5.445e-06 & 2.389e-05 \\
%     \hline
%     skewness & -1.62 & -1.65 & -1.96 & -1.86 & -1.67 \\
%     \hline
%   \end{tabular}
%   \caption{\small \it Variance and skewness of $C_{12}$ and $C_{21}$}
%   \label{tab:garch_rotated_var_skw}
% \end{table}
% From table \ref{tab:garch_rotated_var_skw} and figure
% \ref{fig:garch_rotated_Cij} one can also see that the
% variance steadily increases as $\phi$ increases --- a similar
% observation that has been seen in figure
% \ref{fig:garch_correlated_Cij}.

% \begin{figure}[htb!]
%   \centering
%   \subfigure[]{
%     \includegraphics[scale=0.35, clip=true, trim=107 278 104
%     240]{../pics/garch_rotated_eig1_dist.pdf}
%     \label{fig:garch_rotated_eig1_dist}
%   }
%   \subfigure[]{
%     \includegraphics[scale=0.35, clip=true, trim=103 270 99 
%     227]{../pics/garch_rotated_eig2_dist.pdf}
%     \label{fig:garch_rotated_eig2_dist}
%   }
%   \caption{\small \it Eigenvalue distribution}
%   \label{fig:garch_rotated_eig_dist}
% \end{figure}

% \begin{table}[htb!]
%   \centering
%   \begin{tabular}{|c|c|c|c|c|c|}
%     \hline
%     $\phi$ & 0 & 0.2 & 0.4 & 0.6 & 0.8 \\
%     \hline
%     variance& 3.338e-07 & 4.113e-07 & 8.415e-07 & 2.351e-06 & 9.580e-06 \\
%     \hline
%     skewness & -3.92 & -1.65 & -1.22 & -1.38 & -0.90 \\
%     \hline
%   \end{tabular}
%   \caption{variance and skewness of the distribution of $\lambda_1$}
%   \label{tab:garch_rotated_eig1_var_skw}
% \end{table}

% \begin{table}[htb!]
%   \centering
%   \begin{tabular}{|c|c|c|c|c|c|}
%     \hline
%     $\phi$ & 0 & 0.2 & 0.4 & 0.6 & 0.8 \\
%     \hline
%     variance& 4.947e-07 & 7.852e-07 & 1.767e-06 & 4.813e-06 & 2.258e-05 \\
%     \hline
%     skewness & 2.93 & 2.14 & 2.41 & 2.42 & 2.49 \\
%     \hline
%   \end{tabular}
%   \caption{variance and skewness of the distribution of $\lambda_2$}
%   \label{tab:garch_rotated_eig2_var_skw}
% \end{table}


%%%%%%%%%%%%%%%%%%%%%%%%%%%%%%%%%%%%%%%%%%%%%%%%%%%%%%%%%%%%%%%%%
% To find out more about this series of deformation, we investigate 
% how the characteristic function changes in response to increasing
% auto-correlation (increasing $\phi$). When $\phi = 0$ and T is very
% large, like $8 \times 10^4$ in our simulation, we have found the
% eigenvalue distribution converges to an $\alpha$-stable distribution
% whose characteristic function is known in analytic form.

% The empirical characteristic function $\varphi(k)$ is computed from
% the simulated eigenvalues according to:
% \begin{eqnarray*}
%   \varphi(k) &=& \E(e^{ik\lambda}) \\
%   &=& {1 \over M}\sum_{n=1}^M e^{ik\lambda_n}
% \end{eqnarray*}
% where $M$, the total number of eigenvalues in the sample, is $1\times
% 10^5$ in our simulation, as mentioned earlier.

% \begin{eqnarray}\label{eq:PDF_fourier_expansion}
%   p(\lambda) &=& \sum_{n=-\infty}^\infty c_n \exp\left(
%     -i {2\pi n \lambda \over L}
%   \right) \\ \nonumber
%   &\approx& \sum_{n=-M}^M c_n \exp\left(
%     -i {2\pi n \lambda \over L}
%   \right) \\
%   L &=& \lambda_\M - \lambda_\m \nonumber
% \end{eqnarray}
% where $M$ is a large integer at which the series expansion is
% truncated. For the simulations described in this section, we choose M
% = 500.

% The Fourier coefficients $c_n$ used in the above formula are computed
% as
% \begin{eqnarray*}
%   c_n &=& {1 \over L} \int_{\lambda_\m}^{\lambda_\M} \exp\left(
%     i {2\pi n \lambda \over L}
%   \right) p(\lambda) d\lambda \\
%   &=& {1 \over L} \E \left[
%     \exp\left(
%       i {2\pi n \lambda \over L}
%     \right)
%   \right] \\
%   &=& {1 \over LS} \sum_{k=1}^S \exp\left(
%     i {2\pi n \lambda_k \over L}
%   \right)
% \end{eqnarray*}
% where $S = 1 \times 10^5$ is the aforementioned sample size of the
% simulated eigenvalue. Figure \ref{fig:GarchEigPDFFourierCoef}
% shows the Fourier coefficients computed according to the above
% formula.
% \begin{figure}[htb!]
%   \centering
%   \includegraphics[scale=0.56, clip=true, trim=10 203 6
%   89]{../pics/GarchEigPDFFourierCoef.pdf}
%   \caption{\small \it Fourier Coefficients of the empirical PDF of
%     the Eigenvalue distribution. Blue/Red: Real/imaginary parts of the
%     Fourier coefficients. The $\phi$ values 0, 0.2, 0.5, 0.8, 0.95, 0.96,
%     0.97, 0.98 correspond to correlation time $\tau$ = 0, 0.43, 1.00,
%     3.11, 13.51, 16.98, 22.76, 34.31.}
%   \label{fig:GarchEigPDFFourierCoef}
% \end{figure}

% Before we proceed to investigate the properties of the empirical PDF
% using the Fourier series expansion, we need to ensure the expansion
% does provide a sufficiently accurate approximation to the empirical
% PDF. However, to directly compare the Fourier series expansion
% with the empirical PDF, the empirical PDF will have to be evaluated
% --- the accuracy of the result will then largely depend on the number
% of bins to which one chooses to sort the eigenvalues so as to produce
% the empirical PDF --- a choice that is rather arbitrary and
% subjective. To avoid such a procedure, we can compare the empirical
% CDF with its Fourier expansion instead. Formally we integrate equation
% \ref{eq:PDF_fourier_expansion} to obtain
% \begin{equation}
%   \label{eq:CDF_fourier_expansion}
%   \begin{aligned}
%     F(\lambda) &= \int_{\lambda_\m}^{\lambda} p(x) dx \\
%     &= \sum_{n \in
%       \mathbb{Z}\backslash\{0\}} c_n i {L \over 2\pi n} \left[
%       \exp\left(
%         -i {2\pi n \lambda \over L}
%       \right) - 
%       \exp\left(
%         -i {2\pi n \lambda_\m \over L}
%       \right) \right] \\
%     & + (\lambda - \lambda_\m) c_0
%   \end{aligned}
% \end{equation}
% The CDF on the left-hand-side of equation
% \ref{eq:CDF_fourier_expansion} can be evaluated directly using the
% simulated eigenvalues; meanwhile, the right-hand-side can be evaluated
% for any $\lambda$ using $c_n$.

% Figure \ref{fig:CDF_fourier_expansion} shows how the two sides of
% equation \ref{eq:CDF_fourier_expansion} compare to each other.
% \begin{figure}[htb!]
%   \centering
%   \includegraphics[scale=0.6, clip=true, trim=20 240 15
%   200]{../pics/garch_cdf_fourier_expansion.pdf}
%   \caption{\small \it The empirical cummulative distribution
%     function (CDF) and its Fourier series expansion. Blue: empirical
%     CDF. Red: Fourier series expansion. The plots are on log-log
%     scale. The $\phi$ values 0, 0.2, 0.5, 0.8, 0.95, 0.96,
%     0.97, 0.98 correspond to correlation time $\tau$ = 0, 0.43, 1.00,
%     3.11, 13.51, 16.98, 22.76, 34.31.}
%   \label{fig:CDF_fourier_expansion}
% \end{figure}
% We see that the two curves match fairly well --- they differ from each
% other only at their lower ends, which are statistically insignificant
% due to the very small portion of data points in that region.

% Now that one is assured that the Fourier series expansion is
% sufficiently accurate, it makes sense to study the properties of the
% Fourier coefficients $c_n$. A few features immediately present
% themselves in figure \ref{fig:GarchEigPDFFourierCoef}:
% \begin{itemize}
% \item $\re c_n = \re c_{-n}$, where $\re$ denotes the real part of a
%   complex number;
% \item $\re c_n$ resembles a dumped cosine wave;
% \item $\re c_n$ and $\im c_n$ have the same frequency with a constant
%   phase difference. Here $\im$ denotes the imaginary part of $c_n$.
% \end{itemize}
% The last conjecture about the frequency and the phase difference can
% be verified by looking at the phase plot of $c_n$, which is shown in
% figure \ref{fig:garch_Cn_phase}.
% \begin{figure}[htb!]
%   \centering
%   \includegraphics[scale=0.6, clip=true, trim=21 248 14
%   198]{../pics/garch_Cn_phase.pdf}
%   \caption{\small \it Phases of the Fourier coefficients ($c_n$) of
%     the empirical eigenvalue PDF. Horizonal axis: $n$; Vertical Axis:
%     phases of $c_n$. The $\phi$ values 0, 0.2, 0.5, 0.8, 0.95, 0.96,
%     0.97, 0.98 correspond to correlation time $\tau$ = 0, 0.43, 1.00,
%     3.11, 13.51, 16.98, 22.76, 34.31.}
%   \label{fig:garch_Cn_phase}
% \end{figure}
% From figure \ref{fig:garch_Cn_phase} one can see that the phase of
% $c_n$ ($\ph c_n$) is approximately linear when $\phi$ is
% small --- however, as $\phi$ becomes larger and larger, the
% non-linearity of $\ph c_n$ becomes stronger and stronger.

% In the relatively simple situation of small $\phi$, i.e. weak
% auto-correlation, it lends some insight to consider the following
% approximate formula:
% \begin{eqnarray}
%   \re c_n &=& e^{
%     -u_r(\phi) - v_r(\phi) |n|
%   } \cos[\omega(\phi) n] \label{eq:garch_fourier_coef_real}\\
%   \im c_n &=& e^{
%     -u_i(\phi) - v_i(\phi) |n|
%   } \sin[\omega(\phi) n] \label{eq:garch_fourier_coef_imag}
% \end{eqnarray}
% Apparently, $\omega(\phi)$ can be directly read out from figure
% \ref{fig:garch_Cn_phase}. The results are listed in table
% \ref{tab:garch_phase_velocity}.
% \begin{table}[htb!]
%   \footnotesize
%   \centering
%   \begin{tabular}{|c|c|c|c|c|c|c|c|c|}
%     \hline
%     $\phi$ & 0 & 0.2 & 0.5 & 0.8 & 0.95 & 0.96 & 0.97 & 0.98 \\
%     \hline
%     $\tau$ & 0 & 0.4307 & 1.0000 & 3.1063 &  13.5134 & 16.9797 &
%     22.7566 & 34.3096 \\
%     \hline
%     $\omega(\phi)$ & 0.0442 & 0.0538 & 0.0348 & 0.0414 & 0.0180 &
%     0.0320 & 0.0154 & 0.0242 \\
%     \hline
%   \end{tabular}
%   \caption{\small \it Phase velocity $\omega$ of the Fourier
%     coefficients of the eigenvalue PDF.}
%   \label{tab:garch_phase_velocity}
% \end{table}
% Using these values as initial estimate, the parameters in equation
% \ref{eq:garch_fourier_coef_real} and \ref{eq:garch_fourier_coef_imag},
% namely $u_r(\phi)$, $v_r(\phi)$ and $v_i(\phi)$ can be estimated by
% least-square-error methods. We use the Matlab function ``lsqcurvefit''
% for this purpose and list the estimated parameter values in table
% \ref{tab:garch_fourier_coef_real_param}.
% \begin{table}[htb!]
%   \footnotesize
%   \centering
%   \begin{tabular}{|c|c|c|c|c|c|c|c|c|c|}
%     \hline
%     $\phi$ & 0 & 0.1 & 0.2 & 0.3 & 0.4 & 0.5 & 0.6 & 0.7 & 0.8\\
%     \hline
%     $\tau$ & 0 & 0.3010 & 0.4307 & 0.5757 & 0.7565 & 1.0000 & 1.3569 &
%     1.9434 & 3.1063 \\
%     \hline
%     $u_r(\phi)$ & 0.0938 & 0.0936 & -0.0812 & 0.1649 & 0.2627 & 0.6285 &
%     0.5443 & 0.5720 & 1.1610 \\
%     \hline
%     $v_r(\phi)$ & 0.0044 & 0.0045 & 0.0057 & 0.0046 & 0.0045 & 0.0034 &
%     0.0046 & 0.0059 & 0.0048 \\
%     \hline
%     $\omega(\phi)$ & 0.0420 & 0.0424 & 0.0511 & 0.0428 & 0.0421 &
%     0.0333 & 0.0416 & 0.0497 & 0.0394 \\
%     \hline
%   \end{tabular}
%   % \begin{tabular}{|c|c|c|c|c|c|c|c|c|c|c|}
%   %   \hline
%   %   $\phi$ & 0 & 0.1 & 0.2 & 0.3 & 0.4 & 0.5 & 0.6 & 0.7 & 0.8 & 0.9 \\
%   %   \hline
%   %   $\tau$ & 0 & 0.3010 & 0.4307 & 0.5757 & 0.7565 & 1.0000 & 1.3569 &
%   %   1.9434 & 3.1063 & 6.5788 \\
%   %   \hline
%   %   $u_r(\phi)$ & 0.0938 & 0.0936 & -0.0812 & 0.1649 & 0.2627 & 0.6285 &
%   %   0.5443 & 0.5720 & 1.1610 & 1.2347 \\
%   %   \hline
%   %   $v_r(\phi)$ & 0.0044 & 0.0045 & 0.0057 & 0.0046 & 0.0045 & 0.0034 &
%   %   0.0046 & 0.0059 & 0.0048 & 0.0098 \\
%   %   \hline
%   %   $\omega(\phi)$ & 0.0420 & 0.0424 & 0.0511 & 0.0428 & 0.0421 &
%   %   0.0333 & 0.0416 & 0.0497 & 0.0394 & 0.0667 \\
%   %   \hline
%   % \end{tabular}
%   \caption{\small \it Least-Square-Error estimate of parameter values of $\re
%     c_n$}
%   \label{tab:garch_fourier_coef_real_param}
% \end{table}
% As is said earlier, the equation \ref{eq:garch_fourier_coef_real} and
% \ref{eq:garch_fourier_coef_imag} only apply in the cases of small
% $\phi$, so the estimation has only been done for $\phi$ values up to
% 0.8. To obtain better statistics about the relation between $\phi$ and
% the parameters, a few more values of $\phi$ have been added. The
% corresponding Fourier coefficients have been computed and validated
% against the empirical CDF. Figure \ref{fig:garch_fourier_coef_real}
% shows how the equation \ref{eq:garch_fourier_coef_real} fits to $\re
% c_n$ with the estimated parameter values.
% \begin{figure}[htb!]
%   \centering
%   \includegraphics[scale=0.56, clip=true, trim=10 197 11
%   147]{../pics/garch_fourier_coef_real.pdf}
%   \caption{\small \it A dumped cosine function fit to $\re
%     c_n$. Blue: empirical Fourier coefficients; Green: Dumped cosine
%     function.}
%   \label{fig:garch_fourier_coef_real}
% \end{figure}
% We see that the two curves in figure \ref{fig:garch_fourier_coef_real}
% match fairly well. Obvious deviations are found only in regions of
% large n, where slight inaccuracy in $\omega(\phi)$ is magnified. This
% confirms the validity of formula \ref{eq:garch_fourier_coef_real}.

% % To find out the relation between $\phi$ and the parameters $u_r(\phi)$,
% % $v_r(\phi)$ and $\omega(\phi)$, we plot these parameters' values
% % against $\phi$ in figure \ref{fig:Cn_real_param}.
% % \begin{figure}[htb!]
% %   \centering
% %   \includegraphics[scale=0.56, clip=true, trim=6 311 0
% %   264]{../pics/Cn_real_param.pdf}
% %   \caption{\small \it Linear fit of the parameters in $\re
% %     c_n$. Blue circles: parameter values normalized by the sample
% %     mean; Green line: straight line fitted to the data points.}
% %   \label{fig:Cn_real_param}
% % \end{figure}
% % Instead of directly plotting the parameters' values, figure
% % \ref{fig:Cn_real_param} plots the ratio of these values over their
% % respective sample mean, so that one can see how much each
% % individual data point (blue circles) deviates from the sample mean as
% % well as from the fitted line. The title of each plot in figure
% % \ref{fig:Cn_real_param} is the equation of the fitted line.

% % From figure \ref{fig:Cn_real_param} we see that $u_r(\phi)$ is
% % approximately linear in $\phi$ --- the coefficient of the linear term
% % in its equation is an order of magnitude larger than the constant
% % term. In contrast, $v_r(\phi)$ and $\omega(\phi)$ are approximately
% % constant --- the coefficient of the linear term in their equation is
% % an order of magnitude smaller than the constant term. So, also
% % considering $0 < \phi < 1$ and that we have a rather small sample
% % size, we estimate $v_r(\phi)$ and $\omega(\phi)$ to be constant. In
% % summary
% % \begin{eqnarray*}
% %   u_r(\phi) &=& 1.2364 \phi - 0.1124 \\
% %   v_r &=& 4.7148 \times 10^{-3} \\
% %   \omega &=& 3.3476 \times 10^{-2} \\
% % \end{eqnarray*}
% % Now that $v_r(\phi)$ and $\omega(\phi)$ are constant, we shall write
% % them only as $v_r$ and $\omega$ to avoid confusion.

% Turning our attention to the imaginary part of $c_n$($\im c_n$) and
% applying the same methods, we find the corresponding parameter values
% for $\im c_n$ (table \ref{tab:garch_fourier_coef_imag_param}). Figure
% \ref{fig:garch_fourier_coef_imag} shows how these values together with
% equation \ref{eq:garch_fourier_coef_imag} fit to $\im c_n$. Comparing
% table \ref{tab:garch_fourier_coef_real_param} and
% \ref{tab:garch_fourier_coef_imag_param}, we see that $v_r$ and $v_i$
% are essentially equal, while $u_r(\phi)$ is consistently larger than
% $u_i(\phi)$.

% \begin{table}[htb!]
%   \footnotesize
%   \centering
%   \begin{tabular}{|c|c|c|c|c|c|c|c|c|c|}
%     \hline
%     $\phi$ & 0 & 0.1 & 0.2 & 0.3 & 0.4 & 0.5 & 0.6 & 0.7 & 0.8\\
%     \hline
%     $\tau$ & 0 & 0.3010 & 0.4307 & 0.5757 & 0.7565 & 1.0000 & 1.3569 &
%     1.9434 & 3.1063 \\
%     \hline
%     $u_i(\phi)$ & 0.0555 & 0.0565 & -0.1105 & 0.1291 & 0.2236 &
%     0.6007 & 0.5031 & 0.5341 & 1.1273 \\
%     \hline
%     $v_i(\phi)$ & 0.0047 & 0.0047 & 0.0059 & 0.0048 & 0.0048 &
%     0.0035 & 0.0049 & 0.0062 & 0.0049 \\
%     \hline
%     $\omega(\phi)$ & 0.0420 & 0.0424 & 0.0511 & 0.0428 & 0.0421 &
%     0.0333 & 0.0416 & 0.0497 & 0.0394 \\
%     \hline
%   \end{tabular}
%   \caption{\small \it Least-Square-Error estimate of parameter values of $\im c_n$}
%   \label{tab:garch_fourier_coef_imag_param}
% \end{table}

% \begin{figure}[htb!]
%   \centering
%   \includegraphics[scale=0.56, clip=true, trim=18 221 19
%   171]{../pics/garch_fourier_coef_imag.pdf}
%   \caption{\small \it A dumped sine function fit to the imaginary
%     part of $c_n$ ($\im c_n$). Red: empirical Fourier coefficients;
%     Green: Dumped sine function.} 
%   \label{fig:garch_fourier_coef_imag}
% \end{figure}

%  As is done for $\re c_n$, we can also find an approximate
% linear expression for $u_i(\phi)$ by fitting a line to the data. The
% result is
% \begin{eqnarray*}
%   u_i(\phi) &=& 1.2356 \phi - 0.1480
% \end{eqnarray*}
% So, combining the results for $\re c_n$ and $\im c_n$, we may write
% \begin{eqnarray*}
%   c_n &=& e^{-v|n|}\left[
%     e^{-a_r\phi - b_r}\cos(\omega n) + ie^{-a_i\phi - b_i}\sin(\omega n)
%   \right] \\
% \end{eqnarray*}
% where
% \begin{equation}
%   \label{eq:garch_Cn_parameters}
%   \begin{aligned}
%     v &= 4.7148 \times 10^{-3} \\
%     a_r &= 1.2364 \\
%     b_r &= -0.1124 \\
%     a_i &= 1.2356 \\
%     b_i &= -0.1480 \\
%     \omega &= 3.3476 \times 10^{-2}
%   \end{aligned}
% \end{equation}
% It should be emphasized that the above parameters are only constants
% with respect to the GARCH parameters $\alpha_1$, $\beta_1$ and hence
% the resulting tail exponent $\alpha$. If the GARCH parameters are
% changed, the parameters in \ref{eq:garch_Cn_parameters} are expected
% to follow.


%% TODO:
% 1. find the power-exponent of the tail of the SLV (stochastic
% log-volatility) model. Normalize the returns using the sample
% standard deviation when computing the cross-correlation matrix.
% 
% 2. Hows does the power-law return distribution change the eigenvalue
% distribution in comparison to the Marcenko-Pastur law?
%
% 3. How do the maximum and the minimum eigenvalue change
% corresponding to different values of autocorrelation?
%
% 4. Use different combinations of q = N/T and investigate how the
% eigenvalue distribution changes in the case of the SLV model.

\chapter{Summary of the Results}
The results of this thesis project are the following:
\begin{enumerate}
\item ARIMA log-volatility model
  \begin{enumerate}
  \item ARIMA log-volatility returns' models are justified and tested
    against Nordea Bank 15-minute returns and Volvo B 30-minute returns
    during the period 2013/10/10 - 2014/04/04. Historical volatility's
    are estimated using ``realized volatilities'', denoted
    $\hat{\sigma}_t$ here:
    \begin{eqnarray*}
      \hat{\sigma}_t &=& \sqrt{
        \sum_{i = 1}^n \left[\ln {p_{t - \Delta t + i \Delta t / n} \over
            p_{t - \Delta t + (i-1) \Delta t / n}} \right]^2
      }
    \end{eqnarray*}
    where $p_{t'}$ is the price of the asset at time $t'$, and $n$ is
    the number of sub intervals into which the time interval $[t -
    \Delta t, t]$ is divided. A discussion of the choice of $n$ is found
    in \S\ref{sec:SLV_model}.

    Using the first 80\% of the data for estimating the model and the
    last 20\% for comparing with the model forecast, we find the ARIMA
    models out-perform their GARCH competitors in most cases.

  \item The unconditional distribution functions of ARIMA
    log-volatility models are calculated (equation
    \ref{eq:UncondPDFAsymmetric} and \ref{eq:UncondPDFAsymmetric1}):
    The unconditional distribution of  the returns $r_t$ is the same as
    $\mu + e^{\bar{v} + \sigma a} b$, where $(a, b) \sim N(0, \Sigma)$
    \begin{eqnarray*}
      \Sigma &=&
      \begin{pmatrix}
        1 & \psi \\
        \psi & 1
      \end{pmatrix}
    \end{eqnarray*}
    Then the PDF of the unconditional distribution $f_r(x)$ can be expressed as
    \begin{eqnarray*}
      f_{r'}(x) &=& {1 \over 2\pi \sqrt{1 - \psi^2}} \times \\
      && \int_{-\infty}^\infty da \exp
      \left[- {
          a^2 + 2\sigma(1 - \psi^2)a - 2a\psi e^{-a\sigma} x + e^{-2a\sigma} x^2
          \over
          2(1 - \psi^2)
        }
      \right] \\
      f_r(x) &=&  \td{r'}{r} f_{r'}[e^{-\bar{v}} (x - \mu)] \\
      &=& e^{-\bar{v}} f_{r'}[e^{-\bar{v}} (x-\mu)] \\
    \end{eqnarray*}
    where $r' = e^{\sigma a}b$. The moment generating function of the
    unconditional distribution is found as an integral, and all the
    moments can be evaluated in closed form. See equations
    \ref{eq:AsymmetricMoments}.
    
  \end{enumerate}
\item Correlation matrix of Gaussian returns. $N$ series of
  auto-correlated, standard Gaussian distributed returns are simulated
  to form $N\times N$ correlation matrix $C$:
  \begin{eqnarray*}
    r_{it} &=& \phi r_{i,t-1} + a_t \\
    C_{ij} &=& {1 \over T} \sum_{t=1}^T r_{it} r_{jt}
  \end{eqnarray*}
  where $i = 1, 2, ..., N$ and $t = 1,2, ..., T$. $a_t \sim N(0,
  1)$.
  \begin{enumerate}
  \item Distribution of the elements of C. The asymptotic distribution
    of the matrix elements $C_{ii}$ and $C_{ij}$ ($i \neq j$) are 
    Gaussian whose mean and variance are functions of $\phi$:
    \begin{eqnarray*}
      \E(C_{ij}) &=& {\sigma^2 \over \sqrt{2\pi} (1 - \phi^2)(1 -
        \rho^2)^{1/4}} \left[ P^{-3/2}_{-1/2}(-\rho) -
        P^{-3/2}_{-1/2}(\rho)
      \right] \\
      \var(C_{ij}) &=& {2 \sigma^6 \over T (1 - \phi^2)^2} \left[
        {\phi^2 \over 1 - \phi^2} -
        {\phi^2 (1 - \phi^{2T}) \over
          T(1 - \phi^2)}
      \right] + {\sigma^4 (1 - \rho^2)^2 v^2(\rho) \over
        T (1 - \phi^2)^2} \\
      &\approx& {2 \sigma^6 \phi^2 \over T (1 - \phi^2)^3}
      + {\sigma^4 (1 - \rho^2)^2 v^2(\rho) \over
        T (1 - \phi^2)^2}
    \end{eqnarray*}
    where $\rho = \text{corr}(a_{it}, a_{jt})$, $i \neq
    j$. $P^\mu_\nu(\cdot)$ is Ferrers function of the first kind (See
    equation \ref{eq:Ferrers_1st}) and
    \begin{eqnarray*}
      v^2(\rho) &=&
      {4 \over \sqrt{2\pi} (1 - \rho^2)^{7/4}} \left[
        P^{-5/2}_{-1/2}(\rho) + P^{-5/2}_{-1/2}(-\rho)
      \right] \\
      && - {1 \over 2\pi (1 - \rho^2)^{5/2}} \left[
        P^{-3/2}_{-1/2}(-\rho) - P^{-3/2}_{-1/2}(\rho)
      \right]^2
    \end{eqnarray*}
    In comparison,
    \begin{eqnarray*}
      \E(C_{ii}) & \approx & {\sigma^2 \over 1 - \phi_1^2} \left[
        1 - {\phi_1^2 \over T}
      \right] \\
      \var(C_{ii}) &\approx& {2 \sigma^4 \over T (1 - \phi_1^4)} +
      {\sigma^6 \over T (1 -\phi_1^2)^2}
    \end{eqnarray*}

  \item Distribution of the maximum eigenvalue of C. It is shown that
    the maximum eigenvalue has approximately a gamma distribution even
    when auto-correlations are present among the returns. The mean of
    the gamma distribution is approximately quadratic in $\tan{\pi\phi
      \over 2}$.
  \end{enumerate}

\item Covariance matrix C of GARCH(1,1) processes. In the limit of
  very long return series, the covariance matrix of properly
  specified GARCH(1,1) processes (see chapter
  \ref{chp:CrossCorrelationFat}) is 
  shown to be diagonal. The distribution of its eigenvalues thus
  coincides with that of its diagonal elements. Their distribution
  converges to an $\alpha$-stable distribution. If the returns $r_{it}$
  have tail behavior $P(|r_{it}| > x) \sim {c \over x^\alpha}$, the
  asymptotic stable distribution has L\'evy index
  $\alpha/2$. Moreover, the mean of this stable distribution scales
  with the length of the return series (T) as $T^{1 - 2/\alpha}$,
  while the width is independent of T.

  In contrast, the distribution of the non-diagonal elements of the
  covariance matrix converges to a symmetric, zero-centered, stable
  distribution with a different L\'evy index than $\alpha$. Denote it
  $\alpha'$, then the variance of this stable distribution scales with
  T as $T^{1/\alpha' - 2/\alpha}$.

  When the return series only have a modest length, the situation is
  studied in terms of the localization of the eigenvectors. For a
  given T, the largest eigenvalues correspond to the most localized
  eigenvectors, and the extent of localization decreases as the
  eigenvalue decreases. As T increases, the extent of localization
  increases for all eigenvectors.

  Auto-correlations among the returns, on the other hand, reduces the
  extent of localization, and also the certainty of localization with
  respect to the eigenvalues. In other words, if the return series
  have considerable auto-correlations, the eigenvectors corresponding
  to the largest eigenvalues will have a rather fractured composition.
\end{enumerate}

\chapter{Appendix}
\section{PDF of the covariance matrix of auto-correlated Gaussian
  Returns} \label{app:pdf_gaussian1}
When autocorrelations exist among the columns of the Gaussian returns
matrix $\mtx R$, the distribution of the covariance matrix $\mtx C =
\mtx{R} \mtx{R'}$ is no longer Wishart but is nonethelss closely
related to it. In the following we consider the situation where
$r_{i,t}$ can be represented as a vector auto-regressive
process. Specifically, suppose
\begin{eqnarray*}
  \vec{r}_t &=& \sum_{k=1}^p \phi_k \vec{r}_{t-k} + \vec{a}_t \\
  \vec{a}_t &=& \vec{r}_t - \sum_{k=1}^p \phi_k \vec{r}_{t-k}
\end{eqnarray*}
where $\vec{a}_t = (a_{1,t}, a_{2,t}, \cdots, a_{N,t})' \sim N(0,
\Sigma)$ and comprise the columes of $\mtx A$; $\vec{r}_t$ are the
columes of $\mtx R$. Here $N(0, \Sigma)$ denotes the multivarate
normal distribution with covariance matrix $\Sigma$. The last equation
can be written in matrix form
\[
A = R M
\]
For example, in the case of an AR(1) process
\begin{equation*}
  \begin{pmatrix}
    a_{1,1} & a_{1,2} & \cdots & a_{1,T} \\
    \vdots & \ddots & \vdots \\
    a_{N,1} & a_{N,2} & \cdots & a_{N,T} \\
  \end{pmatrix} =
  \begin{pmatrix}
    r_{1,1} & r_{1,2} & \cdots & r_{1,T} \\
    \vdots & \ddots & \vdots \\
    r_{N,1} & r_{N,2} & \cdots & r_{N,T} \\
  \end{pmatrix}
  \begin{pmatrix}
    1 & -\phi_1 &   &   & \\
      & 1 & -\phi_1 &   & \\
      &   & \ddots  & \ddots &   \\
      &   &   & 1 & -\phi_1 \\
      &   &   &   & 1 \\
  \end{pmatrix}
\end{equation*}
Let $QR = RM$, then $Q = RMR^{-1}$. Since the set of $\{r_{ij}\}$ for
which $\text{det } R = 0$ has probability zero, a matrix $Q$
satisfying the above equation almost surely exists. Thus $A = RM =
QR$ and $AA' = QRR'Q'$ Since the columns of A are not auto-correlated,
$AA' \sim W(\Sigma, T)$. Then $RR' \sim W(Q^{-1} \Sigma Q'^{-1}, T)$
follows from \cite{Anderson2003}, \S7.3.3.

Now we observe that $\Sigma$ enters $f(RR')$ through $\tr (\Sigma^{-1}
RR')$ and $\det \Sigma$. $RR'$ enters through $\tr (\Sigma^{-1}
RR')$ and $\det (RR')$. Clearly
\begin{eqnarray*}
  \det (Q^{-1}\Sigma Q'^{-1}) &=& \det \Sigma \over (\det Q)^2 \\
  &=& \det \Sigma \over (\det M)^2 \\
  &=& \det \Sigma
\end{eqnarray*}
As to $\tr (\Sigma^{-1} C)$, we have
\begin{eqnarray*}
  && \tr\left[(Q^{-1}\Sigma Q'^{-1})^{-1} RR'\right] \\
  &=& \tr\left[Q'\Sigma^{-1} Q RR'\right] \\
  &=& \tr\left[ R'^{-1}M'R'\Sigma^{-1} RMR^{-1} RR'\right] \\
  &=& \tr\left[ \Sigma^{-1} RM (RM)'\right] \\
  &=& \tr\left[ \Sigma^{-1} AA'\right] \\
\end{eqnarray*}

Moreover, $\det RR' = \det AM^{-1} M'^{-1} A' = \det AA'$.

Thus if we use $f_R(\cdot)$ to denote the joint probability density of
the entries of RR' and $f_A(\cdot)$ to denote that of AA', we can
write
\begin{equation}\label{eq:cross-corr-matrix-PDF}
  f_R(RR') = f_A(RMM'R')
\end{equation}

Substituting for $f_A$ the Wishart probability density function (PDF)
\ref{eq:WishartPDF}, we get the PDF of $RR'$.

\bibliographystyle{plain}
\bibliography{econophysics}
\end{document}


