\documentclass{book}
\usepackage{graphicx}
\usepackage{subfigure}
\usepackage{amssymb}
\usepackage{enumerate}
\usepackage[bookmarks=true]{hyperref}
\usepackage{bookmark}

\usepackage{amssymb,amsmath,amsthm}
%\newtheorem{mdef}{Definition}
%\newtheorem{theorem}{Theorem}
\newcommand{\eqsplit}[2]{
  \begin{equation}\label{#2}
    \begin{split}
      #1
    \end{split}
  \end{equation}}
\newcommand{\eqnsplit}[1]{
  \begin{eqnarray*}
    #1
  \end{eqnarray*}}
\newcommand{\tran}[1]{
  \tilde{#1}
}
\newcommand{\td}[2]{
  \frac{d #1}{d #2}
}
\newcommand{\pd}[2]{
  \frac{\partial #1}{\partial #2}
}
\newcommand{\ppd}[2]{
  \frac{\partial^2 #1}{\partial #2^2}
}
\newcommand{\pdd}[3]{
  \frac{\partial^2 #1}{\partial #2 \partial #3}
}
\newcommand{\otd}[1]{
  \frac{d}{d #1}
}
\newcommand{\opd}[1]{
  \frac{\partial}{\partial #1}
}
\newcommand{\oppd}[1]{
  \frac{\partial^2}{\partial #1^2}
}
\newcommand{\opdd}[2]{
  \frac{\partial^2}{\partial #1 \partial #2}
}
\newcommand{\ket}[1]{
  |#1\rangle
}
\newcommand{\bra}[1]{
  \langle#1|
}
\newcommand{\inn}[1]{
  \langle#1\rangle
}
\newcommand{\mean}[1]{
  \langle#1\rangle
}
\newcommand{\tr}{
  \text{tr}\,
}
\newcommand{\re}{
  \text{Re}\,
}
\newcommand\im{
  \text{Im}\,
}
\newcommand{\var}{
  \text{var}
}
\newcommand{\arcsinh}{
  \sinh^{-1}
}
\newcommand{\arccosh}{
  \cosh^{-1}
}
\newcommand{\erfc}{
  \text{erfc}
}
\newcommand{\E}{
  \mathbb{E}
}
\renewcommand{\P}{
  \mathbb{P}
}
\newcommand{\I}[1]{
  \mathbf{1}_{\{#1\}}
}
\newcommand{\1}[1]{
  \mathds{1}_{\{#1\}}
}
\newcommand{\diag}{
  \text{diag\,}
}
\newcommand{\M}{
  {\text{max}}
}
\newcommand{\m}{
  {\text{min}}
}
\newcommand{\ph}{
  {\text{arg}\,}
}
\newcommand\erf{
  \text{erf}
}
\renewcommand\vec[1]{
  \mathbf{#1}
}
\newcommand\mtx[1]{
  \mathbf{#1}
}
\newcommand\ed{
  \,{\buildrel d \over =}\,
}



\DeclareGraphicsExtensions{.pdf,.png,.jpg}

\title{Mathematical Aspects of the Capital Market}
\author{Xie Xiaolei}
\date{\today}
\begin{document}

\maketitle
\tableofcontents

\chapter{Introduction}
$\mathbb{E}(x)$

\section{Stylized facts}

\section{GARCH models}
The autocorrelation function of $\epsilon_t^2$ is given by
(c.f. \cite{Bollerslev87})
$$
\rho_n = \sum_{i=1}^{max(p,q)} (\alpha_i + \beta_i) \rho_{n-i}
\;n > p
$$
where $\alpha_i$ with $i > q$ and $\beta_i$ with $i > p$ are taken as
zeros. From these equations, it is clear that the partial
autocorrelation function cuts off at $max(p, q)$.


\chapter{Case Study of Stocks}
\section{Nordea Bank}
\begin{figure}[ht]
  \centering
  \includegraphics[scale=0.5, clip=true, trim=103 236 118
  200]{../pics/nordea_price_20120116-20120420.pdf}
  \caption{Nordea Bank paid prices 2012/01/16-2012/04/20}
  \label{fig:Nordea}
\end{figure}
Figure \ref{fig:Nordea} shows the paid prices of Nordea Bank from
2012/01/16 to 2012/04/20.

\subsection{15min Returns}
\subsubsection{GARCH model}
The autocorrelation function (ACF) of $\epsilon_t^2$ are shown in figure
\ref{fig:nordea_15min_acf}. At the absence of GARCH effects, the ACF
will have an asymptotic Gaussian distribution with mean 0 and variance
1/T. The first 5 autocorrelations have considerable sizes and do not
fall off as in the Gaussian case. This observation suggests a GARCH
model. Moreover, from \ref{fig:nordea_15min_vlt_acf}, one sees
apparent autocorrelations between volatilities of consecutive time
intervals. This again points to a GARCH model.
\begin{figure}[htb!]
  \centering
  \subfigure[ACF of squared returns]{
    \includegraphics[scale=0.4, clip=true, trim=95 236 118
    200]{../pics/nordea_15min_acf.pdf}
    \label{fig:nordea_15min_acf}
  }
  \subfigure[ACF of volatilities]{
    \includegraphics[scale=0.4, clip=true, trim=95 236 118
    200]{../pics/nordea_15min_vlt_acf.pdf}
    \label{fig:nordea_15min_vlt_acf}
  }
  \caption{Nordea Bank 15m autocorrelations}
\end{figure}

To obtain the functional form of the conditional distribution of
$\epsilon_t$ \footnote{We use the realized volatility as an estimate of the
conditional volatility. See \cite{Andersen03}.}, we look at the
qq-plot of $\epsilon_t/\sigma_t$ \ref{fig:nordea_15min_epOversig_qq},
which is consistent with a standard Gaussian distribution except for a
few outliers at the two ends. Hence the fat tails of the probability
density function can be entirely accounted for by GARCH effects,
implying a Gaussian conditional distribution is appropriate for $\epsilon_t$.
\begin{figure}[htb!]
  \centering
    \includegraphics[scale=0.6, clip=true, trim=100 236 118
    200]{../pics/nordea_15min_epOversig_qq.pdf}
  \caption{QQ-plot of $\epsilon_t / \sigma_t$. $\epsilon_t$ are
    derived from Nordea Bank 15m returns while $\sigma_t$ are realized
    volatilities calculated using 30s returns within each 15m
    interval.}
  \label{fig:nordea_15min_epOversig_qq}
\end{figure}

To determine the orders p and q of the GARCH model, we again check
the partial autocorrelation function of $\epsilon_t^2$. As just
mentioned, the first 5 partial autocorrelations are significant. Thus
we can tentatively estimate a GARCH(1, 5) model. Then by applying the
Lagrange multiplier test and accordingly removing insignificant
parameters, we obtain a GARCH(1, 3) model
\ref{tab:nordea_15min_garch}:
\begin{table}[htb!]
  \centering
  \begin{tabular}{c|c|c|c}
    parameter & value & std. error & t statistic \\
    \hline
     $\alpha_0$ &   1.05668e-06 &   3.56871e-07 &    2.96096\\
     \hline
     $\beta_1$  &     0.687212  &   0.0299314   &    22.9596\\
     \hline
     $\alpha_1$ &      0.128614 &   0.0268144   &    4.79643\\
     \hline
     $\alpha_3$ &    0.0568022  &   0.0301503   &    1.88397
  \end{tabular}
  \caption{GARCH(1, 3) model of Nordea Bank 15m returns}
  \label{tab:nordea_15min_garch}
\end{table}

\subsubsection{$(0,1,1)\times(0, 1, 1)$ seasonal volatility model}
With the availability of transactional data, estimating conditional
volatility using returns sampled at a higher frequency (realized
volatility) gives superior acuracy and reliability. However, at which
frequency the time series should be sampled in order to give an
unbiased and consistent estimate of the volatility is not a trivial
question. Naively one would argue by the doctrine of statistics that
the often the series is sampled, the better the estimate. This is also
well supported in theory, since the quadratic variation which is the
theoretial variance of the time series (see \cite{Andersen03}, theorem
1 and corollary 1), is approximated as (see \cite{Protter05}, ch.II,
sec.6, theorem 22):
\[
\var(p_{t+h} - p_t) = [r, r]_{t+h} - [r, r]_t = \lim_{n \to \infty}
\sum_{T^i \in \pi_n} (p^{T_{i+1}} - p^{T_i})^2
\]
where $p_t$ is the log price process and $r_t = p_t - p_0$. $\pi_n$
is a refining partition of the interval $[t, t+h]$.

However, due to the noise introduced by market micro-structure, there
is actually an optimal sampling frequency relative to $h$. While the
method to determine this optimal frequency is a subject of debate (see
for example \cite{Sahalia05}), it is not difficult to find a fairly
satisfactory frequency in practice.

Andersen and Bollerslev et al proved in \cite{Andersen03} that, with
the conditional volatility well approximated by the realized
volatility, $r_{t+h, h} = p_{t+h} - p_t$ is governed by a Gaussian
distribution (see theorem 2 of \cite{Andersen03}):
\[
r_{t+h, h} | \sigma(\mu_{t+s}, \sigma_{t+s})_{s \in [0, h]} \sim
N(\int_{0}^h \mu_{t+s} ds, \int_{0}^h \sigma_{t+s}^2 ds)
\]

Thus one can try a few frequencies and compare the distribution of
$r_{t+h, h}/\hat{\sigma_{t+h, h}}$ with the standard Gaussian. If the
two match, $\hat{\sigma_{t+h, h}}$ is a good estimate of $\sigma_{t+h,
  h}$, the volatility of $r_{t+h, h}$.

For the present case with $h = 15\text{min}$, a reasonally good
sampling frequency is 1/30sec, as confirmed by the QQ-plot
\ref{fig:nordea_15min_epOversig_qq} as well as by the autocorrelation
functions of $r_{t+h, h} / \sigma_{t+h, h}$ and of $(r_{t+h, h} /
\sigma_{t+h, h})^2$ which are shown in
\ref{fig:nordea_15min_quotient_acf} and
\ref{fig:nordea_15min_quotient_squared_acf} respectively.
\begin{figure}[htb!]
  \centering
  \subfigure[ACF of $r_{t+h, h} / \sigma_{t+h, h}$]{
    \includegraphics[scale=0.4, clip=true, trim=95 236 118
    200]{../pics/nordea_15min_quotient_acf.pdf}
    \label{fig:nordea_15min_quotient_acf}
  }
  \subfigure[ACF of $(r_{t+h, h} / \sigma_{t+h, h})^2$]{
    \includegraphics[scale=0.4, clip=true, trim=95 236 118
    200]{../pics/nordea_15min_quotient_squared_acf.pdf}
    \label{fig:nordea_15min_quotient_squared_acf}
  }
  \caption{Nordea 15min $r_t/\sigma_t$ and $(r_t/\sigma_t)^2$ ACF.}
\end{figure}

We see in figure \ref{fig:nordea_15min_quotient_acf} and
\ref{fig:nordea_15min_quotient_squared_acf} there is essentially no
autocorrelation in the series $r_{t+h, h} / \sigma_{t+h, h}$ and
$(r_{t+h, h} / \sigma_{t+h, h})^2$, supporting the assumption that the
series is i.i.d Gaussian distributed. As a result, the problem of
modeling $r_{t+h, h}$ boils down to modeling $\sigma_{t+h,
  h}$. Andersen and Bollerslev et al reported that, for the exchange
rates between Deutch mark, yen and dollar, $\ln \sigma_{t+h, h}$ is
gaussian distributed \cite{Andersen03}. This is, however, not the case
for our series in question. In fact, in our case, $\ln \sigma_{t+h,
  h}$ is right skewed (skewness 0.3342) and leptokurtic (kurtosis
6.1006). See figure \ref{fig:nordea_15min_logvol_pdf} and
\ref{fig:nordea_15min_logvol_qq}
\begin{figure}[htb!]
  \centering
  \subfigure[PDF of $\ln\sigma_{t+h,h}$]{
    \includegraphics[scale=0.32, clip=true, trim=52 173 46
    43]{../pics/nordea_15min_logvol_hist.pdf}
    \label{fig:nordea_15min_logvol_pdf}
  }
  \subfigure[QQ-plot of $\ln\sigma_{t+h,h}$]{
    \includegraphics[scale=0.4, clip=true, trim=95 236 118
    200]{../pics/nordea_15min_logvol_qq.pdf}
    \label{fig:nordea_15min_logvol_qq}
  }
  \caption{Nordea 15min $\ln\sigma_{t+h, h}$ unconditional
    distribution}
\end{figure}

Moreover, the autocorrelation function of $\ln\sigma_{t+h, h}$
shows long-lasting and periodic autocorrelations in the time series
with an apparent period of 33 (see figure
\ref{fig:nordea_15min_logvol_acf}). This suggests simplifying the
series by differencing \cite{BoxJenkins94}:
\[
w_t = (1-B)(1-B^s)\ln\sigma_{t+h, h}
\]
where $B$ is the back-shift operator. The length of the time interval
h, which is 15min in the present case, is left implicit.

The autocorrelation function of the differenced process $w_t$, as shown in
figure \ref{fig:nordea_15min_w_acf}, clearly points to a seasonal $(0,
0, 1)\times(0, 0, 1)$ model: There are only 4 non-zero
autocorrelations in the plot, located at lags 1, 32, 33, 34,
respectively; furthermore, the two at 32 and 34 are approximately
equal. Thus we can write down the model as
\begin{eqnarray*}
  w_t = (1 - \theta B)(1 - \Theta B^s) y_t
\end{eqnarray*}
where $\theta$ and $\Theta$ are parameters to be determined and $y_t$
is a noise process with constant variance $\sigma_y^2$ and mean
0. $y_t$ is often refered to as the residuals.

The above $(0, 0, 1)\times(0, 0, 1)$ model has the following
autocovariance structure \cite{BoxJenkins94}:
\begin{eqnarray*}
  \gamma_0 &=& \sigma_y^2 (1 + \theta^2)(1 + \Theta^2) \\
  \gamma_1 &=& -\sigma_y^2\theta(1 + \Theta^2) \\
  \gamma_s &=& -\sigma_y^2\Theta(1 + \theta^2) \\
  \gamma_{s+1} &=& \gamma_{s-1}\;=\;\sigma_y^2\theta\Theta
\end{eqnarray*}
These equations together with the measured autocorrelations make
possible an initial estimate of the parameters $\theta$ and $\Theta$:
\begin{eqnarray*}
  {\rho_{s+1}/\rho_s} &=& {\gamma_{s+1}/\gamma_s} \;=\; -{\theta \over
    1 + \theta^2} \\
  {\rho_{s+1}/\rho_1} &=& {\gamma_{s+1}/\gamma_1} \;=\; -{\Theta \over
    1 + \Theta^2} \\
\end{eqnarray*}
Substituting in the measured values shown in table
\ref{tab:nordea_15min_w_acf},
\begin{table}[htb!]
  \centering
  \begin{tabular}{c|c|c|c}
    $\rho_1$ & $\rho_{s-1}$ & $\rho_s$ & $\rho_{s+1}$ \\
    \hline
    -0.4703 &  0.2053 & -0.4564 &  0.2212
  \end{tabular}
  \caption{Nordea 15min $w_t$ autocorrelations}
  \label{tab:nordea_15min_w_acf}
\end{table}
we get
\begin{eqnarray*}
  \theta &=& 0.6890 \\
  \Theta &=& 0.6378
\end{eqnarray*}
Among the two roots of each of the 2nd order equations in the above,
we have chosen the one in the range $(-1, 1)$ so as to ensure
invertibility of the model \cite{BoxJenkins94}.

By inverting the seasonal moving average model of $w_t$ we can then
infer the noise process $y_t$ \footnotemark, i.e. the residuals:
\begin{eqnarray*}
  y_t &=& {1 \over 1 - \theta B}{1 \over 1 - \Theta B^s} w_t \\
  &=& \sum_{i=0}^{\infty}\sum_{j=0}^{\infty} \theta^i \Theta^j B^{i+j}
  w_t \\
  &=& w_t + \sum_{k=1}^{\infty} \sum_{j=0}^{\bot(k/s)} \theta^{k-sj} \Theta^j
  w_{t-k}
\end{eqnarray*}
where $\bot(\cdot)$ denotes the floor function, i.e. the largest
integer smaller than its argument.

In order to forecst the $w_t$ process, and hence the price process
itself, we must also know the distribution of $y_t$. Moreover, to
properly estimate the parameters of the model in the sense of maximum
likelihood, we are also in need of the distribution of $y_t$.

If $y_t$ is indeed i.i.d random variables as we assumed, we only need
to work out its unconditional distribution.

\footnotetext{
  The autoregressive process resulting from the inversion is inifinite
  in extent. Hence one must choose a cut-off value for k. For the
  present case of Nordea 15min log-volatility, I choose the value
  $k=sj$ for which $|\Theta|^j < 0.02$. This value of k, call it
  $k_{\top}$, is around 550. To avoid data reduction of this amount,
  we add to the beginning of $\{w_t\}_{t=1}^n$ $k_{\top}$ zeros
  corresponding to t=0, -1, ..., -$k_{\top}$+1. Recall $E(w_t) = E(y_t
  - \theta y_{t-1} - \Theta y_{t-s} + \theta \Theta y_{t-s-1}) = 0$.
}
\begin{figure}[htb!]
  \centering
  \subfigure[ACF of $\ln\sigma_{t+h,h}$]{
    \includegraphics[scale=0.4, clip=true, trim=95 230 112
    210]{../pics/nordea_15min_logvol_acf.pdf}
    \label{fig:nordea_15min_logvol_acf}
  }
  \subfigure[ACF of $w_t$]{
    \includegraphics[scale=0.4, clip=true, trim=95 230 112
    210]{../pics/nordea_15min_w_acf.pdf}
    \label{fig:nordea_15min_w_acf}
  }
  \caption{Nordea 15min $\ln\sigma_{t+h, h}$ and $w_t$ autocorrelations}
\end{figure}
\subsection{45min Returns}


\bibliographystyle{plain}
\bibliography{econophysics}
\end{document}


