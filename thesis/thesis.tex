\documentclass{report}
\usepackage{graphicx}
\usepackage{subfigure}
\usepackage{multirow}
\usepackage{wrapfig}
\usepackage{amssymb}
\usepackage[usenames,dvipsnames]{color}
\usepackage{amsmath}
\usepackage{mathrsfs}
\usepackage{enumerate}
\usepackage[bookmarks=true]{hyperref}
\usepackage[acronym]{glossaries}
\usepackage{bookmark}

\usepackage{amssymb,amsmath,amsthm,amsfonts}
\usepackage{mathrsfs}
\usepackage{dsfont}
\usepackage{enumerate}

%\newtheorem{mdef}{Definition}
%\newtheorem{theorem}{Theorem}
\newcommand{\eqsplit}[2]{
  \begin{equation}\label{#2}
    \begin{split}
      #1
    \end{split}
  \end{equation}}
\newcommand{\eqnsplit}[1]{
  \begin{eqnarray*}
    #1
  \end{eqnarray*}}
\newcommand{\tran}[1]{
  \tilde{#1}
}
\newcommand{\td}[2]{
  \frac{d #1}{d #2}
}
\newcommand{\pd}[2]{
  \frac{\partial #1}{\partial #2}
}
\newcommand{\ppd}[2]{
  \frac{\partial^2 #1}{\partial #2^2}
}
\newcommand{\pdd}[3]{
  \frac{\partial^2 #1}{\partial #2 \partial #3}
}
\newcommand{\otd}[1]{
  \frac{d}{d #1}
}
\newcommand{\opd}[1]{
  \frac{\partial}{\partial #1}
}
\newcommand{\oppd}[1]{
  \frac{\partial^2}{\partial #1^2}
}
\newcommand{\opdd}[2]{
  \frac{\partial^2}{\partial #1 \partial #2}
}
\newcommand{\ket}[1]{
  |#1\rangle
}
\newcommand{\bra}[1]{
  \langle#1|
}
\newcommand{\inn}[1]{
  \langle#1\rangle
}
\newcommand{\mean}[1]{
  \langle#1\rangle
}
\newcommand{\tr}{
  \text{tr}\,
}
\newcommand{\re}{
  \text{Re}\,
}
\newcommand\im{
  \text{Im}\,
}
\newcommand{\var}{
  \text{var}
}
\newcommand{\arcsinh}{
  \sinh^{-1}
}
\newcommand{\arccosh}{
  \cosh^{-1}
}
\newcommand{\erfc}{
  \text{erfc}
}
\newcommand{\E}{
  \mathbb{E}
}
\renewcommand{\P}{
  \mathbb{P}
}
\newcommand{\I}[1]{
  \mathbf{1}_{\{#1\}}
}
\newcommand{\1}[1]{
  \mathds{1}_{\{#1\}}
}
\newcommand{\diag}{
  \text{diag\,}
}
\newcommand{\M}{
  {\text{max}}
}
\newcommand{\m}{
  {\text{min}}
}
\newcommand{\ph}{
  {\text{arg}\,}
}
\newcommand\erf{
  \text{erf}
}
\renewcommand\vec[1]{
  \mathbf{#1}
}
\newcommand\mtx[1]{
  \mathbf{#1}
}
\newcommand\ed{
  \,{\buildrel d \over =}\,
}




\newacronym{acf}{ACF}{Auto-correlation Function}
\newacronym{arima}{ARIMA}{Integrated Autoregressive Moving Average}
\newacronym{arma}{ARMA}{Autoregressive Moving Average}
\newacronym{mgf}{MGF}{Moment Generating Function}
\newacronym{garch}{GARCH}{Generalized Autoregressive Conditional Heteroscedasticity}

\newacronym{cdf}{CDF}{Cummulative Distribution Function}
\newacronym{mle}{MLE}{Maximum Likelihood Estimate}
\newacronym{pdf}{PDF}{Probability Density Function}
\newacronym{sv}{SV}{Stochastic Volatility}
\newacronym{ar}{AR}{Autoregressive}
\newacronym{ipr}{IPR}{Inverse Participation Ratio}
\newacronym{pr}{PR}{Participation Ratio}


\DeclareGraphicsExtensions{.pdf,.png,.jpg}

\makeglossaries
\author{Xie Xiaolei}
\date{\today}
\begin{document}

\begin{titlepage}
\begin{center}

% Upper part of the page. The '~' is needed because \\
% only works if a paragraph has started.
\includegraphics[width=0.5\textwidth]{../pics/lund_uni-logo_s}~\\[1cm]

% \textsc{\LARGE Lund University}\\[1.5cm]

\textsc{\Large Master's Thesis Project}\\[0.5cm]

% Title
%\HRule \\[0.4cm]
{ \huge \bfseries Return Models and Covariance Matrices
  \\[0.4cm] }

%\HRule \\[1.5cm]

% Author and supervisor
\begin{minipage}{0.4\textwidth}
\begin{flushleft} \large
\emph{Author:}\\
\textsc{Xie} Xiaolei
\end{flushleft}
\end{minipage}
\begin{minipage}{0.4\textwidth}
\begin{flushright} \large
\emph{Supervisor:} \\
Prof. Sven \textsc{\AA berg}
\end{flushright}
\end{minipage}

\vfill

% Bottom of the page
{\large \today}

\end{center}
\end{titlepage}
% \maketitle
\begin{abstract}
Return models and covariance matrices of return series have been
studied. In particular, \gls{garch} and \gls{sv} models
are compared with respect to their forecasting accuracy when applied
to intraday return series. \gls{sv} models are found to be considerably more
accurate and more consistent in accuracy of forecasting.

Covariance matrices formed from Gaussian and \gls{garch} return series, and
in particular, return series auto-correlated as an AR(1) process, have
been studied. In the case of Gaussian returns, the largest eigenvalue
is found to approximately follow a gamma distribution also when the
returns are auto-correlated. Expressions relating auto-correlation
strength with the mean and the variance of the asymptotic Gaussian
distribution of the matrix elements are derived. In the case of
\gls{garch} returns, both the largest and the smallest eigenvalues of
the covariance matrix are seen to increase with increasing
auto-correlation. The matrix elements are found to follow L\'evy
distributions with different L\'evy indices for the diagonal and the
non-diagonal elements.

Localization of eigenvectors of covariance matrices of
returns from \gls{garch} processes has been investigated. It is
found that the localization is reduced as the auto-correlation is
increased. Quantitatively, the number of localized eigenvectors
decreases approximately as a quadratic function with the
auto-correlation strength, i.e. the autoregressive coefficient of the
AR(1) process.

%% Popular Abstract
% Can we predict the movements of the financial market? The direct answer is "no", but there is more to be said. One cannot predict the price movement in an efficient market, but one can indeed predict how "volatile" the prices are (volatility). This does not lead to profits, but is useful for risk management --- if one pursues a profit, one must take a risk. Knowing and quantitatively managing the risks associated with investments, one often makes wiser decisions.

% This report entails, first of all, a comparison of return models in the context of intraday returns. 6 return series are studied: Nordea, Volvo and Ericsson in the time scales of 15 and 30 minutes. Stochastic Volatility (SV) models and GARCH models are fitted to these series and the resulting volatility forecasts are compared. SV models are found to be more accurate in most cases.

% Covariance matrices, which give the correlation between different companies, are also studied. In particular, the eigenvectors and eigenvalues of a covariance matrix, which respectively represent a set of driving factors of the companies' shares and the variances of these factors, are studied. It is found that when the return series are auto-correlated, i.e. when observations at different times are correlated, the eigenvectors are delocalized --- their expansion as a linear combination of the basis vectors involve more appreciable coefficients.
\end{abstract}

\section*{Acknowledgements}
First of all, I thank my supervisor, Prof. Sven \AA berg, for the many
hours of mentoring, discussion, encouragement and guidance. Without
them, this report will not exist. Then I wish to give my gratitude to
the people in the division of mathematical physics. The casual
discussions that I had with them truly inspired me, and
the friendly and lively atmosphere that they create has helped bring
peace and clarity to my mind.

Last but not least, I want to thank my wife, Wang Xiaodan, for her
unconditional support and understanding.

\tableofcontents

% \chapter*{List of Acronyms}

\printglossaries

\chapter{Introduction} \label{chp:introduction}
The mathematics of the financial market has always been a topic that
arouses interest and imagination, and with no doubt, has been
studied from many aspects and in many different ways.

Central to all these studies are the concepts of probability
distributions and correlations. The values of stocks, futures,
options, etc. are stochastic in nature and are governed by the laws of
stochastic processes, which are expressed in terms of probability
distributions and correlations.

Meanwhile, the observables of the market are prices, volumes,
turn-over (the amount of money paid in a trade), names of the brokers,
and time of the trades. These quantities don't make much sense
by themselves but do reveal the probabilistic dynamics of the market
when put together and turned into statistics.

The dynamics of an asset is affected by its own history as well as by
the histories and current values of other assets in the market. The
influence from the asset's own history is termed autocorrelations,
i.e. correlations in time, while the influence from other assets are
termed cross correlations.

Quite often, instead of an asset's price, the relative price change,
i.e. the return, is studied. Analytically solvable models often assume
Gaussian return distribution, although data suggest fatter tails
\footnotemark. By numerical methods, statistical features of
historical returns data, such as fat tails, can be reproduced.
\footnotetext{
  While the exact nature of the tails of the distribution of
  historical returns $r$ is debatable, it is agreed $P(|r| > x) > a
  e^{-bx^2}$, for any positive constants $a$ and $b$. Here $P(\cdot)$
  denotes probability. We adopt this as the definition of ``fat''
  tails.
}

Statistical models built on historical data describe the time
evolution of the aforementioned observables in terms of probability
distributions, autocorrelations and cross correlations. They have been
used to forecast future volatilities and help determine a pair price
of a contingent claim, e.g. an option. Here the term ``volatility''
refers to the standard deviation of the distribution of returns, and
measures how volatile the returns are. The volatility of an asset
constitutes a major risk of investments on the asset and hence is very
important.


% In the domain of discrete-time returns' models,
% \gls{garch} and stochastic volatility models (SV) are the most popular, so in
% this thesis project we study a total of 7 intraday return series 
% using \gls{garch} and SV models and compare their performance in volatility
% forecast.
In addition to the model of a single asset, the correlation between
a group of assets is also of great interest. For example, in principle
component analysis, one wishes to identify a number of factors that
``drive'' the price evolution of a group of assets in the sense that
each of the assets' returns can be expressed as a linear combination
of the factors' returns. In this scenario, the eigenvalues of the
covariance matrix are the variances of the factors' returns and their
corresponding orthogonalized eigenvectors give the composition of the
factors, i.e. the coefficients with which the factors are contructed
as a linear combination of the assets.

Therefore, in this thesis we also study the elements and the
eigenvalues distribution, as well as the eigenvectors composition of
the covariance matrix. When the matrix is constructed from returns
with simple Gaussian distribution, the matrix is termed a Wishart matrix
and has been studied extensively in the literature. If the returns
have L\'evy distributions, the matrix is termed Wishart-L\'evy and has
been studied to some extent, particularly regarding its eigenvalue
distribution \cite{politi2010}.

However, it is understood that real stock/index returns are much more
complicated than a straight-forward Gaussian or L\'evy distribution can
describe --- instead, one needs structured models. For this reason, we
are particularly interested in a covariance matrix obtained for
realistic return models. The so called \gls{garch}(1,1) model is a realistic
return model proven to have regularly varying tails
\cite{mikosch2000}. So we study properties of eigenvalues and
eigenvectors of such covariance matrices. Moreover, we also study how
auto-correlations in the returns influence the aforementioned
properties. Such auto-correlations, known as second-order
auto-correlations decay exponentially but may still leave footprints
in the covariance matrix.

This report is organized as follows: Chapter \ref{chp:PriceModels}
reviews some of the most influential return model. Parameters of the
models are fitted to a few intraday return series and the predictive
power of the models is compared. A calculation of the
unconditional distribution functions of \gls{sv} models,
especially in the case where the residual of the log-volatility
and the innovation of the return are correlated normal variates, is
also presented. In chapter \ref{chp:Gaussian} we investigate
distributions of eigenvalues of the Wishart matrix, and study the
influence of auto-correlated returns. In chapter
\ref{chp:CrossCorrelationFat} distributions of elements and
eigenvalues of the covariance matrix of identically specified
\gls{garch}(1,1) series are studied. Finally, chapter \ref{chp:summary}
summarizes the results. Supplementary material is provided in the
appendices.

\chapter{Return Models}\label{chp:PriceModels}
In this chapter we review some of the discrete-time return models and
fit them to intraday returns. The intention is to compare these models
in terms of forecast accuracy and to understand their statistical
properties. In the following we first describe these models briefly,
then in section \ref{chp:nordea_15min}, section \ref{sec:volvo_30} and appendix
\ref{chp:appendix2} we describe how the \gls{garch} and the \gls{sv} models
are fitted to intraday returns and compare their forecasts. In section
\ref{sec:XieCalc} we calculate the unconditional distribution
functions of the \gls{sv} model.

% \gls{frt} is good.
% section \ref{sec:Garch_model} reviews the \gls{garch} model using the Nordea
% 15-minute returns as an example. Section \ref{sec:SLV_model} examines the
% stochastic log-volatility model and discusses the possibility of
% describing the residuals of the log-volotility series using the
% Johnson Su distribution. The same Nordea return series is fitted as an
% example. Section \ref{sec:XieCalc} derives the unconditional distribution
% function of the stochastic log-volotility model and compares the
% analytic results to the 30-minute returns of a number of Swedish
% stocks.

\begin{enumerate}
\item{\bf Gaussian Distribution}

The justification of modeling return series as independent,
identically distributed Gaussian variates comes from imagining the
price process $S(t)$ as a geometric Brownian motion whose increment
$\sigma dw_t$ at each time step is independent and scales as $\sigma
\sqrt{dt}$. Then, by adding a drift term $\mu dt$ that represents some
deterministic trend in the price process, one can express the
price $S(t)$ as a stochastic differential equation:
\begin{eqnarray*}
  dS &=& S\mu dt + S\sigma dw_t \\
\end{eqnarray*}
Then by It\^o's lemma, a stochastic differential equation for the
logarithmic price $\ln S$ can be obtained
\begin{eqnarray*}
  d(\ln S) &=& (\mu - \frac{1}{2} \sigma^2)dt + \sigma dw_t \\
\end{eqnarray*}
It follows from this equation that $\ln S(t) - \ln S_0$, where $S_0$
is the price at time 0, has Gaussian distribution with mean $(\mu -
\sigma^2/2)t$ and variance $\sigma^2 t$ \cite{Bernt2000}. Therefore,
in a discrete-time model, where the length of each time step $\Delta
t$ is fixed, the return $r_t = \ln S(t) - \ln S(t - \Delta t)$ is
assumed to be Gaussian distributed and have mean and variance that are
functions of $\Delta t$.

The picture depicted above is of course overly simplified, and the
distribution of returns is not really Gaussian. Nevertheless, the
assumption of Gaussian distributed returns underlies such important
theories as Black and Scholes theory of option pricing. Thus, albeit
inaccurate in describing very large returns, the Gaussian distribution
as a return model shall not be forgotten. In the next few sections,
we discuss some more realistic return models.

\item{\bf \gls{garch} models}

``GARCH'' is the acronym for ``Generalized Autoregressive Conditional
Heteroscedasticity''. A \gls{garch}(p, q) model is defined by the following
equation system \cite{Bollerslev86}:
\begin{equation}
  \label{eq:garch_def}
  \begin{aligned}
    r_t &= \mu_t + \epsilon_t \\
  \end{aligned}
\end{equation}
where $r_t$ is the return as mentioned earlier; $p$ and $q$ are
constant integers; $\mu_t$ is the mean process and considered a small
constant for intraday returns. $\epsilon_t$ is called the innovation
of the return and is a variate whose conditional distribution
\footnotemark is Gaussian and has variance $\sigma^2_t$:
\begin{eqnarray*}
  \sigma_t^2 &=& \alpha_0 + \sum_{i=1}^q \alpha_i \epsilon_{t-i}^2 +
  \sum_{i=1}^p \beta_i \sigma_{t-i}^2
\end{eqnarray*}
\footnotetext{
By conditional distribution we mean the distribution conditional on
the variate's history: The conditional distribution at time $t$ means
the distribution conditional on the history up to time $t-1$
in the discrete-time setting.
}

It can be shown that the autocorrelation function $\varrho_n =
\text{corr}(\epsilon_t, \epsilon_{t-n})$ for $n > \text{max}(p,q)$ of
$\epsilon_t^2$ is given by \cite{Bollerslev87}:
\begin{eqnarray*}
\varrho_n &=& \sum_{i=1} ^{p \vee q}
(\alpha_i + \beta_i) \varrho_{n-i} \text{ for $n > p$}
\end{eqnarray*}
where $\alpha_i$ with $i > q$ and $\beta_i$ with $i > p$ are taken as
zeros. $p \vee q$ denotes the maximum of p and q. From these
equations, it is clear that the partial autocorrelation function cuts
off at $\max(p, q)$.

Here we note that, in the \gls{garch} model, the volatility $\sigma_t$ is
$\mathcal{F}_{t-1}$ measurable: given the history up to time $t-1$,
$\sigma_t$ is a deterministic quantity. In contrast, \gls{sv} models,
which we discuss shortly, treat $\sigma_t$ as a random variable even
given the same history.

\item {\bf Stochastic Volatility Models}\label{sec:SLV_model}

For the purpose of intraday returns, we specify the \gls{sv} model as
\begin{eqnarray}
   r_{t, t-h} &=& \ln S_{t} - \ln S_{t-h} \nonumber \\
   r_{t, t-h} &=& \mu + \sigma_{t, t-h} b_t \label{eq:SLV_spec}
\end{eqnarray}
where $S_t$ is the price of the asset at time $t$; $b_t \sim N(0,
1)$; $r_{t, t-h}$ is the return over the time interval $[t-h, t]$.
For simplicity, in the rest of this chapter we shall just write
$t$ for the subscript ``t, t-h'', since the time interval $h$ is fixed
for each time series and is a known constant.

Andersen et al proved the following in \cite{Andersen03} (theorem 2):
\begin{equation}
  \label{eq:normal_r}
  r_t|\mathcal{F}_{t-h} \sim N(\int_{0}^h \mu_{t-h+s} ds, \int_{0}^h
  \sigma_{t-h+s}^2 ds)
\end{equation}
In plain words, conditional on the information up to time $t-h$, the
distribution of $r_t$ is Gaussian with integrated mean and variance.
The variance of this conditional distribution, $\int_{0}^h
\sigma_{t-h+s}^2 ds$, can be approximated by \cite{Protter05,
  Andersen03}:
\begin{equation}
  \label{eq:rv_def}
  \int_{0}^h \sigma_{t-h+s}^2 ds = \sum_{k=1}^n \left(
    \ln S_{t-h+kh/n} - \ln S_{t-h+(k-1)h/n} \right)^2
\end{equation}
where $n$ is a chosen constant. The square root of the right hand side
of equation \ref{eq:rv_def} is the realized volatility, call it
$\hat{\sigma}_t$.

With the availability of transaction data, estimating conditional
volatility using returns sampled at a higher frequency (realized
volatility) gives superior accuracy and reliability. However, at which
frequency the time series should be sampled (the choice of $n$) in
order to give an unbiased and consistent estimate of the volatility is
not a trivial question. Naively one would believe that the often the
series is sampled, the better the estimate, but in fact, due to noise
introduced by market micro-structure, there is an optimal sampling
frequency, depending on $h$. While the method to determine this
optimal frequency is a subject of debate (see for example
\cite{Sahalia05}), it is not difficult to find a fairly satisfactory
frequency in practice:

Keeping equation \ref{eq:normal_r} in mind, one can simply try a few
frequencies and compare the distribution of $(r_t -
\E(r_t))/\hat{\sigma}_t$ with the standard Gaussian. If the two match,
$\hat{\sigma}_t$ is a good approximation of
\[
\sigma_t = \left(\int_{0}^h \sigma_{t-h+s}^2 ds \right)^{1/2}
\]

Once a good approximation of $\sigma_t$ has been found, it is
convenient to model $\ln\sigma_t$ so that the positivity of $\sigma_t$
is implied by construction \cite{Mikosch2009}. As is seen in later
sections, \gls{arima} models often serve well for this purpose. An
\gls{arima}(p, d, q) model, where $p,d,q$ are integers, is defined as
\cite{BoxJenkins94}
\begin{eqnarray*}
  (1 - \sum_{i=1}^p \phi_i B^i) (1 - B)^d \ln \sigma_t &=& (1 -
  \sum_{i=1}^q \theta_i B^i) y_t
\end{eqnarray*}
where $B$ is the back-shift operator such that $B x_t = x_{t-1}$ for
any time series $x_t$. $y_t$ are termed the residuals of the
model, $d$ is the order of integration, and $p, q$  are orders of
autoregression and moving average, respectively. $\phi_i$ and
$\theta_i$ are constant parameters.

Quite often, the auto-correlation function of the time series in
question manifests periodic patterns, thus seasonal components are
added to the above model to account for the seasonality:
\begin{eqnarray*}
&& (1 - \sum_{i=1}^P \Phi_i B^{is}) (1 - \sum_{i=1}^p \phi_i B^i) (1 - B^s)^D (1 -
B)^d\ln \sigma_t \\
&& = (1 - \sum_{i=1}^Q \Theta_i B^{is}) (1 -
\sum_{i=1}^q \theta_i B^i)y_t
\end{eqnarray*}
where $D$, analogous to $d$ in the non-seasonal model, is the order of
seasonal integration; $P, Q$ are orders of seasonal autoregression and
moving average, respectively. $\Phi_i$ and $\Theta_i$ are constant
parameters. In the most general situations, the seasonal and the
non-seasonal components do not necessarily combine in the above
multiplicative fashion, so the following model is also of interest:
\begin{eqnarray*}
(1 - \sum_{i=1}^{p+P} \phi_i B^i) (1 - B^s)^D(1 - B)^d \ln
\sigma_t &=& (1 - \sum_{i=1}^{q+Q} \theta_i B^i) y_t
\end{eqnarray*}
In all the above cases, if $d = 0$ and $D = 0$, the model does not
involve integration and hence is called an \gls{arma}(p,q) model. An
even simpler case is where $q = 0$ and $Q = 0$ in addition to $d = 0$
and $D = 0$. These conditions make the model a pure \gls{ar}
process, denoted AR(p).
\end{enumerate}

To compare \gls{garch} and \gls{sv} models at the face of intraday returns,
we study the {\it Nordea Bank} 15-minute returns during the period
2012/01/16 - 2012/04/20 in section \ref{chp:nordea_15min}, 
and {\it Volvo B} 30-minute returns during 2013/10/10 - 2014/04/04 in
section \ref{sec:volvo_30}. Another 4 intraday series are also studied and
collected in appendix \ref{chp:appendix2}. We have chosen these
particular stocks because they have the largest trading volumes in the
Swedish market and hence provide the largest amount of data for
analysis. The time intervals of 15 and 30 minutes are chosen because
they are relatively short and hence give a large amount of data and
yet they are not too short for noise of market
micro-structure\footnotemark to become a concern.
\footnotetext{
Noise of market micro-structure is due to, most importantly,
the difference between the bid and asked prices (bid-ask spread), and
the discreteness of price changes.
}

\section{Case Study: Nordea 15-minute Returns}
\label{chp:nordea_15min}
In this section we investigate the Nordea 15-minute returns sampled
during the period 2012/01/16 - 2012/04/20. In total, these amount to
2022 returns. We use the first 80\% (1617) for model estimation and  the
remaining 20\% (405) for comparing with model forcasts. In section
\ref{sec:nordea_15min_garch} we study the series with a \gls{garch} model 
and in section \ref{sec:nordea_15min_arima} we study it with a
\gls{sv} model.

\subsection{GARCH Model}\label{sec:nordea_15min_garch}
When volatilities are auto-correlated or squared returns and
volatilities are correlated, a \gls{garch}(p, q) model may be
appropriate for the return series under investigation. To find out
whehter this is true in our case, we plot the \gls{acf} of the squared
returns. This is shown in figure \ref{fig:nordea_15min_acf}. If the
aforementioned features are absent from the series, the
auto-correlations are expected to be Gaussian distributed with mean 0
and variance 1/T, and hence mostly reside within the confidence bounds
set by $\pm 2/\sqrt{T}$ \cite{Bollerslev86, Bollerslev87}.

Clearly this is not the case in figure \ref{fig:nordea_15min_acf} ---
the first 5 auto-correlations are rather significant. Moreover, figure
\ref{fig:nordea_15min_vlt_acf} shows even more clearly that the
conditional variances of the series are correlated. These observations
suggest a \gls{garch}(p, q) model can be appropriate.
\begin{figure}[htb!]
  \centering
  \subfigure[]{
    \includegraphics[scale=0.4, clip=true, trim=90 257 105
    220]{../pics/nordea_15min_acf.pdf}
    \label{fig:nordea_15min_acf}
  }
  \subfigure[]{
    \includegraphics[scale=0.4, clip=true, trim=90 259 104
    218]{../pics/nordea_15min_vlt_acf.pdf}
    \label{fig:nordea_15min_vlt_acf}
  }
  \caption{\small \it \ref{fig:nordea_15min_acf}: Auto-correlations
    (ACF) of the squared returns; \ref{fig:nordea_15min_vlt_acf}:
    Auto-correlations (ACF) of squared realized volatilities
    ($\hat{\sigma}^2_t$). The blue lines are confident bounds set at
    $\pm 2/\sqrt{T}$. $T$ is the length of the time series.}
\end{figure}

Starting with a \gls{garch}(1,1) model and taking advantage of the knowledge
that the log-volatility $\ln \sigma_t$ has seasonality $s=33$ (see
figure \ref{fig:nordea_15min_logvol_acf}), we fit to the return series
a \gls{garch}(33, 33) model, limiting to lags 1 and 33 for both ARCH and
\gls{garch} parameters.
\begin{eqnarray*}
  r_t &=& \mu + \epsilon_t \\
  \epsilon_t &=& \sigma_t z_t \\
  \sigma^2_t &=& \alpha_0 + \alpha_1 \epsilon^2_{t-1} + \alpha_s
  \epsilon^2_{t-s} + \beta_1 \sigma^2_{t-1} + \beta_s \sigma^2_{t-s}
\end{eqnarray*}
where the mean process of $r_t$, denoted $\mu_t$ earlier, is simplified
to a mere constant $\mu$ owing to its smallness. by \gls{mle},
parameter values listed in table \ref{tab:nordea_15min_garch} are obtained.
\begin{table}[htb!]
  \centering
  \begin{tabular}{|c|c|c|c|c|c|}
    \hline
    Parameter & $\alpha_0$ & $\alpha_1$ & $\alpha_s$ & $\beta_1$ &
    $\beta_s$ \\
    \hline
    Value & $4.7833 \times 10^{-7}$ & 0.1600 & 0.0667 & 0.6846 &
    0.0342 \\
    \hline
  \end{tabular}
  \caption{\small \it GARCH model parameters}
  \label{tab:nordea_15min_garch}
\end{table}

\subsection{Stochastic Volatility Model}\label{sec:nordea_15min_arima}
For the Nordea Bank 15-minute returns under consideration, it can be
verified that the square root of the sum of squared 30-second returns
makes a good proxy for the volatility. This can be seen from the
probability plot of $z_t = (r_t - \E(r_t))/\hat{\sigma}_t$ (figure
\ref{fig:nordea_bank_15min_z_prob}), i.e. the quotient of the 
15-minute returns over the volatility proxy.
\begin{figure}[htb!]
  \centering
    \includegraphics[scale=0.4, clip=true, trim=80 258 104
    220]{../pics/nordea_bank_15min_z_prob.pdf}
  \caption{\small \it Probability plot of $z_t =
    (r_t-\E(r_t))/\sigma_t$. $\epsilon_t$ are derived 
      from Nordea Bank 15min returns while $\sigma_t$ are realized
      volatilities calculated using 30s returns within each 15min
      interval. Horizontal axis: $z_t$; Vertical axis: CDF of
    $z_t$, arranged on such a scale that the CDF of the standard
    Gaussian is a straight line.}
  \label{fig:nordea_bank_15min_z_prob}
\end{figure}
% In addition, one can see
% from figure \ref{fig:nordea_15min_quotient_acf} and
% \ref{fig:nordea_15min_quotient_squared_acf} that there is essentially
% no auto-correlation in the $z_t$ or the $z_t^2$ series.
% \begin{figure}[htb!]
%   \centering
%   \subfigure[ACF of $z_t$]{
%     \includegraphics[scale=0.4, clip=true, trim=95 236 118
%     200]{../pics/nordea_15min_quotient_acf.pdf}
%     \label{fig:nordea_15min_quotient_acf}
%   }
%   \subfigure[ACF of $z_t^2$]{
%     \includegraphics[scale=0.4, clip=true, trim=95 236 118
%     200]{../pics/nordea_15min_quotient_squared_acf.pdf}
%     \label{fig:nordea_15min_quotient_squared_acf}
%   }
%   \caption{\small \it Nordea 15min $z_t$ and $z_t^2$ ACF.}
% \end{figure}

Andersen and Bollerslev et al reported that, for the exchange
rates between Deutch mark, yen and dollar, $\ln \sigma_t$ is
Gaussian distributed \cite{Andersen03}. This is, however, not the case
for our modestly sized series. In fact, in our case, $\ln \sigma_t$ is
right skewed (skewness 0.3342) and leptokurtic (kurtosis 6.1006). See
figure \ref{fig:nordea_15min_logvol_prob}
\begin{figure}[htb!]
  \centering
  %\vspace{-15mm}
  \includegraphics[scale=0.4, clip=true, trim=80 223 107
  4]{../pics/nordea_15min_logvol_prob.pdf}
  \caption{\small \it Probability plot of Nordea 15min log-volatility
    $\ln\sigma_t$ unconditional distribution}
  \label{fig:nordea_15min_logvol_prob}
\end{figure}
Moreover, the series of $\ln\sigma_t$ shows long-lasting and
periodic autocorrelations with an apparent period of 33 (see figure
\ref{fig:nordea_15min_logvol_acf}). This suggests the series may be
described by a seasonal \gls{arima} model. Thus we first simplify the
series by differencing \cite{BoxJenkins94}:
\begin{equation}
  \label{eq:differenced_lv}
  w_t = (1-B)(1-B^s)\ln\sigma_t  
\end{equation}
where $B$ is the back-shift operator\footnote{For example, $B\,x_t =
  x_{t-1}$} and $s=33$ is the seasonality.

The autocorrelation function of the differenced process $w_t$, as
shown in figure \ref{fig:nordea_15min_w_acf}, clearly points to a seasonal
moving-average model: There are only 4 non-zero autocorrelations in
the plot, located at lags 1, 32, 33, 34, respectively; furthermore,
the two at 32 and 34 are approximately equal. Thus we can write down
the model as
\begin{eqnarray}
  w_t &=& (1 - \theta B)(1 - \Theta B^s) y_t \label{eq:nordea_w}
\end{eqnarray}
where $\theta$ and $\Theta$ are parameters to be determined and $y_t$
is a noise process with constant variance $\sigma_y^2$ and mean
0. $y_t$ is often refered to as the residuals.
\begin{figure}[htb!]
  \centering
  \subfigure[ACF of log-volatility ($\ln\sigma_t$)]{
    \includegraphics[scale=0.4, clip=true, trim=95 230 112
    235]{../pics/nordea_15min_logvol_acf.pdf}
    \label{fig:nordea_15min_logvol_acf}
  }
  \subfigure[ACF of differenced log-volatility ($w_t$)]{
    \includegraphics[scale=0.4, clip=true, trim=95 230 112
    235]{../pics/nordea_15min_w_acf.pdf}
    \label{fig:nordea_15min_w_acf}
  }
  \caption{\small \it Auto-correlations of Nordea 15min log-volatility
    ($\ln\sigma_t$) and differenced log-volatility ($w_t$).}
  \label{fig:nordea1_15min_acf}
\end{figure}

The above seasonal moving average model has the following
autocovariance structure \cite{BoxJenkins94}:
\begin{eqnarray*}
  \gamma_0 &=& \sigma_y^2 (1 + \theta^2)(1 + \Theta^2) \\
  \gamma_1 &=& -\sigma_y^2\theta(1 + \Theta^2) \\
  \gamma_s &=& -\sigma_y^2\Theta(1 + \theta^2) \\
  \gamma_{s+1} &=& \gamma_{s-1}\;=\;\sigma_y^2\theta\Theta
\end{eqnarray*}
These equations together with the measured autocorrelations make
possible an initial estimate of the parameters $\theta$ and $\Theta$:
\begin{eqnarray*}
  {\varrho_{s+1}/\varrho_s} &=& {\gamma_{s+1}/\gamma_s} \;=\; -{\theta \over
    1 + \theta^2} \\
  {\varrho_{s+1}/\varrho_1} &=& {\gamma_{s+1}/\gamma_1} \;=\; -{\Theta \over
    1 + \Theta^2} \\
\end{eqnarray*}
Substituting in the measured values shown in table
\ref{tab:nordea_15min_w_acf},
\begin{table}[htb!]
  \centering
  \begin{tabular}{|c|c|c|c|}
    \hline
    $\varrho_1$ & $\varrho_{s-1}$ & $\varrho_s$ & $\varrho_{s+1}$ \\
    \hline
    -0.4703 &  0.2053 & -0.4564 &  0.2212 \\
    \hline
  \end{tabular}
  \caption{\small \it autocorrelations of differenced log-volatility
    ($w_t$)}
  \label{tab:nordea_15min_w_acf}
\end{table}
we get
\begin{eqnarray*}
  \theta &=& 0.6890 \\
  \Theta &=& 0.6378
\end{eqnarray*}
Among the two roots of each of the 2nd order equations in the above,
we have chosen the one in the range $(-1, 1)$ so as to ensure
invertibility of the model \cite{BoxJenkins94}.

With an estimate of $\theta$ and $\Theta$, one can then infer the
noise process i.e. the residuals $y_t$:
\begin{equation}
  \label{eq:infer_y}
  y_t = w_t + \theta y_{t-1} + \Theta y_{t-s} - \theta \Theta y_{t-s-1}
\end{equation}
where we substitute $y_t\;(t \leq 0)$ with their unconditional
expectation 0.

In order to forecast the $w_t$ process, and hence the return process
itself, we must also know the distribution of $y_t$. Moreover, to
properly estimate the parameters of the model in the sense of maximum
likelihood, we are also in need of the distribution of $y_t$.

Figure \ref{fig:nordea_15min_y_qq} shows the normal probability plot of
$y_t$. It is evident from this figure that $y_t$ has fat tails.
\begin{figure}[htb!]
  \centering
  \includegraphics[scale=0.4, clip=true, trim=78 255 109
  123]{../pics/nordea2_y_normplot.pdf}
  \caption{\small \it Normal probability plot of residuals of
    log-volatility ($y_t$)}
  \label{fig:nordea_15min_y_qq}
\end{figure}
In addition, a simple calculation reveals that the distribution of
$y_t$ has skewness 0.2988 (shown in table
\ref{tab:nordea_15min_y_moments}).
\begin{table}[htb!]
  \centering
  \begin{tabular}{|c|c|c|c|}
    \hline
    mean & variance & skewness & kurtosis \\
    \hline
    0.0012 & 0.0935 & 0.2988 & 6.8691 \\
    \hline
  \end{tabular}
  \caption{\small \it Moments of log-volatility residuals ($y_t$)}
  \label{tab:nordea_15min_y_moments}
\end{table}
Based on this information, we find that $y_t$ can be well
described by a Johnson Su distribution \cite{Shang2004}:
\[
  y_t = \xi + \lambda\sinh{z_t - \gamma \over \delta}
\]
where $\gamma, \delta, \lambda, \xi$ are parameters to be determined
and $z_t \sim N(0, 1)$. The goodness of fitting is demonstrated in
figure \ref{fig:nordea_15min_y_js_fit} by the
empirical cummulative distribution function in comparison to the
theoretical one.
\begin{figure}[htb!]
  %\vspace{-18mm}
  \centering
    \includegraphics[scale=0.4, clip=true, trim=92 229 116
    133]{../pics/nordea_15min_y_js_fit.pdf}
    \caption{\small \it Log-volatility residuals $y_t$ fitted to a
      Johnson Su distribution. Horizontal: values of $y_t$, denoted x;
      Vertical: $\ln\left(P(y_t < x)\right)$.}
    \label{fig:nordea_15min_y_js_fit}
\end{figure}

The first 4 moments of the Johnson Su distribution are expressible
in closed form in $\gamma, \delta, \lambda, \xi$ \cite{Shang2004}.
% \begin{eqnarray*}
%   w &=& \exp{1 \over \delta^2} \\
%   \Omega &=& {\gamma \over \delta} \\
%   \text{E}(y) &=& -w^{1/2} \lambda \sinh\Omega + \xi\\
%   \text{std}(y) &=& \lambda \left[{1 \over 2}(w-1)(w\cosh 2\Omega +
%     1)\right]^{1/2} \\
%   \text{skewness}(y) &=& {
%     \sqrt{(1/2)w(w-1)} [w(w+2)\sinh 3\Omega + 3\sinh\Omega]
%     \over
%     (w\cosh 2\Omega + 1)^{3/2}} \\
%   \text{kurtosis}(y) &=& {
%     w^2(w^4 + 2w^3 + 3w^2 - 3)\cosh 4\Omega + 4w^2 (w+2) \cosh 2\Omega
%     + 3(2w+1) \over
%     2(w\cosh 2\Omega + 1)^2 }
% \end{eqnarray*}
By matching the theoretical expressions of the moments with their
measured values, and taking help from published tables
\cite{Johnson1965}, one can solve for the parameters $\gamma, \delta,
\lambda, \xi$.

Under the assumption of i.i.d Johnson Su distributed residuals, the
log-likelihood function of the parameters $\theta, \Theta$
conditional on the sample $w_t$ can be written as
\[
L(\theta, \Theta) = -{1 \over 2}\sum_{t=1}^n z_t^2 + n \ln{\delta
  \over \lambda \sqrt{2\pi}} - {1 \over 2}\sum_{t=1}^n \ln\left[
  1 + \left({y_t - \xi \over \lambda}\right)^2
\right]
\]
where $y_t$ are inferred from $w_t$ using eq.\ref{eq:infer_y}
and $z_t$ from $y_t$ using
\[
z_t = \delta \sinh^{-1}{y - \xi \over \lambda} + \gamma
\]
Note that $\gamma, \delta, \lambda, \xi$ are not really free
parameters but rather are implied by $\theta$ and $\Theta$: Once the
latter have been chosen and the corresponding $y_t$ inferred, the
former are determined by the moments of $y_t$.

% The eventual MLE is done in Matlab with the ``active set''
% algorithm. The initial as well as the final estimation results are
% listed in table \ref{tab:nordea_15min_js_param}:
% \begin{table}[htb!]
%   \centering
%   \begin{tabular}{|c|c|c|c|c|c|c|}
%     \hline
%     & $\gamma$ & $\delta$ & $\lambda$ & $\xi$ & $\theta$ & $\Theta$ \\
%     \hline
%     initial estimate & 0.1476 & 1.5121 & 0.3633 & 0.0454 & 0.6890 &
%     0.6378 \\
%     \hline
%     MLE estimate & 0.1319 & 1.5266 & 0.3735 & 0.0410 & 0.6639 & 0.6025
%     \\
%     \hline
%   \end{tabular}
%   \caption{\small \it Nordea 15min estimation results}
%   \label{tab:nordea_15min_js_param}
% \end{table}

\subsection{Comparison of the Forecasts}
In this section we compare the one-step-ahead forecasts from the \gls{garch}
model and from the \gls{sv} model. For this purpose, we
compute the difference between a forecast $\ln \sigma^F_t$ and its measured
counterpart, i.e. the realized volatility of the same period $\ln
\hat{\sigma}_t$. As a reference, we also consider the results obtained
by taking the mean of the realized volatilities of the first 80\% of
the data set as forecast for the volatilities of the remaining 20\%. We
call this naive forecast the ``sample mean''.

First of all, we look at the means and standard deviations of $\ln \sigma^F_t -
\ln \hat{\sigma}_t$, which are listed in table \ref{tab:nordea_2012}.
\begin{table}[htb!]
  \centering
  \begin{tabular}{|c|c|c|c|}
    \hline
    & \gls{sv} & \gls{garch} & Sample mean \\
    \hline
    $\E(\ln \sigma^F_t - \ln \hat{\sigma}_t)$ & 0.0040 & -0.0008 &
    -0.2210 \\
    \hline
    $\text{std}(\ln \sigma^F_t - \ln \hat{\sigma}_t)$ & 0.2659 & 0.3011 &
    0.2893 \\
    \hline
  \end{tabular}
  \caption{\small \it Mean and standard Deviation of the forecasts'
    distribution}
  \label{tab:nordea_2012}
\end{table}
It is seen from table \ref{tab:nordea_2012} that, on average, the \gls{sv}
model over-estimates while \gls{garch} under-estimates. In terms of the
standard deviation of $\ln \sigma^F_t - \ln \hat{\sigma}_t$, the \gls{sv}
model wins with a small margin. In contrast, the sample mean forecast
clearly under-estimates the log-volatilities to a large extent --- the
efforts of building models has not been wasted.

Figure \ref{fig:nordea_2012} compares the 3 kinds of forecasts by
plotting the distribution function and the complementary distribution
function of $\ln \sigma^F_t - \ln \hat{\sigma}_t$. Here one can see
that the \gls{sv} model yields a better quality of forecasts than does \gls{garch} 
with respect to both under-estimates and over-estimates.

\begin{figure}[htb!]
  \centering
    \includegraphics[scale=0.55, clip=true, trim=40 296 16
    260]{../pics/nordea_2012.pdf}
  \caption{\small \it Blue: SV forecasts; Green: GARCH forecasts; Red:
    sample mean forecasts. Left: $P(\ln \sigma^F_t - \ln \hat{\sigma}_t < x)$;
    Right: $P(\ln \sigma^F_t - \ln \hat{\sigma}_t > x)$. Horizontal: $x$.}
  \label{fig:nordea_2012}
\end{figure}

Another measure of the forecasts' quality can be the percentage of good
forecasts, where the criterion of ``good'' is defined, respectively,
as the forecast lying within 1\%, 5\%, or 10\% of the corresponding
realized volatility. Table \ref{tab:nordea_2012_good} shows the
respective percentage of the 3 kinds of forecasts. Again in this table
it is seen that the \gls{sv} model gives more accurate forecasts than does
\gls{garch}. For example, defining a ``good'' forecast as one that lies
within 5\% of the realized volatility, the probability of obtaining
such a good forecast is 75\% using the \gls{sv} model, 71\% using the
\gls{garch} model, and only 53\% using the sample mean.
\begin{table}[htb!]
  \centering
  \begin{tabular}{|c|c|c|c|}
    \hline
    ${|\ln \sigma^F_t - \ln \hat{\sigma}_t| \over |\ln \hat{\sigma}_t|}$
    & \gls{sv} & \gls{garch} & sample mean \\
    \hline
    1\% & 22\% & 14\% & 11\% \\
    \hline
    5\% & 75\% & 71\% & 53\% \\
    \hline
    10\% & 97\% & 94\% & 89\% \\
    \hline
  \end{tabular}
  % \begin{tabular}{|c|c|c|c|}
  %   \hline
  %   ${\ln \sigma^F_t - \ln \hat{\sigma}_t \over |\ln \hat{\sigma}_t|}$
  %   & \gls{sv} & GARCH & sample mean \\
  %   \hline
  %   1\% & 21.73\% & 13.83\% & 11.36\% \\
  %   \hline
  %   5\% & 74.57\% & 70.62\% & 53.33\% \\
  %   \hline
  %   10\% & 97.28\% & 93.83\% & 89.14\% \\
  %   \hline
  % \end{tabular}
  \caption{\small \it The percentage of ``good'' forecasts when the
    criterion of being good is deviating no more than 1\%, 5\% or
    10\%.}
  \label{tab:nordea_2012_good}
\end{table}

\section{Case study: Volvo 30-minute Returns}
\label{sec:volvo_30}
In this section we model the log-volatility of {\it Volvo B} 30-minute
returns during the period 2013/10/10 - 2014/04/04. This series
contains 1884 log-volatilities computed using 2-minute returns in each
30-minute interval. Among them we use the first 1507 for model
estimation and the last 377 for forecast and model verification.

The left plot of figure \ref{fig:volvo_inno_and_lv_acf} shows the
distribution of $(r_t - \E(r_t))/\sigma_t$. We observe in the figure a
nice Gaussian variate, so we can be sure that the sum of squared
2-minute returns makes a good approximation to the variance of
30-minute returns in this particular case.
\begin{figure}[htb!]
  \centering
  \includegraphics[scale=0.4, clip=true, trim=0 256 0
  151]{../pics/volvo_inno_and_lv_acf.pdf}
  \caption{\small \it Left: probability plot of $(r_t -
    \E(r_t))/\sigma_t$; Right: auto-correlations of $\ln \sigma_t$}
  \label{fig:volvo_inno_and_lv_acf}
\end{figure}

Guided by the auto-correlations of $\ln\sigma_t$ shown in the right
plot of figure \ref{fig:volvo_inno_and_lv_acf} we find the following
model:
\begin{eqnarray}
  && (1 - B)(1 - B^s) \ln \sigma_t \nonumber \\
  &=& (1 - \theta_1 B - \theta_2B^2 -
  \theta_3B^3 - \theta_4B^4) (1 - \Theta B^s)
  y_t \label{eq:volvo_lv_model}
\end{eqnarray}
where $s = 16$ is the seasonality and is apparent from the
auto-correlations of $\ln\sigma_t$. Fitting this model to the
measured realized volatilities yields the parameter values listed in
table
\ref{tab:volvo_params}.
\begin{table}[htb!]
  \centering
  \begin{tabular}{|c|c|c|c|c|c|c|}
    \hline
    Parameter & $\theta_1$ & $\theta_2$ & $\theta_3$ & $\theta_4$ &
    $\Theta$ & residual variance \\
    \hline
    Value & 0.7305 & 0.0575 & 0.0574 & 0.0346 & 0.8324 & 0.1340\\
    \hline
  \end{tabular}
  \caption{\small \it Volvo B log-volatility parameters}
  \label{tab:volvo_params}
\end{table}
Forecasting using the estimated model parameters gives the forecast
series $\ln \sigma^{\text{SV}}_t$. To access the quality of the
forecast, we also estimate a \gls{garch} model using the returns. The result
is a \gls{garch}(1, 1) model, whose parameter values are listed in table
\ref{tab:volvo_garch}.
\begin{table}[htb!]
  \centering
  \begin{tabular}{|c|c|c|c|}
    \hline
    Parameter & $\alpha_0$ & $\alpha_1$  & $\beta_1$ \\
    \hline
    Value & $3.125 \times 10^{-7}$ & 0.05 & 0.90 \\
    \hline
  \end{tabular}
  \caption{\small \it GARCH(1, 1) model of Volvo B 30-minute returns}
  \label{tab:volvo_garch}
\end{table}

The forecasts from \gls{sv}, \gls{garch}, and the sample mean are compared
using the difference $\ln \sigma^F_t - \ln\hat{\sigma}_t$, where $\ln
\sigma^F_t$ stands for the forecast. The distributions of this
difference is plotted in figure \ref{fig:volvo_slv_garch_cmp}; the
mean and the standard deviation of the distributions are listed in
table \ref{tab:volvo_slv_garch_cmp}.
\begin{table}[htb!]
  \centering
  \begin{tabular}{|c|c|c|c|}
    \hline
    & \gls{sv} & \gls{garch} & Sample mean \\
    \hline
    $\E(\ln \sigma^F_t - \ln \hat{\sigma}_t)$ & -0.0123 &
    0.0242 & -0.1055 \\
    \hline
    $\text{std}(\ln \sigma^F_t - \ln \hat{\sigma}_t)$ & 0.3261 &
    0.4250 & 0.3708 \\
    \hline
  \end{tabular}
  \caption{\small \it Standard deviation of $\ln\sigma^F_t -
    \ln\hat{\sigma}_t$}
  \label{tab:volvo_slv_garch_cmp}
\end{table}

\begin{figure}[htb!]
  \centering
  \includegraphics[scale=0.5, clip=true, trim=10 282 0
  243]{../pics/volvo_slv_garch_cmp.pdf}
  \caption{\small \it Blue: SV forecasts; Green: GARCH forecasts; Red:
    sample mean forecasts. Left: $P(\ln \sigma^F_t - \ln
    \hat{\sigma}_t < x)$; Right: $P(\ln \sigma^F_t - \ln
    \hat{\sigma}_t > x)$. Horizontal: $x$.}
  \label{fig:volvo_slv_garch_cmp}
\end{figure}
Figure \ref{fig:volvo_slv_garch_cmp} shows, as in the previous case of
Nordea Bank 15-minute returns, the \gls{sv} model performs the best, \gls{garch}
the second, and the sample mean the worst. However, when it comes to
over-estimates, the sample mean appears to be the best estimator,
while \gls{sv} excels over \gls{garch}. But a check of $\E(\ln\sigma_t)$ over
the sample for estimation (1507 data points) and over the sample for
comparison (377 data points) reveals that the first sample has mean
-6.3127 while the second has -6.2072. This increment explains the low
probability of over-estimation when using the first sample mean as
forecast. 

Table \ref{tab:volvo_good_percentage} compares the fraction of
``good'' estimates as measured by ${|\ln\sigma^F_t -
  \ln\hat{\sigma}_t| \over |\ln \hat{\sigma}_t|}$ being less than 1\%,
5\% and 10\%.
\begin{table}[htb!]
  \centering
  \begin{tabular}{|c|c|c|c|}
    \hline
    ${|\ln \sigma^F_t - \ln \hat{\sigma}_t| \over |\ln
      \hat{\sigma}_t|}$ &
    \gls{sv} & \gls{garch} & Sample Mean \\
    \hline
    1\% & 22\% & 12\% & 14\% \\
    \hline
    5\% & 72\% & 62\% & 66\% \\
    \hline
    10\% & 92\% & 88\% & 90\% \\
    \hline
  \end{tabular}
  % \begin{tabular}{|c|c|c|c|}
  %   \hline
  %   ${|\ln \sigma^F_t - \ln \hat{\sigma}_t| \over |\ln
  %     \hat{\sigma}_t|}$ &
  %   \gls{sv} & \gls{garch} & Sample Mean \\
  %   \hline
  %   1\% & 21.75\% & 12.20\% & 14.06\% \\
  %   \hline
  %   5\% & 71.62\% & 61.54\% & 66.31\% \\
  %   \hline
  %   10\% & 92.31\% & 88.06\% & 89.92\% \\
  %   \hline
  % \end{tabular}
  \caption{\small \it Fraction of ``good'' forecasts as defined by
    ${|\ln \sigma^F_t - \ln \hat{\sigma}_t| \over |\ln
      \hat{\sigma}_t|}$ being less than 1\%, 5\% and 10\%.}
  \label{tab:volvo_good_percentage}
\end{table}
We see from the table that the \gls{sv} model consistently excels over
the other two alternatives. In addition, it is also noted that the
\gls{garch} forecast is even worse than the sample mean. This is surprising
but not totally unexpected --- with only 3 parameters, the \gls{garch}(1,1)
model can only describe the most prominent auto-correlations in the
volatility. When the volatility is influenced by relatively weak
auto-correlations at several different time lags, the \gls{garch} forecast
cannot be expected to have good accuracy.

In this particular case, we see that the \gls{arima} model has 3 relatively
small moving average coefficients, located at lags 2, 3, and 4 and
evaluated to 0.06, 0.06, 0.03, suggesting a scattered auto-correlation
structure, so the \gls{garch} model cannot be expected to perform very
well. In contrast, the Volvo 15-minute returns studied in
section \ref{sec:volvo} has more concentrated auto-correlations --- 0.12 and
0.06 at lags 2 and 3 --- thus the \gls{garch}(1,1) model is also found to
perform better and even marginally better than the \gls{sv} model for the
same series.

\section{Unconditional Distribution Functions of SV Models}
\label{sec:XieCalc}
In this section we study the unconditional distribution function of
the \gls{sv} model specified as equation
\ref{eq:SLV_spec}. As is discussed at the beginning of this chapter,
$\ln\sigma_t$ can be described by an \gls{arma} or \gls{arima} model,
possibly with seasonal components. Here we note that all these models
can be re-written as a moving average model, which is infinite in extent if
autoregressive components are present:
\begin{eqnarray*}
  \ln \sigma_t &=& y_t + \sum_{n=1}^\infty c_n y_{t-n} + \text{Const.}
\end{eqnarray*}
Since the $y_t$ are independent and identically distributed,
\begin{eqnarray*}
  y_t + \sum_{n=1}^\infty c_n y_{t-n}  
\end{eqnarray*}
has Gaussian distribution by the central limit theorem, on condition
that $y_t$ for all $t$ have finite second moment --- this is what we
assume in the rest of this section. It follows that the unconditional
distribution of $\ln \sigma_t$ is the same as the distribution of
$\bar{v} + v$ where $v \sim N(0, \sigma)$ and $\bar{v}$, $\sigma$ are
constants. Now we can state that the unconditional distribution of the
returns $r_t$
\begin{eqnarray*}
  r_t &=& \mu + \sigma_t b_t\\
  &=& \mu + b_t \exp\left(
    y_t + \sum_{n=1}^\infty c_n y_{t-n} + \text{Const.}
  \right)
\end{eqnarray*}
 is the same as
\begin{equation}  \label{eq:UnconditionalPdf}
  \begin{aligned}
    r &= \mu + e^{\bar{v} + v} b \\
  \end{aligned}
\end{equation}
where $b \sim N(0, 1)$. For convenience, let $r' = e^v b$

In section \ref{sec:SLV_Symmetric} we first study the model in the
relatively simple case when $v$ and $b$ are uncorrelated and $\mu
= 0$. If this simplified version proves inadequate, one may resort to
the general model studied in section \ref{sec:SLV_Asymmetric}.

\subsection{The Simplified model}\label{sec:SLV_Symmetric}
In the following we derive the unconditional \gls{pdf} of $r'$, the
de-meaned and rescaled return. Denote this \gls{pdf} $f_{r'}(x)$. Then
the \gls{pdf} of $r$ is $e^{-\bar{v}}f_{r'}(e^{-\bar{v}}x)$. First we
consider
\begin{eqnarray*}
  P(r' < x) &=& P(b < xe^{-v}) \\
  f_{r'}(x) &=& f_b(xe^{-v}) e^{-v}
\end{eqnarray*}
Averaging over all $v$, we get
\begin{equation}\label{eq:UncondPDFSymmetric}
  \begin{aligned}
    f_{r'}(x) =& \int_{-\infty}^{\infty} dv (2\pi\sigma^2)^{-1/2}
    e^{-v^2/2\sigma^2}(2\pi)^{-1/2} \exp(-x^2e^{-2v}/2) e^{-v} \\
    =& {1 \over 2\pi\sigma} \int_{-\infty}^{\infty} dv
    \exp\left(
      -{1 \over 2\sigma^2} v^2 - v -{1 \over 2} x^2 e^{-2v}
    \right)
  \end{aligned}
  \end{equation}
The last part of the integrand, $e^{-x^2 e^{-2v} / 2}$, is plotted in
figure \ref{fig:DoubleExp}.
\begin{figure}[htb!]
  \centering
  \includegraphics[scale=0.5, clip=true, trim=85 252 100
  231]{../pics/DoubleExp.pdf}
  \caption{\small \it Plot of $\exp(-{1 \over 2} x^2 e^{-2v})$}
  \label{fig:DoubleExp}
\end{figure}
Therefore we make the following approximation:
\[
\exp\left(-{1 \over 2} x^2 e^{-2v}\right) \approx \left\{
  \begin{array}{lr}
    0 & \text{if } v < \ln|x| -{1 \over 2} \ln(2\ln 2) \\
    1 & \text{otherwise}
  \end{array}
\right.
\]
Here we note that $\exp\left(-{1 \over 2} x^2 e^{-2v}\right) = 1/2$ at
$v = \ln|x| -{1 \over 2} \ln(2\ln 2)$.

With this approximation we have
\begin{eqnarray*}
  f_{r'}(x) &=& {1\over C}{1 \over 2\pi\sigma} \int_{\ln(|x|/\sqrt{\ln
      4})}^{\infty} dv
  \exp\left(-{1 \over 2\sigma^2} v^2 - v\right) \\
  &=& {1\over C}{e^{\sigma^2 / 2} \over \sqrt{8\pi}} \text{erfc} \left(
    {1 \over \sqrt{2}\sigma} \ln{|x| \over \sqrt{\ln 4}} + {\sigma
      \over \sqrt{2}}
  \right)
\end{eqnarray*}
where $1/C$ has been added for the purpose of normalization.

At large $x$, we may use the asymptotic expansion of $\mathrm{erfc}$
to write
\begin{eqnarray*}
  Cf_{r'}(x) &=& {e^{\sigma^2 / 2} \over \sqrt{8\pi}} \mathrm{erfc}(\xi) \\
  &=& {e^{\sigma^2 / 2} \over \sqrt{8 \pi}}
  \frac{e^{-\xi^2}}{\xi\sqrt{\pi}}\left[
    1 +
    \sum_{n=1}^N (-1)^n \frac{(2n-1)!!}{(2\xi^2)^n} \right] +
  O(\xi^{-2N-1} e^{-\xi^2})
\end{eqnarray*}
where $\xi = {1 \over \sqrt{2}\sigma} \ln{|x| \over \sqrt{\ln 4}} +
{\sigma \over \sqrt{2}}$. The slowest-decaying term is
\[
f_{r,0}(x) = {e^{\sigma^2 / 2} \over \sqrt{8\pi}}
\frac{e^{-\xi^2}}{\xi\sqrt{\pi}}
\]
Let $\zeta = {|x| \over \sqrt{\ln 4}}$. With a bit manipulation one
obtains
\begin{equation*}
  f_{r,0}(x) = {1 \over \pi \sqrt{8}}{1 \over
    \left(\ln\zeta/\sigma\sqrt{2} +
      \sigma/\sqrt{2}\right)\zeta\zeta^{\ln \zeta / 2\sigma^2}
  }
\end{equation*}
From the last equation one can see that, at any neighborhood of large
$x$, $f_{r'}(x)$ may be approximated by $C/|x|^\alpha$, i.e. a power law.
The normalization constant $C$ is calculated in appendix
\ref{chp:symmetric_SV_norm_const} to be $C = \sqrt{2 \ln 4\over
  \pi}$. So, in summary, we can write:
\begin{eqnarray*}
  f_r(x) &=& e^{-\bar{v}} f_{r'}(e^{-\bar{v}} x) \\
    &=& {e^{-\bar{v}} \over C}{e^{\sigma^2 / 2} \over \sqrt{8\pi}}
    \text{erfc} \left({1 \over \sqrt{2}\sigma} \ln{|e^{-\bar{v}}x| \over \sqrt{\ln
          4}} + {\sigma \over \sqrt{2}}
    \right) \\
\end{eqnarray*}
Using the same technique for integration as for normalization, the
cummulative distribution function of $r$ is found to be $F(x)$,
which is the following:
\begin{enumerate}
\item if $x < 0$
  \begin{eqnarray*}
    F(x) &=& {\sqrt{\ln 4} \over C}{e^{\sigma^2 / 2} \over \sqrt{8\pi}}
    \left[
      {e^{-\bar{v}}x \over \sqrt{\ln 4}}\text{erfc}\left(
        {1 \over \sigma \sqrt 2} \ln{-e^{-\bar{v}}x \over \sqrt{\ln
            4}} + {\sigma \over \sqrt 2}
      \right) \right.\\
      && \left. + e^{-\sigma^2 / 2} \text{erfc}\left(
        {1 \over \sigma \sqrt 2} \ln{-e^{-\bar{v}}x \over \sqrt{\ln 4}}
      \right)
    \right]
  \end{eqnarray*}
\item if $x \geq 0$
  \begin{eqnarray*}
    F(x) &=& \frac{1}{2} + {\sqrt{\ln 4} \over C}{e^{\sigma^2 / 2} \over
      \sqrt{8\pi}} \left[
      {e^{-\bar{v}}x \over \sqrt{\ln 4}} \text{erfc}\left(
        {1 \over \sigma \sqrt 2} \ln{e^{-\bar{v}}x \over \sqrt{\ln
            4}} + {\sigma \over \sqrt 2}
      \right) \right. \\
      && \left. + e^{-\sigma^2/2} \text{erfc}\left(
        -{1 \over \sigma \sqrt 2} \ln{e^{-\bar{v}}x \over \sqrt{\ln 4}}
      \right)
    \right]
  \end{eqnarray*}
\end{enumerate}

To verify the validity of the model, we fit the above probability
density function to the de-meaned 30min returns of Volvo B
\footnote{By ``de-meaned returns'' we mean the quantity $r_t - 
  \mean{r_t}$, where $r_t$ are the measured returns and $\mean{r_t}$
  is the sample mean. The data set covers the transaction records of
  Volvo B on the OMX market (Stockholm) between 2013-10-10 and
  2014-03-12. The returns are computed using 1-minute mean prices.}
by means of \gls{mle} using the MATLAB function ``mle''. Then for the
parameters $\sigma$ and $\bar{v}$ we get
\begin{eqnarray*}
  \sigma &=& 0.6355 \\
  \bar{v} &=& -6.0308
\end{eqnarray*}
Then we plot $P(r' > x)$ of the model against its empirical
counterpart on a log-log scale, as shown in figure
\ref{fig:volvo_30min_ret}.
\begin{figure}[htb!]
  % \vspace{-15mm}
  \begin{center}
    \includegraphics[scale=0.4, clip=true, trim=98 231 116
    126]{../pics/volvo_30min_ret.pdf}
  \end{center}
  %\vspace{-5mm}
  \caption{\small \it{$P(r' > x)$ for $x > 0$. Blue: empirical
      probabilities. Red: probabilities predicted by the model.}}
  \label{fig:volvo_30min_ret}
\end{figure}

In general figure \ref{fig:volvo_30min_ret} shows a good fit, but a
closer look reveals that deviations are significant in the regions $r
\in (0, \sigma_r)$ and $r \in (2\sigma_r, 3\sigma_r)$, where
$\sigma_r$ stands for the empirical standard deviation of the
de-meaned returns. These observations are
shown in greater details in figure \ref{fig:volvo_30min_ret2}.
\begin{figure}[htb!]
  \centering
  \subfigure[$\ln(P(r > x))$ with $x \in (0, \sigma_r)$]{
    \includegraphics[scale=0.4, clip=true, trim=93 224 115
    125]{../pics/volvo_30min_ret_0-sigma.pdf}
  }
  \subfigure[$\ln(P(r > x))$ with $x \in (2\sigma_r, 3\sigma_r)$]{
    \includegraphics[scale=0.4, clip=true, trim=93 224 115
    125]{../pics/volvo_30min_ret_2sigma-3sigma.pdf}
  }
  \caption{\small \it Surviving probabilities. Blue: empirical values of $\ln(P(r
    > x))$. Red: Predicted values of $\ln(P(r > x))$.}
  \label{fig:volvo_30min_ret2}
\end{figure}

Moreover, the inefficiency of the model is also manifest in the
skewness of the data. For the Volvo 30min returns, the data has a
skewness of 0.2419, but our model is strictly symmetric, since the
return $x$ appears in equation \ref{eq:UncondPDFSymmetric} only as $x^2$.
Hence the above model needs to be improved to accommodate the non-zero
skewness as well as to account for the discrepancies shown in figure
\ref{fig:volvo_30min_ret2}. This is the subject of the next section.

\subsection{The General model}\label{sec:SLV_Asymmetric}
It has long been hypothesized in the liturature that skewness is the
result of price-volatility correlation (for example
\cite{Potters2003}). Therefore, the most apparent modification is to
allow $v$, the zero-mean Gaussian variate in the log-volatility, and
$b$, the zero-mean Gaussian variate in the innovation of the return,
to be correlated. For convenience we decompose $v \sim N(0, \sigma)$
as $v = \sigma a$ and assume
\begin{eqnarray*}
  \begin{pmatrix}
    a \\
    b
  \end{pmatrix} \sim N(0, \Sigma)
\end{eqnarray*}
with a covariance matrix
\begin{eqnarray*}
    \Sigma &=&
  \begin{pmatrix}
    1 & \psi \\
    \psi & 1
  \end{pmatrix}
\end{eqnarray*}
where $|\psi| < 1$. Then we rewrite equation \ref{eq:UnconditionalPdf}
as
\begin{equation}
  \label{eq:r_t}
  \begin{aligned}
    r &= \mu + e^{\bar{v}} r' \\
    r' &= e^{\sigma a} b \\
  \end{aligned}
\end{equation}
We then get
\begin{eqnarray*}
  && P(r' < x)\\
  &=& P(b < e^{-a\sigma}x) \\
  &=& {1 \over 2\pi |\det(\Sigma)|^{1/2}}
  \int_{-\infty}^{\infty} da \int_{-\infty}^{e^{-a\sigma}x} db
  \exp\left[
    -{1 \over 2} (a, b) \Sigma^{-1}
    \begin{pmatrix}
      a \\
      b
    \end{pmatrix}
  \right] \\
\end{eqnarray*}
After some manipulations we get
\begin{equation}\label{eq:UncondCDFAsymmetric}
  \begin{aligned}
    & P(r' < x) \\  
    &= {1 \over 2\pi \sqrt{1 - \psi^2}} \int_{-\infty}^{\infty} da
    e^{-a^2/2} \int_{-\infty}^{e^{-a\sigma}x} db
    \exp\left[
      - {(b - a\psi)^2 \over 2(1 - \psi^2)}
    \right] \\
    &= {1 \over 2\sqrt{2\pi}} \int_{-\infty}^{\infty} e^{-a^2/2}
    \text{erfc}{a\psi - e^{-a\sigma} x \over \sqrt{2(1-\psi^2)}} da
  \end{aligned}
\end{equation}
Differentiating with respect to $x$ yields the \gls{pdf}
\begin{eqnarray}
  f_{r'}(x) &=& {1 \over 2\pi \sqrt{1 - \psi^2}} \times \nonumber \\
  && \int_{-\infty}^\infty da \exp
  \left[- {
      a^2 + 2\sigma(1 - \psi^2)a - 2a\psi e^{-a\sigma} x + e^{-2a\sigma} x^2
      \over
      2(1 - \psi^2)
  }
  \right] \label{eq:UncondPDFAsymmetric} \\
  f_r(x) &=&  \td{r'}{r} f_{r'}[e^{-\bar{v}} (x - \mu)] \nonumber \\
  &=& e^{-\bar{v}} f_{r'}[e^{-\bar{v}}
  (x-\mu)] \label{eq:UncondPDFAsymmetric1}
\end{eqnarray}
Unlike \ref{eq:UncondPDFSymmetric} where a relatively simple
approximation can be found and thus leads to an analytic result of the
integral, no such approximation has been found by the
author for the integral \ref{eq:UncondPDFAsymmetric}. However, the
\gls{mgf} of \gls{pdf} \ref{eq:UncondPDFAsymmetric}
and \ref{eq:UncondPDFAsymmetric1} are easy enough to find and give the
moments about the origin in closed form. Then by matching the analytic
expressions of the moments with their statistical values from the
sample, the parameters $\sigma$, $\psi$ and $\bar{v}$ corresponding to the
sample can be obtained.

The \gls{mgf} of \ref{eq:UncondPDFAsymmetric1} can be found as follows:
\begin{eqnarray*}
  && M_{r}(t) \\
  &=& \E(e^{tr}) \\
  &=& {1 \over 2\pi \sqrt{1 - \psi^2}} \int_{-\infty}^{\infty} da \exp \left(
    -{a^2 \over 2} - \sigma a
  \right) \int_{-\infty}^{\infty} dx \exp\left[
    t(e^{\bar{v}}x + \mu) - {
      (a\psi - e^{-a \sigma} x)^2 \over 2 (1 - \psi^2)
    }
  \right] \\
  &=& {1 \over \sqrt{2 \pi}} \int_{-\infty}^{\infty} da \exp\left[
    -{a^2 \over 2} + {1 \over 2} e^{2 a \sigma + 2\bar{v}} (1 - \psi^2) t^2 +
    a \psi e^{a \sigma + \bar{v}} t + \mu t
  \right]
\end{eqnarray*}
With the \gls{mgf}, the first 4 moments of $r$ can be computed:
\begin{equation}
  \label{eq:AsymmetricMoments}
  \begin{aligned}
    \E(r) &= \mu +e^{\bar{v}+\frac{\sigma ^2}{2}} \sigma  \psi\\
    \E(r^2) &= \mu ^2+2 e^{\bar{v}+\frac{\sigma ^2}{2}} \mu  \sigma  \psi
    +e^{2 \left(\bar{v}+\sigma ^2\right)} \left(1+4 \sigma ^2 \psi^2\right)\\
    \E(r^3) &= \mu ^3+3 e^{\bar{v}+\frac{\sigma ^2}{2}} \mu ^2 \sigma \psi
    +9 e^{3 \bar{v}+\frac{9 \sigma ^2}{2}} \sigma \psi \left(1+3 \sigma ^2
      \psi ^2\right)
    +3e^{2 \left(\bar{v}+\sigma ^2\right)} \mu  \left(1+4 \sigma ^2 \psi
      ^2\right) \\
    \E(r^4) &= \mu ^4+4 e^{\bar{v}+\frac{\sigma ^2}{2}} \mu ^3 \sigma  \psi
    +36 e^{3 \bar{v}+\frac{9 \sigma ^2}{2}} \mu  \sigma  \psi  \left(1+3
      \sigma ^2 \psi ^2\right)\\ 
    & +6 e^{2 \left(\bar{v}+\sigma ^2\right)} \mu ^2 \left(1+4 \sigma ^2 \psi
      ^2\right)
    +e^{4 \bar{v}+8 \sigma ^2} \left(3+96 \sigma ^2 \psi ^2+256 \sigma ^4
      \psi ^4\right)
  \end{aligned}
\end{equation}
These equations are rather complicated and directly solving them to
obtain the parameters $\mu$, $\sigma$, $\psi$ and $\bar{v}$ is
infeasible. However, in practice, $\E(r)$ is often very small --- so, to
get a rough estimate of the parameters, we may set $\mu = 0$ in the
above equations and, with a bit of manipulation, find the following
equation for $\psi\sigma$:
\begin{equation}
  \label{eq:moment1}
  {\E(r^3) \E(r^4)^{-3/8} \over \E(r^2)^{3/4}} =
  {9\sigma \psi (1 + 3\sigma^2\psi^2) \over (1 +
    4\sigma^2\psi^2)^{3/4}}
  {1
    \over
    (3 + 96\sigma^2\psi^2 + 256\sigma^4 \psi^4)^{3/8}
  }
\end{equation}
This equation can be solved numerically for a given sample to yield an
estimate for $\sigma\psi$, which in turn can be substituted in
equations \ref{eq:AsymmetricMoments}, where $\mu$ has been set to 0,
to give estimates for all the 4 parameters. These estimates then serve
as initial values in a numerical solution to
\ref{eq:AsymmetricMoments} where $\mu$ is kept as a free
variable. This full solution can now be used as the initial estimate
in an \gls{mle} procedure.

It is particularly important in our situation to have a good initial
estimate, because, as has been shown earlier, the \gls{pdf} and the
\gls{cdf} cannot be obtained in closed forms and consequently have to
be evaluated by numerical integration. This is a rather costly
procedure especially in the context of \gls{mle}. So the computation of the
moments under the assumption of $\mu = 0$ is worthwhile.

Following the aforementioned procedure, i.e. computing the initial
estimate by matching the moments and then refining the estimate by
\gls{mle}, we obtain the parameter values for a number of return series,
including the Volvo B 30-minute returns described in
section \ref{sec:SLV_Symmetric}. Table \ref{tab:assets_params} and
\ref{tab:assets_moments} present the obtained parameter values and the
resulting moments, respectively. Figure \ref{fig:Volvo_B_30m_returns},
\ref{fig:Nordea_30m_returns}, and \ref{fig:Ericsson_30m_returns}
compare the empirical distribution functions with the analytic
distribution functions evaluated at these parameters.
\begin{figure}[htb!]
  \centering
  \includegraphics[scale=0.5, clip=true, trim=50 233 63
  136]{../pics/Volvo_B_30m_returns.pdf}
  \caption{\small \it Volvo B 30min returns' unconditional distribution fit to
    the SV model. The time series runs from
    2013/10/10 to 2014/03/12. Left: $P(r < x)$ with $x < 0$; Right:
    $P(r > x)$ with $x > 0$. Blue: empirical CDF obtained from data;
    Green: model CDF computed using equation
    \ref{eq:UncondCDFAsymmetric} with $x$ replaced by
    $(x-\mu)e^{-\bar{v}}$. Both plots are on log-log scale.}
  \label{fig:Volvo_B_30m_returns}
\end{figure}
\begin{figure}[htb!]
  \centering
  \includegraphics[scale=0.5, clip=true, trim=65 241 72
  143]{../pics/Nordea_30m_returns.pdf}
  \caption{\small \it Nordea Bank 30min returns' unconditional
    distribution fit to the SV model. The time
    series runs from 2013/10/10 to 2014/03/12. Left: $P(r < x)$ with
    $x < 0$; Right: $P(r > x)$ with $x > 0$. Blue: empirical CDF
    obtained from data; Green: model CDF computed using equation
    \ref{eq:UncondCDFAsymmetric} with $x$ replaced by
    $(x-\mu)e^{-\bar{v}}$. Both plots are on log-log scale.}
  \label{fig:Nordea_30m_returns}
\end{figure}
\begin{figure}[htb!]
  \centering
  \includegraphics[scale=0.5, clip=true, trim=21 241 28
  146]{../pics/Ericsson_30m_returns.pdf}
  \caption{\small \it Ericsson B 30min returns' unconditional
    distribution fit to the SV model. The time
    series runs from 2013/10/10 to 2014/03/12. Left: $P(r < x)$ with
    $x < 0$; Right: $P(r > x)$ with $x > 0$. Blue: empirical CDF
    obtained from data; Green: model CDF computed using equation
    \ref{eq:UncondCDFAsymmetric} with $x$ replaced by
    $(x-\mu)e^{-\bar{v}}$. Both plots are on log-log scale.}
  \label{fig:Ericsson_30m_returns}
\end{figure}
\begin{table}[htb!]
  \centering
  \begin{tabular}{|c|c|c|c|c|}
    \hline
    & $\psi$ & $\sigma$ & $\bar{v}$ & $\mu$ \\
    \hline
    Volvo B & -1.6$\times 10^{-2}$ & 4.7$\times 10^{-1}$ & -6.2&
    -5.3$\times 10^{-5}$ \\
    Nordea Bank & 1.7$\times 10^{-2}$ & 4.3$\times 10^{-1}$ & -6.4 &
    7.5$\times 10^{-5}$ \\
    Ericsson B & 1.8$\times 10^{-2}$ & 4.5$\times 10^{-1}$ & -6.3 &
    -9.3$\times 10^{-5}$ \\
    \hline
  \end{tabular}
  % \begin{tabular}{|c|c|c|c|c|}
  %   \hline
  %   & $\psi$ & $\sigma$ & $\bar{v}$ & $\mu$ \\
  %   \hline
  %   Volvo B & -1.5747e-02 & 4.6929e-01 & -6.2304e+00 &
  %   -5.3040e-05 \\
  %   Nordea Bank & 1.7234e-02 & 4.2780e-01 & -6.3758e+00 & 7.5230e-05
  %   \\
  %   Ericsson B & 1.8486e-02 & 4.4905e-01 & -6.2834e+00 & -9.2817e-05 \\
  %   \hline
  % \end{tabular}
  \caption{\small \it Parameter Values of Selected Assets'
    returns. The time series are 30-minute returns and run from
    2013/10/10 to 2014/03/12.}
  \label{tab:assets_params}
\end{table}
\begin{table}[htb!]
  \centering
  \begin{tabular}{|c|c|c|c|c|}
    \hline
    & mean & std & skewness & kurtosis \\
    \hline
    \multirow{2}{*}{Volvo B} & -8.7$\times 10^{-5}$ & 2.5$\times 10^{-3}$ &
    -2.7$\times 10^{-2}$ & 7.3 \\
    & -6.9$\times 10^{-5}$ & 2.5$\times 10^{-3}$ & -7.3$\times
    10^{-2}$ & 7.2 \\
    \hline
    \multirow{2}{*}{Nordea Bank} & 7.5$\times 10^{-5}$ & 2.0$\times 10^{-3}$ &
    6.9$\times 10^{-2}$ & 6.7 \\
    & 8.9$\times 10^{-5}$ & 2.0$\times 10^{-3}$ &
    6.7$\times 10^{-2}$ & 6.2 \\
    \hline
    \multirow{2}{*}{Ericsson B} & -8.3$\times 10^{-5}$ & 2.3$\times 10^{-3}$ &
    2.7$\times 10^{-1}$ & 7.7 \\
    & -7.6$\times 10^{-5}$ & 2.3$\times 10^{-3}$ &
    7.9$\times 10^{-2}$ & 6.7 \\
    \hline
  \end{tabular}
  \caption{\small \it Moments of Selected Assets' returns. For each return
    series, the 1st row contains the sample moments, while the 2nd
    constains those computed with MLE parameters. The time series are
    30-minute returns and run from 2013/10/10 to 2014/03/12.}
  \label{tab:assets_moments}
\end{table}

Clearly these figures show a fairly good match of the empirical and
the analytic distribution functions. However, we also see that the
skewness of the returns of Volvo B and Ericsson B have rather
different values when computed from the sample and from the
model. This could be the consequence of the limited sample size or
deficiencies in the estimation procedure described above. We leave
these issues to future studies.

%  This indicates the parameter $\psi$, which is the
% correlation between the log-volatility and the return, may be more
% complicated than the simple constant that has been assumed so far. We
% leave this to later studies.

\subsection{Relation to Conditional Distribution Functions}
\label{sec:SV_Conditional}
In the last section we have shown that the unconditional distribution of
the returns are skewed, which implies the zero-mean Gaussian variate $a$ in the
log-volatility is correlated to the zero-mean Gaussian variate $b$ in
the return (c.f. equation \ref{eq:UnconditionalPdf}).

In the context of conditional distributions and forecast, this
correlation translates to the correlation between the residual of the
log-volatility, denoted $y_t$ in section \ref{chp:nordea_15min} and
section \ref{sec:volvo}, and $b_t$. Now let us consider the forecast
function of an \gls{arima} model:
\begin{eqnarray*}
  \ln \sigma_t &=& y_t + \sum_{i=1}^P \phi_i \ln \sigma_{t-i} -
  \sum_{i=1}^Q \theta_i y_{t-i}
\end{eqnarray*}
where $\phi_i < 2$ if the model involves integration, and $\phi_i <
1$ otherwise. Comparing this equation with equation \ref{eq:r_t}, one
immediately realizes that the conditional \gls{pdf} of $r_t$
\begin{eqnarray*}
  r_t &=& \mu + \sigma_t b_t \\
  &=& \mu + \exp\left(
    y_t + \sum_{i=1}^P \phi_i \ln \sigma_{t-i} -
    \sum_{i=1}^Q \theta_i y_{t-i}
  \right) b_t
\end{eqnarray*}
is given by equation \ref{eq:UncondPDFAsymmetric1} and its moments
given by the equations \ref{eq:AsymmetricMoments} if one makes the
following substitutions:
\begin{eqnarray*}
  \sigma &\to& \text{std}(y_t) \\
  \bar{v} &\to& \sum_{i=1}^P \phi_i \ln \sigma_{t-i} - \sum_{i=1}^Q
  \theta_i y_{t-i} \\
  \psi &\to& \text{corr}(y_t, b_t)
\end{eqnarray*}

For the Nordea series considered in section \ref{chp:nordea_15min},
$\text{corr}(y_t, b_t)$ is found to be $3.56 \times 10^{-2}$; while
for the Volvo series considered in section \ref{sec:volvo},
$\text{corr}(y_t, b_t)$ is found to be $-7.3 \times 10^{-3}$. These
values are comparable to those in table \ref{tab:assets_params}, where
$\psi$ is $1.7 \times 10^{-2}$ for the Nordea series and $-1.6
\times 10^{-2}$ for the Volvo series. The similarity in these values
provides some evidence about the validity of the model.

% It is certainly impossible to compare a predicted conditional
% distribution with observations, but conditional distributions are of
% great interest in the context of risk management and derivative
% pricing, so we point out how they may be calculated for ARIMA
% log-volatility models.

\chapter{Covariance Matrix of Gaussian Returns}
\label{chp:Gaussian}
In chapter \ref{chp:PriceModels} we have studied \gls{garch} and
\gls{sv} models and seen their power of forecasting future
volatilities. However, we have not considered the important fact that
a financial market comprises many assets and the volatilities of these
assets are correlated to each other in a complicated
manner. Practically useful volatility forecasts require good
understanding of these correlations.

In the literature, covariance matrices of Gaussian and L\'evy
distributed returns have been studied (see e.g. \cite{politi2010,
  Chiani2012, Lalley2013}). However, as is seen in chapter
\ref{chp:PriceModels}, real-world returns are not described by any
particular distribution but rather by stochastic processes that
account for auto-correlations in the returns and the volatilities.

Therefore, the focus of this and the next chapter is on covariance
matrices of realistic return series, and especially covariance matrices
in the case when the return series have considerable
auto-correlations. In particular, we study covariance matrices of
\gls{garch} (1,1) return series in chapter
\ref{chp:CrossCorrelationFat} and show the influence of
auto-correlations on these matrices. But before that, it is useful to
first understand the influence of auto-correlations on covariance
matrices of Gaussian return series. When auto-correlations are absent,
these matrices are called Wishart matrices and have been studied
extensively. The results we obtain in this chapter will provide a
reference to the studies in chapter \ref{chp:CrossCorrelationFat}.


% In this chapter we present some results about the covariance
% matrix of Gaussian returns. The assumption of Gaussian returns is of
% course an over-simplification of reallity, but nevertheless lends some
% insight into the problem of elements' and eigenvalues' distributions of
% a covariance matrix.
% Our primary interest is in the influence that auto-correlations in
% returns exert on the covariance matrix.

In section \ref{sec:GCC-analytical} we discuss how the distributions
of the matrix elements are affected by autocorrelations, and in
section \ref{sec:GCC-numerical} we investigate the distribution of the
eigenvalues.


\section{Distribution of the Matrix Elements}
\label{sec:GCC-analytical}
In this section we study the distribution of the elements of a
covariance matrix $C = \mtx{RR'}/T$, where
\begin{eqnarray*}
  \mtx R &=&
  \begin{pmatrix}
    r_{11} & r_{12} & \cdots & r_{1T} \\
    r_{21} & r_{22} & \cdots & r_{2T} \\
    \vdots & \vdots & \ddots & \vdots \\
    r_{N1} & r_{N22} & \cdots & r_{NT} \\
  \end{pmatrix}
\end{eqnarray*}
and $\mtx R'$ denotes the transpose of $\mtx R$.
In words, an element $r_{it}$ of $\mtx R$ is the return of asset $i$
at time $t$, with $i=1,\cdots, N$ and $t = 1, \cdots, T$. If each
column of $\mtx R$ follows a zero-mean Gaussian distribution, i.e.
\begin{eqnarray*}
  \begin{pmatrix}
    r_{1t} \\
    \vdots \\
    r_{Nt}
  \end{pmatrix} \sim N(0, \mtx \Sigma)
\end{eqnarray*}
for all $t = 1, \cdots, T$, and none of the return series is
auto-correlated, i.e. $\text{corr}(r_{it}, r_{i,t'}) = 0$ for all $i =
1, \cdots, N$ and $t \neq t'$, then $\mtx{RR'}$ is a Wishart matrix whose
probability density function is well known \cite{Anderson2003}.

When auto-correlations are indeed present in the returns, $\mtx{RR'}$
no longer follows the Wishart distribution. However, the joint distribution
function of its elements can be expressed in terms of the Wishart \gls{pdf}
and the auto-correlations. We show this in appendix
\ref{app:pdf_gaussian1}. Also, in appendix
\ref{chp:gaussian_elements_dist} we derive an approximate expression 
for the asymptotic distribution of these matrix elements, assuming
$r_{it}$ is an \gls{ar}(1) process \footnotemark, i.e. $r_{it} =
\phi r_{i, t-1} + a_{it}$, and
\footnotetext{
  It is straight forward to derive the auto-correlation function
  $\varrho_k$ of an \gls{ar}(1) process with autoregressive
  coefficient $\phi$:
  \begin{eqnarray*}
    \varrho_k &=& \text{corr}(r_{it}, r_{i, t-k}) \\
    &=& \phi \varrho_{k-1} \\
    &=& \phi^k
  \end{eqnarray*}
  Let $\tau$ denote the time lag at which $\varrho_{\tau} = 1/2$, then
  it follows $\tau = -\ln 2/\ln \phi$.
}
\begin{eqnarray*}
  \begin{pmatrix}
    a_{1t} \\
    \vdots \\
    a_{Nt}
  \end{pmatrix} \sim N(0, \mtx \Sigma)
\end{eqnarray*}
where
\begin{eqnarray*}
  \mtx \Sigma &=& \sigma^2
  \begin{pmatrix}
    1 & \rho & \cdots & \rho \\
    \rho & 1 & \cdots & \rho \\
    \vdots & \vdots & \ddots & \vdots \\
    \rho & \rho & \cdots & 1
  \end{pmatrix}
\end{eqnarray*}
where $-1 < \rho < 1$ is a constant parameter describing the correlation
between $a_{it}$ and $a_{jt}$ --- so, by construction, we assume such
correlations are constant across all asset pairs and over all
time.

The elements $C_{ij}$ ($i \neq j$) of the covariance matrix, $C
= \mtx{RR'}/T$, are found to be normally distributed with mean
\begin{eqnarray}
  \mu'_X &=& {\sigma^2 \over \sqrt{2\pi} (1 - \phi^2)(1 -
    \rho^2)^{1/4}} \left[ P^{-3/2}_{-1/2}(-\rho) -
    P^{-3/2}_{-1/2}(\rho)
  \right] \label{eq:gaussian_mean2}
\end{eqnarray}
and variance
\begin{eqnarray}
  \sigma'^2_X &=& {1 \over (1 - \phi^2)^2}\left[
    \sum_{t=1}^T \sum_{k=1}^{t-1} 2\left(
      \phi^k \over T
    \right)^2 \sigma^6 + \sum_{t=1}^T
    {\sigma^4 (1 - \rho^2)^2 \over T^2} v^2(\rho)
  \right] \nonumber \\
  &=& {2 \sigma^6 \over T (1 - \phi^2)^2} \left[
    {\phi^2 \over 1 - \phi^2} -
    {\phi^2 (1 - \phi^{2T}) \over
      T(1 - \phi^2)}
  \right] + {\sigma^4 (1 - \rho^2)^2 v^2(\rho) \over
    T (1 - \phi^2)^2} \nonumber \\
  &\approx& {2 \sigma^6 \phi^2 \over T (1 - \phi^2)^3}
  + {\sigma^4 (1 - \rho^2)^2 v^2(\rho) \over
    T (1 - \phi^2)^2} \label{eq:gaussian_variance2}
\end{eqnarray}
where $P^\mu_\nu(\cdot)$ is Ferrer's function of the first
kind. See appendix \ref{chp:gaussian_elements_dist}. For the
definition of Ferrer's function, see equation
\ref{eq:Ferrers_1st}.

Equation \ref{eq:gaussian_mean2} tells that, if
two return series $i$ and $j$ are not correlated, i.e. $\rho = 0$,
auto-correlation in the returns does not introduce a bias into the
estimation of their covariance, i.e. $\mu'_X$, since the difference
between the two Ferrer's functions evaluate to 0 in equation
\ref{eq:gaussian_mean2}; if, however, the return series are indeed
correlated, auto-correlation in the returns rescales the
covariance through a multiplicative factor $1/(1 - \phi^2)$.

In addition, equation \ref{eq:gaussian_variance2} tells that
auto-correlation in the returns always makes the covariance
estimation more noisy --- auto-correlation not only rescales the
variance of the no-autocorrelation estimation by $1/(1 - \phi^2)^2$
but even adds an extra term ${2 \sigma^6 \phi^2 \over T (1 -
  \phi^2)^3}$.

For the diagonal elements of the covariance matrix $C$ we have
\begin{eqnarray}
  \E(C_{ii}) &=& {1 \over T}\left[
    \sum_{k=0}^{t-1} \phi^{2k} \sigma^2
  \right] \nonumber \\
  &=& {\sigma^2 \over (1 - \phi^2) T} \left[
    T - {\phi^2 (1 - \phi^{2T}) \over 1 - \phi^2}
  \right] \nonumber \\
  & \approx & {\sigma^2 \over 1 - \phi^2} \left[
    1 - {\phi^2 \over T (1 - \phi^2)}
  \right] \label{eq:gaussian_cii_mean2}
\end{eqnarray}
and
\begin{eqnarray}
  \var(C_{ii}) &=& \sum_{t=1}^T \left[
    \sum_{k=0}^{t-1} {\phi^{4k} \sigma^4 \over T^2} 2 +
    \sum_{k,l=0}^{t-1} {\phi^{2(k+l)} \over T^2} \sigma^6
  \right] \nonumber \\
  &=& \sum_{t=1}^T \left[
    {2 \sigma^4 \over T^2} {1 - \phi^{4t} \over 1 - \phi^4} +
    {\sigma^6 \over T^2} \left(
      {1 - \phi^{2t} \over 1- \phi^2}
    \right)^2 \right] \nonumber \\
  &=& {2 \sigma^4 \over T (1 - \phi^4)} -
  {2 \sigma^4 \phi^4 (1 - \phi^{4T}) \over T^2(1 - \phi^4)^2} +
  \nonumber \\
  && {\sigma^6 \over T (1 -\phi^2)^2} -
  {2 \sigma^6 \phi^2 (1 - \phi^{2T}) \over T^2 (1 - \phi^2)^3} +
  {\sigma^6 \phi^4 (1 - \phi^{4T}) \over T^2 (1 - \phi^2)^2 (1 -
    \phi^4)} \nonumber \\
  &\approx& {2 \sigma^4 \over T (1 - \phi^4)} + {\sigma^6 \over T (1
    -\phi^2)^2} \label{eq:gaussian_cii_variance2}
\end{eqnarray}

From equation \ref{eq:gaussian_cii_mean2} we see that auto-correlation
in the returns increases the variance of the return series; and from
equation \ref{eq:gaussian_cii_variance2} we see that the variance of
that variance estimation is also increased by
auto-correlations. Moreover, we note that $\var(C_{ii})$ scales with T
approximately as $1/T$, similar to the behavior of
$\var(C_{ij})$. This is to be compared with the case of \gls{garch}
returns discussed in chapter \ref{chp:CrossCorrelationFat}.

\section{Distribution of the Eigenvalues}
\label{sec:GCC-numerical}
For a Wishart matrix $\mtx{RR'}$, theoretical results are available
for the eigenvalue distribution, and we summarize them in appendix
\ref{sec:wishart_eigen_dist}. In short, the joint probability density
function of the eigenvalues is given by equation
\ref{eq:wishart_eigen_pdf} when neither auto-correlation nor
cross-correlation is present in $\mtx R$. Moreover, the mariginal
distribution of the largest eigenvalue is approximated by a gamma
distribution \cite{Chiani2012}.

However, as detailed in the derivation leading to equation
\ref{eq:cross-corr-matrix-PDF}, the distribution of $\mtx{RR'}$ is not
Wishart when the columns of $\mtx{R}$ are correlated
(auto-correlation). Deriving the eigenvalue distribution analytically
in this case is beyond the scope of this thesis. Instead, we resort to
numerical methods.

As before we consider the \gls{ar}(1) process:
\begin{equation*}
  \vec{r}_t = \phi \vec{r}_{t-1} + \vec{a}_{t}
\end{equation*}
where $\vec{a}_t \sim N(0, \mtx{I})$, i.e. the elements of $\vec{a}_t$ are
independent Gaussian random variable with zero mean and unit
variance. Now we investigate how the eigenvalue distribution depends on
the auto-correlation strength parameter $\phi$. Figure
\ref{fig:GaussianMarkovSpectrumPDF} shows the results of the
simulation.
\begin{figure}[htb!]
  \begin{center}
    \includegraphics[scale=0.4, clip=true, trim=90 228 115
    226]{../pics/GaussianMarkovSpectrumPDF.pdf}
  \end{center}
  \vspace{-10mm}
  \caption{\small \it
      Eigenvalue distribution with correlation time $\tau$ ranging
      from 0 to 3. The 1st blue line, which is shown as stairs, is the
      theoretical eigenvalue distribution according to the
      Marcenko-Pastur law (see eq.\ref{eq:MP_pdf}). In the simulation
      we have chosen $q = N/T = 50/1000 = 0.05$ and the standard
      deviation of the returns $\sigma=1$. For each value of $\tau$ we
      generate 2000 instances of $N \times T$ random matrix $R$, and
      compute C as $C=RR'/T$. Hence each curve in the figure is
      constructed from 2000 sets of eigenvalues. The correspondance
      between the correlation time $\tau$ and the auto-regressive
      coefficient $\phi$ is $\tau = -\ln 2 / \ln \phi$.
    }
  \label{fig:GaussianMarkovSpectrumPDF}
\end{figure}
It is clear from the figure that the maximum eigenvalue moves
consistently to the right as the value of $\phi$ increases, and, as
shown in figure \ref{fig:Gaussian_mineig}, the minimum eigenvalue
also increases with $\phi$.
\begin{figure}[htb!]
  \centering
  \includegraphics[scale=0.4, clip=true, trim=100 226 116
  133]{../pics/Gaussian_mineig.pdf}
  \caption{\small \it The minimum eigenvalue versus
    auto-correlation strength $\phi$. For each value of $\phi$ 2000
    random matrices are generated and their eigenvalues are
    calculated. The minimum eigenvalue of each random matrix is noted
    and the mean of the 2000 such minimum eigenvalues are plotted
    against the chosen value of $\phi$. 20 values of $\phi$ are
    included in the plot, ranging from 0 to 0.95 with step size 0.05.}
  \label{fig:Gaussian_mineig}
\end{figure}

Chiani showed that the marginal distribution of the maximum eigenvalue
($\lambda_1$) is approximately gamma when neither cross-correlation
nor auto-correlation is present\cite{Chiani2012}. So we compare
in figure \ref{fig:GaussianMarkov005MaxEigCDF_loglog} the empirical
cumulative distribution function (CDF) of the maximum eigenvalue 
with the CDF of a gamma distribution. The cases where
auto-correlations are present ($\phi > 0$) have also been included.
\begin{figure}
  \begin{center}
    \includegraphics[scale=0.5, clip=true, trim=0 238 0
    197]{../pics/GaussianMarkov005MaxEigCDF_loglog.pdf}
  \end{center}
  \caption{\small \it Cummulative distribution function (CDF) of the
    maximum eigenvalue ($\lambda_1$) from numerical simulations (blue
    lines) are compared to the fitted gamma distribution (red
    lines). Each pair of CDFs correspond to a fixed autocorrelation
    strength ($\phi$). The parameters $k$ and $\theta$ of the gamma
    distribution are fit to data by matching the 2nd and the 3rd
    moments of the gamma distribution to the corresponding moments of
    the empirical distribution. Then the parameter $\alpha$ in
    equation \ref{eq:TracyWidom-Gamma} is chosen to be $\alpha =
    k\theta - \E\left({\lambda_1 - \mu_{NT} \over
        \sigma_{NT}}\right)$. The curves are plotted on log-log
    scale.}
  \label{fig:GaussianMarkov005MaxEigCDF_loglog}
  %\vspace{-10mm}
\end{figure}
It is seen in the figure that gamma distributions with different
parameters fit fairly well in all cases. So we conclude that a
gamma distribution not only approximates the maximum eigenvalue
distribution at the absence of autocorrelations but does so even at
the {\it presence} of autocorrelations.

Since the maximum eigenvalue distribution is approximated by a gamma
distribution characterized by parameters $k$, $\theta$, and $\alpha$
\footnotemark, the influence of the autocorrelations can be
characterized by the dependence of $k$, $\theta$, and  $\alpha$ on
$\phi$. While these dependences are rather intricate, good support can
be found in the data for the following approximate relation:
\footnotetext{
  The mean, variance and skewness of the gamma distribution are given
  by
  \begin{eqnarray*}
    \text{mean} &=& k\theta \\
    \text{variance} &=& k\theta^2 \\
    \text{skewness} &=& 2/\sqrt{k}
  \end{eqnarray*}
}
\begin{eqnarray}
  k\theta &=& a \tan^2{\pi \phi \over 2} + b\tan{\pi \phi \over 2} +
  c \label{eq:k_theta-phi}
\end{eqnarray}
Here we note that $k\theta$ is the mean of the gamma distribution.
To verify this relation, we first fit a 2nd order polynomial and
obtain the coefficients $a$, $b$, $c$; then for each data point
$k_n\theta_n$ we solve the quadratic equation
\begin{eqnarray}
  a \tan^2{\pi \phi'_n \over 2} + b\tan{\pi \phi'_n \over 2} + c -
  k_n\theta_n &=& 0\label{eq:k_theta-phi_2}
\end{eqnarray}
for $\tan{\pi \phi'_n \over 2}$. If relation \ref{eq:k_theta-phi} is a
good approximation, a close match between $\tan{\pi \phi'_n \over 2}$
and $\tan{\pi \phi_n \over 2}$ is expected. From figure 
\ref{fig:phi_and_roots} one can see this is indeed the case.

\begin{figure}[htb!]
  \centering
    \includegraphics[scale=0.5, clip=true, trim=37 217 39
    170]{../pics/phi_and_roots.pdf}
  \caption{\small \it Upper plot: $\tan{\pi \phi'_n \over 2}$ against
    $\tan{\pi \phi_n \over 2}$. The fitted line has equation $y_n =
    0.995 \tan{\pi \phi_n \over 2} + 0.0146$. Lower plot: Residuals of
    the linear fit, i.e. $\tan{\pi \phi'_n \over 2} - y_n$. 20 values
    of $\phi$ are included in the plot, ranging from 0 to 0.95 with
    step size 0.05.}
  \label{fig:phi_and_roots}
\end{figure}

% Figure \ref{fig:GaussianMarkovMaxEig_k-phi} shows how the mean of the
% fitted gamma distribution, i.e. $k\theta$, varies as the
% autocorrelation strengthens ($\phi$ increases). 
% \begin{figure}[htb!]
%   %\vspace{-10mm}
%   \centering
%     \includegraphics[scale=0.5, clip=true, trim=100 223 112
%     141]{../pics/GaussianMarkov05MaxEig_k-phi.pdf}
%     \caption{\small \it The parameter $k$ against autocorrelation strength
%       $\phi$. Blue crosses: Empirical values of $k$. Red: best fitting
%       line in terms of {\it Least Square Errors}.}
%   \label{fig:GaussianMarkovMaxEig_k-phi}
% \end{figure}
% From the equation of the fitting line we can directly read out
% \begin{eqnarray*}
%   k &=& a\phi + b \\
%   &=& a\left(1 \over 2\right)^{1/\tau} + b
% \end{eqnarray*}
% where $a = -26$ and $b = 47$.

% For the parameter $\theta$, its behavior is more conveniently
% described in terms of the correlation time $\tau$. Figure
% \ref{fig:GaussianMarkovMaxEig_theta-tau} plots the values of $\theta$
% against those of $\tau$ together with a fitting quadratic
% function. Higher order polynomials provide a slightly better
% fit, but coefficients of the 3rd order and above are less than 1/1000
% times the coefficients of the 2nd and the 1st order. Therefore a 2nd
% order polynomial has been chosen.
% \begin{figure}[htb!]
%   \vspace{-15mm}
%   \centering
%   \includegraphics[scale=0.5, clip=true, trim=99 230 114
%   139]{../pics/GaussianMarkov05MaxEig_theta-tau.pdf} 
%   \caption{\small \it Vertical axis: $\theta$; Horizontal axis:
%     correlation time $\tau$. Blue: empirical values of $\theta$. Cyan:
%     Fitting quadratic function.}
%   \label{fig:GaussianMarkovMaxEig_theta-tau}
% \end{figure}
% The equation of this polynomial is
% \begin{eqnarray*}
%   \theta &=& A\tau^2 + B\tau + C
% \end{eqnarray*}
% where $A = 2.1\times 10^{-4}$, $B = -1.1\times 10^{-4}$, $C =
% 3.2\times 10^{-4}$.

% The parameter $\alpha$ shifts the gamma distribution $\mathscr{G}(k,
% \theta)$ to match the mean of ${(\lambda_1 -
%   \mu_{NT})/\sigma_{NT}}$. The behavior of $\alpha$ with respect to
% changing autocorrelation is shown in figure
% \ref{fig:GaussianMarkovMaxEig_alpha-tau}.
% \begin{figure}[htb!]
%   \vspace{-15mm}
%   \centering
%   \includegraphics[scale=0.5, clip=true, trim=104 229 114
%   139]{../pics/GaussianMarkovMaxEig_alpha-tau.pdf}
%   \caption{\small \it The mean-shift parameter $\alpha$ against
%     correlation time $\tau$. Blue: empirical values of $\alpha$. Cyan:
%     quadratic fit.}
%   \label{fig:GaussianMarkovMaxEig_alpha-tau}
% \end{figure}
% The equation of $\alpha$ is then inferred from the fitting line:
% \begin{equation*}
%   \alpha = D\tau^2 + E\tau + F
% \end{equation*}
% where $D = -3.4\times 10^{-3}$, $E = -4.6\times 10^{-2}$ and $F = 69$.

% So combining the results of $\alpha$, $k$ and $\theta$ we can express
% the moments of the Tracy-Widom variate ${(\lambda_1 -
%   \mu_{NT})/\sigma_{NT}}$:
% \begin{equation}\label{eq:lambda1_MeanVariance}
%   \begin{aligned}
%     \E({\lambda_1 - \mu_{NT} \over \sigma_{NT}}) &= k\theta -
%     \alpha \\
%     &= \left[-a\left(1 \over 2\right)^{1/\tau} + b\right](A\tau^2 +
%     B\tau - C) - (D\tau^2 + E\tau + F) \\
%     \var({\lambda_1 - \mu_{NT} \over \sigma_{NT}}) &=
%     k\theta^2 \\
%     &= \left[-a\left(1 \over 2\right)^{1/\tau} + b\right](A\tau^2 +
%     B\tau - C)^2 \\
%   \end{aligned}
% \end{equation}
% Figure \ref{fig:GaussianMarkovMaxEig_tw_moments} shows the empirical
% moments of ${(\lambda_1 - \mu_{NT})/\sigma_{NT}}$ together with the
% corresponding values computed using the above formulas.
% \begin{figure}[htb!]
%   \vspace{-15mm}
%   \centering
%   \includegraphics[scale=0.46, clip=true, trim=48 243 6
%   134]{../pics/GaussianMarkovMaxEig_tw_moments.pdf}
%   \caption{\small \it empirical moments of ${(\lambda_1 -
%       \mu_{NT})/\sigma_{NT}}$ against their theoretical
%     counterparts. Left: Empirical/theoretical mean against $\tau$;
%     Right: Empirical/theoretical variance against $\tau$. Blue
%     crosses: empirical values; Red line: fitting curves. Horizontal
%     axis: correlation time $\tau$.}
% \label{fig:GaussianMarkovMaxEig_tw_moments}
% \end{figure}
% The good fitness shown in figure
% \ref{fig:GaussianMarkovMaxEig_tw_moments} allows us to conclude 
% that, within the range of the correlation time that we have studied,
% namely $\tau \in [0, 13.51]$, the $k$ parameter is a linear function
% of $\phi = 2^{-1/\tau}$ while $\theta$ and $\alpha$ are quadratic
% functions of $\tau$. The moments of the transformed maximum
% eigenvalue, namely ${(\lambda_1 - \mu_{NT})/\sigma_{NT}}$, are thus
% expressed as functions of the correlation time $\tau$ via the
% parameters $k$, $\theta$ and $\alpha$.

% Figure \ref{fig:GaussianMarkovMaxEigPDF_original} shows the PDF of the
% maximum eigenvalue ($\lambda_1$) for a range of values of the
% correlation time $\tau$. One can clearly see that the mean of the
% distribution moves to the right and the width of the distribution
% increases as $\tau$ takes on larger and larger values. However, one
% must not forget that the behavior of the mean and the width is random
% in nature. Eq. \ref{eq:lambda1_MeanVariance} describes such behavior in
% the asymptotic limit, i.e. if an infinite number of random matrices
% are generated and their respective maximum eigenvalues computed for
% each and every value of correlation time $\tau$.
% \begin{figure}[htb!]
%   \begin{center}
%     \includegraphics[scale=0.5, clip=true, trim=98 228 112
%     221]{../pics/GaussianMarkovMaxEigPDF_original.pdf}
%     \caption{\small \it Probability density function of the maximum eigenvalue
%       ($\lambda_1$).}
%   \end{center}
%   \label{fig:GaussianMarkovMaxEigPDF_original}
% \end{figure}



\chapter{Covariance Matrix of GARCH(1,1) Returns}
\label{chp:CrossCorrelationFat}
\gls{garch} models and particularly \gls{garch}(1,1) models are widely used to
model financial return series of various time scales, ranging from
daily and monthly returns that have been studied extensively in the
literature to intraday returns that we have selectively investigated
in chapter \ref{chp:PriceModels}. One particularly nice feature of
\gls{garch}(1,1) models is that they have regularly varying tails (power-law
tails) \cite{Mikosch2009, mikosch2000} even when the innovations
(denoted $z_t$ in the following text) are normally distributed ---
something not shared by other classes of models (e.g. not by
\gls{sv} models, as shown in section \ref{sec:XieCalc}) but well
documented for realistic returns data by empirical studies
\cite{Mantegna2000, Potters2003, Guhr2007}. However, what this tail
behavior implies for the covariance matrix is much less understood,
especially when the return series of the covariance matrix are
auto-correlated.

So in this chapter we consider the covariance matrix of N
identically specified, possibly auto-correlated \gls{garch}(1, 1)
processes:
\begin{eqnarray}
  r_{it} &=& \phi r_{i, t-1} + \epsilon_{it} \nonumber \\
  \epsilon_{it} &=& \sigma_{it} z_{it} \label{eq:garch_spec}
\end{eqnarray}
where $i=1,2,...,N$; $t=1,2,...,T$; $z_{it}$ is independent,
identically distributed, and
\begin{eqnarray*}
  \sigma_{it}^2 &=& \alpha_0 + \alpha_1 z_{i, t-1}^2 + \beta_1
  \sigma_{i,t-1}^2
\end{eqnarray*}
Mikosch and Starica showed in \cite{mikosch2000} that a
\gls{garch}(1,1) process satisfying
\begin{eqnarray*}
  \alpha_0 &>& 0 \\
  \E\ln(\alpha_1 Z^2 + \beta_1) &<& 0 \\
\end{eqnarray*}
and
\begin{eqnarray*}
  \E[(\alpha_1 Z^2 + \beta_1)^{p/2}] &\geq& 1 \\
  \E|Z|^p \ln|Z| &\leq& \infty
\end{eqnarray*}
for some $p > 0$, is stationary and has regularly varying tails. The
tail exponent $\alpha$ is determined by:
\begin{equation}\label{eq:garch_alpha}
  \E[(\alpha_1 Z^2 + \beta_1)^{\alpha/2}] = 1
\end{equation}
Here $Z$ is a random variable that has the same distribution as
$z_{it}$. In our simulations described hereafter, $z_{it}$ and $Z$
have standard Gaussian distribution. In this section and the next, we
first study situations where no auto-correlations are present among
the returns, i.e. $\phi = 0$; then in section \ref{sec:garch_nonzero_phi} we
look at how auto-correlations change the picture.

With regularly varying tails, the eigenvalue distribution of a
covariance matrix built from \gls{garch}(1,1) returns is expected to
differ from the Marcenko-Pastur law discussed in section
\ref{sec:wishart_eigen_dist}. In figure \ref{fig:garch_ev_cii} we
simulate N=50 independent \gls{garch}(1,1) 
returns series, each with identical parameters, namely $\alpha_0 =
2.3\times 10^{-6}$, $\alpha_1 = 0.15$, $\beta_1 = 0.84$, $\phi = 0$
and T=$8\times10^4$ time steps, then we build the covariance
matrix as
\begin{eqnarray*}
  \mtx C &=& {1 \over T^{2/\alpha}} \mtx{RR'}
\end{eqnarray*}
where $\mtx R$ is an $N\times T$ matrix, whose elements $r_{it}$ are
specified by equation \ref{eq:garch_spec} with $\phi = 0$. The
normalization factor $1 \over T^{2/\alpha}$ has been chosen such that the
eigenvalue distribution is independent of $T$ in the limit $T \to
\infty$ \cite{politi2010, Cizeau1994}.

The \gls{pdf} of the eigenvalue distribution of C is plotted in figure
\ref{fig:garch_ev_cii_linear} and \gls{cdf} of the distribution is
plotted on log-log scale in \ref{fig:garch_ev_cii_loglog}.
\begin{figure}[htb!]
  \centering
  \subfigure[]{
    \includegraphics[scale=0.40, clip=true, trim=104 271 111
    216]{../pics/garch_ev_cii.pdf}
    \label{fig:garch_ev_cii_linear}
  }
  \subfigure[]{
    \includegraphics[scale=0.40, clip=true, trim=104 274 104
    226]{../pics/garch_ev_cii_loglog.pdf}
    \label{fig:garch_ev_cii_loglog}
  }
  \caption{\small \it \ref{fig:garch_ev_cii_linear}: Eigenvalues and
    Diagonal elements' distribution of a covariance matrix
    built from independent GARCH return series. Blue: PDF of
    eigenvalues; Green: PDF of diagonal elements; Red: PDF of a
    $\alpha$-stable distribution fitted to the diagonal
    elements. \ref{fig:garch_ev_cii_loglog}: CDF of the same
    quantities in 10-based log-log scale.}
  \label{fig:garch_ev_cii}
\end{figure}
Also plotted in the same figure is the distribution of the diagonal
elements of $\mtx C$. It is clear from the figure that the two PDFs
coincide, implying $\mtx C$ is diagonal. This is further confirmed by
figure \ref{fig:garch_cij} which shows the distribution of the
non-diagonal elements of C. One can see the non-diagonal elements are
distributed symmetrically around 0 with a very small width in
comparison to the distribution of the diagonal elements --- in fact, 1
order of magnitude smaller ($2.14\times10^{-5}$ v.s. $7.31\times
10^{-4}$). Hence $\mtx C$ is very close to a diagonal matrix.
\begin{figure}[htb!]
  \centering
    \includegraphics[scale=0.5, clip=true, trim=95 253 91
    223]{../pics/garch_cij.pdf}
  \caption{\small \it Distribution of the non-diagonal elements of
    the covariance matrix. Blue: PDF of the non-diagonal
    elements; Green: $\alpha$-stable distribution fitted to the
    non-diagonal elements' PDF.}
  \label{fig:garch_cij}
\end{figure}

Figure \ref{fig:garch_ev_cii} also shows the two curves are well fitted
by an $\alpha$-stable distribution. An estimate of the L\'evy index
$\alpha$ of the stable distribution is also obtained via fitting,
$\alpha \approx 1.38$, as shown in figure \ref{fig:garch_ev_cii}. This
is really an expected result for the diagonal elements. Using
$\alpha_1 = 0.15$, $\beta_1 = 0.84$, which are the values used for
simulating the \gls{garch} returns, one can obtain, by solving equation
\ref{eq:garch_alpha}, $\alpha=2.96$. Then according to Mikosch and Starica
\cite{mikosch2000}
\begin{eqnarray*}
  P(|r_t| > x) &\sim& {\E(z^\alpha) c_0 \over x^\alpha} \\
  P(|r_t|^2 > x) &\sim& {\E(z^\alpha) c_0 \over x^{\alpha/2}} \\
\end{eqnarray*}
for some constant $c_0$. Now that $P(|r_t|^2 > x)$ has power-law tail
behavior with power $\alpha/2 < 2$, one can deduce
\begin{equation}
  \label{eq:stable_CLT}
  \begin{aligned}
    \sum_{t=1}^T r_t^2 &\xrightarrow{d} S(\alpha/2,
    1, \gamma, \mu) \text{ as $T \to \infty$}
  \end{aligned}
\end{equation}
where $\xrightarrow{d}$ denotes convergence in distribution, and
$S(\alpha/2, 1, \gamma, \mu)$ denotes an $\alpha$-stable distribution
with parameters $(\alpha/2, 1, \gamma, \mu)$. Here $\alpha/2$ is the
L\'evy index, 1 is the asymmetry, $\gamma$ is the scaling parameter
and $\mu$ is the mean value of the distribution. Asymmetry being 1
means a random variable so distributed only takes positive values
\cite{Bilik2008, Embrechts1997}.

The mean $\mu$ in equation \ref{eq:stable_CLT} is given by
\begin{eqnarray*}
  \mu &=& T \E(|r_t|^2) \\
  &=& T{\alpha_0 \over 1 - \alpha_1 - \beta_1}
\end{eqnarray*}
where we have used the result $\E(|r_t|^2) = \alpha_0 / (1 - \alpha_1
- \beta_1)$ \cite{Bollerslev86}. The scaling parameter $\gamma$ in
\ref{eq:stable_CLT} is determined by the limit \cite{Bilik2008}
\begin{eqnarray*}
  \lim_{T\to\infty} {T \E(z^{\alpha/2}) c_0 \over \gamma^{\alpha/2}}
  &=& C_{\alpha/2}
\end{eqnarray*}
where
\begin{eqnarray*}
  C_{\alpha/2} &=& \left( \int_0^\infty {\sin x \over x^{\alpha/2}} dx
  \right)^{-1} \\
  &\approx& {1 \over \sqrt{2 \pi}}
\end{eqnarray*}
Therefore
\begin{eqnarray*}
  \gamma^{\alpha/2} &=& \sqrt{2\pi} T \E\left(
    |z|^{\alpha/2}
  \right) c_0 \\
  \gamma &=& (2\pi)^{1/\alpha} T^{2/\alpha} \left(\E
    |z|^{\alpha/2}
  \right)^{2/\alpha} c_0^{2/\alpha}
\end{eqnarray*}
Here we note that an $\alpha$-stable distribution $S(\alpha, \beta,
\gamma, \mu)$ has characteristic function \cite{Guhr2007}
\begin{eqnarray*}
  \varphi(k; \alpha, \beta, \gamma, \mu) &=& \exp\left[
    i\mu k - \gamma^\alpha |k|^\alpha \left(
      1 - i \beta {k \over |k|} \tan{\pi \alpha \over 2}
    \right) \right] \text{ for $\alpha \neq 1$}
\end{eqnarray*}
from which we see $\varphi(ak; \alpha, \beta, \gamma, \mu) = \varphi(k;
\alpha, \beta, a\gamma, a\mu)$, implying that, if $x \sim S(\alpha,
\beta, \gamma, \mu)$, then $ax \sim S(\alpha, \beta, a\gamma, a\mu)$.

Now that
\begin{eqnarray*}
  \sum_{t=1}^T r_t^2 &\xrightarrow{d} S(\alpha/2,
  1, \gamma, \mu) \text{ as $T \to \infty$}  
\end{eqnarray*}
we have
\begin{eqnarray}
  C_{ii} &=& {1 \over T^{2/\alpha}}\sum_{t=1}^T r_{it}^2 \\
  &\xrightarrow{d}&
  S(\alpha/2, 1, \gamma_D, \mu_D) \text{ as $T \to \infty$}
  \label{eq:garch_cii_dist}
\end{eqnarray}
where
\begin{eqnarray}\label{eq:garch_wishart_cij_params}
  \gamma_D &=& (2\pi)^{1/\alpha} \E\left(|z|^{\alpha/2}
  \right)^{2/\alpha} c_0^{2/\alpha} \\
  \mu_D &=& {\alpha_0 \over 1 - \alpha_1 - \beta_1} T^{1 - 2/\alpha}
  \nonumber
\end{eqnarray}
So the diagonal elements of the covariance matrix converge to
an $\alpha$-stable distribution with L\'evy index $\alpha/2 \approx
1.48$. This is comparable to the index value 1.38 obtained by
fitting. Considering the slow convergene of regularly varying tails,
this is a reasonably good match.

Now we look at the distribution of the non-diagonal elements. Figure
\ref{fig:garch_cij} shows that an $\alpha$-stable distribution fits
rather well, and additionally, figure \ref{fig:garch_nondiag_probplot}
shows that the distribution of these non-diagonal elements has fat
tails, supporting an $\alpha$-stable distribution.
\begin{figure}[htb!]
  \centering
  \includegraphics[scale=0.5, clip=true, trim=81 229 114
    121]{../pics/garch_cij_prob.pdf}
    \caption{\small \it Probability plot of the non-diagonal
      elements. Blue: The accumulative probability function
      (CDF) of the non-diagonal elements. Black, dashed: CDF of the
      Gaussian distribution that has the same mean and variance as the
      sample. The graph is arranged on such a scale that the Gaussian
      CDF is a straight line.
    }
  \label{fig:garch_nondiag_probplot}
\end{figure}

The parameters of the fitted $\alpha$-stable distribution $S(\alpha',
\beta', \gamma', \mu')$ are obtained in the procedure of fitting. The
results have been shown in the legend of figure
\ref{fig:garch_cij}. In table \ref{tab:garch_wishart_cij_params} we
list them with a higher precision.
\begin{table}[htb!]
  \centering
  \begin{tabular}{|c|c|c|c|}
    \hline
    $\alpha'$ & $\beta'$ & $\gamma'$ & $\mu'$ \\
    \hline
    1.9453 & 0.0018 & $2.1381 \times 10^{-5}$ & $2.0637\times 10^{-9}$ \\
    \hline
  \end{tabular}
  \caption{\small \it Parameters of the non-diagonal elements' distribution}
  \label{tab:garch_wishart_cij_params}
\end{table}
Since $r_{it}$ and $r_{jt}$ are independent of each other, $\beta'$
and $\mu'$ are expected to be 0 --- but with a finite T, some deviation
from 0 is not surprising.

Now that the parameters' values have been obtained, the way they scale
with T, i.e. the length of the return series, can be deduced. Consider
\begin{equation}\label{eq:garch_cij1}
  T^{2/\alpha} C_{ij} = \sum_{t=1}^T r_{it} r_{jt} \xrightarrow{d} S(\alpha',
  0, \gamma', 0)
\end{equation}
where the width $\gamma'$ is determined by \cite{Bilik2008},
\begin{eqnarray*}
  \lim_{T\to\infty} {T C \over \gamma'^{\alpha'}}
  &=& C_{\alpha'} \\
  C_{\alpha'} &=& \left( \int_0^\infty {\sin x \over x^{\alpha'}} dx
  \right)^{-1} \\
\end{eqnarray*}
Here the constant C is such that
\begin{equation*}
  P(|r_{it}r_{jt}| > x) \sim {C \over x^{\alpha'}} \text{ as $x \to \infty$}
\end{equation*}
So we have
\begin{eqnarray*}
  \gamma' &=& {\left( CT \over C_{\alpha'}\right)^{1/\alpha'}}
\end{eqnarray*}
Divide throughout equation \ref{eq:garch_cij1} by $T^{2/\alpha}$ then
gives
\begin{equation}\label{eq:garch_cij2}
  C_{ij} = {1 \over T^{2/\alpha}}\sum_{t=1}^T r_{it} r_{jt}
  \xrightarrow{d}
  S(\alpha', 0, \gamma_N, 0)
\end{equation}
where
\begin{eqnarray*}
  \gamma_N &=& {\left(C \over C_{\alpha'}\right)^{1/\alpha'}}
  T^{1/\alpha'-2/\alpha} \\
\end{eqnarray*}
So we see that distribution of the non-diagonal elements has a width
that scales with T as $T^{1/\alpha' - 2/\alpha} \approx T^{-1/6}$,
while the width of the diagonal elements' distribution, as shown in
equation \ref{eq:garch_wishart_cij_params}, does not scale with
T. This is to be compared with the Wishart case where the return
series have Gaussian distribution, and hence the asymptotic
distributions of both the diagonal and the non-diagonal elements of the
covariance matrix are Gaussian with a variance that scales as $1/T$
(c.f. \ref{sec:GCC-analytical}).

% We also notice that, given our particular choice of $\alpha_1$ and
% $\beta_1$, the covariance matrix's convergence to a diagonal matrix
% is slow, since the non-diagonal elements decay to zero only 

% In the Wishart case, the diagonal and the non-diagonal elements
% scale in the same way with T and thus have comparable sizes at large
% T --- so the matrix is not diagonal and the eigenvalue distribution has
% the Marcenko-Pastur law. In the GARCH case, the diagonal elements do
% not scale with T while the non-diagonal elements scale as $T^{-1/6}$,
% thus at large T, the non-diagonal elements are significantly smaller
% than the diagonal ones --- so a diagonal matrix arises.

\section{Implications of Finite Number of Observations}
In the last section we have discussed the limiting situation where the
return series of the covariance matrix have an infinite number of
observations ($T \to \infty$), and in the simulation studies we have
generated a large number of observations for each return series,
namely $T = 8 \times 10^4$. However, in practice, one often does not
have such a large number of observations available. Therefore, it is
useful to investigate situations where $T$ only has a modest size.

Figure \ref{fig:GarchCiiEig1} shows the distributions of the diagonal
elements as well as the eigenvalues when the number of return series
(N) is fixed at 250 and the number of observations in each series (T)
is gradually increased from 3000 to 6000.
\begin{figure}[htb!]
  \centering
    \includegraphics[scale=0.7, clip=true, trim=81 248 75
    195]{../pics/GarchCiiEig1.pdf}
  \caption{\small \it The Diagonal elements' and the eigenvalues'
    distributions with modest T. Blue: eigenvalues' CDF;
    Green: Diagonal elements' CDF; Red: CDF of the $\alpha$-stable
    distribution fitted to the diagonal elements. All the curves are
    drawn on 10-based log-log scale.
  }
  \label{fig:GarchCiiEig1}
\end{figure}
It is seen from the figure that, compared to the earlier case where $T
= 8 \times 10^4$, the eigenvalue distribution and the diagonal
elements' distribution do not coincide as well but differ rather
siginificantly for small values. For large values of the diagonal
elements and the eigenvalues, the two do coincide and comply with the
fitted $\alpha$-stable distribution. Moreover, the difference between
the eigenvalues' distribution and the diagonal elements' distribution,
as well as that between the diagonal elements' distribution and the
fitting $\alpha$-stable distribution, is also seen to diminish as T
increases.

The convergence of the eigenvalues ($\lambda$) towards the diagonal
elements $C_{ii}$ as $\lambda \to \infty$ is really an anticipated
result. For convenience, we order the diagonal elements so that
\begin{eqnarray*}
  C_{11} < C_{22} < \cdots < C_{NN}
\end{eqnarray*}
where, as before, $N$ is the dimension of the covariance matrix
$C$. Then, as $x \to \infty$,
\begin{eqnarray*}
  P(C_{ii} > x) &\sim& {c \over x^{\alpha/2}} \\
  f(C_{ii}) &\sim& {c \over x^{\alpha/2 + 1}} \\
\end{eqnarray*}
where $f(\cdot)$ denotes the \gls{pdf} of the diagonal elements and $c$ is
some constant. Thus, at the limit $C_{ii} \to \infty$, the distance
between two adjacent diagonal elements $C_{ii}$ and $C_{i+1, i+1}$ can
be expressed as
\begin{eqnarray*}
{1 \over N (C_{i+1, i+1} - C_{ii})} &=& f(C_{ii})\\
C_{i+1, i+1} - C_{ii} &\sim& {C_{ii}^{\alpha/2 + 1} \over c N}
\end{eqnarray*}

Thus as $C_{ii} \to \infty$, $C_{i+1, i+1} - C_{ii} \to \infty$ while
$C_{ij} \to 0$ ($i \neq j$). Now that the spacing between adjacent
diagonal elements become wider and the non-diagonal elements become
smaller, the eigenstates considered as a mixture of the basis states,
become more and more localized to a prominent basis state. This
localization can be measured by the size of the component in each
eigenvector that has the largest absolute value($|c|_\M$), provided
that the eigenvectors have been normalized.

Figure \ref{fig:garch_eigenvec_Cmax} shows how $|c|_\M$ changes
in response to increasing $\lambda$. It is seen that as T increases,
localization of the eigenstates proceeds from those with very
large eigenvalues towards those with relatively smaller
eigenvalues. At the same time, the minimum of $|c|_\M$ increases
and so does the minimum of the eigenvalues. The last point here is
further illustrated in figure \ref{fig:garch_eigenvec_Cmax_dist} where
the \gls{pdf} of $|c|_\M$ is plotted. We see in this figure that an
increased value of T leads to advancement of $\min(|c|_\M)$ to larger
values as well as to an increased proportion of large $|c|_\M$. The
mean of $|c|_\M$ has apparently been increased too.
\begin{figure}[htb!]
  \centering
  \subfigure[]{
  \includegraphics[scale=0.29, clip=true, trim=14 199 31
  155]{../pics/garch_eigenvec_Cmax.pdf}
    \label{fig:garch_eigenvec_Cmax}
  }
  \subfigure[]{
  \includegraphics[scale=0.41, clip=true, trim=86 257 112
  136]{../pics/garch_eigenvec_Cmax_dist.pdf}
  \label{fig:garch_eigenvec_Cmax_dist}
  }
  \caption{\small \it \ref{fig:garch_eigenvec_Cmax}: Horizontal:
    eigenvalues ($\log \lambda$); Vertical: $\log |c|_\M$ of the
    corresponding eigenvector. From left to right and 1st to 2nd row,
    T has values 800, 1000, 1200,
    80000. \ref{fig:garch_eigenvec_Cmax_dist}: PDF of the largest
    component ($|c|_\M$) of each eigenvector. In both plots the number
    of returns (N) is 50.}
%  \label{fig:garch_eigenvec_Cmax_both}
%   \includegraphics[scale=0.4, clip=true, trim=14 199 31
%   155]{../pics/garch_eigenvec_Cmax.pdf}
%   \caption{\small \it largest component ($|c|_\M$) corresponding to
%     the eigenvalue. N is fixed to 50.}
%   \label{fig:garch_eigenvec_Cmax}
% \end{figure}

% \begin{figure}[htb!]
%   \centering
%   \includegraphics[scale=0.4, clip=true, trim=86 257 112
%   136]{../pics/garch_eigenvec_Cmax_dist.pdf}
%   \caption{\small \it PDF of the largest component ($|c|_\M$) of
%     each eigenvector. N is fixed to 50.}
%   \label{fig:garch_eigenvec_Cmax_dist}
\end{figure}

Another informative quantity that measures the localization is the
``Inverse Participation Ratio'' (IPR). For a given normalized eigenvector
$\vec{c}_i = (c_{1, i}, c_{2, i}, ..., c_{N, i})$, the \gls{ipr} is defined
as \cite{Aberg2013}
\begin{eqnarray}
  \text{IPR}(\vec{c}_i) &=& \sum_{k=1}^N c_{k,i}^4 \label{eq:IPR_def}
\end{eqnarray}
Figure \ref{fig:garch_eigenvec_PR} shows ${1 \over
  N \cdot \text{IPR}(\vec{c}_i)}$ in correspondence to the
eigenvalues. This quantity is sometimes termed the
normalized \gls{pr} and measures the proportion of
basis vectors that contribute considerably to the eigenvector in
question. From this figure we see that, for all values of T, if an
eigenvalue is larger than $10^{-1.5} \approx 0.03$, its corresponding
participation ratio is less than $10^{-1.6} = 2.5\%$, meaning less
than $50 \times 0.025 = 1.26$ basis vectors contribute --- each of the
corresponding eigenvectors is localized to a single basis vector and
hence the distribution of such large eigenvalues is the same as the
diagonal elements' distribution.

Figure \ref{fig:garch_eigenvec_PR_dist} shows the \gls{pdf} of the
normalized \gls{pr}. We see that as T increases, the distribution of
\gls{pr} is compressed towards 0, suggesting increased localization of
the eigenvectors.
\begin{figure}[htb!]
  \centering
  \subfigure[]{
    \label{fig:garch_eigenvec_PR}
    \includegraphics[scale=0.34, clip=true, trim=49 208 70
    162]{../pics/garch_eigenvec_PR.pdf}
  }
  \subfigure[]{
    \includegraphics[scale=0.4, clip=true, trim=97 259 113
    226]{../pics/garch_eigenvec_PR_dist.pdf}
  \label{fig:garch_eigenvec_PR_dist}
  }
  \caption{\small \it \ref{fig:garch_eigenvec_PR}: Normalized
    participation ratio (PR) versus eigenvalue
    ($\lambda$). Horizontal: $\log \lambda$; Vertical: $\log
    \text{PR}$ of the corresponding
    eigenvector. From left to right and 1st to 2nd row, T has values
    800, 1000, 1200, 80000. \ref{fig:garch_eigenvec_PR_dist}: PDF of
    the normalized PR. In both plots, the number of returns (N) is
    50.}
\end{figure}
In conclusion, localization of the eigenvectors, which implies 
coincident eigenvalue and diagonal elements' distributions, begins with
those associated to large eigenvalues. Increased observation points
lead to increased localization and hence increased coincident sections
of the eigenvalue and diagonal elements' distributions. However, this
increment with T is slow, because the diagonal elements mean $\mu_D$
increases only as a fractional power of $T$, namely $T^{1 -
  2/\alpha}$, and the non-diagonal elements' variance decreases only
as a fractional power too, namely $T^{1/\alpha' - 2/\alpha}$. These
have been detailed in equations \ref{eq:garch_wishart_cij_params} and
\ref{eq:garch_cij2}.

\section{Influence of auto-correlations}
\label{sec:garch_nonzero_phi}
In the previous two sections we have studied situations where
$\phi=0$ in the specification \ref{eq:garch_spec}, i.e. no
auto-correlation is in the returns. In this section we investigate how
auto-correlations change the picture.

% When auto-correlations are present among the returns, the
% covariance matrix is expected to change. How exactly it changes
% is the subject of this section. Here we consider a model specified
% as follows:
% \begin{eqnarray*}
%   r_{it} &=& \phi r_{i, t-1} + \epsilon_{it} \\
%   \epsilon_{it} &=& \sigma_{it} z_{it} \\
%   \sigma_{it}^2 &=& \alpha_0 + \alpha_1 \epsilon_{i, t-1}^2 + \beta_1
%   \sigma_{i, t-1}^2 \\
%   z_{it} &\sim& N(0, 1)
% \end{eqnarray*}
% where $r_{it}$ is the element of the R matrix at the $i$-th row and the
% $t$-th column. In the simulations described below R has N=50 rows and
% $T = 8 \times 10^4$ columns. The parameters $\alpha_1$ and $\beta_1$
% are the same as in the previous case of zero auto-correlations, namely
% $\alpha_1 = 0.15$ and $\beta_1 = 0.84$. The covariance matrix C
% is built from R using
% \begin{eqnarray*}
%   C &=& {1 \over T^{2/\alpha}} RR'
% \end{eqnarray*}
% where $\alpha$ is 2.96 as before.

Figure \ref{fig:GarchEigDiag1} shows the eigenvalue as well as the
diagonal elements' distribution when $\phi = 0.95$, i.e. $\tau =
13.51$. The values of N and T are 50 and $8\times 10^4$ as before. The
\gls{garch}(1,1) parameters are also unchanged, namely $\alpha_0 = 2.3\times
10^{-6}$, $\alpha_1 = 0.15$, and $\beta_1 = 0.84$. Figure
\ref{fig:GarchNondiag1} shows the non-diagonal elements' distribution
in the same setup. Included in these plots are $50 \times 2000 =
1\times 10^5$ eigenvalues and diagonal elements, as well as ${50
  \choose 2} \times 2000 = 2,450,000$ non-diagonal elements. These
data come from 2000 simulated matrices.
\begin{figure}[htb!]
  \centering
  \subfigure[]{
    \includegraphics[scale=0.42, clip=true, trim=112 272 103
    217]{../pics/GarchEigDiag1.pdf}
    \label{fig:GarchEigDiag1}
  }
  \subfigure[]{
    \includegraphics[scale=0.42, clip=true, trim=105 272 101
    213]{../pics/GarchNondiag1.pdf}
    \label{fig:GarchNondiag1}
  }
  \caption{\small \it \ref{fig:GarchEigDiag1}: Eigenvalues' and
    diagonal elements' distribution when $\phi = 0.95$, i.e. $\tau$ =
    13.51; \ref{fig:GarchNondiag1}: Non-diagonal elements'
    distribution in the same situation.}
\end{figure}

From figure \ref{fig:GarchEigDiag1} we see that, as auto-correlations
become significant, the distribution of the eigenvalues no longer
coincides with the diagonal elements' distribution --- instead it
becomes wider and fatter on the tails. We also notice that the widths
of both the diagonal and the non-diagonal elements' PDF's have
increased.  In figure \ref{fig:garch_ev_cii} we see that, when no
auto-correlation is present, the \gls{pdf} of the diagonal elements has
width ($\gamma$) $7.31 \times 10^{-4}$, while in figure
\ref{fig:GarchEigDiag1} we see that the width has become $7.91 \times
10^{-3}$ as $\tau$ becomes 13.51. Similarly the non-diagonal elements'
\gls{pdf} has width $2.14 \times 10^{-5}$ when $\tau=0$, as shown in figure
\ref{fig:garch_cij}, and this width becomes $9.60 \times 10^{-4}$ when
$\tau = 13.51$, as shown in figure \ref{fig:GarchNondiag1}.

Figure \ref{fig:GarchSpectrumAutocorrelated} shows the eigenvalues'
distribution corresponding to a range of $\phi$ values. The number of
eigenvalues in each curve is the same as in figure
\ref{fig:GarchEigDiag1}.
\begin{figure}[htb!]
  \centering
  \includegraphics[scale=0.4, clip=true, trim=85 258 103
  229]{../pics/GarchSpectrumAutocorrelated.pdf}
  \caption{\small \it Eigenvalue distribution of covariance
    matrix built from auto-correlated GARCH processes. $\phi$ = 0,
    0.5, 0.8, 0.955, 0.97 correspond to correlation time $\tau$ =
    0, 1.00, 3.11, 15.05, 22.76.}
  \label{fig:GarchSpectrumAutocorrelated}
\end{figure}
From this figure one can see that, as auto-correlation strengthens,
\begin{itemize}
\item the \gls{pdf} of the eigenvalue distribution flattens and widens;
\item the minimum as well as the maximum eigenvalues increase.
\end{itemize}

To find out more about this series of deformation, we first look at
how the largest component and the normalized participation ratio of
the eigenvectors change as $\phi$ takes on larger values. Figure
\ref{fig:garch_eigenvec_Cmax_corr} shows the largest eigenvector
component $|c|_\M$ in correspondence to the eigenvalue. Apparently, as
auto-correlation strengthens, the eigenvectors' composition
fractures, leading to a reduced degree of localization and even
reduced certainty of localization --- for a fixed eigenvalue, $|c|_\M$
now varies in a larger range than it does with smaller $\phi$.
\begin{figure}[htb!]
  \centering
  \includegraphics[scale=0.4, clip=true, trim=0 197 0
  154]{../pics/garch_eigenvec_Cmax_corr.pdf}
  \caption{\small \it Eigenvectors' largest component versus
    eigenvalue, for 4 auto-correlation strengths $\phi$ = 0, 0.6, 0.9,
    and 0.99. The number of returns (N) is 50.}
  \label{fig:garch_eigenvec_Cmax_corr}
\end{figure}
The same story of reduced localization is also evident from the plot
of the normalized \gls{pr}, shown in figure
\ref{fig:garch_eigenvec_PR_dist_corr}, and from the \gls{pdf} of $|c|_\M$
shown in figure \ref{fig:garch_eigenvec_Cmax_dist_corr}. It is seen
in \ref{fig:garch_eigenvec_PR_dist_corr} that the peak at the left 
of the plot, representing the group of localized eigenvectors, falls
with increased auto-correlation, and essentially disappears when
$\phi$ reaches the extreme value 0.99. Figure
\ref{fig:garch_eigenvec_Cmax_dist_corr} shows the proportion of large
$|c|_\M$ values is severely reduced and the mean of $|c|_\M$ is pushed
to smaller values by increased auto-correlation.
\begin{figure}[htb!]
  \centering
  \subfigure[]{
    \includegraphics[scale=0.4, clip=true, trim=95 260 111
    232]{../pics/garch_eigenvec_PR_dist_corr.pdf}
    \label{fig:garch_eigenvec_PR_dist_corr}
  }
  \subfigure[]{
    \includegraphics[scale=0.4, clip=true, trim=95 260 111
    232]{../pics/garch_eigenvec_Cmax_dist_corr.pdf}
    \label{fig:garch_eigenvec_Cmax_dist_corr}
  }
  \caption{\small \it \ref{fig:garch_eigenvec_PR_dist_corr}: PDF of
    the normalized participation ratio
    (PR). \ref{fig:garch_eigenvec_Cmax_dist_corr}: PDF of the largest
    component of the eigenvectors. The number of returns (N) is
    50. $\phi$ values of 0, 0.6000, 0.9000, 0.9900 correspond to
    correlation time $\tau$ = 0, 1.3569, 6.5788, 68.9676}
\end{figure}


It is also useful to look at how the fraction of localized eigenvectors
changes with the auto-correlation. For definiteness, we classify
an eigenvector as being localized when (1) its number of participating
basis vectors is less than 2, or (2) the largest of its components'
absolute values is larger than 0.9.

Figure \ref{fig:localization_ratio} shows how the ratio of localized
eigenvectors depends on the auto-correlation strength $\phi$. In
either way of classification, the ratio falls with $\phi$ in
accordance with a power law, the power exponent lying a bit below 2.
This is further confirmed in plot \ref{fig:localization_ratio2}, where
the ratios are plotted versus $\phi$ on log-log scale.
\begin{figure}[htb!]
  \centering
  \subfigure[]{
    \includegraphics[scale=0.38, clip=true, trim=93 229 115
    134]{../pics/localization_ratio.pdf}
    \label{fig:localization_ratio}
  }
  \subfigure[]{
    \includegraphics[scale=0.38, clip=true, trim=87 227 115
    133]{../pics/localization_ratio2.pdf}
    \label{fig:localization_ratio2}
  }
  \caption{\small \it \ref{fig:localization_ratio}: Ratio of localized
    eigenvectors versus the auto-correlation strength $\phi$. Black
    ``+'': ratio of localized eigenvectors as measured by the number
    of participating basis vectors being lower than 2; Black ``x'':
    ratio of localized eigenvectors as measured by the largest
    component being larger than 0.9. Blue curve: quadradic function
    fitted to ``+''. Green curve: quadratic function fitted to
    ``x''. There are 27 data points in the plot, corresponding to 27
    $\phi$ values: 0 to 0.8 with step size 0.05, and 0.9 to 0.99 with
    step size 0.01. The corresponding values of the correlation time
    $\tau$ range from 0 to 69. \ref{fig:localization_ratio2}: $\ln
    (f_\M - f)$ is plotted versus $\ln \phi$, where $f$ stands for the
    ratio of localized eigenvectors. The fitted curves are linear.}
\end{figure}

\chapter{Results}
\label{chp:summary}
Realistic dynamic models for relative price changes (returns) of
financial assets have been studied. In particular, the \gls{garch} and
\gls{sv} models have been utilized and compared in their forecast
accuracies by case studying of 15- and 30-minute returns of Nordea,
Volvo and Ericsson. The following are the conclusions from this
investigation:
\begin{itemize}
\item In all the 7 studied series, the log-volatility $\ln \sigma_t$
  is well described by a seasonally integrated moving average
  model. Long memory in these series is accounted for by the compounded
  difference operator $(1-B)(1-B^s)$ where $B$ denotes the back-shift
  operator. $s$ stands for seasonality, which is 33 in the cases of
  15-minute returns and 16 in the cases of 30-minute returns.

\item An \gls{sv} model generally yields more accurate forecasts than does
  the \gls{garch} model for the same series. This is certainly well expected,
  considering that the \gls{sv} model incorporates much more data than does
  \gls{garch}. However, it must be noted that the validity of the
  aforementioned comparison is underlain by the accuracy of realized
  volatility as a proxy to the true conditional volatility.

\item \gls{sv} models perform more consistently in terms of forecast
  accuracy than does \gls{garch}, as shown in appendix
  \ref{sec:forecast_volatility}.
\end{itemize}

Regarding the covariance matrix of Gaussian return series, we have
arrived at the following results:
\begin{itemize}
\item Auto-correlations in the returns rescale the covariance of two
  series by a factor of $1/(1 - \phi^2)$, where $\phi =
  \text{corr}(r_t, r_{t-1})$. See
  eq. \ref{eq:gaussian_mean2}. Moreover, they rescale the variance of
  each series by ${1 \over (1 - \phi^2)}\left[
    1- {\phi^2 \over T(1 - \phi^2)}
  \right]$. See eq. \ref{eq:gaussian_cii_mean2}.

\item Auto-correlations in the returns increase the variance of the
  covariance and also the variance of the variance. See
  eq. \ref{eq:gaussian_variance2} and
  \ref{eq:gaussian_cii_variance2}.

\item The largest eigenvalue obeys approximately a gamma distribution
  even when the returns are auto-correlated. It was shown by Chiani in
  2012 that the largest eigenvalue obeyed gamma distribution when the
  returns were {\it not} auto-correlated \cite{Chiani2012}. In
  addition, we find that the mean of the gamma distribution is
  approximately quadratic in $\tan{\pi\phi \over 2}$.
\end{itemize}

For the covariance matrix of \gls{garch}(1,1) series, our
contributions are the following:
\begin{itemize}
\item The diagonal and non-diagonal matrix elements both have L\'evy
  distributions but with different L\'evy indices.
\item The power-law tails of \gls{garch}(1,1) returns lead to a group
  of localized eigenvectors that correspond to large eigenvalues.
\item Auto-correlations in the returns reduce the localization of the
  eigenvectors. The fraction of localized eigenvectors decreases
  approximately as a quadratic function of $\phi = \text{corr}(r_t,
  r_{t-1})$. See figure \ref{fig:localization_ratio} and
  \ref{fig:localization_ratio2}.
\end{itemize}

In summary, auto-correlations in the returns create illusory
cross-correlations. To assess the true cross-correlations among the
assets in a market, one has to adopt a model for each of the return
series, infer the residuals and then assess the cross-correlations
among the residuals instead (c.f. chapter \ref{chp:PriceModels}).

However, in an efficient market, auto-correlations in the returns are
necessarily as weak as indistinguishable from measurement errors 
so that exploitable arbitrage opportunities do not exist. Hence
auto-correlations cannot be completely eliminated by taking residuals
of the returns. When estimating the covariance matrix and its
eigenvalues and eigenvectors, the illusory effects caused by
auto-correlations must be considered as an inherent source of
uncertainty.

% Correlations between sequences of stock returns from different assets
% have been studied by forming a covariance matrix. The stock returns
% have been simulated using different dynamical return models. Since a
% \gls{sv} model with Gaussian innovations have an unconditional
% distribution with all moments finite (see section \ref{sec:XieCalc}),
% we have focused our attention on \gls{garch}(1,1) models that were
% proven to have regularly varying (power-law) tails. We have
% investigated how a covariance matrix of \gls{garch}(1,1) returns differs from a
% covariance matrix of Gaussian returns, in terms of matrix elements and
% eigenvalues distributions. In particular, we have considered how the
% two matrices contrast with each other when the return series that they
% derive from are auto-correlated.

% In chapter \ref{chp:Gaussian} we studied how the distributions of
% elements and eigenvalues of a covariance matrix of Gaussian returns
% were altered by auto-correlations in the returns. For the matrix elements
% distribution, asymptotic Gaussian distributions were obtained. It was
% seen that the mean and the variance of the matrix elements
% distributions were changed by auto-correlations
% (Eq. \ref{eq:gaussian_mean2} to \ref{eq:gaussian_cii_variance2}). As 
% for the distribution of the maximum eigenvalue of C, it was shown that
% the maximum eigenvalue had approximately a gamma distribution even
% when the returns were auto-correlated. The mean of the gamma
% distribution was approximately quadratic in $\tan{\pi\phi \over 2}$,
% where $\phi$ was the coefficient of an \gls{ar}(1) process and
% characterized the strength of auto-correlation. On the other hand, the
% minimum eigenvalue was also increased by increased
% auto-correlations. As a whole, the eigenvalue spectrum widened as
% auto-correlation increased. See figure
% \ref{fig:GaussianMarkovSpectrumPDF}.

% In chapter \ref{chp:CrossCorrelationFat} we studied a covariance
% matrix of asset returns described by the \gls{garch}(1,1) 
% model. We found that both the diagonal and the non-diagonal matrix
% elements followed $\alpha$-stable (L\'evy) distributions, but had
% different L\'evy indices and scaling parameters (equations
% \ref{eq:garch_cii_dist} and \ref{eq:garch_cij2}). The eigenvalue
% distribution of the covariance matrix, as in the case of 
% Gaussian returns, was found to widen as auto-correlations
% increased. Meanwhile, both the largest and the smallest eigenvalues
% were seen to increase. See figure
% \ref{fig:GarchSpectrumAutocorrelated}.

% We have also investigated how localization of eigenvectors,
% i.e. dominance of one or a few basis vectors in the composition of an
% eigenvector, is affected by auto-correlations in the returns. Firstly,
% it is seen that localized eigenvectors correspond to large
% eigenvalues; secondly, it is found that strengthened auto-correlations
% reduce the extent of localization. These are consistent with what \AA
% berg found for Wishart-L\'evy matrices \cite{Aberg2013}.

% In particular, we find that the fraction of localized eigenvectors, as
% defined by an eigenvector's participation radio (see equation
% \ref{eq:IPR_def}) being lower than a given threshold or by its largest
% component (in the sense of absolute value) being higher than a
% threshold, falls approximately as a quadratic function of the
% auto-correlation strength (autoregressive coefficient of an
% \gls{ar}(1) process). See figure \ref{fig:localization_ratio} and
% \ref{fig:localization_ratio2}.

\chapter{Outlook}
Continuing from the obtained results, areas of future research may
include, first of all, a covariance matrix formed from \gls{sv} models
with power-law tails, which may be obtained by assuming, for example,
Student's t distribution or Normal Inverse Gaussian distribution for
the return's innovation.

Another interesting area could be models that treat the conditional
covariance matrix as an inherent part of the model specification
rather than treating it as an inferred quantity, which is the approach
taken by the current work. Predecessors on this path are the
multivariate \gls{garch} and \gls{sv} models
\cite{Mikosch2009}. To have sufficient flexibility, many of these
models involve a large number of parameters, which make it hard to fit
them to data and increase the chances of model mis-specification. A
model that strikes a good balance between flexibility and complexity
will be of great interest.
These approaches may well lead to more accurate volatility forecasts
and improved estimation of the uncertainties introduced by
auto-correlations in the returns.

Analytically deriving the eigenvalue and eigenvector distributions of
a covariance matrix formed from return series described by a realistic
model will be very challenging but interesting. It is also interesting to
see whether the methods developed for Wishart matrices, for example
the holonomic gradient method \cite{Hashiguchi2012}, can be applied to
the aforementioned covariance matrices.
Advancement in these areas will undoubtedly find applications in
e.g. principle component analysis and portfolio management.



\chapter{Self-reflection}
First of all, during the thesis project I learned a lot about models
of time series in finance, such as \gls{garch}, and \gls{sv} models,
as well as traditional time series analysis, for example, \gls{arima}
models. In addition, a study of continuous-time models gave me an
opportunity to increase my knowledge of stochastic processes,
probability theory and statistics.

Moreover, I have introduced myself to the random matrix theory,
multivariate analysis, and extreme value theory. These became relevant
when I worked on the covariance matrix of returns described by
Gaussian or \gls{garch} models.

In addition, I have become better at Matlab programming and dealing
with databases, and have grown more confident in numerical analysis.


\appendix
% \section{Fundamental Concepts \& Notations}
% \label{sec:FundamentalConcepts}
% This section is a list of a few concepts that may be unfamiliar to the
% reader and that we will often refer to later in the thesis:
% \begin{itemize}
% \item Return. Given a fixed time interval $[t - \Delta t, t]$, for example, a
%   day, a week, a month, etc, and a particular asset, for example, a
%   share in company ABC, the return of this asset over the time
%   interval is defined as
%   \begin{eqnarray*}
%     r_{t, \Delta t} &=& \ln S_{t} - \ln S_{t - \Delta t} \\
%     &=& \ln \left(1 + {S_{t} - S_{t - \Delta t} \over S_{t - \Delta t}}\right) \\
%     &\approx& {S_{t} - S_{t - \Delta t} \over S_{t - \Delta t}}
%   \end{eqnarray*}
%   where $S_t$ is the price of the asset at time $t$. $\Delta t$ is
%   sometimes called the time-lag of the return. Quite often, where
%   confusion is not possible, we will just write $r_t$ to mean $r_{t,
%     \Delta t}$, the time lag either does not matter or is clear from
%   the context.

% \item Autocorrelation. By autocorrelation, denoted $\rho_k$ here, we
%   mean the correlation between two temporally separated observations
%   of the same time series:
%   \begin{eqnarray*}
%     \rho_k &=& {
%       \E\left[(a_t -\E(a_t))(a_{t-k} - \E(a_{t-k})\right]
%       \over
%       \sqrt{\text{var}(a_t)}\sqrt{\text{var}(a_{t-k})}
%     }
%   \end{eqnarray*}
%   Here $\E(x)$ stands for the expectation value of $x$, and
%   $\text{var}(x)$ stands for the variance of $x$. $k$ is called the
%   time-lag and is the temporal seperation of the two observations
%   measured by the number of observations in between. For example, the
%   time-lag between the 1st and 3rd observation is 2.

% \item Covariance Matrix. Correlations between the returns of a
%   group of assets are described by the covariance matrix. When
%   the asset returns are described by a stable distribution law with
%   L\'evy index $\alpha$, the empirical covariance matrix is
%   constructed as
%   \begin{equation}
%     \label{eq:covariance}
%     \begin{aligned}
%       C_{ij} &= {1 \over T^{2/\alpha}} \sum_{t=1}^T [r_{i,t}-\E(r_i)]
%       [r_{j,t}-\E(r_j)] \\
%       &= {1 \over T^{2/\alpha}} RR'
%     \end{aligned}
%   \end{equation}
%   where $r_{i,t}$ is the return of the i-th asset at time t and is
%   placed at the entry (i, t) of matrix R; $R'$ denotes the
%   transpose of R. In most practical situations, one has abundant data
%   for each and every asset. Thus $T \geq N$ is assumed throughout this
%   thesis.

% \item Auto-regressive processes. A time series $r_t$ is called an
%   auto-regressive process of order $p$ and denoted AR(p), if it can be
%   written in the following form:
%   \begin{eqnarray*}
%     r_t &=& \sum_{i=1}^p \phi_i r_{t-i} + a_t
%   \end{eqnarray*}
%   where, for all $i$, $\phi_i \in (-1, 1)$, and the $a_t$'s are
%   independent and identically distributed (iid.) random variables with
%   zero mean. Obviously this implies the mean of $r_t$ is zero
%   too. Apart from this, their distribution of $a_t$ is not restricted
%   to any particular form.
  
%   Of particular interest to this thesis is the AR(1) process:
%   \begin{eqnarray*}
%     r_t = \phi r_{t-1} + a_t
%   \end{eqnarray*}
%   Its autocorrelation function $\rho_k = \text{corr}(r_t, r_{t-k})$
%   ($k = 0, 1, 2, \cdots$) can be easily shown to fall off
%   exponentially:
%   \begin{eqnarray*}
%     \rho_k &=& {\E(r_tr_{t-k}) - \E(r_{t-k})\E(r_{t}) \over
%       \sqrt{\var(r_t) \var(r_{t-k})}} \\
%     &=& {\phi \E(r_{t-1}r_{t-k}) + \E(a_t r_{t-k})
%       \over
%       \sqrt{\var(r_t) \var(r_{t-k})}
%     } \\
%     &=& \phi \rho_{k-1}
%   \end{eqnarray*}
%   where $\E(a_t r_{t-k}) = 0$ follows from the fact that any return
%   $r_{t1}$ must not depend on disturbances $a_{t2}$ that occur later
%   in time. The last equation means $\rho_k$ is a geometric series. Since
%   $\rho_0 = 1$, we have
%   \begin{eqnarray*}
%     \rho_k &=& \phi^k \\
%     |\rho_k| &=& e^{k\ln|\phi|} \\
%   \end{eqnarray*}
%   Note that $|\phi| < 1$ and hence $\ln|\phi| < 0$.

%   Although $k$ can only take integer values, it is still useful to
%   define a correlation time $\tau$ such that $\phi^\tau = 1/2$. Such a
%   quantity is more intuitive and constitutes a measure of
%   autocorrelations that is universal and comparable among different
%   time series' models. From the definition of $\tau$ we get
%   \begin{equation}
%     \label{eq:tau_def}
%     \begin{aligned}
%       \phi^\tau &= 1/2 \\
%       \tau &= -{\ln 2 \over \ln\phi} \\
%       \phi &= 2^{-1/\tau}
%     \end{aligned}
%   \end{equation}
%   Figure \ref{fig:AR1-autocorrelation} shows the autocorrelation
%   function of the AR(1) model. As proven above, this function decays
%   exponentially.
%   \begin{figure}[htb!]
%     %\vspace{-15mm}
%     \centering
%     \includegraphics[scale=0.5, clip=true, trim=113 229 115
%     139]{../pics/AR1-autocorrelation.pdf}
%     \caption{\small \it Autocorrelation function of the AR(1) model
%       with $\phi=1/\sqrt{2}$. Red circles: autocorrelations at $k=0, 1,
%       2, \cdots$. Blue line: $e^{t\ln\phi}$.}
%     \label{fig:AR1-autocorrelation}
%   \end{figure}
%   For more details about conventional time series' models, see
%   \cite{BoxJenkins94}.
% \end{itemize}

% % For purposes of later reference, we also list some notations that may
% % cause confusion to the reader:
% % \begin{itemize}
% % \item $\E(x)$ or $\mean{x}$: The expectation value of $x$.
% % \item $\var(x)$: the variance of $x$.
% % \item $\text{cov}(x,y)$: the covariance of $x$ and $y$.
% % \item $\text{corr}(x,y)$: the correlation between $x$ and $y$, i.e.
% %   \begin{equation*}
% %     \text{corr}(x,y) = {\text{cov}(x,y) \over \sqrt{\var(x)\var(y)}}
% %   \end{equation*}
% % \end{itemize}

% \section{Stylized facts}\label{sec:StylizedFacts}
% This section presents and explains a few ``stylized facts'',
% i.e. phenomena that are widely observed and accepted as true.
% \begin{itemize}
% \item Fat tails. It has been consistently reported by various studies
%   - for example \cite{Potters2003} and \cite{Mantegna2000} - that the
%   probability density function (PDF) of stock/index returns are not
%   Gaussian. Unlike the Gaussian PDF, which is symmetric and falls off
%   very quickly as its argument moves from the center to the outskirts
%   (tails), the PDF of stock/index returns are higher on the tails,
%   i.e. the probability of large fluctuations is higher than is dictated
%   by a Gaussian distribution - in fact, even higher than dictated by
%   an exponential function. Empirical studies suggest the distribution
%   function of the returns follows a power-law on the tails --- the
%   exponent of the power depends on the specific stock/index.

%   Figure \ref{fig:FatTail} illustrates this feature.
%   \begin{figure}[htb!]
%     \centering
%     \includegraphics[scale=0.6, clip=true, trim=92 229 110
%     140]{../pics/FatTail.pdf}
%     \caption{\small \it Fat tails of {\it Nordea Bank} 15-minute
%       returns during the period 2013-10-10 and 2014-01-29. The returns
%       are computed using minute-by-minute average prices.}
%     \label{fig:FatTail}
%   \end{figure}

% \item Non-zero skewness. Apart from fat tails, the PDF of stock/index
%   returns are often also skewed. If the skewness is positive
%   (negative), the probability of very large positive (negative)
%   returns is higher than that of very large negative (positive)
%   returns, even though the mean of the returns is 0 or extremely
%   close to 0.

%   Table \ref{tab:EmpiricalSkewness} lists the skewness of a few
%   Swedish stocks traded on the Stockholm OMX market.
%   \begin{table}[htb!]
%     \footnotesize
%     \centering
%     \begin{tabular}{|c|c|c|c|c|c|c|}
%       \hline
%       Nordea Bank & Volvo B & Boliden & ABB Ltd & H\&M & Scania
%       B & Ericsson B \\
%       \hline
%       0.1362 & 0.0471 & 0.0567 & -0.0364 & -0.0168 &
%       -1.1554 & 0.2086 \\
%       \hline
%     \end{tabular}
%     \caption{\small \it Skewness of Stock Returns. All the returns
%       have time-lag of 15 minutes and are computed using paid prices
%       between 2013-10-10 and 2014-01-29. }
%     \label{tab:EmpiricalSkewness}
%   \end{table}

% \item Higher-order autocorrelation. To the lowest order, return series
%   are not auto-correlated --- if they are, the auto-correlations would
%   present an obvious opportunity of making easy profit and be
%   exploited and vanish as soon as they appear. However, the squared
%   returns do have significant auto-correlations, as figure
%   \ref{fig:nordea_15min_acf} illustrates. These auto-correlations
%   suggest the variances of the returns at subsequent time steps are
%   correlated. These are referred to as higher-order auto-correlations
%   and make a subjet of returns' models.
  
% \end{itemize}
\chapter{Case Study of Some Intraday Series}
\label{chp:appendix2}
\section{Nordea 30-minute Returns}
\label{sec:nordea2_30min}
In this section we study the volatility of Nordea Bank 30-minute
returns in the period 2013/10/10 - 2014/04/04. The realized
volatilities that approximate the volatilities of these 30-minute
returns are computed using 1-minute returns. As in the previous cases,
this choice of 1-minute returns is confirmed by the normality of the
quotient $(r_t - \E(r_t))/\sigma_t$, which is shown in figure
\ref{fig:nordea3_quotient}.
\begin{figure}[htb!]
  \centering
  \includegraphics[scale=0.4, clip=true, trim=78 255 108
  120]{../pics/nordea3_quotient.pdf}
  \caption{\small \it Normal probability plot of $(r_t -
    \E(r_t))/\sigma_t$}
  \label{fig:nordea3_quotient}
\end{figure}
The auto-correlations of $\ln \sigma_t$ is shown in figure
\ref{fig:nordea3_lv_acf}, where we see an apparent seasonality of
16. By differencing we get $w_t = (1-B)(1-B^s)\ln\sigma_t$ where
$s=16$. Its auto-correlations are shown in figure
\ref{fig:nordea3_w_acf}.
\begin{figure}[htb!]
  \centering
  \subfigure[]{
    \includegraphics[scale=0.4, clip=true, trim=86 256 101
    121]{../pics/nordea3_lv_acf.pdf}
    \label{fig:nordea3_lv_acf}
  }
  \subfigure[]{
    \includegraphics[scale=0.4, clip=true, trim=86 256 103
    122]{../pics/nordea3_w_acf.pdf}
    \label{fig:nordea3_w_acf}
  }
  \caption{\small \it \ref{fig:nordea3_lv_acf}: Auto-correlations
    (ACF) of $\ln \sigma_t$; \ref{fig:nordea3_w_acf}: Auto-correlations
    (ACF) of $w_t = (1-B)(1-B^s)\ln\sigma_t$.}
\end{figure}
Once again, this auto-correlation structure
points to a seasonal moving average model:
\begin{eqnarray*}
  w_t &=& (1-\theta B)(1-\Theta B^s) y_t
\end{eqnarray*}
where $y_t$ is assumed to have Gaussian distribution, neglecting
slight non-normality as before. Maximum likelihood estimation gives
the parameter values listed in table \ref{tab:nordea3_sv_param}.
\begin{table}[htb!]
  \centering
  \begin{tabular}{|c|c|c|c|}
    \hline
    Parameter & $\theta$ & $\Theta$ & $\var(y_t)$\\
    \hline
    Value & 0.7612 & 0.8036 & 0.0736\\
    \hline
  \end{tabular}
  \caption{\small \it Parameter values of the Seasonal Moving Average
    model}
  \label{tab:nordea3_sv_param}
\end{table}
To fit a \gls{garch} model to the same series, we assume Gaussian
innovations (c.f. equation \ref{eq:garch_def}). The result is a
\gls{garch}(1,1) model. Its parameters are listed in table
\ref{tab:nordea3_garch_param}.
\begin{table}[htb!]
  \centering
  \begin{tabular}{|c|c|c|c|}
    \hline
    Parameter & $\alpha_0$ & $\alpha_1$ & $\beta_1$ \\
    \hline
    Value & $2.3 \times 10^{-7}$ & 0.05 & 0.9\\
    \hline
  \end{tabular}
  \caption{\small \it Parameter values of GARCH(1,1) model fitted to
    Nordea 30-minute returns.}
  \label{tab:nordea3_garch_param}
\end{table}

As before, we compare the accuracies of the \gls{garch} and the stochastic
volatility model by comparing their one-ste-ahead forecasts. We
compute the difference between their forecasts and the realized
volatilities and then look at the statistics of these difference
values. Firstly we show the mean and the standard deviation of the
differences in table \ref{tab:nordea3_diff_1}.
\begin{table}[htb!]
  \centering
  \begin{tabular}{|c|c|c|c|}
    \hline
     & \gls{sv} & \gls{garch} & Sample mean \\
     \hline
    $\E(\ln \sigma^F_t - \ln \hat{\sigma}_t)$ & -0.0047 & 0.0130 & 0.0584 \\
    \hline
     $\text{std}(\ln \sigma^F_t - \ln \hat{\sigma}_t)$ & 0.2602 &
     0.3194 & 0.3093 \\
    \hline
 \end{tabular}
  \caption{\small \it Mean and standard deviation of the 3 kinds of
    forecasts}
  \label{tab:nordea3_diff_1}
\end{table}
It is seen in this table that the \gls{sv} forecasts have a mean whose
absolute value is just above 1/3 of that of the \gls{garch} forecasts. The
standard deviation of the \gls{sv} forecasts is also smaller than that of
\gls{garch}. In addition, we check the distribution of
$\ln \sigma^F_t - \ln \hat{\sigma}_t$, and the percentage of ``good''
forecasts with respect to different criteria of goodness. These are
shown in figure \ref{fig:nordea3_diff} and table
\ref{tab:nordea3_diff_2}, respectively.
\begin{figure}[htb!]
  \centering
    \includegraphics[scale=0.54, clip=true, trim=21 312 0
    179]{../pics/nordea3_diff.pdf}
  \caption{\small \it Blue: SV forecasts; Green: GARCH forecasts; Red:
    sample mean forecasts. Left: $P(\ln \sigma^F_t - \ln \hat{\sigma}_t < x)$;
    Right: $P(\ln \sigma^F_t - \ln \hat{\sigma}_t > x)$. Horizontal: $x$.}
  \label{fig:nordea3_diff}
\end{figure}
\begin{table}[htb!]
  \centering
  \begin{tabular}{|c|c|c|c|}
    \hline
    ${|\ln \sigma^F_t - \ln \hat{\sigma}_t| \over |\ln
      \hat{\sigma}_t|}$ &
    \gls{sv} & \gls{garch} & Sample Mean \\
    \hline
    1\% & 18\% & 18\% & 15\% \\
    \hline
    5\% & 81\% & 71\% & 69\% \\
    \hline
    10\% & 98\% & 93\% & 95\% \\
    \hline
  \end{tabular}
  % \begin{tabular}{|c|c|c|c|}
  %   \hline
  %   ${|\ln \sigma^F_t - \ln \hat{\sigma}_t| \over |\ln
  %     \hat{\sigma}_t|}$ &
  %   \gls{sv} & \gls{garch} & Sample Mean \\
  %   \hline
  %   1\% & 17.77\% & 17.77\% & 15.38\% \\
  %   \hline
  %   5\% & 80.90\% & 71.09\% & 69.23\% \\
  %   \hline
  %   10\% & 98.14\% & 93.10\% & 95.23\% \\
  %   \hline
  % \end{tabular}
  \caption{\small \it Percentage of ``good'' forecasts as defined by
    deviating nore more than 1\%, 5\% and 10\% from the measured
    realized volatility.}
  \label{tab:nordea3_diff_2}
\end{table}
It is clear that the \gls{sv} forecasts are considerably more accurate.

\section{Volvo 15-minute returns in 2013/14}
\label{sec:volvo}
In this section we study the volatility of Volvo B 15-minute returns
during the period 2013/10/10 - 2014/04/04.
The subject of modeling is the realized volatility
computed as the square root of the sum of squared 1-minute
returns. The normal probability plot of the quotient $(r_t -
\E(r_t))/\sigma_t$ is shown in figure \ref{fig:volvo_15_quotient},
from which one can see it is normally distributed, confirming the
choice of 1-minute returns for computing the realized volatility.
\begin{figure}[htb!]
  \centering
  \includegraphics[scale=0.4, clip=true, trim=83 259 110 
  220]{../pics/volvo_15_quotient.pdf}
  \caption{\small \it Normal probability plot of $(r_t -
    \E(r_t))/\sigma_t$}
  \label{fig:volvo_15_quotient}
\end{figure}
The auto-correlations of $\ln \sigma_t$ and $w_t = (1-B)(1-B^s) \ln
\sigma_t$ are plotted in figure \ref{fig:volvo_15_lv_acf} and
\ref{fig:volvo_15_w_acf}. The former clearly shows the seasonality
$s = 33$ in the auto-correlations of $\ln \sigma_t$.

\begin{figure}[htb!]
  \centering
  \subfigure[]{
  \includegraphics[scale=0.4, clip=true, trim=90 259 103
  220]{../pics/volvo_15_lv_acf.pdf}
  \label{fig:volvo_15_lv_acf}
  }
  \subfigure[]{
  \includegraphics[scale=0.4, clip=true, trim=90 259 103
  220]{../pics/volvo_15_w_acf.pdf}
  \label{fig:volvo_15_w_acf}
  }
  \caption{\small \it Left: auto-correlations of $\ln \sigma_t$;
    Right: auto-correlations of $w_t = (1-B)(1-B^s) \ln \sigma_t$.}
\end{figure}

Based on the auto-correlations of $w_t$, a seasonal moving average
model is estimated:
\begin{eqnarray*}
  (1-B)(1-B^s) \ln\sigma_t &=& (1-\theta_1B - \theta_2B^2 - \theta_3
  B^3)(1 - \Theta B^s)y_t
\end{eqnarray*}
By maximum likelihood estimation, the parameter values listed in table
\ref{tab:volvo_15_sv_param} are obtained.
\begin{table}[htb!]
  \centering
  \begin{tabular}{|c|c|c|c|c|c|}
  \hline
  Parameter & $\theta_1$ & $\theta_2$ & $\theta_3$ & $\Theta$ & 
 $\var(y_t)$ \\
 \hline
 Value & 0.7491 & 0.1170 & 0.0571 & 0.8646 & 0.1245 \\
  \hline
  \end{tabular}
  \caption{\small \it Parameter values of the stochastoc volatility
    (SV) model.}
  \label{tab:volvo_15_sv_param}
\end{table}

A \gls{garch} model (c.f. equation \ref{eq:garch_def}) assuming Gaussian
innovations is also fitted to the same series. The parameter values
are listed in table \ref{tab:volvo_15_garch_param}:
\begin{table}[htb!]
  \centering
  \begin{tabular}{|c|c|c|c|c|c|c|}
  \hline
  Parameter & $\alpha_0$ & $\alpha_1$ & $\beta_{1}$ \\
  \hline
  Value & $6.54 \times 10^{-7}$ & 0.1565 & 0.6445\\
  \hline
  \end{tabular}
  \caption{\small \it Parameter values of the GARCH model.}
  \label{tab:volvo_15_garch_param}
\end{table}

Performing one-step-ahead forecasts using both models gives the series
$\ln\sigma^F_t - \ln\hat{\sigma}_t$, where $\ln\hat{\sigma}_t$ are the
measured realized volatilities. The mean and standard deviation of
$\ln\sigma^F_t - \ln\hat{\sigma}_t$ are shown in table
\ref{tab:volvo_15_stat}.
\begin{table}[htb!]
  \centering
  \begin{tabular}{|c|c|c|c|}
    \hline
     & \gls{sv} & \gls{garch} & Sample mean \\
     \hline
    $\E(\ln \sigma^F_t - \ln \hat{\sigma}_t)$ & 0.0035 & 0.0313 & -0.1448 \\
    \hline
     $\text{std}(\ln \sigma^F_t - \ln \hat{\sigma}_t)$ & 0.3356 &
     0.3894 & 0.3826 \\
    \hline
 \end{tabular}
  \caption{\small \it Mean and standard deviation of the 3 kinds of
    forecasts}
  \label{tab:volvo_15_stat}
\end{table}
We see that both the \gls{sv} and the \gls{garch} model give biased forecasts, but
the \gls{sv} forecasts are only 1/10 as biased as those of \gls{garch} (0.0035
vs. 0.0313). In addition, the standard deviation of the \gls{sv} forecasts
is smaller too. These results are confirmed by the distribution of
$\ln \sigma^F_t - \ln \hat{\sigma}_t$ and the fraction of ``good''
forecasts, which are shown in figure \ref{fig:volvo_15_diff} and table
\ref{tab:volvo_15_diff}, respectively.
\begin{figure}[htb!]
  \centering
    \includegraphics[scale=0.54, clip=true, trim=27 276 0
    236]{../pics/volvo_15_diff.pdf}
  \caption{\small \it Blue: SV forecasts; Green: GARCH forecasts; Red:
    sample mean forecasts. Left: $P(\ln \sigma^F_t - \ln \hat{\sigma}_t < x)$;
    Right: $P(\ln \sigma^F_t - \ln \hat{\sigma}_t > x)$. Horizontal: $x$.}
  \label{fig:volvo_15_diff}
\end{figure}
\begin{table}[htb!]
  \centering
  \begin{tabular}{|c|c|c|c|}
    \hline
    ${|\ln \sigma^F_t - \ln \hat{\sigma}_t| \over |\ln
      \hat{\sigma}_t|}$ &
    \gls{sv} & \gls{garch} & Sample Mean \\
    \hline
    1\% & 58\% & 60\% & 39\% \\
    \hline
    5\% & 84\% & 84\% & 66\% \\
    \hline
    10\% & 97\% & 95\% & 91\% \\
    \hline
  \end{tabular}
  \caption{\small \it Percentage of ``good'' forecasts as defined by
    deviating nore more than 1\%, 5\% and 10\% from the measured
    realized volatility.}
  \label{tab:volvo_15_diff}
\end{table}

\section{Ericsson 15-minute Returns}
\label{sec:ericsson_15min}
In this section we model the volatility of {\it Ericsson B}
15-minute returns during the period 2013/10/10 - 2014/04/04.
Using the same method as with other series, we first find the right
higher-frequency returns that best approximate the volatility of the
15-minute returns. These turn out to be 50-second returns, as figure
\ref{fig:ericsson_15_quotient} shows.
\begin{figure}[htb!]
  \centering
  \includegraphics[scale=0.4, clip=true, trim=79 259 108
  121]{../pics/ericsson_15_quotient.pdf}
  \caption{\small \it Normal probability plot of $(r_t -\E(r_t))/
    \sigma_t$. $\sigma_t^2$ is computed as the sum of squared
    50-second returns.}
  \label{fig:ericsson_15_quotient}
\end{figure}

Following the procedure described in previous sections, an ARIMA model is
found for this series:
\begin{eqnarray*}
  (1-B)(1-B^s) \ln\sigma_t &=& (1- \theta_1B - \theta_2B^2 -
  \theta_3B^3)(1 - \Theta B^s) y_t
\end{eqnarray*}
where the seasonality $s$ is 33 and the other parameter values are
estimated to be those listed in table \ref{tab:ericsson_15_params}.
\begin{table}[htb!]
  \centering
  \begin{tabular}{|c|c|c|c|c|c|c|}
    \hline
    Parameter & $\theta_1$ & $\theta_2$ & $\theta_3$ & $\Theta$ &
    $\var(y_t)$ \\
    \hline
    Value & 0.8078 & 0.0454 & 0.0943 & 0.8798 & 0.1242 \\
    \hline
  \end{tabular}
  \caption{\small \it Ericsson B log-volatility parameters}
  \label{tab:ericsson_15_params}
\end{table}

Then a \gls{garch} model is also found with parameter values listed in
table \ref{tab:ericsson_15_garch_params}:
\begin{table}[htb!]
  \centering
  \begin{tabular}{|c|c|c|c|c|}
    \hline
    Parameter & $\alpha_0$ & $\alpha_1$ & $\alpha_{s}$ & $\beta_{s}$ \\
    \hline
    Value & $2.4181 \times 10^{-7}$ & 0.1513 & 0.1409 & 0.6027 \\
    \hline
  \end{tabular}
  \caption{\small \it GARCH model of Volvo B 30-minute returns}
  \label{tab:ericsson_15_garch_params}
\end{table}

The forecasts from both models are compared as follows: Table
\ref{tab:ericsson_15_diff1} shows the mean and standard deviation of
the 3 kinds of forecasts:
\begin{table}[htb!]
  \centering
  \begin{tabular}{|c|c|c|c|}
    \hline
    & \gls{sv} & \gls{garch} & Sample mean \\
    \hline
    $\E(\ln \sigma^F_t - \ln \hat{\sigma}_t)$ & 0.0035 &
    0.0600 & 0.1116 \\
    \hline
    $\text{std}(\ln \sigma^F_t - \ln \hat{\sigma}_t)$ & 0.3278 &
    0.3667 & 0.3590 \\
    \hline
  \end{tabular}
  \caption{\small \it Standard deviation of $\ln\sigma^F_t -
    \ln\hat{\sigma}_t$}
  \label{tab:ericsson_15_diff1}
\end{table}
Consistent with previous results, the bias introduced by the \gls{sv} model
is significantly smaller than that introduced by \gls{garch}. Figure
\ref{fig:ericsson_15_diff2} shows the distribution of the difference
between a forecast and its corresponding measured realized volatility,
i.e. $\ln\sigma^F_t - \ln\hat{\sigma}_t$; table
\ref{tab:ericsson_15_diff3} compares the percentage of ``good''
forecasts using the 3 alternatives.
\begin{figure}[htb!]
  \centering
    \includegraphics[scale=0.5, clip=true, trim=45 298 21
    165]{../pics/ericsson_15_diff2.pdf}
  \caption{\small \it Blue: SV forecasts; Green: GARCH forecasts; Red:
    sample mean forecasts. Left: $P(\ln \sigma^F_t - \ln \hat{\sigma}_t < x)$;
    Right: $P(\ln \sigma^F_t - \ln \hat{\sigma}_t > x)$. Horizontal: $x$.}
  \label{fig:ericsson_15_diff2}
\end{figure}

\begin{table}[htb!]
  \centering
  \begin{tabular}{|c|c|c|c|}
    \hline
    ${|\ln \sigma^F_t - \ln \hat{\sigma}_t| \over |\ln
      \hat{\sigma}_t|}$ &
    \gls{sv} & \gls{garch} & Sample Mean \\
    \hline
    1\% & 17\% & 17\% & 16\% \\
    \hline
    5\% & 73\% & 69\% & 69\% \\
    \hline
    10\% & 96\% & 93\% & 94\% \\
    \hline
  \end{tabular}
  \caption{\small \it Fraction of ``good'' forecasts as defined by
    ${|\ln \sigma^F_t - \ln \hat{\sigma}_t| \over |\ln
      \hat{\sigma}_t|}$ being less than 1\%, 5\% and 10\%.}
  \label{tab:ericsson_15_diff3}
\end{table}
It is clear from figure \ref{fig:ericsson_15_diff2} and table
\ref{tab:ericsson_15_diff3} that the \gls{sv} model out-performs \gls{garch}. We
see that the \gls{sv} model yields considerably higher fractions of good
forecasts by all criteria, and in particular, gives much few
under-estimates.

\section{Ericsson 30-minute Returns}\label{sec:ericsson_30min}
In this section we look at the 30-minute returns of Ericsson B during
the period 2013/10/10 - 2014/04/04. Since the methods are the same as
with other series, we shall only present the results here.

First of all, the volatility of this series is found to be well
approximated by realized volatilities computed from 75-second
returns. The normal probability plot is shown in figure
\ref{fig:ericsson_30_quotient}.
\begin{figure}[htb!]
  \centering
  \includegraphics[scale=0.4, clip=true, trim=79 259 108
  121]{../pics/ericsson_30_quotient.pdf}
  \caption{\small \it Normal probability plot of $(r_t -\E(r_t))/
    \sigma_t$. $\sigma_t^2$ is computed as the sum of squared
    75-second returns.}
  \label{fig:ericsson_30_quotient}
\end{figure}
The following ARIMA model is found for the log-volatility $\ln
\sigma_t$:
\begin{eqnarray*}
  (1-B)(1-B^s) \ln\sigma_t &=& (1- \theta_1B - \theta_2B^2) (1 -
  \Theta B^s) y_t 
\end{eqnarray*}
where the seasonality $s = 16$. The parameter values are estimated to
be those listed in table
\ref{tab:ericsson_30_params}:
\begin{table}[htb!]
  \centering
  \begin{tabular}{|c|c|c|c|c|c|c|}
    \hline
    Parameter & $\theta_1$ & $\theta_2$ & $\Theta$ &
    $\var(y_t)$ \\
    \hline
    Value & 0.6842 & 0.2470 & 0.8391 & 0.0918 \\
    \hline
  \end{tabular}
  \caption{\small \it Ericsson B log-volatility parameters}
  \label{tab:ericsson_30_params}
\end{table}
A \gls{garch} model is also found with parameters listed in table
\ref{tab:ericsson_30_garch_params}.
\begin{table}[htb!]
  \centering
  \begin{tabular}{|c|c|c|c|c|}
    \hline
    Parameter & $\alpha_0$ & $\alpha_1$ & $\alpha_{s}$ & $\beta_{s}$ \\
    \hline
    Value & $7.4856 \times 10^{-7}$ & 0.0532 & 0.1506 & 0.6778 \\
    \hline
  \end{tabular}
  \caption{\small \it GARCH(1, 1) model of Ericsson B 30-minute
    returns}
  \label{tab:ericsson_30_garch_params}
\end{table}

Table \ref{tab:ericsson_30_diff1} shows the mean and standard
deviation of the difference series $\ln\sigma^F_t -
\ln\hat{\sigma}_t$, where $\ln\sigma^F_t$ is the forecast
log-volatility and $\ln\hat{\sigma}_t$ is the measured realized
volatility.
\begin{table}[htb!]
  \centering
  \begin{tabular}{|c|c|c|c|}
    \hline
    & \gls{sv} & \gls{garch} & Sample mean \\
    \hline
    $\E(\ln \sigma^F_t - \ln \hat{\sigma}_t)$ & 0.0004 &
    0.0526 & 0.1039 \\
    \hline
    $\text{std}(\ln \sigma^F_t - \ln \hat{\sigma}_t)$ & 0.2519 &
    0.3019 & 0.2963 \\
    \hline
  \end{tabular}
  \caption{\small \it Standard deviation of $\ln\sigma^F_t -
    \ln\hat{\sigma}_t$}
  \label{tab:ericsson_30_diff1}
\end{table}
Apparently, the \gls{garch} forecasts have a fairly large bias compared with
those of the \gls{sv} model. Turning to figure \ref{fig:ericsson_30_diff2}
and table \ref{tab:ericsson_30_diff3}, we also see the \gls{sv} model
performs consistently better --- a higher percentage of ``good''
forecasts are delivered and, while it under-estimates to around the
same extent as does \gls{garch}, it certainly over-estimates a lot
less. These are confirmative to what we have observed for the other
series.
\begin{figure}[htb!]
  \centering
    \includegraphics[scale=0.5, clip=true, trim=33 281 11
    150]{../pics/ericsson_30_diff2.pdf}
  \caption{\small \it Blue: SV forecasts; Green: GARCH forecasts; Red:
    sample mean forecasts. Left: $P(\ln \sigma^F_t - \ln \hat{\sigma}_t < x)$;
    Right: $P(\ln \sigma^F_t - \ln \hat{\sigma}_t > x)$. Horizontal: $x$.}
  \label{fig:ericsson_30_diff2}
\end{figure}

\begin{table}[htb!]
  \centering
  \begin{tabular}{|c|c|c|c|}
    \hline
    ${|\ln \sigma^F_t - \ln \hat{\sigma}_t| \over |\ln
      \hat{\sigma}_t|}$ &
    \gls{sv} & \gls{garch} & Sample Mean \\
    \hline
    1\% & 22\% & 16\% & 16\% \\
    \hline
    5\% & 78\% & 69\% & 69\% \\
    \hline
    10\% & 98\% & 97\% & 96\% \\
    \hline
  \end{tabular}
  \caption{\small \it Fraction of ``good'' forecasts as defined by
    ${|\ln \sigma^F_t - \ln \hat{\sigma}_t| \over |\ln
      \hat{\sigma}_t|}$ being less than 1\%, 5\% and 10\%.}
  \label{tab:ericsson_30_diff3}
\end{table}

\section{Volatility of Forecast accuracy}
\label{sec:forecast_volatility}
In this section we compare the volatility of the accuracy of
\gls{garch} and \gls{sv} models' forecasts. Table
\ref{tab:5percent_values} lists the the percentage of forecasts
that deviate less than 5\% from the corresponding realized
volatilities, and also gives the standard deviation of the 6 numbers.
It is seen that the forecasts of \gls{sv} models have standard
deviation 0.0432 while those of \gls{garch} have 0.0722. Clearly the
accuracy of \gls{sv} forecasts varies considerably less than does that
of \gls{garch} forecasts. In other words, a \gls{sv} model performs
more consistently around a level of accuracy.
  \begin{table}[htb!]
    \centering
    \begin{tabular}{|c|c|c|c|c|}
      \hline
      & Volvo 15m & Volvo 30m & Ericsson 15m & Ericsson 30m \\
      \hline
      SV & 84\% & 72\% & 73\% & 78\% \\
      \hline
      GARCH & 84\% & 62\% & 69\% & 69\% \\
      \hline
      \hline
      & Nordea 1 15m & Nordea 2 30m & \multicolumn{2}{c|}{standard deviation}\\
      \hline
      SV & 75\% & 81\% & \multicolumn{2}{c|}{0.0432} \\
      \hline
      GARCH & 71\% & 71\% & \multicolumn{2}{c|}{0.0722} \\
      \hline
    \end{tabular}
    % \begin{tabular}{|c|c|c|c|c|}
    %   \hline
    %   & Volvo 15m & Volvo 30m & Ericsson 15m & Ericsson 30m \\
    %   \hline
    %   SV & 0.8351 & 0.7162 & 0.7268 & 0.7844 \\
    %   \hline
    %   GARCH & 0.8363 & 0.6154 & 0.6850 & 0.6900 \\
    %   \hline
    %   \hline
    %   & Nordea 1 15m & Nordea 2 30m & std.\\
    %   \hline
    %   SV & 0.7457 & 0.8090 & 0.0432 \\
    %   \hline
    %   GARCH & 0.7062 & 0.7109 & 0.0722 \\
    %   \hline
    % \end{tabular}
    \caption{\small \it Comparison of the percentage of forecasts that
      lie within 5\% of the corresponding realized
      volatilities. Nordea 1 refers to the Nordea data set of
      2012/01/16 - 2012/04/20 (see section \ref{chp:nordea_15min})
      and Nordea 2 refers to the data set of 2013/10/10 - 2014/04/04
      (see section \ref{sec:nordea2_30min}).}
    \label{tab:5percent_values}
  \end{table}


\chapter{Normalization Constant in the Unconditional PDF of the
  Symmetric SV model}
\label{chp:symmetric_SV_norm_const}
In the following we work out the normalization constant C used in
section \ref{sec:SLV_Symmetric}:
\begin{eqnarray*}
  \int_{-\infty}^{\infty} f_{r'}(x) dx &=& {1 \over C}{e^{\sigma^2 / 2} \over
    \sqrt{8\pi}} \int_{-\infty}^{\infty} \text{erfc} \left({1 \over
      \sqrt{2}\sigma} \ln{|x| \over \sqrt{\ln 4}} + {\sigma \over
      \sqrt{2}} \right) dx\\
  &=& {2 \over C}{e^{\sigma^2 / 2} \over
    \sqrt{8\pi}} \int_{0}^{\infty} \text{erfc} \left({1 \over
      \sqrt{2}\sigma} \ln{x \over \sqrt{\ln 4}} + {\sigma \over
      \sqrt{2}} \right) dx\\
\end{eqnarray*}
Let
\begin{eqnarray*}
  a &=& {1 \over \sigma \sqrt 2} \\
  b &=& -{1 \over 2 \sigma \sqrt 2}\ln\ln 4 + {\sigma \over \sqrt 2} \\
  y &=& a\ln|x|+b \\
\end{eqnarray*}
Then
\begin{eqnarray*}
  \int_{-\infty}^{\infty} f_{r'}(x) dx &=& {2 \over C}{e^{\sigma^2 / 2} \over
    \sqrt{8\pi}} \int_{-\infty}^{\infty} dy {e^{(y-b)/a} \over a}
  \text{erfc}(y) \\
  &=& {2 \over C}{e^{\sigma^2 / 2} \over
    \sqrt{8\pi}} \left.e^{(y-b)/a}
    \text{erfc}(y)\right|_{y=-\infty}^{\infty}
  + {2 \over C}{e^{\sigma^2 / 2} \over
    \sqrt{8\pi}} \int_{-\infty}^{\infty} dy {2 \over \sqrt \pi}
  e^{(y-b)/a} e^{-y^2} \\
  &=& {2 \over C}{e^{\sigma^2 / 2} \over \sqrt{8\pi}}  2 e^{\ln\ln 4/2
    - \sigma^2/2}\\
  C &=& \sqrt{2 \ln 4\over \pi} \\
  &\approx& 0.9394
\end{eqnarray*}

\chapter{PDF of the covariance matrix of
auto-correlated Gaussian Returns}
\label{app:pdf_gaussian1}
When autocorrelations exist among the columns of the Gaussian returns
matrix $\mtx R$, the distribution of the covariance matrix $\mtx C =
\mtx{R} \mtx{R'}$ is no longer Wishart but is nonethelss closely
related to it. In the following we consider the situation where
$r_{i,t}$ can be represented as a vector auto-regressive
process. Specifically, suppose
\begin{eqnarray*}
  \vec{r}_t &=& \sum_{k=1}^p \phi_k \vec{r}_{t-k} + \vec{a}_t \\
  \vec{a}_t &=& \vec{r}_t - \sum_{k=1}^p \phi_k \vec{r}_{t-k}
\end{eqnarray*}
where $\vec{a}_t = (a_{1,t}, a_{2,t}, \cdots, a_{N,t})' \sim N(0,
\mtx \Sigma)$ and comprise the columes of $\mtx A$; $\vec{r}_t$ are the
columes of $\mtx R$. Here $N(0, \mtx \Sigma)$ denotes the multivarate
normal distribution with covariance matrix $\mtx \Sigma$. The last equation
can be written in matrix form
\[
\mtx{A} = \mtx{R M}
\]
For example, in the case of an AR(1) process
\begin{equation*}
  \begin{pmatrix}
    a_{1,1} & a_{1,2} & \cdots & a_{1,T} \\
    \vdots & \ddots & \vdots \\
    a_{N,1} & a_{N,2} & \cdots & a_{N,T} \\
  \end{pmatrix} =
  \begin{pmatrix}
    r_{1,1} & r_{1,2} & \cdots & r_{1,T} \\
    \vdots & \ddots & \vdots \\
    r_{N,1} & r_{N,2} & \cdots & r_{N,T} \\
  \end{pmatrix}
  \begin{pmatrix}
    1 & -\phi &   &   & \\
      & 1 & -\phi &   & \\
      &   & \ddots  & \ddots &   \\
      &   &   & 1 & -\phi \\
      &   &   &   & 1 \\
  \end{pmatrix}
\end{equation*}
Let $\mtx{QR} = \mtx{RM}$, then $\mtx{Q} = \mtx{RMR^{-1}}$. Since the
set of $\{r_{ij}\}$ for which $\text{det} \mtx{R} = 0$ has probability
zero, a matrix $\mtx Q$ satisfying the above equation almost surely
exists. Thus $\mtx{A} = \mtx{RM} = \mtx{QR}$ and $\mtx{AA'} =
\mtx{QRR'Q'}$ Since the columns of A are not auto-correlated, $\mtx{AA'}
\sim W(\mtx{\Sigma}, T)$. Then $\mtx{RR'} \sim W(\mtx{Q^{-1} \Sigma
  Q'^{-1}}, T)$ follows from \cite{Anderson2003}, section 7.3.3. Now we
observe the Wishart \gls{pdf}
\begin{equation}
  \label{eq:WishartPDF}
  f_W(\mtx{X}) = { |\det X|^{(T-N-1)/2} \exp\left(-{1 \over 2}\tr
      \mtx{\Sigma^{-1}X} \right)
    \over
    2^{NT/2}\pi^{N(N-1)/4}|\det \mtx{\Sigma}|^{T/2}
    \prod_{i=1}^N \Gamma\left[(n+1-i)/2\right]
  }
\end{equation}
where, assuming $\mtx X = \mtx{UU'}$, $\mtx \Sigma$ is the covariance
matrix of the columns of $\mtx U$. we notice in \ref{eq:WishartPDF}
that $\mtx \Sigma$ enters $f_W(\cdot)$ through $\tr (\mtx{\Sigma^{-1}
  RR'})$ and $\det \mtx{\Sigma}$. $\mtx{RR'}$ enters through $\tr
\mtx{\Sigma^{-1} RR'}$ and $\det \mtx{RR'}$. Clearly
\begin{eqnarray*}
  \det \mtx{Q^{-1}\Sigma Q'^{-1}} &=& \det \mtx{\Sigma \over (\det Q)^2} \\
  &=& \det \mtx{\Sigma} \over (\det \mtx{M})^2 \\
  &=& \det \mtx{\Sigma}
\end{eqnarray*}
As to $\tr \mtx{\Sigma^{-1} C}$, we have
\begin{eqnarray*}
  && \tr\left[\mtx{(Q^{-1}\Sigma Q'^{-1})^{-1} RR'}\right] \\
  &=& \tr\left[\mtx{Q'\Sigma^{-1} Q RR'}\right] \\
  &=& \tr\left[\mtx{R'^{-1}M'R'\Sigma^{-1} RMR^{-1} RR'}\right] \\
  &=& \tr\left[\mtx{\Sigma^{-1} RM (RM)'}\right] \\
  &=& \tr\left[\mtx{\Sigma^{-1} AA'}\right] \\
\end{eqnarray*}

Moreover, $\det \mtx{RR'} = \det \mtx{AM^{-1} M'^{-1} A'} = \det
\mtx{AA'}$. Thus if we use $f_R(\cdot)$ to denote the joint
probability density of the entries of $\mtx{RR'}$ and $f_A(\cdot)$ to
denote that of $\mtx{AA'}$, we can write
\begin{equation}\label{eq:cross-corr-matrix-PDF}
  f_R(\mtx{RR'}) = f_A(\mtx{RMM'R'})
\end{equation}
Substituting for $f_A$ the Wishart \gls{pdf} \ref{eq:WishartPDF}, we
get the \gls{pdf} of $\mtx{RR'}$.

\chapter{Asymptotic Distributions of the Elements of a Covariance
  Matrix of Autocorrelated Gaussian Returns}
\label{chp:gaussian_elements_dist}
In this appendix we derive the approximate distributions of the
elements of a Covariance Matrix of autocorrelated Gaussian returns.

In the following we denote the diagonal elements as $C_{ii}$ and the
non-diagonal elements as $C_{ij}$ ($i \neq j$). We assume that
$\var(a_{i,t}) = \sigma^2$ for any $i$ and $t$, and $a_{i,t}$ are not
autocorrelated, i.e. $\text{corr}(a_{i,t}, a_{i, t'}) = 0$ for any
$i$, $t$ and $t'$. Then we can write
\begin{eqnarray*}
  r_{i,t} r_{j, t} &=& \left[
    \phi r_{i, t-1} + a_{i, t}
  \right] \left[
    \phi r_{j, t-1} + a_{j, t}
  \right] \\
  C_{ij} &=& {1 \over T}\sum_{t=1}^T r_{i,t} r_{j, t} \\
  &=& \phi^2 {1 \over T}\sum_{t=1}^Tr_{i,t-1} r_{j, t-1} +
  \phi{1 \over T}\sum_{t=1}^T\left(
    r_{i, t-1} a_{j, t} + r_{j, t-1} a_{i, t}
  \right) + {1 \over T}\sum_{t=1}^T a_{i,t} a_{j,t}
\end{eqnarray*}
We note that the two sums $\sum_{t=1}^T r_{i,t} r_{j, t}$ and
$\sum_{t=1}^T r_{i,t-1} r_{j, t-1}$ only differ by the first and the
last addend, which is negligible for sufficiently large T. Thus we
have
\begin{equation}\label{eq:Offdiag1}
  \begin{aligned}
    (1 - \phi^2)C_{ij} &\approx& \phi {1 \over T} \sum_{t=1}^T
    \left(r_{i, t-1} a_{j, t} + r_{j, t-1} a_{i, t} \right)
    + {1 \over T}\sum_{t=1}^T a_{i,t} a_{j,t}
  \end{aligned}
\end{equation}
Now we write the AR(1) process $r_{i,t}$ as an infinite moving-average
process:
\begin{eqnarray*}
  r_{i, t} &=& \phi r_{i, t-1} + a_{i,t} \\
  (1 - \phi B) r_{i, t} &=& a_{i,t} \\
\end{eqnarray*}
where $B$ is the back-shift operator. Then it follows from the above
equation
\begin{eqnarray*}
  r_{i,t} &=& {1 \over 1 - \phi B} a_{i,t} \\
  &=& \sum_{k=0}^\infty \phi^k B^k a_{i,t} \\
  &=& \sum_{k=0}^\infty \phi^k a_{i,t-k} \\
\end{eqnarray*}
where it is left implicit that $a_{i, t}$ with $t \leq 0$ is zero (in
words, this implies that the $r_{i,t}$ process is not affected at all
by events before $t = 1$).

Substituting this into eq.\ref{eq:Offdiag1} for $r_{i,t-1}$ yields
\begin{equation}
  \label{eq:Offdiag2}
  \begin{aligned}
  (1 - \phi^2)C_{ij} &\approx
  {1 \over T} \sum_{t=1}^T \sum_{k=0}^\infty\phi^{k+1} (a_{i, t-k-1}
  a_{j, t} + a_{j, t-k-1} a_{i, t})
  + {1 \over T} \sum_{t=1}^T a_{i,t} a_{j,t} \\
  &=
  {1 \over T} \sum_{t=1}^{T} \sum_{k=1}^{t-1}\phi^k (a_{i, t-k}
  a_{j, t} + a_{j, t-k} a_{i, t})
  + {1 \over T} \sum_{t=1}^{T} a_{i,t} a_{j,t}\\
  \end{aligned}
\end{equation}

Given two Gaussian random variables $x$ and $y$ with zero mean and
covariance matrix 
\begin{equation*}
  \Sigma = \sigma^2
  \begin{pmatrix}
    1 & \rho \\
    \rho & 1 \\
  \end{pmatrix}
\end{equation*}

The \gls{pdf} of $xy$ can be found by considering $P(xy < z)$:
\begin{eqnarray*}
  P(xy < z) &=& \left(\int_0^\infty dx \int_{-\infty}^{z/x} dy
    +\int_{-\infty}^0 dx \int_{z/x}^\infty dy \right)
    {1 \over 2\pi\sigma\sqrt{1 - \rho^2}} \\
    &&
    \exp\left[
      -{x^2 -2\rho xy + y^2
        \over
        2\sigma^2 (1 - \rho^2)} 
    \right] \\
  f(z; \sigma, \rho) &=& {d \over dz} P(xy < z)\\
  &=& {1 \over \pi \sigma \sqrt{1 - \rho^2}} \exp\left[
    {\rho z \over \sigma^2 (1 - \rho^2)}\right] K_0\left[
    {|z| \over \sigma^2 (1 - \rho^2)}
  \right]
\end{eqnarray*}
where $K_n(z')$ is the modified Bessel function of the second
kind. It is worth taking note that, when $\rho \neq 0$,
$f(z; \sigma, \rho)$ is not symmetric with respect to
$z$. As a result, if $\rho > 0$, the mean of $f(z;
\sigma, \rho)$ is positive, and vice versa.

Secondly, because $|\rho| < 1$ and
\begin{equation*}
  K_0(z) \sim \sqrt{\pi \over 2z} e^{-z}
\end{equation*}
as $z \to \infty$ \cite{Olver:2010:NHMF}, all the moments of
$f(z; \sigma, \rho)$ are finite, implying the
applicability of the Lyapunov central limit theorem provided that $T$
is large, which is very often the case and what we assume here.

With this in mind, we observe that $\phi$ only affects the first sum in
\ref{eq:Offdiag2}. If $a_{i,t-k}$ and $a_{j,t}$ are not correlated for
non-zero $k$, $\phi$ will not affect the mean of
$C_{ij}$. Furthermore, it is also clear from equation
\ref{eq:Offdiag2} that the variance of $C_{ij}$ is
always increased by a non-zero $\phi$, regardless of the sign of
$\phi$. We compute this increment in the following.

In light of the above expression for $f(z; \sigma,
\rho)$, we rewrite equation \ref{eq:Offdiag2} as
\begin{equation}\label{eq:Cij_dist}
  \begin{aligned}
    (1 - \phi^2)C_{ij} \approx &
    {1 \over T}
    \sum_{t=1}^{T} \sum_{k=1}^{t-1}\phi^k (a_{i, t-k} a_{j, t} + a_{j,
      t-k} a_{i, t}) \\
    & + {1 \over T} \sum_{t=1}^{T} \sigma^2 (1 - \rho^2) {a_{i,t} \over
      \sigma \sqrt{1 - \rho^2}} {a_{j,t} \over \sigma \sqrt{1 - \rho^2}} \\
  \end{aligned}
\end{equation}
In addition, we assume
\begin{equation*}
  \text{corr}(a_{i, t-k}, a_{j, t}) = \left\{
    \begin{array}{l l}
      1 & \text{ if $i=j$ and $k = 0$ }\\
      \rho & -1 < \rho < 1. \text{ if $i \neq j$ and $k = 0$ }\\
      0 & \text{otherwise}
    \end{array}
  \right.
\end{equation*}
Then, because $a_{i, t-k}$ and $ a_{j, t}$ with $i \neq j$ and $k
> 0$ are not correlated, the mean of $a_{i, t-k} a_{j, t}$ is 0
($f(z; \sigma, 0)$ is symmetric), and the variance of it
can be found to be $\sigma^6$ using formula (10.43.19) of \cite{NIST:DLMF}:
\begin{equation*}
  \int_0^\infty dt K_\nu(t) t^{\mu-1} = 2^{\mu-2}
  \Gamma\left(
    {\mu + \nu \over 2}
  \right) \Gamma\left(
    {\mu - \nu \over 2}
  \right)
\end{equation*}
On the other hand, $a_{i,t}/\sigma \sqrt{1 - \rho^2}$ and
$a_{j,t}/\sigma \sqrt{1 - \rho^2}$ have variance $1/(1 - \rho^2)$
and are correlated - $\text{corr}(a_{i,t}, a_{j,t}) = \rho$. The mean
of $a_{i,t}a_{j,t}/\sigma^2 (1 - \rho^2)$ can be found using formula
(10.43.22) of \cite{NIST:DLMF}, given that $-1 < \rho < 1$:
\begin{eqnarray*}
  \int_0^\infty t^{\mu - 1} e^{-at} K_\nu(t) dt &=& (\pi/2)^{1/2}
  \Gamma(\mu + \nu) \Gamma(\mu - \nu)(1 - a^2)^{-\mu/2 + 1/4} \times\\
  && P^{-\mu+1/2}_{\nu-1/2} (a)
\end{eqnarray*}
where $P^\mu_\nu(\cdot)$ is Ferrers function of the first
kind\footnote{ Ferrers function of the first kind is defined through
  the hypergeometric functon $F(a, b; c; z)$ \cite{NIST:DLMF}:
  \begin{equation}
    \label{eq:Ferrers_1st}
    \mathop{\mathsf{P}^{\mu}_{\nu}\/}\nolimits\!\left(x\right)=\left(\frac{1+x}{1-
        x}\right)^{\mu/2}\mathop{\mathbf{F}\/}\nolimits\!\left(\nu+1,-\nu;1-\mu;\tfrac
      {1}{2}-\tfrac{1}{2}x\right).
  \end{equation}
}. The result is
\begin{eqnarray*}
  \E\left[{a_{i,t}a_{j,t} \over \sigma^2 (1 - \rho^2)}\right]
  &=&
  {1 \over \sqrt{2\pi} (1 - \rho^2)^{5/4}} \left[
    P^{-3/2}_{-1/2}(-\rho) - P^{-3/2}_{-1/2}(\rho)
  \right]
\end{eqnarray*}
Similarly, the variance of $a_{i,t}a_{j,t}/\sigma^2 (1 - \rho^2)$ is
found to be
\begin{eqnarray*}
  v^2(\rho) &=&
  {4 \over \sqrt{2\pi} (1 - \rho^2)^{7/4}} \left[
    P^{-5/2}_{-1/2}(\rho) + P^{-5/2}_{-1/2}(-\rho)
  \right] \\
  && - \E^2\left[{a_{i,t}a_{j,t} \over
      \sigma^2 (1 - \rho^2)}\right]
\end{eqnarray*}
Now we can apply the Lyapunov central limit theorem
\cite{Billingsley1995} to the sum in equation \ref{eq:Cij_dist}
and write down the asymptotic Gaussian distribution of $C_{ij}$:
\begin{equation*}
C_{ij} \sim N(\mu'_X, \sigma'_X)
\end{equation*}
where
\begin{eqnarray}
  \mu'_X &=& {\sigma^2 \over \sqrt{2\pi} (1 - \phi^2)(1 -
    \rho^2)^{1/4}} \left[ P^{-3/2}_{-1/2}(-\rho) -
    P^{-3/2}_{-1/2}(\rho)
  \right] \label{eq:gaussian_mean}\\
  \sigma'^2_X &=& {1 \over (1 - \phi^2)^2}\left[
    \sum_{t=1}^T \sum_{k=1}^{t-1} 2\left(
      \phi^k \over T
    \right)^2 \sigma^6 + \sum_{t=1}^T
    {\sigma^4 (1 - \rho^2)^2 \over T^2} v^2(\rho)
  \right] \nonumber \\
  &=& {2 \sigma^6 \over T (1 - \phi^2)^2} \left[
    {\phi^2 \over 1 - \phi^2} -
    {\phi^2 (1 - \phi^{2T}) \over
      T(1 - \phi^2)}
  \right] + {\sigma^4 (1 - \rho^2)^2 v^2(\rho) \over
    T (1 - \phi^2)^2} \nonumber \\
  &\approx& {2 \sigma^6 \phi^2 \over T (1 - \phi^2)^3}
  + {\sigma^4 (1 - \rho^2)^2 v^2(\rho) \over
    T (1 - \phi^2)^2} \label{eq:gaussian_variance}
\end{eqnarray}
% Equation \ref{eq:gaussian_mean} tells that, if two return series $i$
% and $j$ are not correlated, auto-correlation in the returns does not
% introduce a bias into the estimation of the covariance; if,
% however, the return series are indeed correlated, auto-correlation
% in the returns rescales the covariance through a multiplicative
% factor $1/(1 - \phi^2)$.

% In addition, equation \ref{eq:gaussian_variance} tells that
% auto-correlation in the returns always makes the covariance
% estimation more noisy. Auto-correlation not only rescales the variance
% of the no-autocorrelation estimation by $1/(1 - \phi^2)$ but even
% adds an extra term ${2 \sigma^6 \phi^2 \over T (1 - \phi^2)^3}$.

We may apply a similar treatment to the diagonal elements of the
covariance matrix, which we denote as $C_{ii}$ here:
\begin{eqnarray*}
  C_{ii} &=& {1 \over T} \sum_{t=1}^T r^2_{it} \\
  &=& {1 \over T} \sum_{t=1}^T \sum_{l=0}^{t-1}\phi^l a_{i, t-l}
  \sum_{k=0}^{t-1}\phi^k a_{i, t-k} \\
  &=& {1 \over T} \sum_{t=1}^T \left[
    \sum_{k=0}^{t-1} \phi^{2k} a^2_{i, t-k} +
    \sum_{k,l = 0}^{t-1} \phi^{k+l} a_{i, t-k} a_{i, t-l}
  \right]
\end{eqnarray*}
By the Lyapunov central limit theorem under the assumption of large T,
the asymptotic distribution of $C_{ii}$ is Gaussian, the mean and
variance being
\begin{eqnarray}
  \E(C_{ii}) &=& {1 \over T}\left[
    \sum_{k=0}^{t-1} \phi^{2k} \sigma^2
  \right] \nonumber \\
  &=& {\sigma^2 \over (1 - \phi^2) T} \left[
    T - {\phi^2 (1 - \phi^{2T}) \over 1 - \phi^2}
  \right] \nonumber \\
  & \approx & {\sigma^2 \over 1 - \phi^2} \left[
    1 - {\phi^2 \over T}
  \right] \label{eq:gaussian_cii_mean}
\end{eqnarray}
and
\begin{eqnarray}
  \var(C_{ii}) &=& \sum_{t=1}^T \left[
    \sum_{k=0}^{t-1} {\phi^{4k} \sigma^4 \over T^2} 2 +
    \sum_{k,l=0}^{t-1} {\phi^{2(k+l)} \over T^2} \sigma^6
  \right] \nonumber \\
  &=& \sum_{t=1}^T \left[
    {2 \sigma^4 \over T^2} {1 - \phi^{4t} \over 1 - \phi^4} +
    {\sigma^6 \over T^2} \left(
      {1 - \phi^{2t} \over 1- \phi^2}
    \right)^2 \right] \nonumber \\
  &=& {2 \sigma^4 \over T (1 - \phi^4)} -
  {2 \sigma^4 \phi^4 (1 - \phi^{4T}) \over T^2(1 - \phi^4)^2} +
  \nonumber \\
  && {\sigma^6 \over T (1 -\phi^2)^2} -
  {2 \sigma^6 \phi^2 (1 - \phi^{2T}) \over T^2 (1 - \phi^2)^3} +
  {\sigma^6 \phi^4 (1 - \phi^{4T}) \over T^2 (1 - \phi^2)^2 (1 -
    \phi^4)} \nonumber \\
  &\approx& {2 \sigma^4 \over T (1 - \phi^4)} + {\sigma^6 \over T (1
    -\phi^2)^2} \label{eq:gaussian_cii_variance}
\end{eqnarray}
% Equation \ref{eq:gaussian_mean} tells that, if two return series $i$
% and $j$ are not correlated, auto-correlation in the returns does not
% introduce a bias into the estimation of the covariance; if,
% however, the return series are indeed correlated, auto-correlation
% in the returns rescales the covariance through a multiplicative
% factor $1/(1 - \phi^2)$.

% In addition, equation \ref{eq:gaussian_variance} tells that
% auto-correlation in the returns always makes the covariance
% estimation more noisy. Auto-correlation not only rescales the variance
% of the no-autocorrelation estimation by $1/(1 - \phi^2)$ but even
% adds an extra term ${2 \sigma^6 \phi^2 \over T (1 - \phi^2)^3}$.

\chapter{Eigenvalues Distribution of Wishart Matrix}
\label{sec:wishart_eigen_dist}
In this section we summarize the analytic results regarding the
eigenvalue distribution of a Wishart matrix $\mtx{RR'}$. In the
simplest case where the elements of $\mtx R$ are all independent of
each other, i.e. neither auto-correlation nor cross-correlation
exists, we have \cite{Chiani2012}
\begin{eqnarray}
  f(x_1, \cdots, x_N) &=& K \prod_{i=1}^N e^{-x_i/2}
  x_i^{(T-N-1)/2} \prod_{i<j}^N (x_i -
  x_j) \label{eq:wishart_eigen_pdf}
\end{eqnarray}
where the eigenvalues have been indexed in descending order: $x_1 \geq
x_2 \geq \cdots \geq x_N$. The normalization constant K is given by
\begin{equation*}
  K = {\pi^{N^2/2} \over 2^{NT/2}\Gamma_N(T/2) \Gamma_N(N/2)}
\end{equation*}
where the function $\Gamma_m(a)$ is defined as
\begin{equation*}
  \Gamma_m(a) = \pi^{m(m-1)/4} \prod_{k=1}^m \Gamma\left(a - {k-1
      \over 2}\right)
\end{equation*}
If the order of the eigenvalues is ignored, the \gls{pdf} of their
distribution is given by the Marcenko-Pastur law \cite{Guhr2007}
\begin{eqnarray}
  f(x) &=& {1 \over 2\pi \sigma^2 q} {
    \sqrt{(x_2 - x)(x - x_1)} \over x
  } \label{eq:MP_pdf}
\end{eqnarray}
where it is assumed $r_{it} \sim N(0, \sigma^2)$ and
\begin{eqnarray*}
  q &=& \lim_{N,T \to \infty} {N \over T} \\
  x_1 &=& \sigma^2 (1 - \sqrt q)^2 \\
  x_2 &=& \sigma^2 (1 + \sqrt q)^2 \\
\end{eqnarray*}
It is imposed that $q = N/T < 1$.

Moreover, K. Johnsson \cite{Johnsson2000} and I. Johnstone
\cite{Johnstone2001} showed that, at the {\it absence} of
autocorrelations and in the asymptotic limit $N, T \to \infty$, $N/T
\to q < \infty$, the maximum eigenvalue  $\lambda_1$ follows the
Tracy-Widom distribution (denote $\mathscr{TW}_1$ here) when properly
relocated and rescaled:
\begin{equation*}
  {\lambda_1 - \mu_{NT} \over \sigma_{NT}} \sim \mathscr{TW}_1
\end{equation*}
where $\mu_{NT}$ and $\sigma_{NT}$ are given by
\begin{eqnarray*}
  \mu_{NT} &=& \left(
    \sqrt{N-1/2} + \sqrt{T - 1/2}
  \right)^2 \\
  \sigma_{NT} &=& \sqrt{\mu_{NT}} \left(
    {1 \over \sqrt{N-1/2}} + {1 \over \sqrt{T-1/2}}
  \right)^{1/3}
\end{eqnarray*}
The $\mathscr{TW}_1$ distribution has the following cummulative
distribution function (CDF) \cite{Chiani2012}
\begin{equation*}
  F_1(x) = \exp\left[
    -{1 \over 2} \int_x^\infty dy \left(q(y) + (y-x)q^2(y)\right)
  \right]
\end{equation*}
where $q(y)$ is defined as the solution to the Painlev\'e II differential
equation
\begin{equation*}
  q''(y) = yq(y) + 2q^3(y)
\end{equation*}
which is unique when imposing the condition
\begin{equation*}
  q(y) \sim \text{Ai}(y) \text{ as } y \to \infty
\end{equation*}

Then Marco Chiani showed recently that the $\mathscr{TW}_1$ distribution
can be well approximated by a gamma distribution based on his proof
that the exact distribution of the maximum eigenvalue is a mixture of
gamma distributions. Specifically,
\begin{equation}\label{eq:TracyWidom-Gamma}
  {\lambda_1 - \mu_{NT} \over \sigma_{NT}} + \alpha \sim
  \mathscr{G}(k ,\theta)
\end{equation}
where $\mathscr{G}(k, \theta)$ denotes the Gamma distribution with
parameters $k$ and $\theta$ \cite{Chiani2012}. The \gls{pdf} of the
gamma distribution is given by
\begin{equation*}
  f_\gamma(x; k, \theta) = {1 \over \Gamma(k) \theta^k} x^{k-1}
  e^{-x/\theta}
\end{equation*}
Moreover, the first 3 moments of the distribution are simple:
\begin{eqnarray*}
  \text{mean} &=& k\theta \\
  \text{variance} &=& k\theta^2 \\
  \text{skewness} &=& 2/\sqrt{k}
\end{eqnarray*}


% \chapter{Collection of Figures and Tables}

% \begin{figure}[htb!]
%   \centering
%   \subfigure[ACF of $z_t$]{
%     \includegraphics[scale=0.4, clip=true, trim=95 236 118
%     200]{../pics/nordea_15min_quotient_acf.pdf}
%     \label{fig:nordea_15min_quotient_acf}
%   }
%   \subfigure[ACF of $z_t^2$]{
%     \includegraphics[scale=0.4, clip=true, trim=95 236 118
%     200]{../pics/nordea_15min_quotient_squared_acf.pdf}
%     \label{fig:nordea_15min_quotient_squared_acf}
%   }
%   \caption{\small \it Nordea 15min $z_t$ and $z_t^2$ auto-correlation
%     function (ACF).}
% \end{figure}

% \begin{figure}[htb!]
%   \centering
%   \subfigure[Normal Probability Plot of $y_t$]{
%     \includegraphics[scale=0.4, clip=true, trim=76 256 109
%     214]{../pics/nordea2_y_normplot.pdf}
%     \label{fig:nordea_15min_y_qq}
%   }
%   \subfigure[ACF of $y_t$]{
%     \includegraphics[scale=0.4, clip=true, trim=92 258 104
%     218]{../pics/nordea_15min_y_acf.pdf}
%     \label{fig:nordea_15min_y_acf}
%   }
%   \caption{\small \it Nordea 15min $y_t$ normal probability plot and
%     autocorrelations}
% \end{figure}




\bibliographystyle{unsrt}
\bibliography{econophysics}
\end{document}


