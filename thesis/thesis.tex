\documentclass{book}
\usepackage{graphicx}
\usepackage{subfigure}
\usepackage{wrapfig}
\usepackage{amssymb}
\usepackage{amsfonts, amsmath}
\usepackage{mathrsfs}
\usepackage{enumerate}
\usepackage[bookmarks=true]{hyperref}
\usepackage{bookmark}

\usepackage{amssymb,amsmath,amsthm}
%\newtheorem{mdef}{Definition}
%\newtheorem{theorem}{Theorem}
\newcommand{\eqsplit}[2]{
  \begin{equation}\label{#2}
    \begin{split}
      #1
    \end{split}
  \end{equation}}
\newcommand{\eqnsplit}[1]{
  \begin{eqnarray*}
    #1
  \end{eqnarray*}}
\newcommand{\tran}[1]{
  \tilde{#1}
}
\newcommand{\td}[2]{
  \frac{d #1}{d #2}
}
\newcommand{\pd}[2]{
  \frac{\partial #1}{\partial #2}
}
\newcommand{\ppd}[2]{
  \frac{\partial^2 #1}{\partial #2^2}
}
\newcommand{\pdd}[3]{
  \frac{\partial^2 #1}{\partial #2 \partial #3}
}
\newcommand{\otd}[1]{
  \frac{d}{d #1}
}
\newcommand{\opd}[1]{
  \frac{\partial}{\partial #1}
}
\newcommand{\oppd}[1]{
  \frac{\partial^2}{\partial #1^2}
}
\newcommand{\opdd}[2]{
  \frac{\partial^2}{\partial #1 \partial #2}
}
\newcommand{\ket}[1]{
  |#1\rangle
}
\newcommand{\bra}[1]{
  \langle#1|
}
\newcommand{\inn}[1]{
  \langle#1\rangle
}
\newcommand{\mean}[1]{
  \langle#1\rangle
}
\newcommand{\tr}{
  \text{tr}\,
}
\newcommand{\re}{
  \text{Re}\,
}
\newcommand\im{
  \text{Im}\,
}
\newcommand{\var}{
  \text{var}
}
\newcommand{\arcsinh}{
  \sinh^{-1}
}
\newcommand{\arccosh}{
  \cosh^{-1}
}
\newcommand{\erfc}{
  \text{erfc}
}
\newcommand{\E}{
  \mathbb{E}
}
\renewcommand{\P}{
  \mathbb{P}
}
\newcommand{\I}[1]{
  \mathbf{1}_{\{#1\}}
}
\newcommand{\1}[1]{
  \mathds{1}_{\{#1\}}
}
\newcommand{\diag}{
  \text{diag\,}
}
\newcommand{\M}{
  {\text{max}}
}
\newcommand{\m}{
  {\text{min}}
}
\newcommand{\ph}{
  {\text{arg}\,}
}
\newcommand\erf{
  \text{erf}
}
\renewcommand\vec[1]{
  \mathbf{#1}
}
\newcommand\mtx[1]{
  \mathbf{#1}
}
\newcommand\ed{
  \,{\buildrel d \over =}\,
}



\DeclareGraphicsExtensions{.pdf,.png,.jpg}

\title{Mathematical Aspects of the Capital Market}
\author{Xie Xiaolei}
\date{\today}
\begin{document}

\maketitle
\tableofcontents

\chapter{Introduction}
The mathematics of the financial market has always been a topic that
arouses interest and imagination, and with little doubt, has been
studied from many aspects and in many different ways.

Central to all these studies are the concepts of probability
distributions and correlations. The values of stocks, futures,
options, etc. are stochastic in nature and are governed by the laws of
stochastic processes, which are expressed in terms of probability
distributions and correlations.

On the other hand, the observables of the market are prices, volumes,
turn-over (the amount of money paid in a trade), names of the brokers,
and time of the trades. These quantities don't make much sense
by themselves but do reveal the probabilistic dynamics of the market
when put together and turned into statistics.

Here by dynamics we mean how the value of an asset is affected by its
own history as well as by the histories and current values of other
assets in the market. The influence from the asset's own history is
termed autocorrelations, i.e. correlations in time, while
the influence from other assets are termed cross correlations.

Mathematical models that describe the aforementioned observables in
terms of probability distributions, autocorrelations and cross
correlations give predictive power and help determine a fair price of
a given asset. Hence they are a central topic of the mathematics of
the financial market.

Once a model has been tentatively constructed, it must be
verified. The verification again relies on the statistics of the
observables. Over the years, a number of phenomena have been
consistently reported and become generally accepted. These are referred
to as stylized facts, which are detailed in \S
\ref{sec:StylizedFacts} and constitute a standardized test suit for
model verification.

In \S \ref{sec:FundamentalConcepts} we present a few very
important concepts and notations that we will often refer to in later
chapters. Then the rest of this thesis is organized as follows:

\section{Fundamental Concepts \& Notations}\label{sec:FundamentalConcepts}
This section is a list of a few concepts that may be unfamiliar to the
reader and that we will often refer to later in the thesis:
\begin{itemize}
\item Return. Given a fixed time interval $[t, t']$, for example, a
  day, a week, a month, etc, and a particular asset, for example, a
  share in company ABC, the return of this asset over the time
  interval is defined as
  \begin{eqnarray*}
    r_{t', \Delta t} &=& \ln p_{t'} - \ln p_t \\
    &=& \ln \left(1 + {p_{t'} - p_t \over p_t}\right) \\
    &\approx& {p_{t'} - p_t \over p_t}
  \end{eqnarray*}
  where $p_t$ is the price of the asset at time $t$ and $\Delta t = t'
  - t$ is sometimes called the time-lag of the return. Quite often,
  where confusion is not possible, we will just write $r_t$ to mean
  $r_{t, \Delta t}$, the time lag either does not matter or is clear
  from the context.

\item Autocorrelation. By autocorrelation, denoted $\rho_k$ here, we
  mean the correlation between two temporally separated observations
  of the same time series:
  \begin{eqnarray*}
    \rho_k &=& {
      E\left[(a_t -E(a_t))(a_{t-k} - E(a_{t-k})\right]
      \over
      \sqrt{\text{var}(a_t)}\sqrt{\text{var}(a_{t-k})}
    }
  \end{eqnarray*}
  Here $E(x)$ stands for the expectation value of $x$, and
  $\text{var}(x)$ stands for the variance of $x$. $k$ is called the
  time-lag and is the temporal seperation of the two observations
  measured by the number of observations in between. For example, the
  time-lag between the 1st and 3rd observation is 2.

\item Cross-Correlation Matrix. Correlations between the returns of a
  group of assets are described by the cross-correlation matrix. When
  the asset returns are described by a stable distribution law with
  L\'evy index $\alpha$, the empirical cross-correlation matrix is
  constructed as
  \begin{equation}
    \label{eq:cross-correlation}
    \begin{aligned}
      C_{ij} &= {1 \over T^{2/\alpha}} \sum_{t=1}^T [r_{i,t}-E(r_i)]
      [r_{j,t}-E(r_j)] \\
      &= {1 \over T^{2/\alpha}} RR'
    \end{aligned}
  \end{equation}
  where $r_{i,t}$ is the return of the i-th asset at time t and is
  placed at the entry (i, t) of matrix R; $R'$ denotes the
  transpose of R. In most practical situations, one has abundant data
  for each and every asset. Thus $T \geq N$ is assumed throughout this
  thesis.

\item Auto-regressive processes. A time series $r_t$ is called an
  auto-regressive process of order $p$ and denoted AR(p), if it can be
  written in the following form:
  \begin{eqnarray*}
    r_t &=& \sum_{i=1}^p \phi_i r_{t-i} + a_t
  \end{eqnarray*}
  where, for all $i$, $\phi_i \in (-1, 1)$, and the $a_t$'s are
  independent and identically distributed (iid.) random variables with
  zero mean. Obviously this implies the mean of $r_t$ is zero
  too. Apart from this, their distribution of $a_t$ is not restricted
  to any particular form.
  
  Of particular interest to this thesis is the AR(1) process:
  \begin{eqnarray*}
    r_t = \phi r_{t-1} + a_t
  \end{eqnarray*}
  Its autocorrelation function $\rho_k = \text{corr}(r_t, r_{t-k})$
  ($k = 0, 1, 2, \cdots$) can be easily shown to fall off
  exponentially:
  \begin{eqnarray*}
    \rho_k &=& {E(r_tr_{t-k}) - E(r_{t-k})E(r_{t}) \over
      \sqrt{\var(r_t) \var(r_{t-k})}} \\
    &=& {\phi E(r_{t-1}r_{t-k}) + E(a_t r_{t-k})
      \over
      \sqrt{\var(r_t) \var(r_{t-k})}
    } \\
    &=& \phi \rho_{k-1}
  \end{eqnarray*}
  where $E(a_t r_{t-k}) = 0$ follows from the fact that any return
  $r_{t1}$ must not depend on disturbances $a_{t2}$ that occur later
  in time. The last equation means $\rho_k$ is a geometric series. Since
  $\rho_0 = 1$, we have
  \begin{eqnarray*}
    \rho_k &=& \phi^k \\
    |\rho_k| &=& e^{k\ln|\phi|} \\
  \end{eqnarray*}
  Note that $|\phi| < 1$ and hence $\ln|\phi| < 0$.

  Although $k$ can only take integer values, it is still useful to
  define a correlation time $\tau$ such that $\phi^\tau = 1/2$. Such a
  quantity is more intuitive and constitutes a measure of
  autocorrelations that is universal and comparable among different
  time series' models. From the definition of $\tau$ we get
  \begin{equation}
    \label{eq:tau_def}
    \begin{aligned}
      \phi^\tau &= 1/2 \\
      \tau &= -{\ln 2 \over \ln\phi} \\
      \phi &= 2^{-1/\tau}
    \end{aligned}
  \end{equation}
  Figure \ref{fig:AR1-autocorrelation} shows the autocorrelation
  function of the AR(1) model. As proven above, this function decays
  exponentially.
  \begin{figure}[htb!]
    \vspace{-15mm}
    \centering
    \includegraphics[scale=0.5, clip=true, trim=113 229 115
    139]{../pics/AR1-autocorrelation.pdf}
    \caption{\footnotesize Autocorrelation function of the AR(1) model
      with $\phi=1/\sqrt{2}$. Red circles: autocorrelations at $k=0, 1,
      2, \cdots$. Blue line: $e^{t\ln\phi}$.}
    \label{fig:AR1-autocorrelation}
  \end{figure}
  For more details about conventional time series' models, see
  \cite{BoxJenkins94}.
\end{itemize}

For purposes of later reference, we also list some notations that may
cause confusion to the reader:
\begin{itemize}
\item $E(x)$ or $\mean{x}$: The expectation value of $x$.
\item $\var(x)$: the variance of $x$.
\item $\text{cov}(x,y)$: the covariance of $x$ and $y$.
\item $\text{corr}(x,y)$: the correlation between $x$ and $y$, i.e.
  \begin{equation*}
    \text{corr}(x,y) = {\text{cov}(x,y) \over \sqrt{\var(x)\var(y)}}
  \end{equation*}
\end{itemize}

\section{Stylized facts}\label{sec:StylizedFacts}
This section presents and explains a few ``stylized facts'',
i.e. phenomena that are widely observed and accepted as true.
\begin{itemize}
\item Fat tails. It has been consistently reported by various studies
  - for example \cite{Potters2003} and \cite{Mantegna2000} - that the
  probability density function (PDF) of stock/index returns are not
  Gaussian. Unlike the Gaussian PDF, which is symmetric and falls off
  very quickly as its argument moves from the center to the outskirts
  (tails), the PDF of stock/index returns are higher on the tails,
  i.e. the probability of large fluctuations is higher than is dictated
  by a Gaussian distribution - in fact, even higher than dictated by
  an exponential function. Empirical studies suggest the PDF of the
  returns follows a power-law on the tails; the exponent of the power
  is around 4, depending on the specific stock/index.

  Figure \ref{fig:FatTail} illustrates this feature.
  \begin{figure}[htb!]
    \centering
    \includegraphics[scale=0.5, clip=true, trim=92 229 110
    140]{../pics/FatTail.pdf}
    \caption{\footnotesize Fat tails. Blue: empirical PDF of 15min returns of
      {\it Nordea Bank}. The returns are computed using paid
      prices on the Stockholm OMX market during the period between
      2013-10-10 and 2014-01-29. Red: Gaussian distribution with equal
      mean and variance.}
    \label{fig:FatTail}
  \end{figure}

\item Non-zero skewness. Apart from fat tails, the PDF of stock/index
  returns are often also skewed. If the skewness is positive
  (negative), the probability of very large positive (negative)
  returns is higher than that of very large negative (positive)
  returns, even though the mean of the returns is 0 or extremely
  close to 0.

  Table \ref{tab:EmpiricalSkewness} lists the skewness of a few
  Swedish stocks traded on the Stockholm OMX market.
  \begin{table}[htb!]
    \footnotesize
    \centering
    \begin{tabular}{c|c|c|c|c|c|c}
      Nordea Bank & Volvo B & Boliden & ABB Ltd & H\&M & Scania
      B & Ericsson B \\
      \hline
      0.1362 & 0.0471 & 0.0567 & -0.0364 & -0.0168 &
      -1.1554 & 0.2086 \\
    \end{tabular}
    \caption{\footnotesize Skewness of Stock Returns. All the returns
      have time-lag of 15 minutes and are computed using paid prices
      between 2013-10-10 and 2014-01-29. }
    \label{tab:EmpiricalSkewness}
  \end{table}

\item Higher-order autocorrelation. 
\end{itemize}


\section{GARCH models}
The autocorrelation function of $\epsilon_t^2$ is given by
(c.f. \cite{Bollerslev87})
$$
\rho_n = \sum_{i=1}^{p \vee q} (\alpha_i + \beta_i) \rho_{n-i}
\;n > p
$$
where $\alpha_i$ with $i > q$ and $\beta_i$ with $i > p$ are taken as
zeros. $p \vee q$ denotes the maximum of p and q. From these
equations, it is clear that the partial autocorrelation function cuts
off at $\max(p, q)$.

\chapter{Price Models}
\section{Nordea Bank}
\begin{figure}[htb!]
  \centering
  \includegraphics[scale=0.5, clip=true, trim=103 236 118
  200]{../pics/nordea_price_20120116-20120420.pdf}
  \caption{\footnotesize Nordea Bank paid prices 2012/01/16-2012/04/20}
  \label{fig:Nordea}
\end{figure}
Figure \ref{fig:Nordea} shows the paid prices of Nordea Bank from
2012/01/16 to 2012/04/20.

\subsection{15min Returns}
\subsubsection{GARCH model}
The autocorrelation function (ACF) of $\epsilon_t^2$ are shown in figure
\ref{fig:nordea_15min_acf}. At the absence of GARCH effects, the ACF
will have an asymptotic Gaussian distribution with mean 0 and variance
1/T \cite{Bollerslev86, Bollerslev87}. The first 5 autocorrelations
have considerable sizes and do not fall off as in the Gaussian
case. This observation suggests a GARCH model. Moreover, from
\ref{fig:nordea_15min_vlt_acf}, one sees apparent autocorrelations
between volatilities of consecutive time intervals. This again points
to a GARCH model.
\begin{figure}[htb!]
  \centering
  \subfigure[ACF of squared returns]{
    \includegraphics[scale=0.4, clip=true, trim=95 236 118
    200]{../pics/nordea_15min_acf.pdf}
    \label{fig:nordea_15min_acf}
  }
  \subfigure[ACF of volatilities]{
    \includegraphics[scale=0.4, clip=true, trim=95 236 118
    200]{../pics/nordea_15min_vlt_acf.pdf}
    \label{fig:nordea_15min_vlt_acf}
  }
  \caption{\footnotesize Nordea Bank 15min autocorrelations}
\end{figure}

To obtain the functional form of the conditional distribution of
$\epsilon_t$ \footnote{We use the realized volatility as an estimate of the
conditional volatility. See \cite{Andersen03}.}, we look at the
qq-plot of $\epsilon_t/\sigma_t$ (figure
\ref{fig:nordea_15min_epOversig_qq}), which is consistent with a
standard Gaussian distribution except for a few outliers at the two
ends. Hence the fat tails of the probability density function can be
entirely accounted for by GARCH effects, implying a Gaussian
conditional distribution is appropriate for $\epsilon_t$.
% \begin{figure}[htb!]
\begin{wrapfigure}{r}{0.5\textwidth}
  \centering
    \includegraphics[scale=0.35, clip=true, trim=100 236 118
    200]{../pics/nordea_15min_epOversig_qq.pdf}
  \caption{\scriptsize{QQ-plot of $\epsilon_t / \sigma_t$. $\epsilon_t$ are
    derived from Nordea Bank 15min returns while $\sigma_t$ are realized
    volatilities calculated using 30s returns within each 15min
    interval.}}
  \label{fig:nordea_15min_epOversig_qq}
\end{wrapfigure}

To determine the orders p and q of the GARCH model, we again check
the partial autocorrelation function of $\epsilon_t^2$. As just
mentioned, the first 5 partial autocorrelations are significant. Thus
we can tentatively estimate a GARCH(1, 5) model. Then by applying the
Lagrange multiplier test and accordingly removing insignificant
parameters, we obtain a GARCH(1, 3) model
\ref{tab:nordea_15min_garch}:
\begin{table}[htb!]
  \centering
  \begin{tabular}{c|c|c|c}
    parameter & value & std. error & t statistic \\
    \hline
     $\alpha_0$ &   1.05668e-06 &   3.56871e-07 &    2.96096\\
     \hline
     $\beta_1$  &     0.687212  &   0.0299314   &    22.9596\\
     \hline
     $\alpha_1$ &      0.128614 &   0.0268144   &    4.79643\\
     \hline
     $\alpha_3$ &    0.0568022  &   0.0301503   &    1.88397
  \end{tabular}
  \caption{\footnotesize GARCH(1, 3) model of Nordea Bank 15min returns}
  \label{tab:nordea_15min_garch}
\end{table}

\subsubsection{$(0,1,1)\times(0, 1, 1)$ seasonal volatility model}
With the availability of transactional data, estimating conditional
volatility using returns sampled at a higher frequency (realized
volatility) gives superior acuracy and reliability. However, at which
frequency the time series should be sampled in order to give an
unbiased and consistent estimate of the volatility is not a trivial
question. Naively one would argue by the doctrine of statistics that
the often the series is sampled, the better the estimate. This is also
well supported in theory, since the quadratic variation which is the
theoretial variance of the time series (see \cite{Andersen03}, theorem
1 and corollary 1), is approximated as (see \cite{Protter05}, ch.II,
sec.6, theorem 22):
\[
\var(p_{t+h} - p_t) = [r, r]_{t+h} - [r, r]_t = \lim_{n \to \infty}
\sum_{T^i \in \pi_n} (p^{T_{i+1}} - p^{T_i})^2
\]
where $p_t$ is the log price process and $r_t = p_t - p_0$. $\pi_n$
is a refining partition of the interval $[t, t+h]$.

However, due to the noise introduced by market micro-structure, there
is actually an optimal sampling frequency relative to $h$. While the
method to determine this optimal frequency is a subject of debate (see
for example \cite{Sahalia05}), it is not difficult to find a fairly
satisfactory frequency in practice.

Andersen and Bollerslev et al proved in \cite{Andersen03} that, with
the conditional volatility well approximated by the realized
volatility, $r_{t+h, h} = p_{t+h} - p_t$ is governed by a Gaussian
distribution (see theorem 2 of \cite{Andersen03}):
\[
r_{t+h, h} | \sigma(\mu_{t+s}, \sigma_{t+s})_{s \in [0, h]} \sim
N(\int_{0}^h \mu_{t+s} ds, \int_{0}^h \sigma_{t+s}^2 ds)
\]

Thus one can try a few frequencies and compare the distribution of
$r_{t+h, h}/\hat{\sigma_{t+h, h}}$ with the standard Gaussian. If the
two match, $\hat{\sigma_{t+h, h}}$ is a good estimate of $\sigma_{t+h,
  h}$, the volatility of $r_{t+h, h}$.

For the present case with $h = 15\text{min}$, a reasonally good
sampling frequency is 1/30sec, as confirmed by the QQ-plot
\ref{fig:nordea_15min_epOversig_qq} as well as by the autocorrelation
functions of $r_{t+h, h} / \sigma_{t+h, h}$ and of $(r_{t+h, h} /
\sigma_{t+h, h})^2$ which are shown in
\ref{fig:nordea_15min_quotient_acf} and
\ref{fig:nordea_15min_quotient_squared_acf} respectively.
\begin{figure}[htb!]
  \centering
  \subfigure[ACF of $r_{t+h, h} / \sigma_{t+h, h}$]{
    \includegraphics[scale=0.4, clip=true, trim=95 236 118
    200]{../pics/nordea_15min_quotient_acf.pdf}
    \label{fig:nordea_15min_quotient_acf}
  }
  \subfigure[ACF of $(r_{t+h, h} / \sigma_{t+h, h})^2$]{
    \includegraphics[scale=0.4, clip=true, trim=95 236 118
    200]{../pics/nordea_15min_quotient_squared_acf.pdf}
    \label{fig:nordea_15min_quotient_squared_acf}
  }
  \caption{\footnotesize Nordea 15min $r_t/\sigma_t$ and $(r_t/\sigma_t)^2$ ACF.}
\end{figure}

We see in figure \ref{fig:nordea_15min_quotient_acf} and
\ref{fig:nordea_15min_quotient_squared_acf} there is essentially no
autocorrelation in the series $r_{t+h, h} / \sigma_{t+h, h}$ and
$(r_{t+h, h} / \sigma_{t+h, h})^2$, supporting the assumption that the
series is i.i.d Gaussian distributed. As a result, the problem of
modeling $r_{t+h, h}$ boils down to modeling $\sigma_{t+h,
  h}$. Andersen and Bollerslev et al reported that, for the exchange
rates between Deutch mark, yen and dollar, $\ln \sigma_{t+h, h}$ is
gaussian distributed \cite{Andersen03}. This is, however, not the case
for our series in question. In fact, in our case, $\ln \sigma_{t+h,
  h}$ is right skewed (skewness 0.3342) and leptokurtic (kurtosis
6.1006). See figure \ref{fig:nordea_15min_logvol_pdf} and
\ref{fig:nordea_15min_logvol_qq}
\begin{figure}[htb!]
  \centering
  \subfigure[PDF of $\ln\sigma_{t+h,h}$]{
    \includegraphics[scale=0.32, clip=true, trim=52 173 46
    113]{../pics/nordea_15min_logvol_hist.pdf}
    \label{fig:nordea_15min_logvol_pdf}
  }
  \subfigure[QQ-plot of $\ln\sigma_{t+h,h}$]{
    \includegraphics[scale=0.4, clip=true, trim=95 236 118
    220]{../pics/nordea_15min_logvol_qq.pdf}
    \label{fig:nordea_15min_logvol_qq}
  }
  \caption{\footnotesize Nordea 15min $\ln\sigma_{t+h, h}$ unconditional
    distribution}
\end{figure}

Moreover, the autocorrelation function of $\ln\sigma_{t+h, h}$
shows long-lasting and periodic autocorrelations in the time series
with an apparent period of 33 (see figure
\ref{fig:nordea_15min_logvol_acf}). This suggests simplifying the
series by differencing \cite{BoxJenkins94}:
\[
w_t = (1-B)(1-B^s)\ln\sigma_{t+h, h}
\]
where $B$ is the back-shift operator. The length of the time interval
h, which is 15min in the present case, is left implicit.

The autocorrelation function of the differenced process $w_t$, as shown in
figure \ref{fig:nordea_15min_w_acf}, clearly points to a seasonal $(0,
0, 1)\times(0, 0, 1)$ model: There are only 4 non-zero
autocorrelations in the plot, located at lags 1, 32, 33, 34,
respectively; furthermore, the two at 32 and 34 are approximately
equal. Thus we can write down the model as
\begin{eqnarray*}
  w_t = (1 - \theta B)(1 - \Theta B^s) y_t
\end{eqnarray*}
where $\theta$ and $\Theta$ are parameters to be determined and $y_t$
is a noise process with constant variance $\sigma_y^2$ and mean
0. $y_t$ is often refered to as the residuals.
\begin{figure}[htb!]
  \centering
  \subfigure[ACF of $\ln\sigma_{t+h,h}$]{
    \includegraphics[scale=0.4, clip=true, trim=95 230 112
    235]{../pics/nordea_15min_logvol_acf.pdf}
    \label{fig:nordea_15min_logvol_acf}
  }
  \subfigure[ACF of $w_t$]{
    \includegraphics[scale=0.4, clip=true, trim=95 230 112
    235]{../pics/nordea_15min_w_acf.pdf}
    \label{fig:nordea_15min_w_acf}
  }
  \caption{\footnotesize Nordea 15min $\ln\sigma_{t+h, h}$ and $w_t$ autocorrelations}
\end{figure}

The above $(0, 0, 1)\times(0, 0, 1)$ model has the following
autocovariance structure \cite{BoxJenkins94}:
\begin{eqnarray*}
  \gamma_0 &=& \sigma_y^2 (1 + \theta^2)(1 + \Theta^2) \\
  \gamma_1 &=& -\sigma_y^2\theta(1 + \Theta^2) \\
  \gamma_s &=& -\sigma_y^2\Theta(1 + \theta^2) \\
  \gamma_{s+1} &=& \gamma_{s-1}\;=\;\sigma_y^2\theta\Theta
\end{eqnarray*}
These equations together with the measured autocorrelations make
possible an initial estimate of the parameters $\theta$ and $\Theta$:
\begin{eqnarray*}
  {\rho_{s+1}/\rho_s} &=& {\gamma_{s+1}/\gamma_s} \;=\; -{\theta \over
    1 + \theta^2} \\
  {\rho_{s+1}/\rho_1} &=& {\gamma_{s+1}/\gamma_1} \;=\; -{\Theta \over
    1 + \Theta^2} \\
\end{eqnarray*}
Substituting in the measured values shown in table
\ref{tab:nordea_15min_w_acf},
\begin{table}[htb!]
  \centering
  \begin{tabular}{c|c|c|c}
    $\rho_1$ & $\rho_{s-1}$ & $\rho_s$ & $\rho_{s+1}$ \\
    \hline
    -0.4703 &  0.2053 & -0.4564 &  0.2212
  \end{tabular}
  \caption{\footnotesize Nordea 15min $w_t$ autocorrelations}
  \label{tab:nordea_15min_w_acf}
\end{table}
we get
\begin{eqnarray*}
  \theta &=& 0.6890 \\
  \Theta &=& 0.6378
\end{eqnarray*}
Among the two roots of each of the 2nd order equations in the above,
we have chosen the one in the range $(-1, 1)$ so as to ensure
invertibility of the model \cite{BoxJenkins94}.

With an estimate of $\theta$ and $\Theta$, one can then infer the
noise process i.e. the residuals $y_t$:
\begin{equation}
  \label{eq:infer_y}
  y_t = w_t + \theta y_{t-1} + \Theta y_{t-s} - \theta \Theta y_{t-s-1}
\end{equation}
where we substitute $y_t\;(t \leq 0)$ with their unconditional
expectation 0.
% By inverting the seasonal moving average model of $w_t$ we can then
% infer the noise process $y_t$ \footnotemark, i.e. the residuals:
% \begin{eqnarray*}
%   y_t &=& {1 \over 1 - \theta B}{1 \over 1 - \Theta B^s} w_t \\
%   &=& \sum_{i=0}^{\infty}\sum_{j=0}^{\infty} \theta^i \Theta^j B^{i+j}
%   w_t \\
%   &=& w_t + \sum_{k=1}^{\infty} \sum_{j=0}^{\bot(k/s)} \theta^{k-sj} \Theta^j
%   w_{t-k}
% \end{eqnarray*}
% where $\bot(\cdot)$ denotes the floor function, i.e. the largest
% integer smaller than its argument.
% \footnotetext{
%   The autoregressive process resulting from the inversion is inifinite
%   in extent. Hence one must choose a cut-off value for k. For the
%   present case of Nordea 15min log-volatility, I choose the value
%   $k=sj$ for which $|\Theta|^j < 0.02$. This value of k, call it
%   $k_{\top}$, is around 550. To avoid data reduction of this amount,
%   we add to the beginning of $\{w_t\}_{t=1}^n$ $k_{\top}$ zeros
%   corresponding to t=0, -1, ..., -$k_{\top}$+1. Recall $E(w_t) = E(y_t
%   - \theta y_{t-1} - \Theta y_{t-s} + \theta \Theta y_{t-s-1}) = 0$.
% }

In order to forecst the $w_t$ process, and hence the price process
itself, we must also know the distribution of $y_t$. Moreover, to
properly estimate the parameters of the model in the sense of maximum
likelihood, we are also in need of the distribution of $y_t$.

If $y_t$ is indeed i.i.d random variables as we assumed, we only need
to work out its unconditional distribution. This is shown in figure
\ref{fig:nordea_15min_y_qq} and \ref{fig:nordea_15min_y_acf}.
\begin{figure}[htb!]
  \centering
  \subfigure[QQ-plot of $y_t$]{
    \includegraphics[scale=0.4, clip=true, trim=95 230 112
    210]{../pics/nordea_15min_y_qq.pdf}
    \label{fig:nordea_15min_y_qq}
  }
  \subfigure[ACF of $y_t$]{
    \includegraphics[scale=0.4, clip=true, trim=95 230 112
    210]{../pics/nordea_15min_y_acf.pdf}
    \label{fig:nordea_15min_y_acf}
  }
  \caption{\footnotesize Nordea 15min $y_t$ QQ-plot and autocorrelations}
\end{figure}
In addition, the measured moments of $y_t$ are shown in table
\ref{tab:nordea_15min_y_moments}:
\begin{table}[htb!]
  \centering
  \begin{tabular}{c|c|c|c}
    mean & variance & skewness & kurtosis \\
    \hline
    0.0012 & 0.0935 & 0.2988 & 6.8691
  \end{tabular}
  \caption{\footnotesize Nordea 15min $y_t$ moments}
  \label{tab:nordea_15min_y_moments}
\end{table}

If we ignore the weak autocorrelations in this crude estimate of
$y_t$ and assume $y_t$ to be i.i.d random variables governed
by a skewed and leptokertic distribution, the parameters of this
distribution together with $\theta$ and $\Theta$ can be determined via
maximum likelihood estimate (MLE).

It turns out that $y_t$ can be well described by a Johnson Su
distribution \cite{Shang2004}:
\[
  y_t = \xi + \lambda\sinh{z_t - \gamma \over \delta}
\]
where $\gamma, \delta, \lambda, \xi$ are parameters to be determined
and $z_t \sim N(0, 1)$. The goodness of fitting is demonstrated in
figure \ref{fig:nordea_15min_y_js_fit} by the
empirical cummulative distribution function in comparison to the
theoretical one.
\begin{figure}[htb!]
  \vspace{-18mm}
  \centering
    \includegraphics[scale=0.35, clip=true, trim=112 272 107
    140]{../pics/nordea_15min_y_js_fit.pdf}
    \caption{\footnotesize Nordea 15min $y_t$ fitted to a Johnson Su distribution}
    \label{fig:nordea_15min_y_js_fit}
\end{figure}

The first 4 moments of the Johnson Su distribution are expressible
in closed form in $\gamma, \delta, \lambda, \xi$ \cite{Shang2004}:
\begin{eqnarray*}
  w &=& \exp{1 \over \delta^2} \\
  \Omega &=& {\gamma \over \delta} \\
  \text{E}(y) &=& -w^{1/2} \lambda \sinh\Omega + \xi\\
  \text{std}(y) &=& \lambda \left[{1 \over 2}(w-1)(w\cosh 2\Omega +
    1)\right]^{1/2} \\
  \text{skewness}(y) &=& {
    \sqrt{(1/2)w(w-1)} [w(w+2)\sinh 3\Omega + 3\sinh\Omega]
    \over
    (w\cosh 2\Omega + 1)^{3/2}} \\
  \text{kurtosis}(y) &=& {
    w^2(w^4 + 2w^3 + 3w^2 - 3)\cosh 4\Omega + 4w^2 (w+2) \cosh 2\Omega
    + 3(2w+1) \over
    2(w\cosh 2\Omega + 1)^2 }
\end{eqnarray*}
By matching the theoretical expressions of the moments with their
measured values, and taking help from published tables
\cite{Johnson1965}, one can solve for the parameters $\gamma, \delta,
\lambda, \xi$.

Under the assumption of i.i.d Johnson Su distributed residuals, the
log-likelihood function of the parameters $\gamma, \delta$
conditional on the sample $w_t$ can be written as
\[
L(\theta, \Theta) = -{1 \over 2}\sum_{t=1}^n z_t^2 + n \ln{\delta
  \over \lambda \sqrt{2\pi}} - {1 \over 2}\sum_{t=1}^n \ln\left[
  1 + \left({y_t - \xi \over \lambda}\right)^2
\right]
\]
where $y_t$ are infered from $w_t$ using eq.\ref{eq:infer_y}
and $z_t$ from $y_t$ using
\[
z_t = \delta \sinh^{-1}{y - \xi \over \lambda} + \gamma
\]
Note that $\gamma, \delta, \lambda, \xi$ are not really free
parameters but rather are implied by $\theta$ and $\Theta$: Once the
latter have been chosen and the corresponding $y_t$ infered, the
former are determined by the moments of $y_t$.

The eventual MLE is done in Matlab with the ``active set''
algorithm. The initial as well as the final estimation results are
listed in table \ref{tab:nordea_15min_js_param}:
\begin{table}[htb!]
  \centering
  \begin{tabular}{c|c|c|c|c|c|c}
    & $\gamma$ & $\delta$ & $\lambda$ & $\xi$ & $\theta$ & $\Theta$ \\
    \hline
    initial estimate & 0.1476 & 1.5121 & 0.3633 & 0.0454 & 0.6890 &
    0.6378 \\
    \hline
    MLE estimate & 0.1319 & 1.5266 & 0.3735 & 0.0410 & 0.6639 & 0.6025
  \end{tabular}
  \caption{\footnotesize Nordea 15min estimation results}
  \label{tab:nordea_15min_js_param}
\end{table}

% The probability density function of the residuals $y_t$ that result
% from the MLE estimate is shown in figure
% \ref{fig:nordea_15min_y1_hist} together with its counterpart from the
% initial estimate.
\subsection{45min Returns}
\section{Fat tail and stochastic volatility}
In this section we show that fat tails of the unconditional PDF are
the results of stochasticity in volatility.

Specifically we consider the model
\begin{equation}  \label{eq:UnconditionalPdf}
  \begin{aligned}
    r_t &= e^{\bar{v} + v_t} b_t \\
    &= e^{\bar{v}} e^{v_t} b_t \\
    &= e^{\bar{v}} r'_t
  \end{aligned}
\end{equation}
where $b_t \sim N(0, 1)$, $\bar{v}$ is a constant and $v_t \sim N(0,
\sigma)$ if there is no auto-correlation in $\{v_t\}$. If there is
auto-correlation, $v_t$ is described by an appropriate time series
model. For example, in the case of Volvo B 102min returns, $v_t$ is
represented by a seasonal auto-regressive model with 5 orders of
auto-regression and 6 orders of seasonal auto-regression with a
seasonality of 4.

Here we note that any stationary ARMA model can be represented as a
pure moving average model, which is infinite in extent if
auto-regressive components are present. Therefore auto-correlations in
$\{v_t\}$ influence the unconditional distribution of $r_t$ only
through $\sigma$.

In the following we derive the unconditional PDF of $r'_t$, denoted
$f_{r'}(x)$. Then the PDF of $r_t$ is $e^{-\bar{v}}f_{r'}(e^{-\bar{v}}x)$.
\begin{eqnarray*}
  P(r'_t < x) &=& P(b_t < xe^{-v_t}) \\
  f_{r'}(x) &=& f_b(xe^{-v_t}) e^{-v_t}
\end{eqnarray*}
Averaging over all $v_t$, we get
\begin{equation}\label{eq:UncondPDFSymmetric}
  \begin{aligned}
    f_{r'}(x) =& \int_{-\infty}^{\infty} dv (2\pi\sigma^2)^{-1/2}
    e^{-v^2/2\sigma^2}(2\pi)^{-1/2} \exp(-x^2e^{-2v}/2) e^{-v} \\
    =& {1 \over 2\pi\sigma} \int_{-\infty}^{\infty} dv
    \exp\left(
      -{1 \over 2\sigma^2} v^2 - v -{1 \over 2} x^2 e^{-2v}
    \right)
  \end{aligned}
  \end{equation}
The last part of the integrand, $e^{-x^2 e^{-2v} / 2}$, is plotted in
figure \ref{fig:DoubleExp}.
\begin{figure}[htb!]
  \centering
  \includegraphics[scale=0.5, clip=true, trim=85 252 100
  231]{../pics/DoubleExp.pdf}
  \caption{\footnotesize Plot of $\exp(-{1 \over 2} x^2 e^{-2v})$}
  \label{fig:DoubleExp}
\end{figure}
Therefore we make the following approximation:
\[
\exp\left(-{1 \over 2} x^2 e^{-2v}\right) \approx \left\{
  \begin{array}{lr}
    0 & \text{if } v < \ln|x| -{1 \over 2} \ln(2\ln 2) \\
    1 & \text{otherwise}
  \end{array}
\right.
\]
Here we note that $\exp\left(-{1 \over 2} x^2 e^{-2v}\right) = 1/2$ at
$v = \ln|x| -{1 \over 2} \ln(2\ln 2)$.

With this approximation we have
\begin{eqnarray*}
  f_{r'}(x) &=& {1\over C}{1 \over 2\pi\sigma} \int_{\ln(|x|/\sqrt{\ln
      4})}^{\infty} dv
  \exp\left(-{1 \over 2\sigma^2} v^2 - v\right) \\
  &=& {1\over C}{e^{\sigma^2 / 2} \over \sqrt{8\pi}} \text{erfc} \left(
    {1 \over \sqrt{2}\sigma} \ln{|x| \over \sqrt{\ln 4}} + {\sigma
      \over \sqrt{2}}
  \right)
\end{eqnarray*}
where $1/C$ has been added for the purpose of normalization.

At large $x$, we may use the asymptotic expansion of $\mathrm{erfc}$
to write
\begin{eqnarray*}
  Cf_{r'}(x) &=& {e^{\sigma^2 / 2} \over \sqrt{8\pi}} \mathrm{erfc}(\xi) \\
  &=& {e^{\sigma^2 / 2} \over \sqrt{8 \pi}}
  \frac{e^{-\xi^2}}{\xi\sqrt{\pi}}\left[
    1 +
    \sum_{n=1}^N (-1)^n \frac{(2n-1)!!}{(2\xi^2)^n} \right] +
  O(\xi^{-2N-1} e^{-\xi^2})
\end{eqnarray*}
where $\xi = {1 \over \sqrt{2}\sigma} \ln{|x| \over \sqrt{\ln 4}} +
{\sigma \over \sqrt{2}}$. The slowest-decaying term is
\[
f_{r,0}(x) = {e^{\sigma^2 / 2} \over \sqrt{8\pi}}
\frac{e^{-\xi^2}}{\xi\sqrt{\pi}}
\]
Let $\zeta = {|x| \over \sqrt{\ln 4}}$. With a bit manipulation one
obtains
\begin{equation}
  f_{r,0}(x) = {1 \over \pi \sqrt{8}}{1 \over
    \left(\ln\zeta/\sigma\sqrt{2} +
      \sigma/\sqrt{2}\right)\zeta\zeta^{\ln \zeta / 2\sigma^2}
  }
\end{equation}
From the last equation one can see that, at any neighborhood of large
$x$, $f_{r'}(x)$ may be approximated by $C/|x|^\alpha$, i.e. a power law.

In the following we work out the normalization constant C.
\begin{eqnarray*}
  \int_{-\infty}^{\infty} f_{r'}(x) dx &=& {1 \over C}{e^{\sigma^2 / 2} \over
    \sqrt{8\pi}} \int_{-\infty}^{\infty} \text{erfc} \left({1 \over
      \sqrt{2}\sigma} \ln{|x| \over \sqrt{\ln 4}} + {\sigma \over
      \sqrt{2}} \right) dx\\
  &=& {2 \over C}{e^{\sigma^2 / 2} \over
    \sqrt{8\pi}} \int_{0}^{\infty} \text{erfc} \left({1 \over
      \sqrt{2}\sigma} \ln{x \over \sqrt{\ln 4}} + {\sigma \over
      \sqrt{2}} \right) dx\\
\end{eqnarray*}
Let
\begin{eqnarray*}
  a &=& {1 \over \sigma \sqrt 2} \\
  b &=& -{1 \over 2 \sigma \sqrt 2}\ln\ln 4 + {\sigma \over \sqrt 2} \\
  y &=& a\ln|x|+b \\
\end{eqnarray*}
Then
\begin{eqnarray*}
  \int_{-\infty}^{\infty} f_{r'}(x) dx &=& {2 \over C}{e^{\sigma^2 / 2} \over
    \sqrt{8\pi}} \int_{-\infty}^{\infty} dy {e^{(y-b)/a} \over a}
  \text{erfc}(y) \\
  &=& {2 \over C}{e^{\sigma^2 / 2} \over
    \sqrt{8\pi}} \left.e^{(y-b)/a}
    \text{erfc}(y)\right|_{y=-\infty}^{\infty}
  + {2 \over C}{e^{\sigma^2 / 2} \over
    \sqrt{8\pi}} \int_{-\infty}^{\infty} dy {2 \over \sqrt \pi}
  e^{(y-b)/a} e^{-y^2} \\
  &=& {2 \over C}{e^{\sigma^2 / 2} \over \sqrt{8\pi}}  2 e^{\ln\ln 4/2
    - \sigma^2/2}\\
  C &=& \sqrt{2 \ln 4\over \pi} \\
  &\approx& 0.9394
\end{eqnarray*}
Thus we can write in summary:
\begin{eqnarray*}
  f_r(x) &=& e^{-\bar{v}} f_{r'}(e^{-\bar{v}} x) \\
    &=& {e^{-\bar{v}} \over C}{e^{\sigma^2 / 2} \over \sqrt{8\pi}}
    \text{erfc} \left({1 \over \sqrt{2}\sigma} \ln{|e^{-\bar{v}}x| \over \sqrt{\ln
          4}} + {\sigma \over \sqrt{2}}
    \right) \\
\end{eqnarray*}
Using the same technique of integration as for normalization, the
cummulative distribution function of $r_t$ is found to be $F(x)$,
which is the following:
\begin{enumerate}
\item if $x < 0$
  \begin{eqnarray*}
    F(x) &=& {\sqrt{\ln 4} \over C}{e^{\sigma^2 / 2} \over \sqrt{8\pi}}
    \left[
      {e^{-\bar{v}}x \over \sqrt{\ln 4}}\text{erfc}\left(
        {1 \over \sigma \sqrt 2} \ln{-e^{-\bar{v}}x \over \sqrt{\ln
            4}} + {\sigma \over \sqrt 2}
      \right) \right.\\
      && \left. + e^{-\sigma^2 / 2} \text{erfc}\left(
        {1 \over \sigma \sqrt 2} \ln{-e^{-\bar{v}}x \over \sqrt{\ln 4}}
      \right)
    \right]
  \end{eqnarray*}
\item if $x \geq 0$
  \begin{eqnarray*}
    F(x) &=& \frac{1}{2} + {\sqrt{\ln 4} \over C}{e^{\sigma^2 / 2} \over
      \sqrt{8\pi}} \left[
      {e^{-\bar{v}}x \over \sqrt{\ln 4}} \text{erfc}\left(
        {1 \over \sigma \sqrt 2} \ln{e^{-\bar{v}}x \over \sqrt{\ln
            4}} + {\sigma \over \sqrt 2}
      \right) \right. \\
      && \left. + e^{-\sigma^2/2} \text{erfc}\left(
        -{1 \over \sigma \sqrt 2} \ln{e^{-\bar{v}}x \over \sqrt{\ln 4}}
      \right)
    \right]
  \end{eqnarray*}
\end{enumerate}

To verify the validity of the model, we fit the above probability
density function to the de-meaned\footnote{By ``de-meaned returns'' we
  mean the quantity $r_t - \mean{r_t}$, where $r_t$ are the measured
  returns and $\mean{r_t}$ is their mean over time.} 30min returns of
Volvo B by means of maximum likelihood estimate using the MATLAB
function ``mle''. Then for the parameters $\sigma$ and $\bar{v}$ we
get
\begin{eqnarray*}
  \sigma &=& 0.6355 \\
  \bar{v} &=& -6.0308
\end{eqnarray*}
Then we plot $P(r'_t > x)$ of the model against its empirical
counterpart on a log-log scale, as shown in figure
\ref{fig:volvo_30min_ret}.
\begin{wrapfigure}{r}{0.5\textwidth}
  \vspace{-20mm}
%\begin{figure}
  \begin{center}
    \includegraphics[scale=0.4, clip=true, trim=98 231 116
    126]{../pics/volvo_30min_ret.pdf}
  \end{center}
  \vspace{-5mm}
  \caption{\footnotesize{$P(r'_t > x)$ for $x > 0$. Blue: empirical
      probabilities. Red: probabilities predicted by the model.}}
  \label{fig:volvo_30min_ret}
\end{wrapfigure}

In general figure \ref{fig:volvo_30min_ret} shows a good fit, but a
closer look reveals that deviations are significant in the regions $r
\in (0, \sigma_r)$ and $r \in (2\sigma_r, 3\sigma_r)$, where
$\sigma_r$ stands for the empirical standard deviation of the
de-meaned returns. These observations are
shown in greater details in figure \ref{fig:volvo_30min_ret2}.
\begin{figure}[htb!]
  \centering
  \subfigure[$\ln(P(r > x))$ with $x \in (0, \sigma_r)$]{
    \includegraphics[scale=0.4, clip=true, trim=93 224 115
    125]{../pics/volvo_30min_ret_0-sigma.pdf}
  }
  \subfigure[$\ln(P(r > x))$ with $x \in (2\sigma_r, 3\sigma_r)$]{
    \includegraphics[scale=0.4, clip=true, trim=93 224 115
    125]{../pics/volvo_30min_ret_2sigma-3sigma.pdf}
  }
  \caption{\footnotesize Surviving probabilities. Blue: empirical values of $\ln(P(r
    > x))$. Red: Predicted values of $\ln(P(r > x))$.}
  \label{fig:volvo_30min_ret2}
\end{figure}

Moreover, the inefficiency of the model is also manifest in the
skewness of the data. For the Volvo 30min returns, the data has a
skewness of 0.2419, but our model is strictly symmetric, since
$x$ appears in equation \ref{eq:UncondPDFSymmetric} only as $x^2$.

Hence the above model needs to be improved to accommodate the non-zero
skewness as well as to account for the discrepancies shown in figure
\ref{fig:volvo_30min_ret2}.

It has long been hypothesized in the liturature that skewness is the
result of price-volatility correlation (for example
\cite{Potters2003}). Therefore, the most apparent modification is to
allow $v_t$ and $b_t$ to be correlated. For convenience we decompose
$v_t \sim N(0, \sigma)$ as $v_t = \sigma a_t$ and assume
\begin{eqnarray*}
  \begin{pmatrix}
    a_t \\
    b_t
  \end{pmatrix} \sim N(0, \Sigma)
\end{eqnarray*}
with a covariance matrix
\begin{eqnarray*}
    \Sigma &=&
  \begin{pmatrix}
    1 & \psi \\
    \psi & 1
  \end{pmatrix}
\end{eqnarray*}
where $|\psi| < 1$. Then we rewrite equation \ref{eq:UnconditionalPdf}
as
\begin{eqnarray*}
    r_t &=& e^{\bar{v} + \sigma a_t} b_t \\
    &=& e^{\bar{v}} r'_t \\
    r'_t &=& e^{\sigma a_t} b_t \\
\end{eqnarray*}
Then
\begin{eqnarray*}
  && P(r'_t < x)\\
  &=& P(b_t < e^{-a_t\sigma}x) \\
  &=& {1 \over 2\pi |\det(\Sigma)|^{1/2}}
  \int_{-\infty}^{\infty} da \int_{-\infty}^{e^{-a\sigma}x} db
  \exp\left[
    -{1 \over 2} (a, b) \Sigma^{-1}
    \begin{pmatrix}
      a \\
      b
    \end{pmatrix}
  \right] \\
\end{eqnarray*}
After some manipulations we get
\begin{equation}\label{eq:UncondCDFAsymmetric}
  \begin{aligned}
    & P(r'_t < x) \\  
    &= {1 \over 2\pi \sqrt{1 - \psi^2}} \int_{-\infty}^{\infty} da
    e^{-a^2/2} \int_{-\infty}^{e^{-a\sigma}x} db
    \exp\left[
      - {(b - a\psi)^2 \over 2(1 - \psi^2)}
    \right] \\
    &= {1 \over 2\sqrt{2\pi}} \int_{-\infty}^{\infty} e^{-a^2/2}
    \text{erfc}{a\psi - e^{-a\sigma} x \over \sqrt{2(1-\psi^2)}} da
  \end{aligned}
\end{equation}
% To evaluate the above integral we taylor-expand $\erfc(u)$ around $u =
% 0$:
% \begin{eqnarray*}
%   \erfc(u) &=& 1 - {2 \over \sqrt \pi} \sum_{k=0}^\infty (-1)^k
%   {x^{2k+1} \over (1 + 2k) k!}
% \end{eqnarray*}
% Substitution in equation \ref{eq:UncondCDFAsymmetric} gives
% \begin{eqnarray*}
%   && P(r'_t < x) \\
%   &=& {1 \over 2} - {1 \over \pi\sqrt{2}} \sum_{k=0}^\infty (-1)^k {1 \over
%     (2k+1)k!} {1 \over \left[2(1 - \psi^2)\right]^{k+1/2}}
%   \int_{-\infty}^\infty da e^{-a^2/2}(a\psi - e^{-a\sigma}x)^{2k+1} \\
%   &=& 
% \end{eqnarray*}
% Binomial expansion of $(a\psi - e^{-a\sigma}x)^{2k+1}$ yields
% \begin{eqnarray*}
%   && P(r'_t < x) \\
%   &=& {1 \over 2} - {1 \over \pi\sqrt{2}} \sum_{k=0}^\infty
%   (-1)^k {1 \over (2k+1)k!} {1 \over \left[2(1 -
%       \psi^2)\right]^{k+1/2}} \sum_{i=0}^{2k+1} {2k+1 \choose i}
%   \psi^i x^{2k+1-i} \times \\
%   && \exp{\sigma^2 (2k+1-i)^2 \over 2} \int_{-\infty}^\infty da a^i
%   \exp\left\{
%     -{1 \over 2} [a + \sigma(2k+1-i)]^2
%     \right\} \\
%   \end{eqnarray*}
%   The trailing integral can be also evaluated by binomial expansion:
%   \begin{eqnarray*}
%     && \int_{-\infty}^\infty da a^i
%     \exp\left\{
%       -{1 \over 2} [a + \sigma(2k+1-i)]^2
%     \right\}\\
%     &=& \int_{-\infty}^\infty \sqrt{2} da \left[
%     \sqrt{2} a - \sigma(2k+1-i)
%   \right]^i e^{-a^2} \\
%   &=& \sum_{j=0}^i {i \choose j} 2^{j/2} \sigma^{i-j} (2k+1-i)^{i-j}
%   \sqrt{2} \int_{-\infty}^\infty a^j e^{-a^2} da \\
%   &=& \sum_{j=0}^i {i \choose j} 2^{j/2} \sigma^{i-j} (2k+1-i)^{i-j}
%   \sqrt{2} \Gamma\left(
%     {j+1 \over 2}
%   \right)
% \end{eqnarray*}
% Finally we get
% \begin{eqnarray*}
%     && P(r'_t < x) \\
%   &=& {1 \over 2} - {1 \over \pi\sqrt{2}} \sum_{k=0}^\infty
%   (-1)^k {1 \over (2k+1)k!} {1 \over \left[2(1 -
%       \psi^2)\right]^{k+1/2}} \sum_{i=0}^{2k+1} {2k+1 \choose i}
%   \psi^i x^{2k+1-i} \times \\
%   && \exp{\sigma^2 (2k+1-i)^2 \over 2} \sum_{j=0}^i {i \choose j}
%   2^{j/2} \sigma^{i-j} (2k+1-i)^{i-j} \sqrt{2} \Gamma\left(
%     {j+1 \over 2}
%   \right)
% \end{eqnarray*}
% The above expression is not in a closed form but nonetheless suffices
% for numerical evaluation. For that purpose we re-arrange the i index
% and rewrite the above expression as
% \begin{eqnarray*}
%     && P(r'_t < x) \\
%   &=& {1 \over 2} - {1 \over \pi\sqrt{2}} \sum_{k=0}^\infty
%   (-1)^k {1 \over (2k+1)k!} {1 \over \left[2(1 -
%       \psi^2)\right]^{k+1/2}} \sum_{i=0}^{2k+1} {2k+1 \choose i}
%   \psi^{2k+1-i} x^i \times \\
%   && \exp{\sigma^2 i^2 \over 2} \sum_{j=0}^{2k+1-i} {2k+1-i \choose j}
%   2^{j/2} (\sigma i)^{2k+1-i-j} \sqrt{2} \Gamma\left(
%     {j+1 \over 2}
%   \right)
% \end{eqnarray*}
% To compare the model with empirical return distributions we need $P(r_t
% < x)$:
% \begin{eqnarray*}
%   P(r_t < x) &=& P(e^{\bar{v}} r'_t < x) \\
%   &=& P(r'_t < e^{-\bar{v}} x) \\
% \end{eqnarray*}
To evaluate this integral we make an approximation of $\erfc(x)$:
\begin{equation*}
  \erfc(u) \approx \left\{
  \begin{array}{ll}
    2 & \text{if } u \leq u_1 \\
    Au^3 + Bu + 1 & \text{if } u_1 < u \leq u_2 \\
    0 & \text{if } u > u_2
  \end{array}
  \right.
\end{equation*}
where $u_1$ and $u_2$ are constants to be determined by OLE methods.

The 3 intervals of $u$ translate to 3 corresponding intervals of $a$:
\begin{enumerate}[I]
\item
  \begin{equation*}
    (-\infty, u_1) \Leftrightarrow \left\{
    \begin{array}{ll}
      (-\infty, a_1) & \text{ if } \psi > 0 \\
      (a_1, \infty) & \text{ if } \psi < 0 \\
    \end{array}
    \right.
  \end{equation*}
  where $a_1$ is defined as the root of the
  transcendental equation $a_1\psi - e^{-a_1\sigma}x = u_1 \sqrt{2(1-\psi^2)}$.
\item
  \begin{equation}\label{eq:RegionI}
    (u_1, u_2) \Leftrightarrow \left\{
    \begin{array}{ll}
      (a_1, a_2) & \text{ if } \psi > 0 \\
      (a_2, a_1) & \text{ if } \psi < 0 \\
    \end{array}
    \right.
  \end{equation}
  where $a_2$ is defined as the root of the
  transcendental equation $a_2\psi - e^{-a_2\sigma}x = u_1 \sqrt{2(1-\psi^2)}$.
\item
  \begin{equation}\label{eq:RegionII}
    (u_2, \infty) \Leftrightarrow \left\{
    \begin{array}{ll}
      (a_2, \infty) & \text{ if } \psi > 0 \\
      (-\infty, a_2) & \text{ if } \psi < 0 \\
    \end{array}
    \right.
  \end{equation}
\end{enumerate}
We denote the parts of the integral \ref{eq:UncondCDFAsymmetric} in
these intervals with I, II and III correspondingly. Then we can write
\begin{equation}\label{eq:DeducedCDF}
    P(r'_t < x) = {1 \over 2\sqrt{2\pi}} \left(
      \text{I + II}
    \right)
\end{equation}
In the following we first consider the case $\psi > 0$. In this case
$a_1 < a_2$
\begin{eqnarray*}
  \text{I} &=& 2 \int_{-\infty}^{a_1} e^{-a^2/2} da \\
  &=& \sqrt{2\pi} \erfc(-a_1/\sqrt 2)
\end{eqnarray*}
\begin{eqnarray*}
  \text{II} &=& \int_{a_1}^{a_2} da e^{-a^2/2} \left[
    A {(a\psi - e^{-a\sigma} x)^3\over 2^{3/2} (1 - \psi^2)^{3/2}} +
    B{a\psi - e^{-a\sigma}x \over \sqrt{2(1-\psi^2)}} + 1
  \right] \\
  &=& \int_{a_1}^{a_2} da e^{-a^2/2} \left[
    A{-e^{-3a\sigma}x^3 - 3a^2\psi^2 x e^{-a\sigma}
      +3 a \psi x^2 e^{-2a\sigma} + a^3\psi^3
      \over
      2^{3/2} (1 - \psi^2)^{3/2}
    } + \right.\\
    && \left. B{a\psi - e^{-a\sigma}x
    \over
    \sqrt{2(1 - \psi^2)}
  } + 1
\right] \\
&=& \text{II1} + \text{II2} + \text{II3} + \text{II4} + \text{II5} + \text{II6} + \text{II7}
\end{eqnarray*}
where
\begin{eqnarray*}
  \text{II1} &=& -{Ax^3 \over 2^{3/2} (1 -
    \psi^2)^{3/2}}\int_{a_1}^{a_2} da e^{-a^2/2} e^{-3a\sigma}x^3 \\
  &=& -{Ax^3e^{9\sigma^2/2}\sqrt{2\pi} \over 2^{5/2} (1 -
    \psi^2)^{3/2}} \left[
    \erfc\left(a_1 + 3\sigma \over \sqrt 2\right)
    - \erfc\left(a_2 + 3\sigma \over \sqrt 2\right)
  \right]
\end{eqnarray*}
\begin{eqnarray*}
  \text{II2} &=& -{3A\psi^2 x \over 2^{3/2} (1 - \psi^2)^{3/2}}
  \int_{a_1}^{a_2} e^{-a^2/2 - a\sigma} a^2 \\
  &=& -{3A\psi^2 x \over 2^{3/2} (1 - \psi^2)^{3/2}} e^{\sigma^2/2}
  \sqrt 2 \int_{(a_1+\sigma)/\sqrt 2}^{(a_2+\sigma)/\sqrt 2} da \left[
    2a^2 - 2\sqrt{2}\sigma a + \sigma^2
  \right] e^{-a^2} \\
  &=& -{3A\psi^2 x \over 2^{3/2} (1 - \psi^2)^{3/2}} e^{\sigma^2/2}
  \sqrt 2 \left[
    2\times\text{II2.1} -2\sqrt{2}\sigma\times\text{II2.2} + \sigma^2\text{II2.3}
  \right]
\end{eqnarray*}
\begin{itemize}
\item II2.1
  \begin{itemize}
  \item if $(a_1 + \sigma)(a_2 + \sigma) < 0$
    \begin{equation*}
      \text{II2.1} =  {1 \over 2} \left\{
        \gamma\left[ 3/2, {(a_2 + \sigma)^2 \over 2} \right] +
        \gamma\left[ 3/2, {(a_1 + \sigma)^2 \over 2} \right]
      \right\}
    \end{equation*}
  \item if $(a_1 + \sigma)(a_2 + \sigma) > 0$
    \begin{equation*}
      \text{II2.1} =  {1 \over 2} \left|
        \gamma\left[3/2, {(a_2 + \sigma)^2 \over 2}\right] -
        \gamma\left[3/2, {(a_1 + \sigma)^2\over 2}\right]
      \right|
    \end{equation*}
  \end{itemize}
  where $\gamma(s, x)$ denotes the lower incomplete gamma function:
  \begin{equation*}
    \gamma(s, x) = \int_0^x x'^{s-1} e^{-x'} dx'
  \end{equation*}
\item II2.2
  \begin{equation*}
    \text{II2.2} = {1 \over 2} \left|
      e^{-(a_1 + \sigma)^2/2} - e^{-(a_2 + \sigma)^2/2}
    \right|
  \end{equation*}
\item II2.3
  \begin{equation*}
    \text{II2.3} = {\sqrt{\pi} \over 2} \left(
      \erfc {a_1 + \sigma \over \sqrt 2} - \erfc {a_2 + \sigma \over \sqrt 2}
    \right)
  \end{equation*}
\end{itemize}
\begin{eqnarray*}
  \text{II3} &=& {3Ax^2\psi \over 2^{3/2}(1 - \psi^2)^{3/2}}
  \int_{a_1}^{a_2} da e^{-a^2/2 - 2a\sigma} a \\
  &=& {3Ax^2\psi \over 2^{3/2}(1 - \psi^2)^{3/2}} e^{2\sigma^2}
  \int_{(a_1 + 2\sigma)/\sqrt 2}^{(a_2 + 2\sigma)/\sqrt 2} e^{-a^2}
  (\sqrt{2} a - 2\sigma) \sqrt{2} da \\
  &=& {3Ax^2\psi \over 2^{3/2}(1 - \psi^2)^{3/2}} e^{2\sigma^2} \left[
    e^{-(a_1 + 2\sigma)^2/2} - e^{-(a_2 + 2\sigma)^2/2} - \right.\\
    &&
    \left. \sqrt{2\pi}\sigma \left(
      \erfc{a_1 + 2\sigma \over \sqrt 2} - \erfc{a_2 + 2\sigma \over \sqrt 2}
    \right)
  \right]
\end{eqnarray*}
\begin{eqnarray*}
  \text{II4} &=& {A\psi^3 \over 2^{3/2}(1 - \psi^2)^{3/2}} \left[
    \gamma\left(2, {a_2^2 \over 2}\right) -
    \gamma\left(2, {a_1^2 \over 2}\right)
  \right]
\end{eqnarray*}
\begin{eqnarray*}
  \text{II5} &=& {B\psi \over \sqrt{2(1 - \psi^2)}} \left(
    e^{-a_1^2/2} - e^{-a_2^2/2}
  \right)
\end{eqnarray*}
\begin{eqnarray*}
  \text{II6} = -{Bx e^{\sigma^2/2}\over \sqrt{1 - \psi^2}}{\sqrt \pi
    \over 2} \left[
    \erfc{a_1 + \sigma \over \sqrt{2}} -
    \erfc{a_2 + \sigma \over \sqrt{2}}
  \right]
\end{eqnarray*}
\begin{eqnarray*}
  \text{II7} &=& \sqrt{\pi \over 2} \left[
    \erfc{a_1^2 \over 2} - \erfc{a_2^2 \over 2}
  \right]
\end{eqnarray*}

So far our results are only valid in the case $\psi > 0$, i.e. $a_t$
and $b_t$ are positively correlated. If, however, $\psi < 0$, then
$a_1$ and $a_2$ as defined in \ref{eq:RegionI} and \ref{eq:RegionII}
will have inverse orders, $a_2 < a_1$. To make the above results valid
even in this case, we only need to exchange $a_1$ and $a_2$ in the
formulas.

To obtain $P(r_t < x)$, we note
\begin{eqnarray*}
P(r_t < x) &=& P(e^{\bar{v}} r'_t < x) \\
&=& P(r'_t < xe^{-\bar{v}}) \\
\end{eqnarray*}
That is, by substituting $xe^{-\bar{v}}$ for $x$ in the formula of
$P(r'_t < x)$ we get $P(r_t < x)$.

\chapter{Cross-Correlation Matrix with Gaussian Returns}
In this chapter we present some analytical and numerical results about
the cross-correlation matrix (cf. \S
\ref{sec:FundamentalConcepts}), in the special case where the involved
asset returns are assumed to be Gaussian distributed. This assumption
is of course an over-simplification, but nevertheless lends some
insight into the problem.

Our primary interest is in the influence of autocorrelations on the
cross-correlation matrix. \S \ref{sec:GCC-analytical} discusses the
distribution of the matrix entries and \S \ref{sec:GCC-numerical} the
distribution of the eigenvalues.

\section{Distribution of the Matrix Entries}\label{sec:GCC-analytical}
If the returns follow a zero-mean Gaussian distribution, 
i.e. $\vec{r}(t) \sim N(0, \Sigma)$, and are not
auto-correlated, the L\'evy index $\alpha$ as appears in equation
\ref{eq:cross-correlation} is 2, and $RR'$ is a Wishart matrix with
the following distribution law \cite{Anderson2003}:
\begin{equation}
  \label{eq:WishartPDF}
  f_{RR'}(X) = { |\det X|^{(T-N-1)/2} \exp\left(-{1 \over 2}\tr
      \Sigma^{-1}X \right)
    \over
    2^{NT/2}\pi^{N(N-1)/4}|\det \Sigma|^{T/2}
    \prod_{i=1}^N \Gamma\left[(n+1-i)/2\right]
  }
\end{equation}
where $\Sigma$ is the true covariance matrix of the returns.

The Wishart matrix $RR'$ is the main subject of study in the
literature (see for example \cite{Anderson2003}, \cite{Chiani2012}),
and we shall make no exception. Since the leading constant $1/T$ is
the only difference, the results about $RR'$ are easily transformed
to those about $RR'/T$.

When autocorrelations do exist among the columns of R, the
distribution of C is no longer Wishart but is nonethelss closely
related to it. In the following we consider the situation where
$r_{i,t}$ can be represented as a vector auto-regressive
process. Specifically, suppose
\begin{eqnarray*}
  \vec{r}_t &=& \sum_{k=1}^p \phi_k \vec{r}_{t-k} + \vec{a}_t \\
  \vec{a}_t &=& \vec{r}_t - \sum_{k=1}^p \phi_k \vec{r}_{t-k}
\end{eqnarray*}
where $\vec{a}_t = (a_{1,t}, a_{2,t}, \cdots, a_{N,t})' \sim N(0,
\Sigma)$. Here $N(0, \Sigma)$ denotes the multivarate normal
distribution with covariance matrix $\Sigma$. The last equation can be
written in matrix form
\[
A = R M
\]
For example, in the case of an AR(1) process
\begin{equation*}
  \begin{pmatrix}
    a_{1,1} & a_{1,2} & \cdots & a_{1,T} \\
    \vdots & \ddots & \vdots \\
    a_{N,1} & a_{N,2} & \cdots & a_{N,T} \\
  \end{pmatrix} =
  \begin{pmatrix}
    r_{1,1} & r_{1,2} & \cdots & r_{1,T} \\
    \vdots & \ddots & \vdots \\
    r_{N,1} & r_{N,2} & \cdots & r_{N,T} \\
  \end{pmatrix}
  \begin{pmatrix}
    1 & -\phi_1 &   &   & \\
      & 1 & -\phi_1 &   & \\
      &   & \ddots  & \ddots &   \\
      &   &   & 1 & -\phi_1 \\
      &   &   &   & 1 \\
  \end{pmatrix}
\end{equation*}
Let $QR = RM$, then $Q = RMR^{-1}$. Since the set of $\{r_{ij}\}$ for
which $\text{det } R = 0$ has probability zero, a matrix $Q$
satisfying the above equation almost surely exists. Thus $A = RM =
QR$ and $AA' = QRR'Q'$ Since the columns of A are not auto-correlated,
$AA' \sim W(\Sigma, T)$. Then $RR' \sim W(Q^{-1} \Sigma Q'^{-1}, T)$
follows from \cite{Anderson2003}, sec.7.3.3.

Now we observe that $\Sigma$ enters $f(RR')$ through $\tr (\Sigma^{-1}
RR')$ and $\det \Sigma$. $RR'$ enters through $\tr (\Sigma^{-1}
RR')$ and $\det (RR')$. Clearly
\begin{eqnarray*}
  \det (Q^{-1}\Sigma Q'^{-1}) &=& \det \Sigma \over (\det Q)^2 \\
  &=& \det \Sigma \over (\det M)^2 \\
  &=& \det \Sigma
\end{eqnarray*}
As to $\tr (\Sigma^{-1} C)$, we have
\begin{eqnarray*}
  && \tr\left[(Q^{-1}\Sigma Q'^{-1})^{-1} RR'\right] \\
  &=& \tr\left[Q'\Sigma^{-1} Q RR'\right] \\
  &=& \tr\left[ R'^{-1}M'R'\Sigma^{-1} RMR^{-1} RR'\right] \\
  &=& \tr\left[ \Sigma^{-1} RM (RM)'\right] \\
  &=& \tr\left[ \Sigma^{-1} AA'\right] \\
\end{eqnarray*}

Moreover, $\det RR' = \det AM^{-1} M'^{-1} A' = \det AA'$.

Thus if we use $f_R(\cdot)$ to denote the joint probability density of
the entries of RR' and $f_A(\cdot)$ to denote that of AA', we can
write
\begin{equation}\label{eq:cross-corr-matrix-PDF}
  f_R(RR') = f_A(RMM'R')
\end{equation}

Substituting for $f_A$ the Wishart probability density function (PDF)
\ref{eq:WishartPDF}, we get the PDF of $RR'$. The result is not
particularly simple. However, a relatively simple approximate
expression for the off-diagonal entries of $RR'/T$ may be
obtained. In the following we assume that $\text{var}(a_{i,t}) =
\sigma^2$ for any $i$ and $t$, and as always, that $a_{i,t}$ are not
autocorrelated, i.e. $\text{corr}(a_{i,t}, a_{i, t'}) = 0$ for any
$i$, $t$ and $t'$. We have
\begin{eqnarray*}
  r_{i,t} r_{j, t} &=& \left[
    \phi_1 r_{i, t-1} + a_{i, t}
  \right] \left[
    \phi_1 r_{j, t-1} + a_{j, t}
  \right] \\
  \mean{r_{i,t} r_{j, t}} &=& {1 \over T}\sum_{t=1}^T r_{i,t} r_{j, t} \\
  &=& \phi_1^2 \mean{r_{i,t-1} r_{j, t-1}} + \phi_1\mean{\left(
    r_{i, t-1} a_{j, t} + r_{j, t-1} a_{i, t}
  \right)} + \mean{ a_{i,t} a_{j,t}}
\end{eqnarray*}
We note that the two sums $\sum_{t=1}^T r_{i,t} r_{j, t}$ and
$\sum_{t=1}^T r_{i,t-1} r_{j, t-1}$ only differ by the first and the
last addend, which is negligible for sufficiently large T. Thus we
have
\begin{eqnarray*}
  (1 - \phi_1^2)\mean{r_{i,t} r_{j, t}} &\approx& \phi_1\mean{\left(
    r_{i, t-1} a_{j, t} + r_{j, t-1} a_{i, t}
  \right)} + \mean{ a_{i,t} a_{j,t}}
\end{eqnarray*}
Now we use the fact that $i$ and $j$ are symmetric on the left-hand
side, and hence must be so as well on the right-hand side. So we write
\begin{equation}\label{eq:Offdiag1}
  (1 - \phi_1^2)\mean{r_{i,t} r_{j, t}} = 2\phi_1\mean{r_{i, t-1}
    a_{j, t}} + \mean{ a_{i,t} a_{j,t}}
\end{equation}
Now we write the AR(1) process $r_{i,t}$ as an infinite moving-average
process:
\begin{eqnarray*}
  r_{i, t} &=& \phi_1 r_{i, t-1} + a_{i,t} \\
  (1 - \phi_1 B) r_{i, t} &=& a_{i,t} \\
\end{eqnarray*}
where $B$ is the back-shift operator. Then it follows from the above
equation
\begin{eqnarray*}
  r_{i,t} &=& {1 \over 1 - \phi_1 B} a_{i,t} \\
  &=& \sum_{k=0}^\infty \phi_1^k B^k a_{i,t} \\
  &=& \sum_{k=0}^\infty \phi_1^k a_{i,t-k} \\
\end{eqnarray*}
where it is remembered that $a_{i, t}$ with $t \leq 0$ is zero, (in
words, this implies that the $r_{i,t}$ process is not affected at all
by events before $t = 1$.)

% we have
% \begin{eqnarray*}
%   r_{i,t} &=& \sum_{k=0}^{t-1} \phi_1^k a_{i,t-k}
% \end{eqnarray*}
Substituting this into eq.\ref{eq:Offdiag1} for $r_{i,t-1}$ yields
\begin{equation}
  \label{eq:Offdiag2}
  \begin{aligned}
  (1 - \phi_1^2)\mean{r_{i,t} r_{j, t}} &\approx
  2 \sum_{k=0}^\infty\phi_1^{k+1} \mean{a_{i, t-k-1} a_{j, t}}
  + \mean{a_{i,t} a_{j,t}} \\
  &=
  2 \sum_{k=1}^\infty\phi_1^k \mean{a_{i, t-k} a_{j, t}}
  + \mean{a_{i,t} a_{j,t}} \\
  &=
  {2 \over T} \sum_{t=1}^{T} \sum_{k=1}^{t-1}\phi_1^k a_{i, t-k} a_{j, t}
  + {1 \over T} \sum_{t=1}^{T} a_{i,t} a_{j,t}\\
  \end{aligned}
\end{equation}

% To proceed any further, one has to assume a particular correlation
% structure among the a's. A simple yet non-trivial assumption is the
% following:
% \begin{equation*}
%   \text{corr}(a_{i, t-k}, a_{j, t}) = \left\{
%     \begin{array}{l l}
%       1 & \text{ if $i=j$ and $k = 0$ }\\
%       \rho & \text{ if $i \neq j$ and $k = 0$ }\\
%       0 & \text{otherwise}
%     \end{array}
%   \right.
% \end{equation*}

% To obtain the PDF of $\mean{r_{i,t} r_{j, t}}$ from the above
% equation, we first find the characteristic function of
% $a_{i,t}a_{j,t'}$ for arbitrary $t, t' \geq 1$. Then the
% characteristic function of the above weighted sum of ${T+1 \choose 2}$
% such random variables can be found as a product of the characteristic
% functions of the addends. In the end we compute the PDF of $(1 -
% \phi_1^2)\mean{r_{i,t} r_{j, t}}$ by an inverse Fourier transform.

Given two Gaussian random variables $x$ and $y$ with zero mean and
covariance matrix 
\begin{equation*}
  \Sigma =
  \begin{pmatrix}
    1 & \rho \\
    \rho & 1 \\
  \end{pmatrix}
\end{equation*}

The PDF of $xy$ can be found by considering $P(xy < z)$:
\begin{eqnarray*}
  P(xy < z) &=& \left(\int_0^\infty dx \int_0^{z/x} dy
    +\int_{-\infty}^0 dx \int_{z/x}^\infty dy \right)
    {1 \over 2\pi \sqrt{1 - \rho^2}} \\
    &&
    \exp\left[
      -{x^2 -2\rho xy + y^2
        \over
        2\sigma^2 (1 - \rho^2)} 
    \right] \\
  f_\text{{INN}}(z; \sigma, \rho) &=& {d \over dz} P(xy < z)\\
  &=& {1 \over \pi \sqrt{1 - \rho^2}} \exp\left[
    {\rho z \over \sigma^2 (1 - \rho^2)}\right] K_0\left[
    {|z| \over \sigma^2 (1 - \rho^2)}
  \right]
\end{eqnarray*}
where $K_n(z')$ is the modified Bessel function of the second
kind. It is worth taking note that, when $\rho \neq 0$,
$f_\text{{INN}}(z; \sigma, \rho)$ is not symmetric with respect to
$z$. As a result, if $\rho > 0$, the mean of $f_\text{{INN}}(z;
\sigma, \rho)$ is positive, and vice versa.

Secondly, because $|\rho| < 1$ and
\begin{equation*}
  K_0(z) \sim \sqrt{\pi \over 2z} e^{-z}
\end{equation*}
as $z \to \infty$ \cite{Olver:2010:NHMF}, all the moments of
$f_\text{{INN}}(z; \sigma, \rho)$ are finite, implying the
applicability of the Lyapunov central limit theorem provided that $T$
is large, which is very often the case and what we assume here.

With this mind, we observe that $\phi_1$ only affects the first sum in
\ref{eq:Offdiag2}. If $a_{i,t-k}$ and $a_{j,t}$ are not correlated for
non-zero $k$, $\phi_1$ will not affect the mean of $\mean{r_{i,t}
  r_{j, t}}$, the estimated covariance between asset $i$ and
$j$. Furthermore, it is also clear from equation \ref{eq:Offdiag2}
that the variance of $\mean{r_{i,t}r_{j, t}}$ is always increased by a
non-zero $\phi_1$, regardless of the sign of $\phi_1$. We compute this
increment in the following.

In light of the above expression for $f_\text{{INN}}(z; \sigma,
\rho)$, we rewrite equation \ref{eq:Offdiag2} as
\begin{eqnarray*}
  (1 - \phi_1^2)\mean{r_{i,t} r_{j, t}} &\approx&
  {2 \over T}
  \sum_{t=1}^{T} \sum_{k=1}^{t-1}\phi_1^k a_{i, t-k} a_{j, t} \\
  && + {1 \over T} \sum_{t=1}^{T} \sigma^2 (1 - \rho^2) {a_{i,t} \over
    \sigma \sqrt{1 - \rho^2}} {a_{j,t} \over \sigma \sqrt{1 - \rho^2}} \\
\end{eqnarray*}
In addition, we assume
\begin{equation*}
  \text{corr}(a_{i, t-k}, a_{j, t}) = \left\{
    \begin{array}{l l}
      1 & \text{ if $i=j$ and $k = 0$ }\\
      \rho & -1 < \rho < 1. \text{ if $i \neq j$ and $k = 0$ }\\
      0 & \text{otherwise}
    \end{array}
  \right.
\end{equation*}
Then, because $a_{i, t-k}$ and $ a_{j, t}$ with $i \neq j$ and $k
> 0$ are not correlated, the mean of $a_{i, t-k} a_{j, t}$ is 0
($f_\text{{INN}}(z; \sigma, 0)$ is symmetric), and the variance of it
can be found to be $\sigma^6$ using formula (10.43.19) of \cite{NIST:DLMF}:
\begin{equation*}
  \int_0^\infty dt K_\nu(t) t^{\mu-1} = 2^{\mu-2}
  \Gamma\left(
    {\mu + \nu \over 2}
  \right) \Gamma\left(
    {\mu - \nu \over 2}
  \right)
\end{equation*}
On the other hand, $a_{i,t}/\sigma \sqrt{1 - \rho^2}$ and
$a_{j,t}/\sigma \sqrt{1 - \rho^2}$ have variance $1/(1 - \rho^2)$
and are correlated - $\text{corr}(a_{i,t}, a_{j,t}) = \rho$. The mean
of $a_{i,t}a_{j,t}/\sigma^2 (1 - \rho^2)$ can be found using formula
(10.43.22) of \cite{NIST:DLMF}, given that $-1 < \rho < 1$:
\begin{eqnarray*}
  \int_0^\infty t^{\mu - 1} e^{-at} K_\nu(t) &=& (\pi/2)^{1/2}
  \Gamma(\mu + \nu) \Gamma(\mu - \nu)(1 - a^2)^{-\mu/2 + 1/4} \times\\
  && P^{-\mu+1/2}_{\nu-1/2} (a)
\end{eqnarray*}
where $P^\mu_\nu(\cdot)$ is Ferrers function of the first kind. The
result is
\begin{eqnarray*}
  E\left[{a_{i,t}a_{j,t} \over \sigma^2 (1 - \rho^2)}\right]
  &=&
  {1 \over \sqrt{2\pi} (1 - \rho^2)^{5/4}} \left[
    P^{-3/2}_{-1/2}(-\rho) - P^{-3/2}_{-1/2}(\rho)
  \right]
\end{eqnarray*}
Similarly, the variance of $a_{i,t}a_{j,t}/\sigma^2 (1 - \rho^2)$ is
found to be
\begin{eqnarray*}
  v^2(\rho) &=&
  {4 \over \sqrt{2\pi} (1 - \rho^2)^{7/4}} \left[
    P^{-5/2}_{-1/2}(\rho) + P^{-5/2}_{-1/2}(-\rho)
  \right] \\
  && - E^2\left[{a_{i,t}a_{j,t} \over
      \sigma^2 (1 - \rho^2)}\right]
\end{eqnarray*}
Now we can write down the asymptotic Gaussian distribution of
$\mean{r_{i,t}r_{j,t}}$, the empirical estimate
$\text{corr}(r_{i,t},r_{j,t})$:
\begin{equation*}
\mean{r_{i,t}r_{j,t}} \sim N(\mu'_X, \sigma'_X)
\end{equation*}
where
\begin{eqnarray*}
  \mu'_X &=& {\sigma^2 \over \sqrt{2\pi} (1 - \phi_1^2)(1 -
    \rho^2)^{1/4}} \left[ P^{-3/2}_{-1/2}(-\rho) -
    P^{-3/2}_{-1/2}(\rho)
  \right] \\
  \sigma'^2_X &=& {1 \over (1 - \phi_1^2)^2}\left[
    \sum_{t=1}^T \sum_{k=1}^{t-1} \left(
      2\phi_1^k \over T
    \right)^2 \sigma^6 + \sum_{t=1}^T
    {\sigma^4 (1 - \rho^2)^2 \over T^2} v^2(\rho)
  \right] \\
  &=& {4 \sigma^6 \over T (1 - \phi_1^2)^2} \left[
    {\phi_1^2 \over 1 - \phi_1^2} -
    {\phi_1^2 (1 - \phi_1^{2T}) \over
      T(1 - \phi_1^2)}
  \right] + {\sigma^4 (1 - \rho^2)^2 v^2(\rho) \over
    T (1 - \phi_1^2)^2} \\
  &\approx& {4 \sigma^6 \phi_1^2 \over T (1 - \phi_1^2)^3}
  + {\sigma^4 (1 - \rho^2)^2 v^2(\rho) \over
    T (1 - \phi_1^2)^2}
\end{eqnarray*}
Considering that, for any $i$ and $t$,
\begin{equation*}
  r_{i,t} = \sum_{k=0}^{t-1} \phi_1^k a_{i, t-k}
\end{equation*}
we have
\begin{equation*}
  \text{var}(r_{i,t}) \approx {\sigma^2 \over 1 - \phi^2}
\end{equation*}
Then it follows that
\begin{eqnarray*}
  \text{corr}(r_{i,t}, r_{j,t}) &=&
  {\mean{r_{i,t}, r_{j,t}} \over
    \sqrt{\text{var}(r_{i,t})\text{var}(r_{j,t})}} \\
  &\sim& N(\mu_X, \sigma_X)
\end{eqnarray*}
where
\begin{eqnarray*}
  \mu_X &=& {1 \over \sqrt{2\pi} (1 -
    \rho^2)^{1/4}} \left[ P^{-3/2}_{-1/2}(-\rho) -
    P^{-3/2}_{-1/2}(\rho)
  \right] \\
  \sigma'^2_X   &\approx& {4 \sigma^2 \phi_1^2 \over T (1 - \phi_1^2)}
  + {(1 - \rho^2)^2 v^2(\rho) \over T} \\
  &=& {4 \sigma^2 \over T (2^{2/\tau} - 1)}
  + {(1 - \rho^2)^2 v^2(\rho) \over T} \\ 
\end{eqnarray*}
In the last line, $\phi$ is replaced by the correlation time
$\tau$, whose definition is in equation \ref{eq:tau_def}.

What we have obtained is a quantitative description of what has been
said before: The mean of the off-diagonal entries of the
cross-correlation matrix is not affected by autocorrelations in the
returns, given that innovations $a_{i,t}$ of different assets and
different times are un-correlated; the variance, however, is always
increased by autocorrelations no matter the autocorrelation is
positive or nagative. The increment is in the form of an additive term
that scales as $1 / (2^{2/\tau} - 1)$.

% However, to write $f_A(RMM'R')$ as a function of the eigen
% values of $RMM'R'$ is very involved and no theoretical results are
% known.

% In the more general case
% \begin{eqnarray*}
%   a_{i,t} &=& \left(
%     1 - \sum_{k=1}^p \phi_{i,k} B^k
%   \right) r_{i,t} \\
%   &=& \Phi_i(B)} r_{i,t
% \end{eqnarray*}
% where $B$ denotes the back shift operator acting on subscript $t$, we
% can write
% \begin{eqnarray*}
%   A &=&
%   \begin{pmatrix}
%     \Phi_1(B) & &\\
%     & \ddots & \\
%     && \Phi_N(B)
%   \end{pmatrix} R \\
%   &=& \Phi(B) R
% \end{eqnarray*}

% Following the same reasoning, $RR' \sim W(\Phi^{-1}(B)
% \Sigma \Phi'^{-1}(B), T)$. Substituting $\Phi^{-1}(B)
% \Sigma \Phi'^{-1}(B)$ for $\Sigma$ in the Wishart
% probability density function, we get
% \begin{eqnarray*}
%   && \tr\left\{\left[\Phi^{-1}(B)\Sigma \Phi'^{-1}(B)\right]^{-1}
%     RR'\right\}\\
%   &=& \tr \left\{\Phi'(B)\Sigma^{-1} \Phi(B) RR'\right\}\\
%   &=& \tr\left\{\Sigma^{-1}AA'\right\}\\
% \end{eqnarray*}
% As to $\det \Phi^{-1}(B)\Sigma \Phi'^{-1}(B)$, we note that $\Sigma$
% is constant over time, and thus a function of $B$ has no effect on
% it. Therefore $\det \Phi^{-1}(B)\Sigma \Phi'^{-1}(B) = \det \Sigma$.

\section{Distribution of the Eigenvalues}\label{sec:GCC-numerical}
For the Wishart matrix, theoretical results are available for the
eigenvalue distribution \cite{Chiani2012}. In the simplest case where
$\Sigma = I$, i.e. no cross-correlations are present, we have
\begin{eqnarray*}
  f_{\lambda}(x_1, \cdots, x_N) = K \prod_{i=1}^N e^{-x_i/2}
  x_i^{(T-N-1)/2} \prod_{i<j}^N (x_i - x_j)
\end{eqnarray*}
where the eigenvalues have been indexed in descending order: $x_1 \geq
x_2 \geq \cdots \geq x_N$. The normalization constant K is given by
\begin{equation*}
  K = {\pi^{N^2/2} \over 2^{NT/2}\Gamma_N(T/2) \Gamma_N(N/2)}
\end{equation*}
where the function $\Gamma_m(a)$ is defined as
\begin{equation*}
  \Gamma_m(a) = \pi^{m(m-1)/4} \prod_{k=1}^m \Gamma\left(a - {k-1
      \over 2}\right)
\end{equation*}
However, as detailed in the derivation leading to equation
\ref{eq:cross-corr-matrix-PDF}, the distribution of $RR'$ is not
Wishart when the columns of $R$ are correlated. Deriving the
eigenvalue distribution analytically in this case is difficult and
beyond the scope of this thesis. Instead we resort to numerical
methods.

As before we consider the AR(1) process (see \S
\ref{sec:FundamentalConcepts}):
\begin{equation*}
  \vec{r}_t = \phi \vec{r}_{t-1} + \vec{a}_{t}
\end{equation*}
where $\vec{a}_t \sim N(0, I)$, i.e. the elements of $\vec{a}_t$ are
independent Gaussian random variable with zero mean and unit
variance.

Now we investigate how the eigenvalue distribution depends on
$\phi$. Figure \ref{fig:GaussianMarkovSpectrumPDF} shows the results
of the simulation.
% \begin{wrapfigure}{r}{0.5\textwidth}
\begin{figure}[htb!]
  \begin{center}
    \includegraphics[scale=0.5, clip=true, trim=90 228 115
    226]{../pics/GaussianMarkovSpectrumPDF.pdf}
  \end{center}
  \vspace{-10mm}
  \label{fig:GaussianMarkovSpectrumPDF}
  \caption{\footnotesize{
    Eigenvalue distribution with $\tau$ ranging from 0 to
    3. The 1st blue line, which is shown as stairs, is the
    theoretical eigenvalue distribution according to the
    Marcenko-Pastur law: $f(\lambda) = \sqrt{(\lambda_+/\lambda - 1)(1
      - \lambda/\lambda_-)}/2\pi q \sigma^2$, where $\lambda_{\pm} =
    \sigma^2(1 \pm \sqrt{q})^2$ In the simulation we have
    chosen $q = N/T = 50/1000 = 0.05$ and $\sigma=1$. For each value
    of $\tau$ we generate 2000 instances of $N \times T$ random matrix
    $R$, and compute C as $C=RR'/T$. Hence each curve in the figure is
    constructed from 2000 sets of eigenvalues.
  }}
\end{figure}
It is clear from the figure that, while the minimum eigenvalue remains
unchanged from its value given by the Marcenko-Pastur law, the maximum
eigenvalue moves consistently to the right as the value of $\phi$
increases. This prompts a study of the maximum eigenvalue distribution.

K. Johnsson \cite{Johnsson2000} and I. Johnstone \cite{Johnstone2001}
showed that, at the {\it absence} of autocorrelations and in the asymptotic
limit $N, T \to \infty$, $N/T \to q < \infty$, the maximum eigenvalue
$\lambda_1$ follows the Tracy-Widom distribution (denote
$\mathscr{TW}_1$ here) when properly relocated and rescaled:
\begin{equation*}
  {\lambda_1 - \mu_{NT} \over \sigma_{NT}} \sim \mathscr{TW}_1
\end{equation*}
where $\mu_{NT}$ and $\sigma_{NT}$ are given by
\begin{eqnarray*}
  \mu_{NT} &=& \left(
    \sqrt{N-1/2} + \sqrt{T - 1/2}
  \right)^2 \\
  \sigma_{NT} &=& \sqrt{\mu_{NT}} \left(
    {1 \over \sqrt{N-1/2}} + {1 \over \sqrt{T-1/2}}
  \right)^{1/3}
\end{eqnarray*}

The $\mathscr{TW}_1$ distribution has the following cummulative
distribution function (CDF) \cite{Chiani2012}
\begin{equation*}
  F_1(x) = \exp\left[
    -{1 \over 2} \int_x^\infty dy \left(q(y) + (y-x)q^2(y)\right)
  \right]
\end{equation*}
where $q(y)$ is defined as the solution to the Painlev\'e II differential
equation
\begin{equation*}
  q''(y) = yq(y) + 2q^3(y)
\end{equation*}
which is unique when imposing the condition
\begin{equation*}
  q(y) \sim \text{Ai}(y) \text{ as } y \to \infty
\end{equation*}

Then Marco Chiani showed recently that the $\mathscr{TW}_1$ distribution
can be well approximated by a gamma distribution based on his proof
that the exact distribution of the maximum eigenvalue is a mixture of
gamma distributions. Specifically,
\begin{equation}\label{eq:TracyWidom-Gamma}
  {\lambda_1 - \mu_{NT} \over \sigma_{NT}} + \alpha \sim
  \mathscr{G}(k ,\theta)
\end{equation}
where $\mathscr{G}(k, \theta)$ denotes the Gamma distribution with
parameters $k$ and $\theta$ \cite{Chiani2012}. The probability
density function (PDF) of the gamma distribution is given by
\begin{equation*}
  f_\gamma(x; k, \theta) = {1 \over \Gamma(k) \theta^k} x^{k-1}
  e^{-x/\theta}
\end{equation*}
Moreover, the first 3 moments of the distribution are simple:
\begin{eqnarray*}
  E &=& k\theta \\
  \text{variance} &=& k\theta^2 \\
  \text{skewness} &=& 2/\sqrt{k}
\end{eqnarray*}

In the following we first verify these results and extend them to the
case {\it with} autocorrelations and then study how the
distribution changes as $\phi$ increases. Figure
\ref{fig:GaussianMarkov005MaxEigCDF_loglog} compares the empirical
maximum eigenvalue CDF with the CDF of a gamma distribution. Each pair
of CDFs correspond to a fixed autocorrelation strength ($\phi$). The
parameters $k$ and $\theta$ of the gamma distribution are fit to
data by matching the 2nd and the 3rd moments of the gamma distribution
to the corresponding moments of the empirical distribution. Then the
parameter $\alpha$ in equation \ref{eq:TracyWidom-Gamma} is chosen to
be
\begin{equation*}
  \alpha = k\theta - \mean{{\lambda_1 - \mu_{NT} \over \sigma_{NT}}}
\end{equation*}

\begin{figure}
  \begin{center}
    \includegraphics[scale=0.5, clip=true, trim=30 198 32
    166]{../pics/GaussianMarkov005MaxEigCDF_loglog.pdf}
    \label{fig:GaussianMarkov005MaxEigCDF_loglog}
    \caption{\footnotesize Cummulative distribution function (CDF) of the maximum
      eigenvalue ($\lambda_1$). Blue: empirical CDF. Red: CDF of the
      fitting gamma distribution. Both axis's are in log-scale.}
  \end{center}
  \vspace{-10mm}
\end{figure}

We can see in figure \ref{fig:GaussianMarkov005MaxEigCDF_loglog} that
the gamma distribution fits fairly well to its corresponding maximum
eigenvalue distribution. So we conclude that a gamma distribution not
only approximates the maximum eigenvalue distribution at the absence of
autocorrelations but does so even at the {\it presence} of
autocorrelations. Autocorrelations influence the maximum eigenvalue
distribution only via the parameters $k$, $\theta$ and $\alpha$. Next
we study this influcence in detail.

% Since the maximum eigenvalue distribution is well approximated by a
% gamma distribution, which is characterized by the parmaters $k$ and
% $\theta$, connecting the maximum eigenvalue distribution with the
% autocorrelation means connecting $k$ and $\theta$ with the
% autocorrelation strength $\phi$ or alternatively with the correlation
% time $\tau$.

Figure \ref{fig:GaussianMarkovMaxEig_k-phi} shows how the parameter $k$
varies as the autocorrelation strengthens ($\phi$ increases). 
\begin{figure}[htb!]
  \vspace{-10mm}
  \centering
    \includegraphics[scale=0.5, clip=true, trim=100 223 112
    141]{../pics/GaussianMarkov05MaxEig_k-phi.pdf}
    \caption{\footnotesize The parameter $k$ against autocorrelation strength
      $\phi$. Blue crosses: Empirical values of $k$. Red: best fitting
      line in terms of {\it Least Square Errors}.}
  \label{fig:GaussianMarkovMaxEig_k-phi}
\end{figure}
From the equation of the fitting line we can directly read out
\begin{eqnarray*}
  k &=& a\phi + b \\
  &=& a\left(1 \over 2\right)^{1/\tau} + b
\end{eqnarray*}
where $a = -26$ and $b = 47$.

For the parameter $\theta$, its behavior is more conveniently
described in terms of the correlation time $\tau$. Figure
\ref{fig:GaussianMarkovMaxEig_theta-tau} plots the values of $\theta$
against those of $\tau$ together with a fitting quadratic
function. Higher order polynomials provide a slightly better
fit, but coefficients of the 3rd order and above are less than 1/1000
times the coefficients of the 2nd and the 1st order. Therefore a 2nd
order polynomial has been chosen.
\begin{figure}[htb!]
  \vspace{-15mm}
  \centering
  \includegraphics[scale=0.5, clip=true, trim=99 230 114
  139]{../pics/GaussianMarkov05MaxEig_theta-tau.pdf} 
  \caption{\footnotesize parameter $\theta$ against correlation time $\tau$. Blue:
    empirical values of $\theta$. Cyan: Fitting quadratic function.}
  \label{fig:GaussianMarkovMaxEig_theta-tau}
\end{figure}
The equation of this polynomial is
\begin{eqnarray*}
  \theta &=& A\tau^2 + B\tau + C
\end{eqnarray*}
where $A = 2.1\times 10^{-4}$, $B = -1.1\times 10^{-4}$, $C =
3.2\times 10^{-4}$.

The parameter $\alpha$ shifts the gamma distribution $\mathscr{G}(k,
\theta)$ to match the mean of ${(\lambda_1 -
  \mu_{NT})/\sigma_{NT}}$. The behavior of $\alpha$ with respect to
changing autocorrelation is shown in figure
\ref{fig:GaussianMarkovMaxEig_alpha-tau}.
\begin{figure}[htb!]
  \vspace{-15mm}
  \centering
  \includegraphics[scale=0.5, clip=true, trim=104 229 114
  139]{../pics/GaussianMarkovMaxEig_alpha-tau.pdf}
  \caption{\footnotesize The mean-shift parameter $\alpha$ against
    correlation time $\tau$. Blue: empirical values of $\alpha$. Cyan:
    quadratic fit.}
  \label{fig:GaussianMarkovMaxEig_alpha-tau}
\end{figure}
The equation of $\alpha$ is then inferred from the fitting line:
\begin{equation*}
  \alpha = D\tau^2 + E\tau + F
\end{equation*}
where $D = -3.4\times 10^{-3}$, $E = -4.6\times 10^{-2}$ and $F = 69$.

So combining the results of $\alpha$, $k$ and $\theta$ we can express
the moments of the Tracy-Widom variate ${(\lambda_1 -
  \mu_{NT})/\sigma_{NT}}$:
\begin{equation}\label{eq:lambda1_MeanVariance}
  \begin{aligned}
    E({\lambda_1 - \mu_{NT} \over \sigma_{NT}}) &= k\theta -
    \alpha \\
    &= \left[-a\left(1 \over 2\right)^{1/\tau} + b\right](A\tau^2 +
    B\tau - C) - (D\tau^2 + E\tau + F) \\
    \text{variance}({\lambda_1 - \mu_{NT} \over \sigma_{NT}}) &=
    k\theta^2 \\
    &= \left[-a\left(1 \over 2\right)^{1/\tau} + b\right](A\tau^2 +
    B\tau - C)^2 \\
  \end{aligned}
\end{equation}
Figure \ref{fig:GaussianMarkovMaxEig_tw_moments} shows the empirical
moments of ${(\lambda_1 - \mu_{NT})/\sigma_{NT}}$ together with the
corresponding values computed using the above formulas.
\begin{figure}[htb!]
  \vspace{-15mm}
  \centering
  \includegraphics[scale=0.46, clip=true, trim=48 243 6
  134]{../pics/GaussianMarkovMaxEig_tw_moments.pdf}
  \caption{\footnotesize empirical moments of ${(\lambda_1 -
      \mu_{NT})/\sigma_{NT}}$ against their theoretical
    counterparts. Left: Empirical/theoretical mean against $\tau$;
    Right: Empirical/theoretical variance against $\tau$. Blue
    crosses: empirical values; Red line: fitting curves. Horizontal
    axis: correlation time $\tau$.}
\label{fig:GaussianMarkovMaxEig_tw_moments}
\end{figure}
The good fitness shown in figure
\ref{fig:GaussianMarkovMaxEig_tw_moments} allows us to conclude 
that, within the range of the correlation time that we have studied,
namely $\tau \in [0, 13.51]$, the $k$ parameter is a linear function
of $\phi = 2^{-1/\tau}$ while $\theta$ and $\alpha$ are quadratic
functions of $\tau$. The moments of the transformed maximum
eigenvalue, namely ${(\lambda_1 - \mu_{NT})/\sigma_{NT}}$, are thus
expressed as functions of the correlation time $\tau$ via the
parameters $k$, $\theta$ and $\alpha$.

Figure \ref{fig:GaussianMarkovMaxEigPDF_original} shows the PDF of the
maximum eigenvalue ($\lambda_1$) for a range of values of the
correlation time $\tau$. One can clearly see that the mean of the
distribution moves to the right and the width of the distribution
increases as $\tau$ takes on larger and larger values. However, one
must not forget that the behavior of the mean and the width is random
in nature. Eq. \ref{eq:lambda1_MeanVariance} describes such behavior in
the asymptotic limit, i.e. if an infinite number of random matrices
are generated and their respective maximum eigenvalues computed for
each and every value of correlation time $\tau$.
\begin{figure}[htb!]
  \begin{center}
    \includegraphics[scale=0.5, clip=true, trim=98 228 112
    221]{../pics/GaussianMarkovMaxEigPDF_original.pdf}
    \label{fig:GaussianMarkovMaxEigPDF_original}
    \caption{Probability density function of the maximum eigenvalue
      ($\lambda_1$).}
  \end{center}
\end{figure}

% The above observations lead to the conclusion that the
% autocorrelations are fully characterized by the parameters $k$ and
% $\theta$, which determine the gamma distribution via the 1st and the
% 2nd moments:
% \begin{eqnarray*}
%   E(\lambda_1) &=& k\theta \\
%   \text{var}(\lambda_1) &=& k\theta^2
% \end{eqnarray*}

% Table \ref{tab:GammaDistParam} shows the values of $k$ and $\theta$
% corresponding to different values of $\phi$.
% \begin{table}
%   \begin{footnotesize}
%   \centering
%   \begin{tabular}{c|c|c || c|c|c }
%     \hline
%     $\phi$ & $k$ & $\theta$ &  $\phi$ & $k$ & $\theta$ \\
%     \hline
%     0  &  48.4730  &   0.0002  &   0.5000  &  66.9245  &   0.0003  \\
%     0.0500  &  32.3891  &   0.0002  &   0.5500  &  21.1857  &   0.0007  \\
%     0.1000  &  54.4910  &   0.0002  &   0.6000  &  49.2886  &   0.0005  \\
%     0.1500  &  42.6755  &   0.0002  &   0.6500  &  39.0294  &   0.0007  \\
%     0.2000  &  39.7918  &   0.0002  &   0.7000  &  37.8492  &   0.0009  \\
%     0.2500  &  30.0915  &   0.0003  &   0.7500  &  27.5861  &   0.0016  \\
%     0.3000  &  60.3928  &   0.0002  &   0.8000  &  29.6914  &   0.0021  \\
%     0.3500  &  26.1387  &   0.0004  &   0.8500  &  19.5804  &   0.0047  \\
%     0.4000  &  36.8579  &   0.0003  &   0.9000  &  28.8284  &   0.0079  \\
%     0.4500  &  29.0491  &   0.0004  &   0.9500  &  18.2793  &   0.0375  \\
%     \hline
%   \end{tabular}
%   \label{tab:GammaDistParam}
%   \caption{Values of $k$ and $\theta$ for different $\phi$}
% \end{footnotesize}
% \end{table}
% The values of $k$ are random while those of $\theta$ have a trend to
% increase. It is hence reasonable to hypothesize that more probability
% is distributed to large values of $\theta$ as $\phi$ gets larger. The
% values of $\theta$ are plotted against those of $\phi$ in figure
% \ref{fig:GaussianMarkovThetaPhi}. One can see that the $\theta$ values
% grow faster than an exponential function after exceeding a critical value.
% \begin{figure}
%   \begin{center}
%     \includegraphics[scale=0.35, clip=true, trim=58 220 87
%     133]{../pics/GaussianMarkovThetaPhi.pdf}
%   \end{center}
%   \caption{$\theta$ values againt $\phi$ values in linear scale
%     (left) and in semi-log scale (right)}
%   \label{fig:GaussianMarkovThetaPhi}
% \end{figure}
% The smallness of the $\theta$ values and their behavior of increasing
% with $\phi$ explain the pattern shown in figure
% \ref{fig:GaussianMarkovMaxEigPDF_original}: The mean as well as the variance
% of the $\lambda_1$ distribution increases with $\phi$; The effect of
% right-shifting dominates before a critical $\phi$ value, while the
% effect of widening dominates after the critical value.

\chapter{Cross-Correlation Matrix with Fat-Tailed Returns}
\bibliographystyle{plain}
\bibliography{econophysics}
\end{document}


