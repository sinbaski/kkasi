\documentclass{book}
\usepackage{graphicx}
\usepackage{subfigure}
\usepackage{enumerate}
\usepackage[bookmarks=true]{hyperref}
\usepackage{bookmark}

\usepackage{amssymb,amsmath,amsthm,amsfonts}
\usepackage{mathrsfs}
\usepackage{dsfont}
\usepackage{enumerate}

%\newtheorem{mdef}{Definition}
%\newtheorem{theorem}{Theorem}
\newcommand{\eqsplit}[2]{
  \begin{equation}\label{#2}
    \begin{split}
      #1
    \end{split}
  \end{equation}}
\newcommand{\eqnsplit}[1]{
  \begin{eqnarray*}
    #1
  \end{eqnarray*}}
\newcommand{\tran}[1]{
  \tilde{#1}
}
\newcommand{\td}[2]{
  \frac{d #1}{d #2}
}
\newcommand{\pd}[2]{
  \frac{\partial #1}{\partial #2}
}
\newcommand{\ppd}[2]{
  \frac{\partial^2 #1}{\partial #2^2}
}
\newcommand{\pdd}[3]{
  \frac{\partial^2 #1}{\partial #2 \partial #3}
}
\newcommand{\otd}[1]{
  \frac{d}{d #1}
}
\newcommand{\opd}[1]{
  \frac{\partial}{\partial #1}
}
\newcommand{\oppd}[1]{
  \frac{\partial^2}{\partial #1^2}
}
\newcommand{\opdd}[2]{
  \frac{\partial^2}{\partial #1 \partial #2}
}
\newcommand{\ket}[1]{
  |#1\rangle
}
\newcommand{\bra}[1]{
  \langle#1|
}
\newcommand{\inn}[1]{
  \langle#1\rangle
}
\newcommand{\mean}[1]{
  \langle#1\rangle
}
\newcommand{\tr}{
  \text{tr}\,
}
\newcommand{\re}{
  \text{Re}\,
}
\newcommand\im{
  \text{Im}\,
}
\newcommand{\var}{
  \text{var}
}
\newcommand{\arcsinh}{
  \sinh^{-1}
}
\newcommand{\arccosh}{
  \cosh^{-1}
}
\newcommand{\erfc}{
  \text{erfc}
}
\newcommand{\E}{
  \mathbb{E}
}
\renewcommand{\P}{
  \mathbb{P}
}
\newcommand{\I}[1]{
  \mathbf{1}_{\{#1\}}
}
\newcommand{\1}[1]{
  \mathds{1}_{\{#1\}}
}
\newcommand{\diag}{
  \text{diag\,}
}
\newcommand{\M}{
  {\text{max}}
}
\newcommand{\m}{
  {\text{min}}
}
\newcommand{\ph}{
  {\text{arg}\,}
}
\newcommand\erf{
  \text{erf}
}
\renewcommand\vec[1]{
  \mathbf{#1}
}
\newcommand\mtx[1]{
  \mathbf{#1}
}
\newcommand\ed{
  \,{\buildrel d \over =}\,
}



\DeclareGraphicsExtensions{.pdf,.png,.jpg}

\title{Mathematical Aspects of the Capital Market}
\author{Xie Xiaolei}
\date{\today}
\begin{document}

\maketitle
\tableofcontents

\chapter{Introduction}
\section{Stylized facts}

\section{GARCH models}
The Auto-correlation function of $\epsilon_t^2$ is given by (c.f. \cite{Bollerslev87})
$$
\rho_n = \sum_{i=1}^{max(p,q)} (\alpha_i + \beta_i) \rho_{n-i}
\;n > p
$$
where $\alpha_i$ with $i > q$ and $\beta_i$ with $i > p$ are taken as
zeros. From these equations, it is clear that the partial
auto-correlation function cuts off at $max(p, q)$.


\chapter{Case Study of Stocks}
\section{Nordea Bank}
\begin{figure}[ht]
  \centering
  \includegraphics[scale=0.5, clip=true, trim=103 236 118
  130]{../pics/nordea_price_20120116-20120420.pdf}
  \caption{Nordea Bank paid prices 2012/01/16-2012/04/20}
  \label{fig:Nordea}
\end{figure}
Figure \ref{fig:Nordea} shows the paid prices of Nordea Bank from
2012/01/16 to 2012/04/20.

\subsection{15min Returns}
\subsubsection{GARCH model}
The auto-correlation function (ACF) of $\epsilon_t^2$ are shown in figure
\ref{fig:nordea_15min_acf}. At the absence of GARCH effects, the ACF
will have an asymptotic Gaussian distribution with mean 0 and variance
1/T. The first 5 auto-correlations have considerable sizes and do not
fall off as in the Gaussian case. This observation suggests a GARCH
model. Moreover, from \ref{fig:nordea_15min_vlt_acf}, one sees
apparent auto-correlations between volatilities of consecutive time
intervals. This again points to a GARCH model.
\begin{figure}[htb!]
  \centering
  \subfigure[ACF of squared returns]{
    \includegraphics[scale=0.4, clip=true, trim=95 236 118
    130]{../pics/nordea_15min_acf.pdf}
    \label{fig:nordea_15min_acf}
  }
  \subfigure[ACF of volatilities]{
    \includegraphics[scale=0.4, clip=true, trim=95 236 118
    130]{../pics/nordea_15min_vlt_acf.pdf}
    \label{fig:nordea_15min_vlt_acf}
  }
  \caption{Nordea Bank 15m autocorrelations}
\end{figure}

To obtain the functional form of the conditional distribution of
$\epsilon_t$, we look at the qq-plot of $\epsilon_t/\sigma_t$
\ref{fig:nordea_15min_epOversig_qq}, which is consistent with a
standard Gaussian distribution except for a few outliers at the two
ends. Hence the fat tails of the probability density function can be
entirely accounted for by GARCH effects, implying a Gaussian
conditional distribution is appropriate for $\epsilon_t$. 
\begin{figure}[htb!]
  \centering
    \includegraphics[scale=0.6, clip=true, trim=100 236 118
    130]{../pics/nordea_15min_epOversig_qq.pdf}
  \caption{QQ-plot of $\epsilon_t / \sigma_t$. $\epsilon_t$ are
    derived from Nordea Bank 15m returns while $\sigma_t$ are realized
    volatilities calculated using 30s returns within each 15m
    interval.}
  \label{fig:nordea_15min_epOversig_qq}
\end{figure}

To determine the orders p and q of the GARCH model, we again check
the partial auto-correlation function of $\epsilon_t^2$. As just
mentioned, the first 5 partial auto-correlations are significant. Thus
we can tentatively estimate a GARCH(1, 5) model. Then by applying the
Lagrange multiplier test and accordingly removing insignificant
parameters, we obtain a GARCH(1, 3) model
\ref{tab:nordea_15min_garch}:
\begin{table}[htb!]
  \centering
  \begin{tabular}{c|c|c|c}
    parameter & value & std. error & t statistic \\
    \hline
     $\alpha_0$ &   1.05668e-06 &   3.56871e-07 &    2.96096\\
     \hline
     $\beta_1$  &     0.687212  &   0.0299314   &    22.9596\\
     \hline
     $\alpha_1$ &      0.128614 &   0.0268144   &    4.79643\\
     \hline
     $\alpha_3$ &    0.0568022  &   0.0301503   &    1.88397
  \end{tabular}
  \caption{GARCH(1, 3) model of Nordea Bank 15m returns}
  \label{tab:nordea_15min_garch}
\end{table}

\subsection{15min Returns}


\bibliographystyle{plain}
\bibliography{econophysics}
\end{document}


