\chapter{Correlation Matrix of Gaussian Returns}\label{chp:Gaussian}
In this chapter we present some results about the cross-correlation
matrix of Gaussian returns. This assumption is of course an
over-simplification, but nevertheless lends some insight into the
problem.

Our primary interest is in the influence of autocorrelations on the
cross-correlation matrix. \S\ref{sec:GCC-analytical} and
\S\ref{sec:GCC-numerical} discuss the distribution of the matrix
elements and the eigenvalues, respectively.

\section{Distribution of the Matrix Elements}\label{sec:GCC-analytical}
If the returns follow a zero-mean Gaussian distribution, 
i.e. $\vec{r}_t \sim N(0, \Sigma)$, and are not
auto-correlated, then $RR'$ is a Wishart matrix with
the following distribution law \cite{Anderson2003}:
\begin{equation}
  \label{eq:WishartPDF}
  f_{RR'}(X) = { |\det X|^{(T-N-1)/2} \exp\left(-{1 \over 2}\tr
      \Sigma^{-1}X \right)
    \over
    2^{NT/2}\pi^{N(N-1)/4}|\det \Sigma|^{T/2}
    \prod_{i=1}^N \Gamma\left[(n+1-i)/2\right]
  }
\end{equation}
where $\Sigma$ is the true covariance matrix of the returns.

When $\vec{r}_t$ are indeed auto-correlated, $RR'$ no longer has the
Wishart distribution. The joint distribution function of its elements
can be expressed in terms of the Wishart PDF and the
auto-correlations. We show this in appendix
\ref{app:pdf_gaussian1}. In the rest of this section, we derive an
approximate expression for the asymptotic distribution of these matrix
elements.

In the following we denote the diagonal elements as $C_{ii}$ and the
non-diagonal elements as $C_{ij}$ ($i \neq j$). We assume that
$\var(a_{i,t}) = \sigma^2$ for any $i$ and $t$, and $a_{i,t}$ are not
autocorrelated, i.e. $\text{corr}(a_{i,t}, a_{i, t'}) = 0$ for any
$i$, $t$ and $t'$. Then we can write
\begin{eqnarray*}
  r_{i,t} r_{j, t} &=& \left[
    \phi_1 r_{i, t-1} + a_{i, t}
  \right] \left[
    \phi_1 r_{j, t-1} + a_{j, t}
  \right] \\
  C_{ij} &=& {1 \over T}\sum_{t=1}^T r_{i,t} r_{j, t} \\
  &=& \phi_1^2 {1 \over T}\sum_{t=1}^Tr_{i,t-1} r_{j, t-1} +
  \phi_1{1 \over T}\sum_{t=1}^T\left(
    r_{i, t-1} a_{j, t} + r_{j, t-1} a_{i, t}
  \right) + {1 \over T}\sum_{t=1}^T a_{i,t} a_{j,t}
\end{eqnarray*}
We note that the two sums $\sum_{t=1}^T r_{i,t} r_{j, t}$ and
$\sum_{t=1}^T r_{i,t-1} r_{j, t-1}$ only differ by the first and the
last addend, which is negligible for sufficiently large T. Thus we
have
\begin{equation}\label{eq:Offdiag1}
  \begin{aligned}
    (1 - \phi_1^2)C_{ij} &\approx& \phi_1 {1 \over T} \sum_{t=1}^T
    \left(r_{i, t-1} a_{j, t} + r_{j, t-1} a_{i, t} \right)
    + {1 \over T}\sum_{t=1}^T a_{i,t} a_{j,t}
  \end{aligned}
\end{equation}
Now we write the AR(1) process $r_{i,t}$ as an infinite moving-average
process:
\begin{eqnarray*}
  r_{i, t} &=& \phi_1 r_{i, t-1} + a_{i,t} \\
  (1 - \phi_1 B) r_{i, t} &=& a_{i,t} \\
\end{eqnarray*}
where $B$ is the back-shift operator. Then it follows from the above
equation
\begin{eqnarray*}
  r_{i,t} &=& {1 \over 1 - \phi_1 B} a_{i,t} \\
  &=& \sum_{k=0}^\infty \phi_1^k B^k a_{i,t} \\
  &=& \sum_{k=0}^\infty \phi_1^k a_{i,t-k} \\
\end{eqnarray*}
where it is left implicit that $a_{i, t}$ with $t \leq 0$ is zero (in
words, this implies that the $r_{i,t}$ process is not affected at all
by events before $t = 1$).

Substituting this into eq.\ref{eq:Offdiag1} for $r_{i,t-1}$ yields
\begin{equation}
  \label{eq:Offdiag2}
  \begin{aligned}
  (1 - \phi_1^2)C_{ij} &\approx
  {1 \over T} \sum_{t=1}^T \sum_{k=0}^\infty\phi_1^{k+1} (a_{i, t-k-1}
  a_{j, t} + a_{j, t-k-1} a_{i, t})
  + {1 \over T} \sum_{t=1}^T a_{i,t} a_{j,t} \\
  &=
  {1 \over T} \sum_{t=1}^{T} \sum_{k=1}^{t-1}\phi_1^k (a_{i, t-k}
  a_{j, t} + a_{j, t-k} a_{i, t})
  + {1 \over T} \sum_{t=1}^{T} a_{i,t} a_{j,t}\\
  \end{aligned}
\end{equation}

Given two Gaussian random variables $x$ and $y$ with zero mean and
covariance matrix 
\begin{equation*}
  \Sigma =
  \begin{pmatrix}
    1 & \rho \\
    \rho & 1 \\
  \end{pmatrix}
\end{equation*}

The PDF of $xy$ can be found by considering $P(xy < z)$:
\begin{eqnarray*}
  P(xy < z) &=& \left(\int_0^\infty dx \int_0^{z/x} dy
    +\int_{-\infty}^0 dx \int_{z/x}^\infty dy \right)
    {1 \over 2\pi \sqrt{1 - \rho^2}} \\
    &&
    \exp\left[
      -{x^2 -2\rho xy + y^2
        \over
        2\sigma^2 (1 - \rho^2)} 
    \right] \\
  f(z; \sigma, \rho) &=& {d \over dz} P(xy < z)\\
  &=& {1 \over \pi \sqrt{1 - \rho^2}} \exp\left[
    {\rho z \over \sigma^2 (1 - \rho^2)}\right] K_0\left[
    {|z| \over \sigma^2 (1 - \rho^2)}
  \right]
\end{eqnarray*}
where $K_n(z')$ is the modified Bessel function of the second
kind. It is worth taking note that, when $\rho \neq 0$,
$f(z; \sigma, \rho)$ is not symmetric with respect to
$z$. As a result, if $\rho > 0$, the mean of $f(z;
\sigma, \rho)$ is positive, and vice versa.

Secondly, because $|\rho| < 1$ and
\begin{equation*}
  K_0(z) \sim \sqrt{\pi \over 2z} e^{-z}
\end{equation*}
as $z \to \infty$ \cite{Olver:2010:NHMF}, all the moments of
$f(z; \sigma, \rho)$ are finite, implying the
applicability of the Lyapunov central limit theorem provided that $T$
is large, which is very often the case and what we assume here.

With this in mind, we observe that $\phi_1$ only affects the first sum in
\ref{eq:Offdiag2}. If $a_{i,t-k}$ and $a_{j,t}$ are not correlated for
non-zero $k$, $\phi_1$ will not affect the mean of
$C_{ij}$. Furthermore, it is also clear from equation
\ref{eq:Offdiag2} that the variance of $C_{ij}$ is
always increased by a non-zero $\phi_1$, regardless of the sign of
$\phi_1$. We compute this increment in the following.

In light of the above expression for $f(z; \sigma,
\rho)$, we rewrite equation \ref{eq:Offdiag2} as
\begin{equation}\label{eq:Cij_dist}
  \begin{aligned}
    (1 - \phi_1^2)C_{ij} \approx &
    {1 \over T}
    \sum_{t=1}^{T} \sum_{k=1}^{t-1}\phi_1^k (a_{i, t-k} a_{j, t} + a_{j,
      t-k} a_{i, t}) \\
    & + {1 \over T} \sum_{t=1}^{T} \sigma^2 (1 - \rho^2) {a_{i,t} \over
      \sigma \sqrt{1 - \rho^2}} {a_{j,t} \over \sigma \sqrt{1 - \rho^2}} \\
  \end{aligned}
\end{equation}
In addition, we assume
\begin{equation*}
  \text{corr}(a_{i, t-k}, a_{j, t}) = \left\{
    \begin{array}{l l}
      1 & \text{ if $i=j$ and $k = 0$ }\\
      \rho & -1 < \rho < 1. \text{ if $i \neq j$ and $k = 0$ }\\
      0 & \text{otherwise}
    \end{array}
  \right.
\end{equation*}
Then, because $a_{i, t-k}$ and $ a_{j, t}$ with $i \neq j$ and $k
> 0$ are not correlated, the mean of $a_{i, t-k} a_{j, t}$ is 0
($f(z; \sigma, 0)$ is symmetric), and the variance of it
can be found to be $\sigma^6$ using formula (10.43.19) of \cite{NIST:DLMF}:
\begin{equation*}
  \int_0^\infty dt K_\nu(t) t^{\mu-1} = 2^{\mu-2}
  \Gamma\left(
    {\mu + \nu \over 2}
  \right) \Gamma\left(
    {\mu - \nu \over 2}
  \right)
\end{equation*}
On the other hand, $a_{i,t}/\sigma \sqrt{1 - \rho^2}$ and
$a_{j,t}/\sigma \sqrt{1 - \rho^2}$ have variance $1/(1 - \rho^2)$
and are correlated - $\text{corr}(a_{i,t}, a_{j,t}) = \rho$. The mean
of $a_{i,t}a_{j,t}/\sigma^2 (1 - \rho^2)$ can be found using formula
(10.43.22) of \cite{NIST:DLMF}, given that $-1 < \rho < 1$:
\begin{eqnarray*}
  \int_0^\infty t^{\mu - 1} e^{-at} K_\nu(t) dt &=& (\pi/2)^{1/2}
  \Gamma(\mu + \nu) \Gamma(\mu - \nu)(1 - a^2)^{-\mu/2 + 1/4} \times\\
  && P^{-\mu+1/2}_{\nu-1/2} (a)
\end{eqnarray*}
where $P^\mu_\nu(\cdot)$ is Ferrers function of the first
kind\footnote{ Ferrers function of the first kind is defined through
  the hypergeometric functon $F(a, b; c; z)$ \cite{NIST:DLMF}:
  \begin{equation}
    \label{eq:Ferrers_1st}
    \mathop{\mathsf{P}^{\mu}_{\nu}\/}\nolimits\!\left(x\right)=\left(\frac{1+x}{1-
        x}\right)^{\mu/2}\mathop{\mathbf{F}\/}\nolimits\!\left(\nu+1,-\nu;1-\mu;\tfrac
      {1}{2}-\tfrac{1}{2}x\right).
  \end{equation}
}. The result is
\begin{eqnarray*}
  \E\left[{a_{i,t}a_{j,t} \over \sigma^2 (1 - \rho^2)}\right]
  &=&
  {1 \over \sqrt{2\pi} (1 - \rho^2)^{5/4}} \left[
    P^{-3/2}_{-1/2}(-\rho) - P^{-3/2}_{-1/2}(\rho)
  \right]
\end{eqnarray*}
Similarly, the variance of $a_{i,t}a_{j,t}/\sigma^2 (1 - \rho^2)$ is
found to be
\begin{eqnarray*}
  v^2(\rho) &=&
  {4 \over \sqrt{2\pi} (1 - \rho^2)^{7/4}} \left[
    P^{-5/2}_{-1/2}(\rho) + P^{-5/2}_{-1/2}(-\rho)
  \right] \\
  && - \E^2\left[{a_{i,t}a_{j,t} \over
      \sigma^2 (1 - \rho^2)}\right]
\end{eqnarray*}
Now we can apply the Lyapunov central limit theorem
\cite{Billingsley1995} to the sum in equation \ref{eq:Cij_dist}
and write down the asymptotic Gaussian distribution of $C_{ij}$:
\begin{equation*}
C_{ij} \sim N(\mu'_X, \sigma'_X)
\end{equation*}
where
\begin{eqnarray}
  \mu'_X &=& {\sigma^2 \over \sqrt{2\pi} (1 - \phi_1^2)(1 -
    \rho^2)^{1/4}} \left[ P^{-3/2}_{-1/2}(-\rho) -
    P^{-3/2}_{-1/2}(\rho)
  \right] \label{eq:gaussian_mean}\\
  \sigma'^2_X &=& {1 \over (1 - \phi_1^2)^2}\left[
    \sum_{t=1}^T \sum_{k=1}^{t-1} 2\left(
      \phi_1^k \over T
    \right)^2 \sigma^6 + \sum_{t=1}^T
    {\sigma^4 (1 - \rho^2)^2 \over T^2} v^2(\rho)
  \right] \nonumber \\
  &=& {2 \sigma^6 \over T (1 - \phi_1^2)^2} \left[
    {\phi_1^2 \over 1 - \phi_1^2} -
    {\phi_1^2 (1 - \phi_1^{2T}) \over
      T(1 - \phi_1^2)}
  \right] + {\sigma^4 (1 - \rho^2)^2 v^2(\rho) \over
    T (1 - \phi_1^2)^2} \nonumber \\
  &\approx& {2 \sigma^6 \phi_1^2 \over T (1 - \phi_1^2)^3}
  + {\sigma^4 (1 - \rho^2)^2 v^2(\rho) \over
    T (1 - \phi_1^2)^2} \label{eq:gaussian_variance}
\end{eqnarray}
Equation \ref{eq:gaussian_mean} tells that, if two return series $i$
and $j$ are not correlated, auto-correlation in the returns does not
introduce a bias into the estimation of the cross-correlation; if,
however, the return series are indeed correlated, auto-correlation
in the returns rescales the cross-correlation through a multiplicative
factor $1/(1 - \phi_1^2)$.

In addition, equation \ref{eq:gaussian_variance} tells that
auto-correlation in the returns always makes the cross-correlation
estimation more noisy. Auto-correlation not only rescales the variance
of the no-autocorrelation estimation by $1/(1 - \phi_1^2)$ but even
adds an extra term ${2 \sigma^6 \phi_1^2 \over T (1 - \phi_1^2)^3}$.

% What we have obtained is a quantitative description of what has been
% said before: The mean of the non-diagonal entries of the
% cross-correlation matrix is not affected by autocorrelations in the
% returns, given that innovations $a_{i,t}$ of different assets and
% different times are un-correlated; the variance, however, is always
% increased by autocorrelations no matter the autocorrelation is
% positive or nagative. The increment is in the form of an additive term
% that scales as $\phi_1^2 / (1 - \phi_1^2)^3$.

We may apply a similar treatment to the diagonal elements of the
covariance matrix, which we denote as $C_{ii}$ here:
\begin{eqnarray*}
  C_{ii} &=& {1 \over T} \sum_{t=1}^T r^2_{it} \\
  &=& {1 \over T} \sum_{t=1}^T \sum_{l=0}^{t-1}\phi_1^l a_{i, t-l}
  \sum_{k=0}^{t-1}\phi_1^k a_{i, t-k} \\
  &=& {1 \over T} \sum_{t=1}^T \left[
    \sum_{k=0}^{t-1} \phi_1^{2k} a^2_{i, t-k} +
    \sum_{k,l = 0}^{t-1} \phi_1^{k+l} a_{i, t-k} a_{i, t-l}
  \right]
\end{eqnarray*}
By the Lyapunov central limit theorem under the assumption of large T,
the asymptotic distribution of $C_{ii}$ is Gaussian, the mean and
variance being
\begin{eqnarray}
  \E(C_{ii}) &=& {1 \over T}\left[
    \sum_{k=0}^{t-1} \phi_1^{2k} \sigma^2
  \right] \nonumber \\
  &=& {\sigma^2 \over (1 - \phi_1^2) T} \left[
    T - {\phi_1^2 (1 - \phi_1^{2T}) \over 1 - \phi_1^2}
  \right] \nonumber \\
  & \approx & {\sigma^2 \over 1 - \phi_1^2} \left[
    1 - {\phi_1^2 \over T}
  \right] \label{eq:gaussian_cii_mean}
\end{eqnarray}
and
\begin{eqnarray}
  \var(C_{ii}) &=& \sum_{t=1}^T \left[
    \sum_{k=0}^{t-1} {\phi_1^{4k} \sigma^4 \over T^2} 2 +
    \sum_{k,l=0}^{t-1} {\phi_1^{2(k+l)} \over T^2} \sigma^6
  \right] \nonumber \\
  &=& \sum_{t=1}^T \left[
    {2 \sigma^4 \over T^2} {1 - \phi_1^{4t} \over 1 - \phi_1^4} +
    {\sigma^6 \over T^2} \left(
      {1 - \phi_1^{2t} \over 1- \phi_1^2}
    \right)^2 \right] \nonumber \\
  &=& {2 \sigma^4 \over T (1 - \phi_1^4)} -
  {2 \sigma^4 \phi_1^4 (1 - \phi_1^{4T}) \over T^2(1 - \phi_1^4)^2} +
  \nonumber \\
  && {\sigma^6 \over T (1 -\phi_1^2)^2} -
  {2 \sigma^6 \phi_1^2 (1 - \phi_1^{2T}) \over T^2 (1 - \phi_1^2)^3} +
  {\sigma^6 \phi_1^4 (1 - \phi_1^{4T}) \over T^2 (1 - \phi_1^2)^2 (1 -
    \phi_1^4)} \nonumber \\
  &\approx& {2 \sigma^4 \over T (1 - \phi_1^4)} + {\sigma^6 \over T (1
    -\phi_1^2)^2} \label{eq:gaussian_cii_variance}
\end{eqnarray}
From equation \ref{eq:gaussian_cii_mean} we see that auto-correlation
in the returns increases the variance of the return series; and from
equation \ref{eq:gaussian_cii_variance} we see that the variance of
that variance estimation is also increased by
auto-correlations. Moreover, we note that $\var(C_{ii})$ scales with T
approximately as $1/T$, similar to the behavior of
$\var(C_{ij})$. This is to be compared with the case of GARCH returns
discussed in chapter \ref{chp:CrossCorrelationFat}.

% As for the diagonal elements of the cross-correlation matrix,
% i.e. $\var(r_{it})$, from the representation of $r_{it}$ as an
% inifinite moving average process, one can deduce immediately
% \begin{equation*}
%   \text{var}(r_{i,t}) \approx {\sigma^2 \over 1 - \phi^2}
% \end{equation*}
% as is mentioned earlier in equation \ref{eq:gaussian_Cii_variance}.

% However, to write $f_A(RMM'R')$ as a function of the eigen
% values of $RMM'R'$ is very involved and no theoretical results are
% known.

% In the more general case
% \begin{eqnarray*}
%   a_{i,t} &=& \left(
%     1 - \sum_{k=1}^p \phi_{i,k} B^k
%   \right) r_{i,t} \\
%   &=& \Phi_i(B)} r_{i,t
% \end{eqnarray*}
% where $B$ denotes the back shift operator acting on subscript $t$, we
% can write
% \begin{eqnarray*}
%   A &=&
%   \begin{pmatrix}
%     \Phi_1(B) & &\\
%     & \ddots & \\
%     && \Phi_N(B)
%   \end{pmatrix} R \\
%   &=& \Phi(B) R
% \end{eqnarray*}

% Following the same reasoning, $RR' \sim W(\Phi^{-1}(B)
% \Sigma \Phi'^{-1}(B), T)$. Substituting $\Phi^{-1}(B)
% \Sigma \Phi'^{-1}(B)$ for $\Sigma$ in the Wishart
% probability density function, we get
% \begin{eqnarray*}
%   && \tr\left\{\left[\Phi^{-1}(B)\Sigma \Phi'^{-1}(B)\right]^{-1}
%     RR'\right\}\\
%   &=& \tr \left\{\Phi'(B)\Sigma^{-1} \Phi(B) RR'\right\}\\
%   &=& \tr\left\{\Sigma^{-1}AA'\right\}\\
% \end{eqnarray*}
% As to $\det \Phi^{-1}(B)\Sigma \Phi'^{-1}(B)$, we note that $\Sigma$
% is constant over time, and thus a function of $B$ has no effect on
% it. Therefore $\det \Phi^{-1}(B)\Sigma \Phi'^{-1}(B) = \det \Sigma$.

\section{Distribution of the Eigenvalues}\label{sec:GCC-numerical}
For the Wishart matrix $\mtx{RR'}$, theoretical results are available
for the eigenvalue distribution, and we summarize them in appendix
\ref{sec:wishart_eigen_dist}. In short, the joint probability density
function of the eigenvalues is given by equation
\ref{eq:wishart_eigen_pdf} when neither auto-correlation nor
cross-correlation is present in $\mtx R$. Moreover, the mariginal
distribution of the largest eigenvalue is approximated by a gamma
distribution \cite{Chiani2012}.

However, as detailed in the derivation leading to equation
\ref{eq:cross-corr-matrix-PDF}, the distribution of $RR'$ is not
Wishart when the columns of $R$ are correlated
(auto-correlation). Deriving the eigenvalue distribution analytically
in this case is beyond the scope of this thesis. Instead, we resort to
numerical methods.

As before we consider the AR(1) process:
\begin{equation*}
  \vec{r}_t = \phi \vec{r}_{t-1} + \vec{a}_{t}
\end{equation*}
where $\vec{a}_t \sim N(0, I)$, i.e. the elements of $\vec{a}_t$ are
independent Gaussian random variable with zero mean and unit
variance.

Now we investigate how the eigenvalue distribution depends on
$\phi$. Figure \ref{fig:GaussianMarkovSpectrumPDF} shows the results
of the simulation. 
It is clear from the figure that the maximum eigenvalue moves
consistently to the right as the value of $\phi$ increases, and, as
shown in figure \ref{fig:Gaussian_mineig}, the minimum eigenvalue
also increases with $\phi$.

% In the following we focus on the maximum eigenvalue, which is shown by
% Chiani to have approximately a gamma distribution \cite{Chiani2012}.
% So we first verify Chiani's result, and then extend it to the
% case {\it with} autocorrelations and study how the
% distribution changes as $\phi$ increases.
Chiani showed that the maximum eigenvalue ($\lambda_1$) had
approximately a gamma distribution \cite{Chiani2012}. So we compare in
figure \ref{fig:GaussianMarkov005MaxEigCDF_loglog} the empirical
cumulative distribution function (CDF) of the maximum eigenvalue 
with the CDF of a gamma distribution. The cases where
auto-correlations are present ($\phi > 0$) have also been included.

It is seen in the figure that gamma distributions with different
parameters fit fairly well in all cases. So we conclude that a
gamma distribution not only approximates the maximum eigenvalue
distribution at the absence of autocorrelations but does so even at
the {\it presence} of autocorrelations.

Since the maximum eigenvalue distribution is characterized by the
parameters $k$, $\theta$, and $\alpha$, the influence of the
autocorrelations can be characterized by the dependence of $k$,
$\theta$, and  $\alpha$ on $\phi$. While these dependences are rather
intricate,
% Figure \ref{fig:param1_phi_dep} shows how the mean, variance,
% and skewness of the fitted gamma distribution, as well as the
% relocation parameter $\alpha$ denpend on $\phi$.
% \begin{figure}[htb!]
%   \centering
%     \includegraphics[scale=0.5, clip=true, trim=37 240 50
%     202]{../pics/param1_phi_dep.pdf}
%   \caption{\small \it  the  mean, variance, and skewness of the fitted
%     gamma distribution as well as the relocation parameter $\alpha$
%     are plotted against the autocorrelation strength $\phi$. 20 values
%     of $\phi$ are included in the plot, ranging from 0 to 0.95 with
%     step size 0.05.}
%   \label{fig:param1_phi_dep}
% \end{figure}
% Despite the monotonic behavior shown by 3 of these curves, the
% relation between each of the aforementioned quantities and $\phi$ is
% more complicated than it apears. However, the mean, i.e. $k \theta$ is
% an exception: 
good support can be found in the data for the following
approximate relation:
\begin{eqnarray}
  k\theta &=& a \tan^2{\pi \phi \over 2} + b\tan{\pi \phi \over 2} +
  c \label{eq:k_theta-phi}
\end{eqnarray}
To verify this relation, we first fit a 2nd order polynomial and
obtain the coefficients $a$, $b$, $c$; then for each data point
$k_n\theta_n$ we solve the quadratic equation
\begin{eqnarray}
  a \tan^2{\pi \phi'_n \over 2} + b\tan{\pi \phi'_n \over 2} + c -
  k_n\theta_n &=& 0\label{eq:k_theta-phi_2}
\end{eqnarray}
for $\tan{\pi \phi'_n \over 2}$. If relation \ref{eq:k_theta-phi} is a
good approximation, a close match between $\tan{\pi \phi'_n \over 2}$
and $\tan{\pi \phi_n \over 2}$ is expected. Figure
\ref{fig:phi_and_roots} plots $\tan{\pi \phi'_n \over 2}$ against
$\tan{\pi \phi_n \over 2}$ and fits a straight line to the data
points. It is seen that the data points lie fairly close to the fitted
line and the line has a slope very close to 1 and an intercept close to
0. This strongly suggests $\tan{\pi \phi'_n \over 2} = \tan{\pi \phi_n
  \over 2}$ and supports the relation \ref{eq:k_theta-phi}.
\begin{figure}[htb!]
  \centering
    \includegraphics[scale=0.5, clip=true, trim=37 217 39
    170]{../pics/phi_and_roots.pdf}
  \caption{\small \it Upper plot: $\tan{\pi \phi'_n \over 2}$ against
    $\tan{\pi \phi_n \over 2}$. The fitted line has equation $y_n =
    0.995 \tan{\pi \phi_n \over 2} + 0.0146$. Lower plot: Residuals of
    the linear fit, i.e. $\tan{\pi \phi'_n \over 2} - y_n$. 20 values
    of $\phi$ are included in the plot, ranging from 0 to 0.95 with
    step size 0.05.}
  \label{fig:phi_and_roots}
\end{figure}

% Figure \ref{fig:GaussianMarkovMaxEig_k-phi} shows how the mean of the
% fitted gamma distribution, i.e. $k\theta$, varies as the
% autocorrelation strengthens ($\phi$ increases). 
% \begin{figure}[htb!]
%   %\vspace{-10mm}
%   \centering
%     \includegraphics[scale=0.5, clip=true, trim=100 223 112
%     141]{../pics/GaussianMarkov05MaxEig_k-phi.pdf}
%     \caption{\small \it The parameter $k$ against autocorrelation strength
%       $\phi$. Blue crosses: Empirical values of $k$. Red: best fitting
%       line in terms of {\it Least Square Errors}.}
%   \label{fig:GaussianMarkovMaxEig_k-phi}
% \end{figure}
% From the equation of the fitting line we can directly read out
% \begin{eqnarray*}
%   k &=& a\phi + b \\
%   &=& a\left(1 \over 2\right)^{1/\tau} + b
% \end{eqnarray*}
% where $a = -26$ and $b = 47$.

% For the parameter $\theta$, its behavior is more conveniently
% described in terms of the correlation time $\tau$. Figure
% \ref{fig:GaussianMarkovMaxEig_theta-tau} plots the values of $\theta$
% against those of $\tau$ together with a fitting quadratic
% function. Higher order polynomials provide a slightly better
% fit, but coefficients of the 3rd order and above are less than 1/1000
% times the coefficients of the 2nd and the 1st order. Therefore a 2nd
% order polynomial has been chosen.
% \begin{figure}[htb!]
%   \vspace{-15mm}
%   \centering
%   \includegraphics[scale=0.5, clip=true, trim=99 230 114
%   139]{../pics/GaussianMarkov05MaxEig_theta-tau.pdf} 
%   \caption{\small \it Vertical axis: $\theta$; Horizontal axis:
%     correlation time $\tau$. Blue: empirical values of $\theta$. Cyan:
%     Fitting quadratic function.}
%   \label{fig:GaussianMarkovMaxEig_theta-tau}
% \end{figure}
% The equation of this polynomial is
% \begin{eqnarray*}
%   \theta &=& A\tau^2 + B\tau + C
% \end{eqnarray*}
% where $A = 2.1\times 10^{-4}$, $B = -1.1\times 10^{-4}$, $C =
% 3.2\times 10^{-4}$.

% The parameter $\alpha$ shifts the gamma distribution $\mathscr{G}(k,
% \theta)$ to match the mean of ${(\lambda_1 -
%   \mu_{NT})/\sigma_{NT}}$. The behavior of $\alpha$ with respect to
% changing autocorrelation is shown in figure
% \ref{fig:GaussianMarkovMaxEig_alpha-tau}.
% \begin{figure}[htb!]
%   \vspace{-15mm}
%   \centering
%   \includegraphics[scale=0.5, clip=true, trim=104 229 114
%   139]{../pics/GaussianMarkovMaxEig_alpha-tau.pdf}
%   \caption{\small \it The mean-shift parameter $\alpha$ against
%     correlation time $\tau$. Blue: empirical values of $\alpha$. Cyan:
%     quadratic fit.}
%   \label{fig:GaussianMarkovMaxEig_alpha-tau}
% \end{figure}
% The equation of $\alpha$ is then inferred from the fitting line:
% \begin{equation*}
%   \alpha = D\tau^2 + E\tau + F
% \end{equation*}
% where $D = -3.4\times 10^{-3}$, $E = -4.6\times 10^{-2}$ and $F = 69$.

% So combining the results of $\alpha$, $k$ and $\theta$ we can express
% the moments of the Tracy-Widom variate ${(\lambda_1 -
%   \mu_{NT})/\sigma_{NT}}$:
% \begin{equation}\label{eq:lambda1_MeanVariance}
%   \begin{aligned}
%     \E({\lambda_1 - \mu_{NT} \over \sigma_{NT}}) &= k\theta -
%     \alpha \\
%     &= \left[-a\left(1 \over 2\right)^{1/\tau} + b\right](A\tau^2 +
%     B\tau - C) - (D\tau^2 + E\tau + F) \\
%     \var({\lambda_1 - \mu_{NT} \over \sigma_{NT}}) &=
%     k\theta^2 \\
%     &= \left[-a\left(1 \over 2\right)^{1/\tau} + b\right](A\tau^2 +
%     B\tau - C)^2 \\
%   \end{aligned}
% \end{equation}
% Figure \ref{fig:GaussianMarkovMaxEig_tw_moments} shows the empirical
% moments of ${(\lambda_1 - \mu_{NT})/\sigma_{NT}}$ together with the
% corresponding values computed using the above formulas.
% \begin{figure}[htb!]
%   \vspace{-15mm}
%   \centering
%   \includegraphics[scale=0.46, clip=true, trim=48 243 6
%   134]{../pics/GaussianMarkovMaxEig_tw_moments.pdf}
%   \caption{\small \it empirical moments of ${(\lambda_1 -
%       \mu_{NT})/\sigma_{NT}}$ against their theoretical
%     counterparts. Left: Empirical/theoretical mean against $\tau$;
%     Right: Empirical/theoretical variance against $\tau$. Blue
%     crosses: empirical values; Red line: fitting curves. Horizontal
%     axis: correlation time $\tau$.}
% \label{fig:GaussianMarkovMaxEig_tw_moments}
% \end{figure}
% The good fitness shown in figure
% \ref{fig:GaussianMarkovMaxEig_tw_moments} allows us to conclude 
% that, within the range of the correlation time that we have studied,
% namely $\tau \in [0, 13.51]$, the $k$ parameter is a linear function
% of $\phi = 2^{-1/\tau}$ while $\theta$ and $\alpha$ are quadratic
% functions of $\tau$. The moments of the transformed maximum
% eigenvalue, namely ${(\lambda_1 - \mu_{NT})/\sigma_{NT}}$, are thus
% expressed as functions of the correlation time $\tau$ via the
% parameters $k$, $\theta$ and $\alpha$.

% Figure \ref{fig:GaussianMarkovMaxEigPDF_original} shows the PDF of the
% maximum eigenvalue ($\lambda_1$) for a range of values of the
% correlation time $\tau$. One can clearly see that the mean of the
% distribution moves to the right and the width of the distribution
% increases as $\tau$ takes on larger and larger values. However, one
% must not forget that the behavior of the mean and the width is random
% in nature. Eq. \ref{eq:lambda1_MeanVariance} describes such behavior in
% the asymptotic limit, i.e. if an infinite number of random matrices
% are generated and their respective maximum eigenvalues computed for
% each and every value of correlation time $\tau$.
% \begin{figure}[htb!]
%   \begin{center}
%     \includegraphics[scale=0.5, clip=true, trim=98 228 112
%     221]{../pics/GaussianMarkovMaxEigPDF_original.pdf}
%     \caption{\small \it Probability density function of the maximum eigenvalue
%       ($\lambda_1$).}
%   \end{center}
%   \label{fig:GaussianMarkovMaxEigPDF_original}
% \end{figure}

