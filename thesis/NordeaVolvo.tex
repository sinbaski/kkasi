\chapter{Nordea in 2013/14}
\label{chp:Nordea2}
In this chapter we model the volatility of {\it Nordea Bank}
15-minute and 30-minute returns during the period 2013/10/10 -
2014/04/04.

\section{15-minute Returns}\label{sec:nordea2_15min}
This return series contains 3879 data points. The squared
volatilities, i.e. the variances, are computed as the sum of squared
50-second returns. The normal probability plot of the quotient $(r_t - \E(r_t))/
\sigma_t$, is shown in figure \ref{fig:nordea2_b}. We see in this figure that
$(r_t - \E(r_t)) / \sigma_t$ is approximately Gaussian, confirming the choice of
50-second returns for the computation of the realized
volatility. Among the 3879 data points, we use the first 80\% for
model estimation and the last 20\%, i.e. 776 data points for forecast
verification.

\begin{figure}[htb!]
  \centering
  \includegraphics[scale=0.4, clip=true, trim=79 259 108
  121]{../pics/nordea2_b.pdf}
  \caption{\small \it Normal probability plot of $(r_t - \E(r_t))/
    \sigma_t$. $\sigma_t^2$ is computed as the sum of squared
    50-second returns.}
  \label{fig:nordea2_b}
\end{figure}

The log-volatility series $\ln \sigma_t$ in this case has an
auto-correlation structure similar to its predecessor studied in
chapter \ref{chp:nordea_15min}. The seasonality is 33 as before. We
show the auto-correlations of $\ln \sigma_t$ in figure
\ref{fig:nordea2_15_lv_acf}, and those of $w_t = (1 - B)(1-B^s)\ln
\sigma_t$ in figure \ref{fig:nordea2_15_w_acf}. The reader may want to
compare them with figure \ref{fig:nordea1_15min_acf}.

\begin{figure}[htb!]
  \centering
  \subfigure[]{
  \includegraphics[scale=0.4, clip=true, trim=86 254 98
  123]{../pics/nordea2_15min_lv_acf.pdf}
  \label{fig:nordea2_15_lv_acf}
  }
  \subfigure[]{
  \includegraphics[scale=0.4, clip=true, trim=95 230 112
  235]{../pics/nordea2_w_acf.pdf}
  \label{fig:nordea2_15_w_acf}
  }
  \caption{\small \it Auto-correlation function of $\ln \sigma_t$ and
    $w_t = (1 - B)(1-B^s)\ln \sigma_t$} 
\end{figure}
The significant auto-correlations of $w_t$ and their respective lags are listed
in table \ref{tab:nordea2_w_acf}.
\begin{table}[htb!]
  \centering
  \begin{tabular}{|c|c|c|c|c|}
    \hline
    lag & 1 & 32 & 33 & 34 \\
    \hline
    ACF &   -0.4805 & 0.2861 &  -0.5034 & 0.2093 \\
    \hline
  \end{tabular}
  \caption{\small \it Significant auto-correlations of $w_t$}
  \label{tab:nordea2_w_acf}
\end{table}
Although the log-volatilities have the seasonality $s = 33$, judging
from the relatively big differences between the auto-correlations at
lags 32 and 34, we don't expect the moving average process to be in the
multiplicative form of equation \ref{eq:nordea_w}. Instead, we model
$\ln\sigma_t$ as
\begin{eqnarray}
  (1 - B)(1 - B^s) \ln\sigma_t &=& (1 - \theta_1 B - \theta_s B^s -
  \theta_{s+1} B^{s+1} ) y_t \label{eq:nordea2_v}
\end{eqnarray}
where $y_t$ is assumed to be normally distributed, neglecting slight
non-normality. By maximum likelihood estimation, we fit the above
model to the realized volatilities. The resulting parameter values are
listed in table \ref{tab:nordea2_15_SV}.

\begin{table}[htb!]
  \centering
  \begin{tabular}{|c|c|c|c|c|}
    \hline
    parameter & $\theta_1$ & $\theta_{33}$ & $\theta_{34}$ & $\var(y_t)$ \\
    \hline
    value & 0.8339 & 0.8448 & -0.7067 & 0.10071 \\
    \hline
  \end{tabular}
  \caption{\small \it: Parameter values of $\ln \sigma_t$}
  \label{tab:nordea2_15_SV}
\end{table}

The auto-correlations of the residuals $y_t$ are insignificant, as
shown in figure \ref{fig:nordea2_y}. Though the Ljung-Box Q
test rejects the null hypothesis of no autocorrelation in the
residuals, we rather neglect these weak auto-correlations in favor of
a relatively simple model.

\begin{figure}[htb!]
  \centering
  \includegraphics[scale=0.4, clip=true, trim=86 256 101
  121]{../pics/nordea2_y_acf.pdf}
  \caption{\small \it Auto-correlations of the residuals ($y_t$).}
  \label{fig:nordea2_y}
\end{figure}

Forecast of $\ln\sigma_t$ is done by reverting equation
\ref{eq:nordea2_v} and write
\begin{eqnarray*}
  \ln\sigma_t &=& \ln\sigma_{t-1} + \ln\sigma_{t-s} -
  \ln\sigma_{t-s-1} + (1 - \theta_1 B - \theta_s B^s + \theta_{s+1}
  B^{s+1} ) y_t \\
  \E(\ln\sigma_t | \mathcal{F}_{t-1}) &=& \ln\sigma_{t-1} + \ln\sigma_{t-s} -
  \ln\sigma_{t-s-1} - \theta_1 y_{t-1} - \theta_s y_{t-s} +
  \theta_{s+1} y_{t-s-1}
\end{eqnarray*}

Thus the forecast, namely $\E(\ln\sigma_t | \mathcal{F}_{t-1})$, is
found in terms of previous values in the same series and of previous
residuals. To compare the performance, we also estimate a GARCH model with
Gaussian conditional distribution. The procedure is standard and we
only list the results here
\begin{table}[htb!]
  \centering
  \begin{tabular}{|c|c|c|c|c|}
    \hline
    Parameter & $\alpha_0$ & $\alpha_1$ & $\alpha_5$ & $\beta_1$ \\
    \hline
    Value & $7.3686 \times 10^{-7}$ & 0.154131 & 0.0360 & 0.491832 \\
    \hline
  \end{tabular}
  \caption{\small \it Parameters of GARCH(1, 5) model of Nordea
    Bank 15-minute returns}
  \label{tab:nordea2_garch}
\end{table}

Forecasting using both models, we get two series of one-step-ahead
forecast, $\ln \sigma^{\text{SV}}_t$ and $\ln \sigma^G_t$,
respectively. We denote the empirical log-volatilities as $\ln
\hat{\sigma}_t$. Then the accuracy of the two models can be compared
by comparing $\ln \sigma^{\text{SV}}_t - \ln \hat{\sigma}_t$ and
$\ln \sigma^G_t - \ln \hat{\sigma}_t$. As a reference, we also include
in the comparison the naive approach of taking the sample mean of
$\ln \sigma_t$ as the forecast.

Table \ref{tab:slv_garch_comparison} shows the mean and the standard
deviation of $\ln \sigma^F_t - \ln \hat{\sigma}_t$ for the 3 kinds of
forecast. Here we use $\ln \sigma^F_t$ to denote the forecast of $\ln
\sigma_t$. Clearly the SV model has the smallest mean and standard
deviation for $\ln \sigma^F_t - \ln \hat{\sigma}_t$, the sample mean
has the largest, and the GARCH model sits in between.
\begin{table}[htb!]
  \centering
  \begin{tabular}{|c|c|c|c|}
    \hline
    & SV & GARCH & Sample mean \\
    \hline
    $\E(\ln \sigma^F_t - \ln \hat{\sigma}_t)$ & -0.0033 & 0.0356 &
    0.0544 \\
    \hline
    $\text{std}(\ln \sigma^F_t - \ln \hat{\sigma}_t)$ &
    0.2876  &  0.3096 &   0.3349 \\
    \hline
  \end{tabular}
  \caption{\small \it Standard deviation of $\ln\sigma^F_t -
    \ln\hat{\sigma}_t$}
  \label{tab:slv_garch_comparison}
\end{table}
Even more informative about the quality of forecast is the
distribution of $\ln \sigma^F_t - \ln \hat{\sigma}_t$, which we show
in figure \ref{fig:slv_garch_comparison}.
\begin{figure}[htb!]
  \centering
  \includegraphics[scale=0.5, clip=true, trim=0 292 0
  255]{../pics/slv_garch_comparison.pdf}
  \caption{\small \it Comparison of the forecasts from SV, GARCH,
    and sample mean. Blue: SV forecasts; Green: GARCH forecasts; Red:
    Sample mean forecasts.}
  \label{fig:slv_garch_comparison}
\end{figure}
The left plot in the figure shows the situation of under-estimates,
i.e. $\ln \sigma^F_t - \ln \hat{\sigma}_t < 0$. We see that, to the
far left, the blue line is below the green, which is in turn below the
red, meaning the SV forecast has the lowest probability of
under-estimate. The GARCH forecast comes second and the sample
mean forecast under-estimates most.

The same ranking of forecast quality is also seen on the right plot of
figure \ref{fig:slv_garch_comparison}, which shows the complementary
distribution of $\ln \sigma^F_t - \ln \hat{\sigma}_t$ for postive
values, i.e. over-estimates: The SV forecast has the lowest
probability of over-estimate; GARCH comes second and the sample mean
is most likely to over-estimate.

As another measure of the quality of forecast, the fraction of
``good'' forecasts that differ no more than a given percentage from
the measured realized volatilities is also computed. Table
\ref{tab:nordea_good_percentage} compares this fraction of good
forecasts from the 3 alternatives with respect to different
criteria of goodness.
\begin{table}[htb!]
  \centering
  \begin{tabular}{|c|c|c|c|}
    \hline
    ${|\ln \sigma^F_t - \ln \hat{\sigma}_t| \over |\ln
      \hat{\sigma}_t|}$ &
    SV & GARCH & Sample Mean \\
    \hline
    1\% & 0.1920 & 0.2010 & 0.1843 \\
    \hline
    5\% & 0.7887 & 0.7436 & 0.7229 \\
    \hline
    10\% & 0.9665 & 0.9601 & 0.9369 \\
    \hline
  \end{tabular}
  \caption{\small \it Fraction of ``good'' forecasts as defined by
    ${|\ln \sigma^F_t - \ln \hat{\sigma}_t| \over |\ln
      \hat{\sigma}_t|}$ being less than 1\%, 5\% and 10\%.}
  \label{tab:nordea_good_percentage}
\end{table}
Clearly, although SV and GARCH rank differently with respect to
different criteria of goodness, they both out-perform the sample
mean forecast.

\section{30-minute Returns}\label{sec:nordea2_30min}
In this section we study the volatility of Nordea Bank 30-minute
returns in the period 2013/10/10 - 2014/04/04. The realized
volatilities that approximate the volatilities of these 30-minute
returns are computed using 1-minute returns. As in the previous cases,
this choice of 1-minute returns is confirmed by the normality of the
quotient $(r_t - \E(r_t))/\sigma_t$, which is shown in figure
\ref{fig:nordea3_quotient}.
\begin{figure}[htb!]
  \centering
  \includegraphics[scale=0.4, clip=true, trim=78 255 108
  120]{../pics/nordea3_quotient.pdf}
  \caption{\small \it Normal probability plot of $(r_t -
    \E(r_t))/\sigma_t$}
  \label{fig:nordea3_quotient}
\end{figure}
The auto-correlations of $\ln \sigma_t$ is shown in figure
\ref{fig:nordea3_lv_acf}, where we see an apparent seasonality of
16. By differencing we get $w_t = (1-B)(1-B^s)\ln\sigma_t$ where
$s=16$. Its auto-correlations are shown in figure
\ref{fig:nordea3_w_acf}.
\begin{figure}[htb!]
  \centering
  \subfigure[]{
    \includegraphics[scale=0.4, clip=true, trim=86 256 101
    121]{../pics/nordea3_lv_acf.pdf}
    \label{fig:nordea3_lv_acf}
  }
  \subfigure[]{
    \includegraphics[scale=0.4, clip=true, trim=86 256 103
    122]{../pics/nordea3_w_acf.pdf}
    \label{fig:nordea3_w_acf}
  }
  \caption{\small \it \ref{fig:nordea3_lv_acf}: Auto-correlations
    (ACF) of $\ln \sigma_t$; \ref{fig:nordea3_w_acf}: Auto-correlations
    (ACF) of $w_t = (1-B)(1-B^s)\ln\sigma_t$.}
\end{figure}
Once again, this auto-correlation structure
points to a seasonal moving average model:
\begin{eqnarray*}
  w_t &=& (1-\theta B)(1-\Theta B^s) y_t
\end{eqnarray*}
where $y_t$ is assumed to have Gaussian distribution, neglecting
slight non-normality as before. Maximum likelihood estimation gives
the parameter values listed in table \ref{tab:nordea3_sv_param}.
\begin{table}[htb!]
  \centering
  \begin{tabular}{|c|c|c|c|}
    \hline
    Parameter & $\theta$ & $\Theta$ & $\var(y_t)$\\
    \hline
    Value & 0.7612 & 0.8036 & 0.0736\\
    \hline
  \end{tabular}
  \caption{\small \it Parameter values of the Seasonal Moving Average
    model}
  \label{tab:nordea3_sv_param}
\end{table}
To fit a GARCH model to the same series, we assume Gaussian
innovations (c.f. equation \ref{eq:garch_def}). The result is a
GARCH(1,1) model. Its parameters are listed in table
\ref{tab:nordea3_garch_param}.
\begin{table}[htb!]
  \centering
  \begin{tabular}{|c|c|c|c|}
    \hline
    Parameter & $\alpha_0$ & $\alpha_1$ & $\beta_1$ \\
    \hline
    Value & $2.3 \times 10^{-7}$ & 0.05 & 0.9\\
    \hline
  \end{tabular}
  \caption{\small \it Parameter values of GARCH(1,1) model fitted to
    Nordea 30-minute returns.}
  \label{tab:nordea3_garch_param}
\end{table}

As before, we compare the accuracies of the GARCH and the stochastic
volatility model by comparing their one-ste-ahead forecasts. We
compute the difference between their forecasts and the realized
volatilities and then look at the statistics of these difference
values. Firstly we show the mean and the standard deviation of the
differences in table \ref{tab:nordea3_diff_1}.
\begin{table}[htb!]
  \centering
  \begin{tabular}{|c|c|c|c|}
    \hline
     & SV & GARCH & Sample mean \\
     \hline
    $\E(\ln \sigma^F_t - \ln \hat{\sigma}_t)$ & -0.0047 & 0.0130 & 0.0584 \\
    \hline
     $\text{std}(\ln \sigma^F_t - \ln \hat{\sigma}_t)$ & 0.2602 &
     0.3194 & 0.3093 \\
    \hline
 \end{tabular}
  \caption{\small \it Mean and standard deviation of the 3 kinds of
    forecasts}
  \label{tab:nordea3_diff_1}
\end{table}
It is seen in this table that the SV forecasts have a mean whose
absolute value is just above 1/3 of that of the GARCH forecasts. The
standard deviation of the SV forecasts is also smaller than that of
GARCH. In addition, we check the distribution of
$\ln \sigma^F_t - \ln \hat{\sigma}_t$, and the percentage of ``good''
forecasts with respect to different criteria of goodness. These are
shown in figure \ref{fig:nordea3_diff} and table
\ref{tab:nordea3_diff_2}, respectively.
\begin{figure}[htb!]
  \centering
    \includegraphics[scale=0.54, clip=true, trim=21 312 0
    179]{../pics/nordea3_diff.pdf}
  \caption{\small \it Comparison of the forecasts from SV, GARCH,
    and sample mean. Blue: SV forecasts; Green: GARCH forecasts; Red:
    Sample mean forecasts.}
  \label{fig:nordea3_diff}
\end{figure}
\begin{table}[htb!]
  \centering
  \begin{tabular}{|c|c|c|c|}
    \hline
    ${|\ln \sigma^F_t - \ln \hat{\sigma}_t| \over |\ln
      \hat{\sigma}_t|}$ &
    SV & GARCH & Sample Mean \\
    \hline
    1\% & 0.1777 & 0.1777 & 0.1538 \\
    \hline
    5\% & 0.8090 & 0.7109 & 0.6923 \\
    \hline
    10\% & 0.9814 & 0.9310 & 0.9523 \\
    \hline
  \end{tabular}
  \caption{\small \it Percentage of ``good'' forecasts as defined by
    deviating nore more than 1\%, 5\% and 10\% from the measured
    realized volatility.}
  \label{tab:nordea3_diff_2}
\end{table}
It is clear that the SV forecasts are considerably more accurate.

\chapter{Volvo in 2013/14}\label{sec:volvo}
In this chapter we study the volatility of Volvo B during the period
2013/10/10 - 2014/04/04. Section \ref{sec:volvo_15} and section \ref{sec:volvo_30}
investigate the 15-minute and 30-minute returns respectively.

\section{15-minute Returns}\label{sec:volvo_15}
In this section we look at the volatility of Volvo 15-minute
returns. The subject of modeling is the realized volatility, which is
computed as the square root of the sum of squared 1-minute
returns. The normal probability plot of the quotient $(r_t -
\E(r_t))/\sigma_t$ is shown in figure \ref{fig:volvo_15_quotient},
from which one can see it is normally distributed, confirming the
choice of 1-minute returns for computing the realized volatility.
\begin{figure}[htb!]
  \centering
  \includegraphics[scale=0.4, clip=true, trim=83 259 110 
  220]{../pics/volvo_15_quotient.pdf}
  \caption{\small \it Normal probability plot of $(r_t -
    \E(r_t))/\sigma_t$}
  \label{fig:volvo_15_quotient}
\end{figure}
The auto-correlations of $\ln \sigma_t$ and $w_t = (1-B)(1-B^s) \ln
\sigma_t$ are plotted in figure \ref{fig:volvo_15_lv_acf} and
\ref{fig:volvo_15_w_acf}. The former clearly shows the seasonality
$s = 33$ in the auto-correlations of $\ln \sigma_t$.
\begin{figure}[htb!]
  \centering
  \subfigure[]{
  \includegraphics[scale=0.4, clip=true, trim=90 259 103
  220]{../pics/volvo_15_lv_acf.pdf}
  \label{fig:volvo_15_lv_acf}
  }
  \subfigure[]{
  \includegraphics[scale=0.4, clip=true, trim=90 259 103
  220]{../pics/volvo_15_w_acf.pdf}
  \label{fig:volvo_15_w_acf}
  }
  \caption{\small \it Left: auto-correlations of $\ln \sigma_t$;
    Right: auto-correlations of $w_t = (1-B)(1-B^s) \ln \sigma_t$.}
\end{figure}
Based on the auto-correlations of $w_t$, a seasonal moving average
model is estimated:
\begin{eqnarray*}
  (1-B)(1-B^s) \ln\sigma_t &=& (1-\theta_1B - \theta_2B^2 - \theta_3
  B^3)(1 - \Theta B^s)y_t
\end{eqnarray*}
By maximum likelihood estimation, the parameter values listed in table
\ref{tab:volvo_15_sv_param} are obtained.
\begin{table}[htb!]
  \centering
  \begin{tabular}{|c|c|c|c|c|c|}
  \hline
  Parameter & $\theta_1$ & $\theta_2$ & $\theta_3$ & $\Theta$ & 
 $\var(y_t)$ \\
 \hline
 Value & 0.7491 & 0.1170 & 0.0571 & 0.8646 & 0.1245 \\
  \hline
  \end{tabular}
  \caption{\small \it Parameter values of the stochastoc volatility
    (SV) model.}
  \label{tab:volvo_15_sv_param}
\end{table}

A GARCH model (c.f. equation \ref{eq:garch_def}) assuming Gaussian
innovations is also fitted to the same series. The parameter values
are listed in table \ref{tab:volvo_15_garch_param}:
\begin{table}[htb!]
  \centering
  \begin{tabular}{|c|c|c|c|c|c|c|}
  \hline
  Parameter & $\alpha_0$ & $\alpha_1$ & $\beta_{1}$ \\
  \hline
  Value & $6.54 \times 10^{-7}$ & 0.1565 & 0.6445\\
  \hline
  \end{tabular}
  \caption{\small \it Parameter values of the GARCH model.}
  \label{tab:volvo_15_garch_param}
\end{table}

Performing one-step-ahead forecasts using both models gives the series
$\ln\sigma^F_t - \ln\hat{\sigma}_t$, where $\ln\hat{\sigma}_t$ are the
measured realized volatilities. The mean and standard deviation of
$\ln\sigma^F_t - \ln\hat{\sigma}_t$ are shown in table
\ref{tab:volvo_15_stat}.
\begin{table}[htb!]
  \centering
  \begin{tabular}{|c|c|c|c|}
    \hline
     & SV & GARCH & Sample mean \\
     \hline
    $\E(\ln \sigma^F_t - \ln \hat{\sigma}_t)$ & 0.0035 & 0.0313 & -0.1448 \\
    \hline
     $\text{std}(\ln \sigma^F_t - \ln \hat{\sigma}_t)$ & 0.3356 &
     0.3894 & 0.3826 \\
    \hline
 \end{tabular}
  \caption{\small \it Mean and standard deviation of the 3 kinds of
    forecasts}
  \label{tab:volvo_15_stat}
\end{table}
We see that both the SV and the GARCH model give biased forecasts, but
the SV forecasts are only 1/10 as biased as those of GARCH (0.0035
vs. 0.0313). In addition, the standard deviation of the SV forecasts
is smaller too. These results are confirmed by the distribution of
$\ln \sigma^F_t - \ln \hat{\sigma}_t$ and the fraction of ``good''
forecasts, which are shown in figure \ref{fig:volvo_15_diff} and table
\ref{tab:volvo_15_diff}, respectively.
\begin{figure}[htb!]
  \centering
    \includegraphics[scale=0.54, clip=true, trim=27 276 0
    236]{../pics/volvo_15_diff.pdf}
  \caption{\small \it Comparison of the forecasts from SV, GARCH,
    and sample mean. Blue: SV forecasts; Green: GARCH forecasts; Red:
    Sample mean forecasts.}
  \label{fig:volvo_15_diff}
\end{figure}
\begin{table}[htb!]
  \centering
  \begin{tabular}{|c|c|c|c|}
    \hline
    ${|\ln \sigma^F_t - \ln \hat{\sigma}_t| \over |\ln
      \hat{\sigma}_t|}$ &
    SV & GARCH & Sample Mean \\
    \hline
    1\% & 0.5760 & 0.5979 & 0.3930 \\
    \hline
    5\% & 0.8351 & 0.8363 & 0.6611 \\
    \hline
    10\% & 0.9716 & 0.9536 & 0.9085 \\
    \hline
  \end{tabular}
  \caption{\small \it Percentage of ``good'' forecasts as defined by
    deviating nore more than 1\%, 5\% and 10\% from the measured
    realized volatility.}
  \label{tab:volvo_15_diff}
\end{table}


\section{30-minute Returns}\label{sec:volvo_30}
In this section we model the log-volatility of {\it Volvo B} 30-minute
returns during the period 2013/10/10 - 2014/04/04. This series
contains 1884 log-volatilities computed using 2-minute returns in each
30-minute interval. Among them we use the first 1507 for model
estimation and the last 377 for forecast and model verification.

The left plot of figure \ref{fig:volvo_inno_and_lv_acf} shows the
distribution of $(r_t - \E(r_t))/\sigma_t$. We observe in the figure a
nice Gaussian variate, so we can be sure that the sum of squared
2-minute returns makes a good approximation to the variance of
30-minute returns in this particular case.
\begin{figure}[htb!]
  \centering
    \includegraphics[scale=0.4, clip=true, trim=0 256 0
    151]{../pics/volvo_inno_and_lv_acf.pdf}
    \label{fig:volvo_inno_and_lv_acf}
  \caption{\small \it Left: probability plot of $(r_t -
    \E(r_t))/\sigma_t$; Right: auto-correlations of $\ln \sigma_t$}
\end{figure}

Guided by the auto-correlations of $\ln\sigma_t$ shown in the right
plot of figure \ref{fig:volvo_inno_and_lv_acf} we find the following
model:
\begin{eqnarray}
  && (1 - B)(1 - B^s) \ln \sigma_t \nonumber \\
  &=& (1 - \theta_1 B - \theta_2B^2 -
  \theta_3B^3 - \theta_4B^4) (1 - \Theta B^s)
  y_t \label{eq:volvo_lv_model}
\end{eqnarray}
where $s = 16$ is the seasonality and is apparent from the
auto-correlations of $\ln\sigma_t$. Fitting this model to the
measured realized volatilities yields the parameter values listed in
table
\ref{tab:volvo_params}.
\begin{table}[htb!]
  \centering
  \begin{tabular}{|c|c|c|c|c|c|c|}
    \hline
    Parameter & $\theta_1$ & $\theta_2$ & $\theta_3$ & $\theta_4$ &
    $\Theta$ & residual variance \\
    \hline
    Value & 0.7305 & 0.0575 & 0.0574 & 0.0346 & 0.8324 & 0.1340\\
    \hline
  \end{tabular}
  \caption{\small \it Volvo B log-volatility parameters}
  \label{tab:volvo_params}
\end{table}
Forecasting using the estimated model parameters gives the forecast
series $\ln \sigma^{\text{SV}}_t$. To access the quality of the
forecast, we also estimate a GARCH model using the returns. The result
is a GARCH(1, 1) model, whose parameter values are listed in table
\ref{tab:volvo_garch}.
\begin{table}[htb!]
  \centering
  \begin{tabular}{|c|c|c|c|}
    \hline
    Parameter & $\alpha_0$ & $\alpha_1$  & $\beta_1$ \\
    \hline
    Value & $3.125 \times 10^{-7}$ & 0.05 & 0.90 \\
    \hline
  \end{tabular}
  \caption{\small \it GARCH(1, 1) model of Volvo B 30-minute returns}
  \label{tab:volvo_garch}
\end{table}

The forecasts from SV, GARCH, and the sample mean are compared
using the difference $\ln \sigma^F_t - \ln\hat{\sigma}_t$, where $\ln
\sigma^F_t$ stands for the forecast. The distributions of this
difference is plotted in figure \ref{fig:volvo_slv_garch_cmp}; the
mean and the standard deviation of the distributions are listed in
table \ref{tab:volvo_slv_garch_cmp}.
\begin{table}[htb!]
  \centering
  \begin{tabular}{|c|c|c|c|}
    \hline
    & SV & GARCH & Sample mean \\
    \hline
    $\E(\ln \sigma^F_t - \ln \hat{\sigma}_t)$ & -0.0123 &
    0.0242 & -0.1055 \\
    \hline
    $\text{std}(\ln \sigma^F_t - \ln \hat{\sigma}_t)$ & 0.3261 &
    0.4250 & 0.3708 \\
    \hline
  \end{tabular}
  \caption{\small \it Standard deviation of $\ln\sigma^F_t -
    \ln\hat{\sigma}_t$}
  \label{tab:volvo_slv_garch_cmp}
\end{table}

\begin{figure}[htb!]
  \centering
    \includegraphics[scale=0.5, clip=true, trim=10 282 0
    243]{../pics/volvo_slv_garch_cmp.pdf}
    \label{fig:volvo_slv_garch_cmp}
  \caption{\small \it Comparison of the forecast from SV, GARCH,
    and sample mean. Blue: SV forecast; Green: GARCH forecast; Red:
    Sample mean forecast.}
\end{figure}
Figure \ref{fig:volvo_slv_garch_cmp} shows, as in the previous case of
Nordea Bank 15-minute returns, the SV model performs the best, GARCH
the second, and the sample mean the worst. However, when it comes to
over-estimates, the sample mean appears to be the best estimator,
while SV excels over GARCH. But a check of $\E(\ln\sigma_t)$ over
the sample for estimation (1507 data points) and over the sample for
comparison (377 data points) reveals that the first sample has mean
-6.3127 while the second has -6.2072. This increment explains the low
probability of over-estimation when using the first sample mean as
forecast. 

Table \ref{tab:volvo_good_percentage} compares the fraction of
``good'' estimates as measured by ${|\ln\sigma^F_t -
  \ln\hat{\sigma}_t| \over |\ln \hat{\sigma}_t|}$ being less than 1\%,
5\% and 10\%.
\begin{table}[htb!]
  \centering
  \begin{tabular}{|c|c|c|c|}
    \hline
    ${|\ln \sigma^F_t - \ln \hat{\sigma}_t| \over |\ln
      \hat{\sigma}_t|}$ &
    SV & GARCH & Sample Mean \\
    \hline
    1\% & 0.2175 & 0.1220 & 0.1406 \\
    \hline
    5\% & 0.7162 & 0.6154 & 0.6631 \\
    \hline
    10\% & 0.9231 & 0.8806 & 0.8992 \\
    \hline
  \end{tabular}
  \caption{\small \it Fraction of ``good'' forecasts as defined by
    ${|\ln \sigma^F_t - \ln \hat{\sigma}_t| \over |\ln
      \hat{\sigma}_t|}$ being less than 1\%, 5\% and 10\%.}
  \label{tab:volvo_good_percentage}
\end{table}
We see from the table that the SV model consistently excels over
the other two alternatives.

\chapter{Ericsson in 2013/14}
\label{chp:ericsson}
In this chapter we model the volatility of {\it Ericsson B}
15-minute and 30-minute returns during the period 2013/10/10 -
2014/04/04.

\section{15-minute Returns}\label{sec:ericsson_15min}
Using the same method as before, we first find the right
higher-frequency returns that best approximate the volatility of the
15-minute returns. These turn out to be 50-second returns, as figure
\ref{fig:ericsson_15_quotient} shows.
\begin{figure}[htb!]
  \centering
  \includegraphics[scale=0.4, clip=true, trim=79 259 108
  121]{../pics/ericsson_15_quotient.pdf}
  \caption{\small \it Normal probability plot of $(r_t -\E(r_t))/
    \sigma_t$. $\sigma_t^2$ is computed as the sum of squared
    50-second returns.}
  \label{fig:ericsson_15_quotient}
\end{figure}

Following the procedure described in previous sections, an ARIMA model is
found for this series:
\begin{eqnarray*}
  (1-B)(1-B^s) \ln\sigma_t &=& (1- \theta_1B - \theta_2B^2 -
  \theta_3B^3)(1 - \Theta B^s) y_t
\end{eqnarray*}
where the seasonality $s$ is 33 and the other parameter values are
estimated to be those listed in table \ref{tab:ericsson_15_params}.
\begin{table}[htb!]
  \centering
  \begin{tabular}{|c|c|c|c|c|c|c|}
    \hline
    Parameter & $\theta_1$ & $\theta_2$ & $\theta_3$ & $\Theta$ &
    $\var(y_t)$ \\
    \hline
    Value & 0.8078 & 0.0454 & 0.0943 & 0.8798 & 0.1242 \\
    \hline
  \end{tabular}
  \caption{\small \it Ericsson B log-volatility parameters}
  \label{tab:ericsson_15_params}
\end{table}

Then a GARCH model is also found with parameter values listed in
table \ref{tab:ericsson_15_garch_params}:
\begin{table}[htb!]
  \centering
  \begin{tabular}{|c|c|c|c|c|}
    \hline
    Parameter & $\alpha_0$ & $\alpha_1$ & $\alpha_{s}$ & $\beta_{s}$ \\
    \hline
    Value & $2.4181 \times 10^{-7}$ & 0.1513 & 0.1409 & 0.6027 \\
    \hline
  \end{tabular}
  \caption{\small \it GARCH model of Volvo B 30-minute returns}
  \label{tab:ericsson_15_garch_params}
\end{table}

The forecasts from both models are compared as follows: Table
\ref{tab:ericsson_15_diff1} shows the mean and standard deviation of
the 3 kinds of forecasts:
\begin{table}[htb!]
  \centering
  \begin{tabular}{|c|c|c|c|}
    \hline
    & SV & GARCH & Sample mean \\
    \hline
    $\E(\ln \sigma^F_t - \ln \hat{\sigma}_t)$ & 0.0035 &
    0.0600 & 0.1116 \\
    \hline
    $\text{std}(\ln \sigma^F_t - \ln \hat{\sigma}_t)$ & 0.3278 &
    0.3667 & 0.3590 \\
    \hline
  \end{tabular}
  \caption{\small \it Standard deviation of $\ln\sigma^F_t -
    \ln\hat{\sigma}_t$}
  \label{tab:ericsson_15_diff1}
\end{table}
Consistent with previous results, the bias introduced by the SV model
is significantly smaller than that introduced by GARCH. Figure
\ref{fig:ericsson_15_diff2} shows the distribution of the difference
between a forecast and its corresponding measured realized volatility,
i.e. $\ln\sigma^F_t - \ln\hat{\sigma}_t$; table
\ref{tab:ericsson_15_diff3} compares the percentage of ``good''
forecasts using the 3 alternatives.
\begin{figure}[htb!]
  \centering
    \includegraphics[scale=0.5, clip=true, trim=45 298 21
    165]{../pics/ericsson_15_diff2.pdf}
  \caption{\small \it Comparison of the forecast from SV, GARCH,
    and sample mean. Blue: SV forecast; Green: GARCH forecast; Red:
    Sample mean forecast.}
  \label{fig:ericsson_15_diff2}
\end{figure}

\begin{table}[htb!]
  \centering
  \begin{tabular}{|c|c|c|c|}
    \hline
    ${|\ln \sigma^F_t - \ln \hat{\sigma}_t| \over |\ln
      \hat{\sigma}_t|}$ &
    SV & GARCH & Sample Mean \\
    \hline
    1\% & 0.1673 & 0.1673 & 0.1634 \\
    \hline
    5\% & 0.7268 & 0.6850 & 0.6941 \\
    \hline
    10\% & 0.9621 & 0.9307 & 0.9399 \\
    \hline
  \end{tabular}
  \caption{\small \it Fraction of ``good'' forecasts as defined by
    ${|\ln \sigma^F_t - \ln \hat{\sigma}_t| \over |\ln
      \hat{\sigma}_t|}$ being less than 1\%, 5\% and 10\%.}
  \label{tab:ericsson_15_diff3}
\end{table}
It is clear from figure \ref{fig:ericsson_15_diff2} and table
\ref{tab:ericsson_15_diff3} that the SV model out-performs GARCH. We
see that the SV model yields considerably higher fractions of good
forecasts by all criteria, and in particular, gives much few
under-estimates.

\section{30-minute Returns}\label{sec:ericsson_30min}
In this section we look at the 30-minute returns of Ericsson B during
the period 2013/10/10 - 2014/04/04. Since the methods are the same as
with other series, we shall only present the results here.

First of all, the volatility of this series is found to be well
approximated by realized volatilities computed from 75-second
returns. The normal probability plot is shown in figure
\ref{fig:ericsson_30_quotient}.
\begin{figure}[htb!]
  \centering
  \includegraphics[scale=0.4, clip=true, trim=79 259 108
  121]{../pics/ericsson_30_quotient.pdf}
  \caption{\small \it Normal probability plot of $(r_t -\E(r_t))/
    \sigma_t$. $\sigma_t^2$ is computed as the sum of squared
    75-second returns.}
  \label{fig:ericsson_30_quotient}
\end{figure}
The following ARIMA model is found for the log-volatility $\ln
\sigma_t$:
\begin{eqnarray*}
  (1-B)(1-B^s) \ln\sigma_t &=& (1- \theta_1B - \theta_2B^2) (1 -
  \Theta B^s) y_t 
\end{eqnarray*}
where the seasonality $s = 16$. The parameter values are estimated to
be those listed in table
\ref{tab:ericsson_30_params}:
\begin{table}[htb!]
  \centering
  \begin{tabular}{|c|c|c|c|c|c|c|}
    \hline
    Parameter & $\theta_1$ & $\theta_2$ & $\Theta$ &
    $\var(y_t)$ \\
    \hline
    Value & 0.6842 & 0.2470 & 0.8391 & 0.0918 \\
    \hline
  \end{tabular}
  \caption{\small \it Ericsson B log-volatility parameters}
  \label{tab:ericsson_30_params}
\end{table}
A GARCH model is also found with parameters listed in table
\ref{tab:ericsson_30_garch_params}.
\begin{table}[htb!]
  \centering
  \begin{tabular}{|c|c|c|c|c|}
    \hline
    Parameter & $\alpha_0$ & $\alpha_1$ & $\alpha_{s}$ & $\beta_{s}$ \\
    \hline
    Value & $7.4856 \times 10^{-7}$ & 0.0532 & 0.1506 & 0.6778 \\
    \hline
  \end{tabular}
  \caption{\small \it GARCH(1, 1) model of Ericsson B 30-minute
    returns}
  \label{tab:ericsson_30_garch_params}
\end{table}

Table \ref{tab:ericsson_30_diff1} shows the mean and standard
deviation of the difference series $\ln\sigma^F_t -
\ln\hat{\sigma}_t$, where $\ln\sigma^F_t$ is the forecast
log-volatility and $\ln\hat{\sigma}_t$ is the measured realized
volatility.
\begin{table}[htb!]
  \centering
  \begin{tabular}{|c|c|c|c|}
    \hline
    & SV & GARCH & Sample mean \\
    \hline
    $\E(\ln \sigma^F_t - \ln \hat{\sigma}_t)$ & 0.0004 &
    0.0526 & 0.1039 \\
    \hline
    $\text{std}(\ln \sigma^F_t - \ln \hat{\sigma}_t)$ & 0.2519 &
    0.3019 & 0.2963 \\
    \hline
  \end{tabular}
  \caption{\small \it Standard deviation of $\ln\sigma^F_t -
    \ln\hat{\sigma}_t$}
  \label{tab:ericsson_30_diff1}
\end{table}
Apparently, the GARCH forecasts have a fairly large bias compared with
those of the SV model. Turning to figure \ref{fig:ericsson_30_diff2}
and table \ref{tab:ericsson_30_diff3}, we also see the SV model
performs consistently better --- a higher percentage of ``good''
forecasts are delivered and, while it under-estimates to around the
same extent as does GARCH, it certainly over-estimates a lot
less. These are confirmative to what we have observed for the other
series.
\begin{figure}[htb!]
  \centering
    \includegraphics[scale=0.5, clip=true, trim=33 281 11
    150]{../pics/ericsson_30_diff2.pdf}
  \caption{\small \it Comparison of the forecast from SV, GARCH,
    and sample mean. Blue: SV forecast; Green: GARCH forecast; Red:
    Sample mean forecast.}
  \label{fig:ericsson_30_diff2}
\end{figure}

\begin{table}[htb!]
  \centering
  \begin{tabular}{|c|c|c|c|}
    \hline
    ${|\ln \sigma^F_t - \ln \hat{\sigma}_t| \over |\ln
      \hat{\sigma}_t|}$ &
    SV & GARCH & Sample Mean \\
    \hline
    1\% & 0.2210 & 0.1617 & 0.1563 \\
    \hline
    5\% & 0.7844 & 0.6900 & 0.6873 \\
    \hline
    10\% & 0.9838 & 0.9650 & 0.9596 \\
    \hline
  \end{tabular}
  \caption{\small \it Fraction of ``good'' forecasts as defined by
    ${|\ln \sigma^F_t - \ln \hat{\sigma}_t| \over |\ln
      \hat{\sigma}_t|}$ being less than 1\%, 5\% and 10\%.}
  \label{tab:ericsson_30_diff3}
\end{table}
