\chapter{Conclusions}
\label{chp:summary2}
In the thesis project we have studied the volatilities of a total of 7
intraday return series: 15-minute returns of Nordea Bank during the
period 2012/01/16 - 2012/04/20 and 15-minute and 30-minute returns of
Nordea Bank, Volvo B and Ericsson B in the period 2013/10/10 -
2014/04/04. Every series is investigated with a stochastic volatility
(SV) model and a GARCH model. In fact, when we fit the GARCH model, 3
GARCH variants have been considered: the basic GARCH, EGARCH, and
GJR. But in all cases, the basic GARCH stands out to be the most
accurate in forecasting the realized volatilities.

For comparison, each model makes one-step-ahead forecasts and
then the two series of forecasts are compared. The quality of a
forecast series $\ln\sigma^F_t$ is judged by $\ln\sigma^F_t - \ln
\hat{\sigma}_t$, where $\hat{\sigma}_t$ is the measured realized
volatility. For each series of $\ln\sigma^F_t - \ln \hat{\sigma}_t$, we
have compared the mean, standard deviation, cumulative distribution
function, complementary distribution function, as well as the
percentage of ``good'' forecasts, where ``good'' is defined by
deviating less than 1\%, 5\% or 10\% from the realized
volatility. What we find through the studies is the following:
\begin{itemize}
\item In all the 7 studied series, the log-volatility $\ln \sigma_t$
  is well described by a seasonally integrated moving average
  model. Long memory in these series is accounted for by the compounded
  difference operator $(1-B)(1-B^s)$ where $B$ denotes the back-shift
  operator. $s$ at here stands for seasonality, which is 33 in the
  cases of 15-minute returns and 16 in the cases of 30-minute
  returns.

\item An SV model generally yields more accurate forecasts than does
  the GARCH model for the same series. This is certainly well expected,
  considering that the SV model incorporates much more data than does
  GARCH. However, it must be noted that the validity of the
  aforementioned comparison is underlain by the accuracy of realized
  volatility as a proxy to the true conditional volatility.

\item The accuracy of forecasts varies more for GARCH than for SV
  models. For example, the percentage of forecasts
  that lie within 5\% of the corresponding realized volatilities has
  the values listed in table \ref{tab:5percent_values}.
  \begin{table}[htb!]
    \centering
    \begin{tabular}{|c|c|c|c|c|}
      \hline
      & Volvo 15m & Volvo 30m & Ericsson 15m & Ericsson 30m \\
      \hline
      SV & 0.8351 & 0.7162 & 0.7268 & 0.7844 \\
      \hline
      GARCH & 0.8363 & 0.6154 & 0.6850 & 0.6900 \\
      \hline
      \hline
      & Nordea 1 15m & Nordea 2 15m & Nordea 2 30m & std.\\
      \hline
      SV & 0.7457 & 0.7887 & 0.8090 & 0.0432 \\
      \hline
      GARCH & 0.7062 & 0.7436 & 0.7109 & 0.0722 \\
      \hline
    \end{tabular}
    \caption{\small \it Comparison of the percentage of forecasts that
      lie within 5\% of the corresponding realized
      volatilities. Nordea 1 refers to the Nordea 15-minute returns
      studied in chapter \ref{chp:nordea_15min} and Nordea 2 refers to
      the 15-minute and 30-minute returns studied in chapter
      \ref{chp:Nordea2}. 15m and 30m are acronyms for 15-minute and
      30-minute, respectively.}
    \label{tab:5percent_values}
  \end{table}
  The standard deviation of this percentage is 0.0432 for SV models
  but 0.0722 for GARCH.

\end{itemize}

In chapter \ref{chp:SLV_unconditional} we have presented our
calculation of the unconditional distribution functions of the
stochastic volatility models used throughout the project, i.e. the
distribution of $\mu + e^{\bar{v} + \sigma a}b$, where $a, b \sim N(0,
\mtx \Sigma)$. In the simplest case where $\mtx \Sigma = \mtx I$, this
is simply the product of a lognormal and a normal variate, but when
$\mtx \Sigma \neq \mtx I$, the distribution becomes more
complicated. In the case $\mtx \Sigma \neq \mtx I$, we obtained the
moment generating function and calculated the first four moments of
the distribution. The parameters of the unconditional distribution are
estimated with respect to the 30-minute returns of Nordea Bank, Volvo
B and Ericsson B. The resulting distributions are seen to fit the data
very nicely.
