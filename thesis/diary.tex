\documentclass{article}
\usepackage[T1]{fontenc}
\usepackage[utf8]{inputenc}
\usepackage[swedish]{babel}

\title{Project Diary}
\author{Xie Xiaolei}
\date{\today}
\begin{document}

\maketitle

\begin{itemize}
\item 2013/08/26 - 2013/09/01

Read Sven's article about Wishart-Levy matrices as well as two
others referenced by it.

\item 2013/09/02 - 2013/09/06

Read Free probability theory which is used to derive the eigenvalue
spectrum of the Wishart-Levy matrix.

\item 2013/09/07 - 2013/09/12

Read the book ``An application to probability theory and its
applications'', trying to understand the concepts of the concepts of
probability measures, measurable sets and sigma algebras.

\item 2013/09/13 - 2013/09/24

Work on the Matlab program that analyzes the intra-day transaction
data. The purpose to study the tail behavior. The main difficulty is
that, in order to obtain good statistics, a lot of data and hence
computation time are needed.

\item 2013/09/25 - 2013/09/29

Improve the program to make full use of data and to make it more
efficient to go through the data.

\item 2013/09/30 - 2013/10/04

Study the auto-correlations of intra-day returns. These are expected to
decay exponentially. It turns out that the auto-correlations between
the returns decay to 0 in approximately the same time as the time lag
of the returns.

\item 2013/10/05 - 2013/10/11

Study the GARCH models. Fit a GARCH model to 15-minute returns of
Nordea Bank.

\item 2013/10/12

Read a paper about using the Johnson SU distribution to model the
residuals in a GARCH model.

\item 2013/10/13 - 2013/10/15

Calculate the log-likelihood function of a GARCH model whose residuals
have Johnson SU distributions.

\item 2013/10/16 - 2013/10/21

Write a program to perform maximum-likelihood estimate for GARCH
models with Johnson SU residuals.

\item 2013/10/22 - 2013/10/30

Read articles about ``realized volatility'':
\begin{enumerate}
\item realized volatility is the square root of the sum of squared
  returns that are sampled at a higher frequency that those one
  intends to model.
\item The square of realized volatility is the sample analogue of the
  quadratic variation of a stochastic process.
\item At what frequency should the returns be sampled to give a good
  estimate of the quadratic variation: Ideally, the higher the
  better. But due to market micro-structure, there is an optimal
  frequency, the determination of which is, however, a subject of
  debate.
\end{enumerate}

\item 2013/11/01 - 2013/11/02

Read a paper about alternative volatility proxies to the realized
volatility. The trading range, i.e. (highest - lowest)/lowest, is
proposed among others. The use of realized volatility relies on the
availability of intra-day transaction data. I have such data but am
limited to a few stocks in disconnected time periods.

\item 2013/11/03 - 2013/11/05

Try out the proposed volatility proxies against real data.

\item 2013/11/06 - 2013/11/07

Study the log-volatility of Nordea Bank. They are not Gaussian
distributed but rather have fat tails.

\item 2013/11/08 - 2013/11/15

Study time series analysis.

\item 2013/11/16 - 2013/11/24

Study the book ``Stochastic integration and differential
equations''. This book is probably meant for mathematicians. Theorems,
corollaries, and lemma follow one another.

\item 2013/11/25 - 2013/11/30
Study the book ``Stochastic Differential equation - An introduction
with Applications''. It is easier than the other book and more
oriented to applications.

\item 2013/12/01 - 2013/12/03
Make a Matlab function that retrieve n-minute returns together with
their realized volatility from transaction data.
Estimate a stochastic log-volatility model for 15min returns of
Nordea Bank. The realized volatilities are calculated using 30sec
returns.

\item 2013/12/04 - 2013/12/07
Analyze the tail behavior of the simplest possible stochastic
log-volatility model: r = m + a exp(s b) where a and b are independent
standard Gaussian r.v. Eventually I can express p(r) as
p(r) = A erfc(a ln|r| + b)

\item 2013/12/08 - 2013/12/15
Read about Wishart matrices and multivariate Gaussian distribution.
It is understood that the elements' distribution of a Wishart
matrix, a GOE, a Wiger ensemble, are characterized by the
eigenvalues.

\item 2013/12/16 - 2013/12/20
Derive an expression for the elements' distribution of RR', where
R is an NxT matrix, each row of which represents a time series.
$R_{it} = \phi R_{i, t-1} + a_{it}$. It turns out The joint
probability density of RR' is simply p(RR') = p(AA') where R = AM and
hence $A = RM^{-1}$.

\item 2013/12/21 - 2013/12/23
Try to find an analytic expression for the eigenvalue distribution
of RR' at the presence of autocorrelations.

\item 2013/12/27 - 2013/12/29
Read an article about the maximum eigenvalue distribution of the
Wishart matrix when the covariance matrix is just the identity
matrix. The paper gives exact expressions for the PDF and the CDF of
the maximum eigenvalue. It also shows that the PDF is a mixture of
gamma distributions with different weights and hence can be
approximated by a gamma distribution, namely the dominant one in the
mixture.

\item 2014/01/03 - 2013/01/10
Try to make a Matlab program that computes the exact maximum
eigenvalue PDF. The difficulty is to deal with quotients of gamma
functions where both the denominator and the numerator tend to
infinity at large T. One also needs to handle a determinant that tends
to 0 at large T.

\item 2014/01/11 - 2013/01/15
Use the approximate gamma distribution instead. Write Matlab programs
to simulate the return matrices, to fit the simulated maximum
eigenvalue distribution to a gamma distribution. Study how the fitting
gamma distributions change in response to increasing auto-correlations
among the returns.

\item 2014/01/16 - 2013/01/21
Even more simulations. We need the autocorrelations to change more
smoothly in order to find out how the fitting gamma distributions
change.

\item 2014/01/22 - 2013/01/28
Conclude the chapter about cross-correlation matrix with Gaussian
returns. Plan the next chapter of correlation matrix with fat-tailed
returns.

\item 2014/01/29 - 2013/02/05
Further develop my simple stochastic log-volatility model
\begin{eqnarray*}
  r_t &=& A e^{\sigma a_t} b_t
\end{eqnarray*}
The current model does not account for skewness. Instinct suggests
that allowing $a_t$ and $b_t$ to be correlated will introduce
skewness. Calculate the analytic PDF in this case. An integral has to
be solved to express the PDF in closed form.

\item 2014/02/06 - 2013/02/10
Still on the integral of the PDF. Have tried different ways to
approximate it. The integrand is really complicated.

\item 2014/02/11 - 2013/02/14
Study the method of steepest descent in order to approximate the
integral. It turns out to be inapplicable. Its main use is in studying
asymptotic behavior of integrals when a large parameter is present in
the exponent and makes it valid to expand the exponent around a few
local maxima.

\item 2014/02/14 - 2013/02/20
Read a paper about regularly varying tails of GARCH(1,1) models, a
book about extreme events that systematically treats different types
of heavy tails, and an article about the heavy-tail central limit
theorem, i.e. convergence to the $\alpha$-stable laws.

\item 2014/02/21 - 2014/02/26
Study the correlation matrix built from GARCH returns, write up the
results.




\end{itemize}
\end{document}
