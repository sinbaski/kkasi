\chapter{Results}
\label{chp:summary}
Realistic dynamic models for relative price changes (returns) of
financial assets have been studied. In particular, the \gls{garch} and
\gls{sv} models have been utilized and compared in their forecast
accuracies by case studying of 15- and 30-minute returns of Nordea,
Volvo and Ericsson. The following are the conclusions from this
investigation:
\begin{itemize}
\item In all the 7 studied series, the log-volatility $\ln \sigma_t$
  is well described by a seasonally integrated moving average
  model. Long memory in these series is accounted for by the compounded
  difference operator $(1-B)(1-B^s)$ where $B$ denotes the back-shift
  operator. $s$ stands for seasonality, which is 33 in the cases of
  15-minute returns and 16 in the cases of 30-minute returns.

\item An \gls{sv} model generally yields more accurate forecasts than does
  the \gls{garch} model for the same series. This is certainly well expected,
  considering that the \gls{sv} model incorporates much more data than does
  \gls{garch}. However, it must be noted that the validity of the
  aforementioned comparison is underlain by the accuracy of realized
  volatility as a proxy to the true conditional volatility.

\item \gls{sv} models perform more consistently in terms of forecast
  accuracy than does \gls{garch}, as shown in appendix
  \ref{sec:forecast_volatility}.
\end{itemize}

Regarding the covariance matrix of Gaussian return series, we have
arrived at the following results:
\begin{itemize}
\item Auto-correlations in the returns rescale the covariance of two
  series by a factor of $1/(1 - \phi^2)$, where $\phi =
  \text{corr}(r_t, r_{t-1})$. See
  eq. \ref{eq:gaussian_mean2}. Moreover, they rescale the variance of
  each series by ${1 \over (1 - \phi^2)}\left[
    1- {\phi^2 \over T(1 - \phi^2)}
  \right]$. See eq. \ref{eq:gaussian_cii_mean2}.

\item Auto-correlations in the returns increase the variance of the
  covariance and also the variance of the variance. See
  eq. \ref{eq:gaussian_variance2} and
  \ref{eq:gaussian_cii_variance2}.

\item The largest eigenvalue obeys approximately a gamma distribution
  even when the returns are auto-correlated. It was shown by Chiani in
  2012 that the largest eigenvalue obeyed gamma distribution when the
  returns were {\it not} auto-correlated \cite{Chiani2012}. In
  addition, we find that the mean of the gamma distribution is
  approximately quadratic in $\tan{\pi\phi \over 2}$.
\end{itemize}

For the covariance matrix of \gls{garch}(1,1) series, our
contributions are the following:
\begin{itemize}
\item The diagonal and non-diagonal matrix elements both have L\'evy
  distributions but with different L\'evy indices.
\item The power-law tails of \gls{garch}(1,1) returns lead to a group
  of localized eigenvectors that correspond to large eigenvalues.
\item Auto-correlations in the returns reduce the localization of the
  eigenvectors. The fraction of localized eigenvectors decreases
  approximately as a quadratic function of $\phi = \text{corr}(r_t,
  r_{t-1})$. See figure \ref{fig:localization_ratio} and
  \ref{fig:localization_ratio2}.
\end{itemize}

In summary, auto-correlations in the returns create illusory
cross-correlations. To assess the true cross-correlations among the
assets in a market, one has to adopt a model for each of the return
series, infer the residuals and then assess the cross-correlations
among the residuals instead (c.f. chapter \ref{chp:PriceModels}).

However, in an efficient market, auto-correlations in the returns are
necessarily as weak as indistinguishable from measurement errors 
so that exploitable arbitrage opportunities do not exist. Hence
auto-correlations cannot be completely eliminated by taking residuals
of the returns. When estimating the covariance matrix and its
eigenvalues and eigenvectors, the illusory effects caused by
auto-correlations must be considered as an inherent source of
uncertainty.

% Correlations between sequences of stock returns from different assets
% have been studied by forming a covariance matrix. The stock returns
% have been simulated using different dynamical return models. Since a
% \gls{sv} model with Gaussian innovations have an unconditional
% distribution with all moments finite (see section \ref{sec:XieCalc}),
% we have focused our attention on \gls{garch}(1,1) models that were
% proven to have regularly varying (power-law) tails. We have
% investigated how a covariance matrix of \gls{garch}(1,1) returns differs from a
% covariance matrix of Gaussian returns, in terms of matrix elements and
% eigenvalues distributions. In particular, we have considered how the
% two matrices contrast with each other when the return series that they
% derive from are auto-correlated.

% In chapter \ref{chp:Gaussian} we studied how the distributions of
% elements and eigenvalues of a covariance matrix of Gaussian returns
% were altered by auto-correlations in the returns. For the matrix elements
% distribution, asymptotic Gaussian distributions were obtained. It was
% seen that the mean and the variance of the matrix elements
% distributions were changed by auto-correlations
% (Eq. \ref{eq:gaussian_mean2} to \ref{eq:gaussian_cii_variance2}). As 
% for the distribution of the maximum eigenvalue of C, it was shown that
% the maximum eigenvalue had approximately a gamma distribution even
% when the returns were auto-correlated. The mean of the gamma
% distribution was approximately quadratic in $\tan{\pi\phi \over 2}$,
% where $\phi$ was the coefficient of an \gls{ar}(1) process and
% characterized the strength of auto-correlation. On the other hand, the
% minimum eigenvalue was also increased by increased
% auto-correlations. As a whole, the eigenvalue spectrum widened as
% auto-correlation increased. See figure
% \ref{fig:GaussianMarkovSpectrumPDF}.

% In chapter \ref{chp:CrossCorrelationFat} we studied a covariance
% matrix of asset returns described by the \gls{garch}(1,1) 
% model. We found that both the diagonal and the non-diagonal matrix
% elements followed $\alpha$-stable (L\'evy) distributions, but had
% different L\'evy indices and scaling parameters (equations
% \ref{eq:garch_cii_dist} and \ref{eq:garch_cij2}). The eigenvalue
% distribution of the covariance matrix, as in the case of 
% Gaussian returns, was found to widen as auto-correlations
% increased. Meanwhile, both the largest and the smallest eigenvalues
% were seen to increase. See figure
% \ref{fig:GarchSpectrumAutocorrelated}.

% We have also investigated how localization of eigenvectors,
% i.e. dominance of one or a few basis vectors in the composition of an
% eigenvector, is affected by auto-correlations in the returns. Firstly,
% it is seen that localized eigenvectors correspond to large
% eigenvalues; secondly, it is found that strengthened auto-correlations
% reduce the extent of localization. These are consistent with what \AA
% berg found for Wishart-L\'evy matrices \cite{Aberg2013}.

% In particular, we find that the fraction of localized eigenvectors, as
% defined by an eigenvector's participation radio (see equation
% \ref{eq:IPR_def}) being lower than a given threshold or by its largest
% component (in the sense of absolute value) being higher than a
% threshold, falls approximately as a quadratic function of the
% auto-correlation strength (autoregressive coefficient of an
% \gls{ar}(1) process). See figure \ref{fig:localization_ratio} and
% \ref{fig:localization_ratio2}.

\chapter{Outlook}
Continuing from the obtained results, areas of future research may
include, first of all, a covariance matrix formed from \gls{sv} models
with power-law tails, which may be obtained by assuming, for example,
Student's t distribution or Normal Inverse Gaussian distribution for
the return's innovation.

Another interesting area could be models that treat the conditional
covariance matrix as an inherent part of the model specification
rather than treating it as an inferred quantity, which is the approach
taken by the current work. Predecessors on this path are the
multivariate \gls{garch} and \gls{sv} models
\cite{Mikosch2009}. To have sufficient flexibility, many of these
models involve a large number of parameters, which make it hard to fit
them to data and increase the chances of model mis-specification. A
model that strikes a good balance between flexibility and complexity
will be of great interest.
These approaches may well lead to more accurate volatility forecasts
and improved estimation of the uncertainties introduced by
auto-correlations in the returns.

Analytically deriving the eigenvalue and eigenvector distributions of
a covariance matrix formed from return series described by a realistic
model will be very challenging but interesting. It is also interesting to
see whether the methods developed for Wishart matrices, for example
the holonomic gradient method \cite{Hashiguchi2012}, can be applied to
the aforementioned covariance matrices.
Advancement in these areas will undoubtedly find applications in
e.g. principle component analysis and portfolio management.

