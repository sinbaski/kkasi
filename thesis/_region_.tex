\message{ !name(draft4.tex)}\documentclass{article}
\usepackage{amssymb,amsmath,amsthm,bm}
\usepackage{graphicx}
\usepackage{subfigure}
\usepackage{multirow}
\usepackage{wrapfig}
\usepackage[usenames,dvipsnames]{color}
\usepackage{mathrsfs}
\usepackage{enumerate}
\usepackage[bookmarks=true]{hyperref}
\usepackage{bookmark}

\usepackage{amssymb,amsmath,amsthm,amsfonts}
\usepackage{mathrsfs}
\usepackage{dsfont}
\usepackage{enumerate}

%\newtheorem{mdef}{Definition}
%\newtheorem{theorem}{Theorem}
\newcommand{\eqsplit}[2]{
  \begin{equation}\label{#2}
    \begin{split}
      #1
    \end{split}
  \end{equation}}
\newcommand{\eqnsplit}[1]{
  \begin{eqnarray*}
    #1
  \end{eqnarray*}}
\newcommand{\tran}[1]{
  \tilde{#1}
}
\newcommand{\td}[2]{
  \frac{d #1}{d #2}
}
\newcommand{\pd}[2]{
  \frac{\partial #1}{\partial #2}
}
\newcommand{\ppd}[2]{
  \frac{\partial^2 #1}{\partial #2^2}
}
\newcommand{\pdd}[3]{
  \frac{\partial^2 #1}{\partial #2 \partial #3}
}
\newcommand{\otd}[1]{
  \frac{d}{d #1}
}
\newcommand{\opd}[1]{
  \frac{\partial}{\partial #1}
}
\newcommand{\oppd}[1]{
  \frac{\partial^2}{\partial #1^2}
}
\newcommand{\opdd}[2]{
  \frac{\partial^2}{\partial #1 \partial #2}
}
\newcommand{\ket}[1]{
  |#1\rangle
}
\newcommand{\bra}[1]{
  \langle#1|
}
\newcommand{\inn}[1]{
  \langle#1\rangle
}
\newcommand{\mean}[1]{
  \langle#1\rangle
}
\newcommand{\tr}{
  \text{tr}\,
}
\newcommand{\re}{
  \text{Re}\,
}
\newcommand\im{
  \text{Im}\,
}
\newcommand{\var}{
  \text{var}
}
\newcommand{\arcsinh}{
  \sinh^{-1}
}
\newcommand{\arccosh}{
  \cosh^{-1}
}
\newcommand{\erfc}{
  \text{erfc}
}
\newcommand{\E}{
  \mathbb{E}
}
\renewcommand{\P}{
  \mathbb{P}
}
\newcommand{\I}[1]{
  \mathbf{1}_{\{#1\}}
}
\newcommand{\1}[1]{
  \mathds{1}_{\{#1\}}
}
\newcommand{\diag}{
  \text{diag\,}
}
\newcommand{\M}{
  {\text{max}}
}
\newcommand{\m}{
  {\text{min}}
}
\newcommand{\ph}{
  {\text{arg}\,}
}
\newcommand\erf{
  \text{erf}
}
\renewcommand\vec[1]{
  \mathbf{#1}
}
\newcommand\mtx[1]{
  \mathbf{#1}
}
\newcommand\ed{
  \,{\buildrel d \over =}\,
}



\DeclareGraphicsExtensions{.eps, .pdf}

\begin{document}

\message{ !name(draft4.tex) !offset(-3) }

\section{effect of heteroscedasticity in covariance matrix}
Consider a covariance matrix constructed as
$$
C_{ij} = {1 \over T}\sum_{t=1}^T r_{it} r_{jt}
$$
where $1 \leq i \leq N$. Adopt a stochastic volatility model for
$r_{it}$, i.e.
$$
r_{it} = \sigma_{it} \eta_{it}
$$
We can write
\begin{eqnarray*}
C_{ij} &=& {1 \over T}\sum_{t=1}^T \sigma_{it} \eta_{it} \sigma_{jt}
\eta_{jt} \\
\bm{C} &=& {1 \over T} (\bm{\sigma * \eta}) (\bm{\sigma * \eta})'
\end{eqnarray*}
where the matrices $\bm{\sigma}$ and $\bm{\eta}$ have elements
$\sigma_{it}$ and $\eta_{it}$ respectively, and * denotes element-wise
multiplication. Due to the cyclic property of matrix trace, the
moment generating function of $\bm{C}$, $M_C(z)$, relates to $M_D(z)$,
the moment generating function of $\bm D$
\begin{eqnarray*}
  \bm{D} &=& {1 \over T} (\bm{\sigma * \eta})' (\bm{\sigma * \eta})
\end{eqnarray*}
through the equation
\begin{eqnarray*}
  M_C(z) &=& {1 \over q} M_D(z)
\end{eqnarray*}
where $q = N/T$. For later convenience, we make use of $T \times T$
matrices $\bm{\tilde{\sigma}}$ and $\bm{\tilde{\eta}}$, as well as a
projector
$$
\bm{P} = \text{diag}(\underbrace{1, \cdots, 1}_{\text{N 1's}}, 
\underbrace{0, \cdots, 0}_{\text{T-N 0's}})
$$
The first N rows of $\bm{\tilde{\sigma}}$ and $\bm{\tilde{\eta}}$ are
precisely those of $\bm{\sigma}$ and $\bm{\eta}$. This way, we have
\begin{eqnarray*}
\bm D &=& {1 \over T} (\bm{\sigma * \eta})' (\bm{\sigma * \eta}) \\
&=& {1 \over T} (\bm{\tilde{\sigma} * \tilde{\eta}})' \bm P'
\bm P (\bm{\tilde{\sigma} * \tilde{\eta}}) \\
&=& {1 \over T} (\bm{\tilde{\sigma} * \tilde{\eta}})'
\bm P (\bm{\tilde{\sigma} * \tilde{\eta}}) \\
\end{eqnarray*}
Again, by the cyclic property of matrix trace, the spectral properties
of the RHS of the last equation is equivalent to those of
\begin{eqnarray*}
  \bm E &=& {1 \over T} (\bm{\tilde{\sigma} * \tilde{\eta}}) (\bm{\tilde{\sigma}
    * \tilde{\eta}})' \bm P \\
\end{eqnarray*}
In the simple situation where $\sigma_{it} =
\sigma_{jt} = \sigma_t$ for some $\sigma_t$ and for all $i, j, t$,
\begin{eqnarray*}
  \bm{\tilde \sigma * \tilde \eta} &=& \bm{\tilde \eta}
  \begin{pmatrix}
    \sigma_1 &        & \\
        & \ddots & \\
        &        & \sigma_T
  \end{pmatrix} \\
  &=& \bm{\tilde \eta \bar \sigma}
\end{eqnarray*}
Thus
\begin{eqnarray*}
  \bm E &=& {1 \over T}\bm{\tilde \eta \bar \sigma^2 \tilde \eta' P}
\end{eqnarray*}
In the large T limit $\bm{\tilde \eta \bar \sigma^2 \tilde \eta'}$ and $P$ are
freely independent, therefore their S-transforms are multiplicative,
i.e.
\begin{eqnarray*}
  S_E(z) &=& S_{\tilde \eta \bar \sigma^2 \tilde \eta'/T}(z) S_P(z) \\
  &=& S_{\tilde \eta' \tilde \eta /T}(z) S_{\bar \sigma^2}(z) S_P(z)
\end{eqnarray*}
For $S_{\tilde \eta' \tilde \eta /T}(z)$ there have been results when
the elements of $\bm{\tilde \eta}$ are identically distributed with finite
second moment \cite{burda2011} or alternatively, follow $\alpha$-stable
distribution \cite{politi2010}.

So our focus hereafter is on the spectral properties of
$\bar \sigma^2$. The Green's function of $\bar \sigma^2$ can be immediately
written down:
\begin{eqnarray*}
  G_{\bar \sigma^2}(z) &=& {1 \over T} \sum_{t=1}^T \frac{1}{z - \sigma_t^2}
\end{eqnarray*}
If the $\sigma_t$'s are independent, $G_{\bar \sigma^2}(z)$ is deterministic and
equal to $\mean{\frac{1}{z - \sigma_t^2}}$; If, however, the $\sigma_t$'s are
correlated, $G_{\bar \sigma^2}(z)$ is random. In the extreme case where the
$\sigma_t$'s have correlation 1 between any pair, $G_{\bar \sigma^2}(z)$ is a
random variable $\frac{1}{z - \sigma_1^2}$. In the latter case, although
the spectrum of $\bm \bar \sigma^2$ is random, it may still make sense to
know the expectation of the spectrum, which one obtains from the
expectation of the random Green's function. In the rest of this
section we consider a few cases in regard to the distribution of
$\sigma_t$'s as well as their correlations.

\subsection{log-normal volatilities with AR(1) autocorrelation}
In this subsection we consider the case
\begin{eqnarray*}
  \ln \sigma_t &=& \phi\ln \sigma_{t-1} + x_t
\end{eqnarray*}
where $x_t \sim N(0, 1)$. Then, assuming the process was started in
the infinite past, i.e. $t = -\infty$, $\mean{\ln \sigma_t \ln \sigma_s} =
\frac{\phi^{|t-s|}}{1 - \phi^2}$.

\bibliographystyle{unsrt}
\bibliography{econophysics}
\end{document}

\message{ !name(draft4.tex) !offset(-127) }
