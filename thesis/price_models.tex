\chapter{Return Models}\label{chp:PriceModels}
In this chapter we review some of the discrete-time return models and
fit them to intraday returns. The intention is to compare these models
in terms of forecast accuracy and to understand their statistical
properties. In the following we first describe these models briefly,
then in 

% \S\ref{sec:Garch_model} reviews the GARCH model using the Nordea
% 15-minute returns as an example. \S\ref{sec:SLV_model} examines the
% stochastic log-volatility model and discusses the possibility of
% describing the residuals of the log-volotility series using the
% Johnson Su distribution. The same Nordea return series is fitted as an
% example. \S\ref{sec:XieCalc} derives the unconditional distribution
% function of the stochastic log-volotility model and compares the
% analytic results to the 30-minute returns of a number of Swedish
% stocks.

\begin{enumerate}
\item{\bf Gaussian Distribution}

The justification of modeling return series as independent,
identically distributed Gaussian variates comes from imagining the
price process $S(t)$ as a Brownian motion whose increment $\sigma
dw_t$ at each time step is independent and scales as $\sigma
\sqrt{dt}$. Then, by adding a drift term $\mu dt$ that represents some
deterministic trend in the price process, one can express the
price $S(t)$ as a stochastic differential equation:
\begin{eqnarray*}
  dS &=& S\mu dt + S\sigma dw_t \\
  d(\ln S) &=& (\mu - \frac{1}{2} \sigma^2)dt + \sigma dw_t \\
\end{eqnarray*}
It follows from this equation that $\ln S(t) - \ln S_0$, where $S_0$
is the price at time 0, has Gaussian distribution with mean $(\mu -
\sigma^2/2)t$ and variance $\sigma^2 t$. Therefore, in a discrete-time
model, where the length of each time step $\Delta t$ is fixed, the
return $r_t = \ln S(t + \Delta t) - \ln S(t)$ is assumed to be
Gaussian distributed and have mean and variance that are functions of
$\Delta t$.

The picture depicted above is of course overly simplified, and as we
discussed in \S\ref{sec:StylizedFacts}, the distribution of returns is
not really Gaussian. Nevertheless, the assumption of Gaussian
distributed returns underlies such important theories as Black and
Scholes theory of option pricing. Thus, albeit hugely inaccurate, the
Gaussian distribution as a returns model shall not be forgotten. In
the next few sections, we discuss some more realistic returns models.

\item{\bf GARCH models}
``GARCH'' is the acronym for ``Generalized Autoregressive Conditional
Heteroscedasticity''. A GARCH(p, q) model is defined by the following
equation system:
\begin{equation}
  \label{eq:garch_def}
  \begin{aligned}
    r_t &= \mu + \epsilon_t \\
    \sigma_t^2 &= \alpha_0 + \sum_{i=1}^q \alpha_i \epsilon_{t-i}^2 +
    \sum_{i=1}^p \beta_i \sigma_{t-i}^2
  \end{aligned}
\end{equation}
where $\epsilon_t$ is termed the innovation of the return $r_t$ and is
a random variable of a prescribed distribution, the most common ones
being Gaussian, Student's t, Asymmetric Student's t, etc; $\sigma_t^2$
is the variance of $\epsilon_t$ \cite{Bollerslev86}. The
autocorrelation function of $\epsilon_t^2$ is given by
(c.f. \cite{Bollerslev87})
\begin{eqnarray*}
\rho_n &=& \sum_{i=1} ^{p \vee q} (\alpha_i + \beta_i) \rho_{n-i}
\text{ for $n > p$}
\end{eqnarray*}
where $\alpha_i$ with $i > q$ and $\beta_i$ with $i > p$ are taken as
zeros. $p \vee q$ denotes the maximum of p and q. From these
equations, it is clear that the partial autocorrelation function cuts
off at $\max(p, q)$.

\item {\bf Stochastic Volatility Models}\label{sec:SLV_model}
For the purpose of intraday returns, we specify the stochastic
volatility model as
\begin{eqnarray}
   r_{t, t-h} &=& \ln p_{t} - \ln p_{t-h} \nonumber \\
   r_{t, t-h} &=& \mu + \sigma_{t, t-h} b_t \label{eq:SLV_spec}
\end{eqnarray}
where $p_t$ is the price of the asset at time $t$; $b_t \sim N(0,
1)$; $r_{t, t-h}$ is the return in the time interval $[t-h, t]$.
For simplicity, in the rest of this chapter we shall just write
$t$ for the subscript ``t, t-h'', since the time interval $h$ is fixed
for each time series and is a known constant.

Andersen and Bollerslev et al proved the following in
\cite{Andersen03} (theorem 2):
\begin{equation}
  \label{eq:normal_r}
  r_t|\mathcal{F}_{t-h} \sim N(\int_{0}^h \mu_{t-h+s} ds, \int_{0}^h
  \sigma_{t-h+s}^2 ds)
\end{equation}
In plain words, given all the information up to time $t-h$, the
distribution of $r_t$ is Gaussian with integrated mean and variance.
The variance of this conditional distribution, $\int_{0}^h
\sigma_{t-h+s}^2 ds$, can be approximated by \cite{Protter05,
  Andersen03}:
\begin{equation}
  \label{eq:rv_def}
  \int_{0}^h \sigma_{t-h+s}^2 ds = \sum_{k=1}^n \left(
    p_{t-h+kh/n} - p_{t-h+(k-1)h/n} \right)^2
\end{equation}
where $n$ is a chosen constant. The square root of the right hand side
of equation \ref{eq:rv_def} is the realized volatility, call it
$\hat{\sigma}_t$.

With the availability of transaction data, estimating conditional
volatility using returns sampled at a higher frequency (realized
volatility) gives superior accuracy and reliability. However, at which
frequency the time series should be sampled (the choice of $n$) in
order to give an unbiased and consistent estimate of the volatility is
not a trivial question. Naively one would believe that the often the
series is sampled, the better the estimate, but in fact, due to noise
introduced by market micro-structure, there is an optimal sampling
frequency, depending on $h$. While the method to determine this
optimal frequency is a subject of debate (see for example
\cite{Sahalia05}), it is not difficult to find a fairly satisfactory
frequency in practice:

Keeping equation \ref{eq:normal_r} in mind, one can simply try a few
frequencies and compare the distribution of $(r_t -
\E(r_t))/\hat{\sigma}_t$ with the standard Gaussian. If the two match,
$\hat{\sigma}_t$ is a good approximation of
\[
\sigma_t = \left(\int_{0}^h \sigma_{t-h+s}^2 ds \right)^{1/2}
\]

\end{enumerate}

To compare GARCH and stochastic log-volatility models at the face of
intraday returns, we study the {\it Nordea Bank} 15-minute returns
during the period 2012/01/16 - 2012/04/20 in \S\ref{chp:nordea_15min},
and {\it Volvo B} 30-minute returns during 2013/10/10 - 2014/04/04.

\section{Case Study: Nordea in 1st quarter 2012}
\label{chp:nordea_15min}
In this section we investigate the Nordea 15-minute returns sampled
during the period 2012/01/16 - 2012/04/20. In total, these amount to
2022 returns. We use the first 80\% (1617) for model estimation and  the
remaining 20\% (405) for comparing with model forcasts. In section
\ref{sec:nordea_15min_garch} we study the series with a GARCH model 
and in section \ref{sec:nordea_15min_arima} we study it with a
stochastic volatility model.

\subsection{GARCH Model}\label{sec:nordea_15min_garch}
First of all, to check for GARCH effects, we plot the auto-correlation
function (ACF) of the squared returns. This is shown in figure
\ref{fig:nordea_15min_acf}. At the absence of GARCH effects, the ACF
is expected to have an asymptotic Gaussian distribution with mean 0
and variance 1/T \cite{Bollerslev86, Bollerslev87}, which is clearly
not the case in figure \ref{fig:nordea_15min_acf} --- the first 5
autocorrelations have comparable sizes and do not fall off as in a
Gaussian scheme. Moreover, figure \ref{fig:nordea_15min_vlt_acf} shows
even more clearly that the conditional variances of the series are
correlated. These observations suggest a GARCH(p, q) model can be
appropriate.

Starting with a GARCH(1,1) model and taking advantage of the knowledge
that the log-volatility $\ln \sigma_t$ has seasonality $s=33$, we fit
to the return series a GARCH(33, 33) model, limiting to lags 1 and 33
for both ARCH and GARCH parameters.
\begin{eqnarray*}
  r_t &=& \mu + €_t \\
  €_t &=& \sigma_t z_t \\
  \sigma^2_t &=& \alpha_0 + \alpha_1 €^2_{t-1} + \alpha_s €^2_{t-s} +
  \beta_1 \sigma^2_{t-1} + \beta_s \sigma^2_{t-s}
\end{eqnarray*}
Via maximum likelihood estimation, parameter values listed in table
\ref{tab:nordea_15min_garch} are obtained.
\begin{table}[htb!]
  \centering
  \begin{tabular}{|c|c|c|c|c|c|}
    \hline
    Parameter & $\alpha_0$ & $\alpha_1$ & $\alpha_s$ & $\beta_1$ &
    $\beta_s$ \\
    \hline
    Value & $4.7833 \times 10^{-7}$ & 0.1600 & 0.0667 & 0.6846 &
    0.0342 \\
    \hline
  \end{tabular}
  \caption{\small \it GARCH model parameters}
  \label{tab:nordea_15min_garch}
\end{table}

\subsection{Stochastic Volatility Model}\label{sec:nordea_15min_arima}
For the Nordea Bank 15-minute returns under consideration, it can be
verified that the square root of the sum of squared 30-second returns
makes a good proxy for the volatility. This can be seen from the
probability plot of $z_t = (r_t - \E(r_t))/\hat{\sigma}_t$ (figure
\ref{fig:nordea_bank_15min_z_prob}), i.e. the quotient of the 
15-minute returns over the volatility proxy.
\begin{figure}[htb!]
  \centering
    \includegraphics[scale=0.4, clip=true, trim=80 258 104
    220]{../pics/nordea_bank_15min_z_prob.pdf}
  \caption{\small \it{Probability plot of $z_t =
      (\epsilon_t-\E(\epsilon_t))/\sigma_t$. $\epsilon_t$ are derived
      from Nordea Bank 15min returns while $\sigma_t$ are realized
      volatilities calculated
      using 30s returns within each 15min interval. Horizontal axis:
      $z_t$}; Vertical axis: cummulative probability function (CDF) of
    $z_t$, arranged on such a scale that the CDF of the standard
    Gaussian is a straight line.}
  \label{fig:nordea_bank_15min_z_prob}
\end{figure}
% In addition, one can see
% from figure \ref{fig:nordea_15min_quotient_acf} and
% \ref{fig:nordea_15min_quotient_squared_acf} that there is essentially
% no auto-correlation in the $z_t$ or the $z_t^2$ series.
% \begin{figure}[htb!]
%   \centering
%   \subfigure[ACF of $z_t$]{
%     \includegraphics[scale=0.4, clip=true, trim=95 236 118
%     200]{../pics/nordea_15min_quotient_acf.pdf}
%     \label{fig:nordea_15min_quotient_acf}
%   }
%   \subfigure[ACF of $z_t^2$]{
%     \includegraphics[scale=0.4, clip=true, trim=95 236 118
%     200]{../pics/nordea_15min_quotient_squared_acf.pdf}
%     \label{fig:nordea_15min_quotient_squared_acf}
%   }
%   \caption{\small \it Nordea 15min $z_t$ and $z_t^2$ ACF.}
% \end{figure}

Andersen and Bollerslev et al reported that, for the exchange
rates between Deutch mark, yen and dollar, $\ln \sigma_t$ is
gaussian distributed \cite{Andersen03}. This is, however, not the case
for our series in question. In fact, in our case, $\ln \sigma_t$ is
right skewed (skewness 0.3342) and leptokurtic (kurtosis 
6.1006). See figure \ref{fig:nordea_15min_logvol_prob}
\begin{figure}[htb!]
  \centering
  %\vspace{-15mm}
  \includegraphics[scale=0.4, clip=true, trim=80 223 107
  4]{../pics/nordea_15min_logvol_prob.pdf}
  \caption{\small \it Probability plot of Nordea 15min
    $\ln\sigma_t$ unconditional distribution}
  \label{fig:nordea_15min_logvol_prob}
\end{figure}
Moreover, the series of $\ln\sigma_t$ shows long-lasting and
periodic autocorrelations with an apparent period of 33 (see figure
\ref{fig:nordea_15min_logvol_acf}). This suggests the series may be
described by a seasonal ARIMA model\footnote{ARIMA ---
  Integrated Autoregressive Moving Average} model. This we first
simplify the series by differencing \cite{BoxJenkins94}:
\[
w_t = (1-B)(1-B^s)\ln\sigma_t
\]
where $B$ is the back-shift operator and $s=33$ is the seasonality.

The autocorrelation function of the differenced process $w_t$, as
shown in figure \ref{fig:nordea_15min_w_acf}, clearly points to a seasonal
moving-average model: There are only 4 non-zero autocorrelations in
the plot, located at lags 1, 32, 33, 34, respectively; furthermore,
the two at 32 and 34 are approximately equal. Thus we can write down
the model as
\begin{eqnarray}
  w_t &=& (1 - \theta B)(1 - \Theta B^s) y_t \label{eq:nordea_w}
\end{eqnarray}
where $\theta$ and $\Theta$ are parameters to be determined and $y_t$
is a noise process with constant variance $\sigma_y^2$ and mean
0. $y_t$ is often refered to as the residuals.
\begin{figure}[htb!]
  \centering
  \subfigure[ACF of $\ln\sigma_t$]{
    \includegraphics[scale=0.4, clip=true, trim=95 230 112
    235]{../pics/nordea_15min_logvol_acf.pdf}
    \label{fig:nordea_15min_logvol_acf}
  }
  \subfigure[ACF of $w_t$]{
    \includegraphics[scale=0.4, clip=true, trim=95 230 112
    235]{../pics/nordea_15min_w_acf.pdf}
    \label{fig:nordea_15min_w_acf}
  }
  \caption{\small \it Nordea 15min $\ln\sigma_t$ and $w_t$
    autocorrelations}
  \label{fig:nordea1_15min_acf}
\end{figure}

The above seasonal moving average model has the following
autocovariance structure \cite{BoxJenkins94}:
\begin{eqnarray*}
  \gamma_0 &=& \sigma_y^2 (1 + \theta^2)(1 + \Theta^2) \\
  \gamma_1 &=& -\sigma_y^2\theta(1 + \Theta^2) \\
  \gamma_s &=& -\sigma_y^2\Theta(1 + \theta^2) \\
  \gamma_{s+1} &=& \gamma_{s-1}\;=\;\sigma_y^2\theta\Theta
\end{eqnarray*}
These equations together with the measured autocorrelations make
possible an initial estimate of the parameters $\theta$ and $\Theta$:
\begin{eqnarray*}
  {\rho_{s+1}/\rho_s} &=& {\gamma_{s+1}/\gamma_s} \;=\; -{\theta \over
    1 + \theta^2} \\
  {\rho_{s+1}/\rho_1} &=& {\gamma_{s+1}/\gamma_1} \;=\; -{\Theta \over
    1 + \Theta^2} \\
\end{eqnarray*}
Substituting in the measured values shown in table
\ref{tab:nordea_15min_w_acf},
\begin{table}[htb!]
  \centering
  \begin{tabular}{|c|c|c|c|}
    \hline
    $\rho_1$ & $\rho_{s-1}$ & $\rho_s$ & $\rho_{s+1}$ \\
    \hline
    -0.4703 &  0.2053 & -0.4564 &  0.2212 \\
    \hline
  \end{tabular}
  \caption{\small \it Nordea 15min $w_t$ autocorrelations}
  \label{tab:nordea_15min_w_acf}
\end{table}
we get
\begin{eqnarray*}
  \theta &=& 0.6890 \\
  \Theta &=& 0.6378
\end{eqnarray*}
Among the two roots of each of the 2nd order equations in the above,
we have chosen the one in the range $(-1, 1)$ so as to ensure
invertibility of the model \cite{BoxJenkins94}.

With an estimate of $\theta$ and $\Theta$, one can then infer the
noise process i.e. the residuals $y_t$:
\begin{equation}
  \label{eq:infer_y}
  y_t = w_t + \theta y_{t-1} + \Theta y_{t-s} - \theta \Theta y_{t-s-1}
\end{equation}
where we substitute $y_t\;(t \leq 0)$ with their unconditional
expectation 0.

In order to forecast the $w_t$ process, and hence the return process
itself, we must also know the distribution of $y_t$. Moreover, to
properly estimate the parameters of the model in the sense of maximum
likelihood, we are also in need of the distribution of $y_t$.

Figure \ref{fig:nordea_15min_y_qq} shows the normal probability plot of
$y_t$. It is evident from this figure that $y_t$ has fat tails.
\begin{figure}[htb!]
  \centering
  \includegraphics[scale=0.4, clip=true, trim=78 255 109
  123]{../pics/nordea2_y_normplot.pdf}
  \caption{\small \it Nordea 15min $y_t$ normal probability plot.}
  \label{fig:nordea_15min_y_qq}
\end{figure}
In addition, a simple calculation reveals that the distribution of
$y_t$ has skewness 0.2988 (shown in table
\ref{tab:nordea_15min_y_moments}).
\begin{table}[htb!]
  \centering
  \begin{tabular}{|c|c|c|c|}
    \hline
    mean & variance & skewness & kurtosis \\
    \hline
    0.0012 & 0.0935 & 0.2988 & 6.8691 \\
    \hline
  \end{tabular}
  \caption{\small \it Nordea 15min $y_t$ moments}
  \label{tab:nordea_15min_y_moments}
\end{table}
Based on this information, we find that $y_t$ can be well
described by a Johnson Su distribution \cite{Shang2004}:
\[
  y_t = \xi + \lambda\sinh{z_t - \gamma \over \delta}
\]
where $\gamma, \delta, \lambda, \xi$ are parameters to be determined
and $z_t \sim N(0, 1)$. The goodness of fitting is demonstrated in
figure \ref{fig:nordea_15min_y_js_fit} by the
empirical cummulative distribution function in comparison to the
theoretical one.
\begin{figure}[htb!]
  %\vspace{-18mm}
  \centering
    \includegraphics[scale=0.4, clip=true, trim=92 229 116
    133]{../pics/nordea_15min_y_js_fit.pdf}
    \caption{\small \it Nordea 15min residuals $y_t$ fitted to a
      Johnson Su distribution. Horizontal: values of $y_t$, denoted x;
      Vertical: $\ln\left(P(y_t < x)\right)$.}
    \label{fig:nordea_15min_y_js_fit}
\end{figure}

The first 4 moments of the Johnson Su distribution are expressible
in closed form in $\gamma, \delta, \lambda, \xi$ \cite{Shang2004}:
\begin{eqnarray*}
  w &=& \exp{1 \over \delta^2} \\
  \Omega &=& {\gamma \over \delta} \\
  \text{E}(y) &=& -w^{1/2} \lambda \sinh\Omega + \xi\\
  \text{std}(y) &=& \lambda \left[{1 \over 2}(w-1)(w\cosh 2\Omega +
    1)\right]^{1/2} \\
  \text{skewness}(y) &=& {
    \sqrt{(1/2)w(w-1)} [w(w+2)\sinh 3\Omega + 3\sinh\Omega]
    \over
    (w\cosh 2\Omega + 1)^{3/2}} \\
  \text{kurtosis}(y) &=& {
    w^2(w^4 + 2w^3 + 3w^2 - 3)\cosh 4\Omega + 4w^2 (w+2) \cosh 2\Omega
    + 3(2w+1) \over
    2(w\cosh 2\Omega + 1)^2 }
\end{eqnarray*}
By matching the theoretical expressions of the moments with their
measured values, and taking help from published tables
\cite{Johnson1965}, one can solve for the parameters $\gamma, \delta,
\lambda, \xi$.

Under the assumption of i.i.d Johnson Su distributed residuals, the
log-likelihood function of the parameters $\gamma, \delta$
conditional on the sample $w_t$ can be written as
\[
L(\theta, \Theta) = -{1 \over 2}\sum_{t=1}^n z_t^2 + n \ln{\delta
  \over \lambda \sqrt{2\pi}} - {1 \over 2}\sum_{t=1}^n \ln\left[
  1 + \left({y_t - \xi \over \lambda}\right)^2
\right]
\]
where $y_t$ are inferred from $w_t$ using eq.\ref{eq:infer_y}
and $z_t$ from $y_t$ using
\[
z_t = \delta \sinh^{-1}{y - \xi \over \lambda} + \gamma
\]
Note that $\gamma, \delta, \lambda, \xi$ are not really free
parameters but rather are implied by $\theta$ and $\Theta$: Once the
latter have been chosen and the corresponding $y_t$ inferred, the
former are determined by the moments of $y_t$.

The eventual MLE is done in Matlab with the ``active set''
algorithm. The initial as well as the final estimation results are
listed in table \ref{tab:nordea_15min_js_param}:
\begin{table}[htb!]
  \centering
  \begin{tabular}{|c|c|c|c|c|c|c|}
    \hline
    & $\gamma$ & $\delta$ & $\lambda$ & $\xi$ & $\theta$ & $\Theta$ \\
    \hline
    initial estimate & 0.1476 & 1.5121 & 0.3633 & 0.0454 & 0.6890 &
    0.6378 \\
    \hline
    MLE estimate & 0.1319 & 1.5266 & 0.3735 & 0.0410 & 0.6639 & 0.6025
    \\
    \hline
  \end{tabular}
  \caption{\small \it Nordea 15min estimation results}
  \label{tab:nordea_15min_js_param}
\end{table}

\subsection{Comparison of the Forecasts}
In this section we compare the one-step-ahead forecasts from the GARCH
model and from the stochastic volatility model. For this purpose, we
compute the difference between a forecast $\ln \sigma^F_t$ and its measured
counterpart, i.e. the realized volatility of the same period $\ln
\hat{\sigma}_t$.

First of all, we look at the means and standard deviations of $\ln \sigma^F_t -
\ln \hat{\sigma}_t$, which are listed in table \ref{tab:nordea_2012}.
\begin{table}[htb!]
  \centering
  \begin{tabular}{|c|c|c|c|}
    \hline
    & SV & GARCH & Sample mean \\
    \hline
    $\E(\ln \sigma^F_t - \ln \hat{\sigma}_t)$ & 0.0040 & -0.0008 &
    -0.2210 \\
    \hline
    $\text{std}(\ln \sigma^F_t - \ln \hat{\sigma}_t)$ & 0.2659 & 0.3011 &
    0.2893 \\
    \hline
  \end{tabular}
  \caption{\small \it Mean and standard Deviation of the forecasts'
    distribution. ``Sample mean`` refers to the mean of the realized
    volatilities in the first 80\% of the data set. ``SV'': Stochastic
    volatility.}
  \label{tab:nordea_2012}
\end{table}
It is seen from table \ref{tab:nordea_2012} that, on average, the SV
model over-estimates while GARCH under-estimates. In terms of the
standard deviation of $\ln \sigma^F_t - \ln \hat{\sigma}_t$, the SV
model wins with a small margin. In contrast, the sample mean forecast
clearly under-estimates the log-volatilities to a large extent --- the
efforts of building models has not been wasted.

Figure \ref{fig:nordea_2012} compares the 3 kinds of forecasts by
plotting the distribution function and the complementary distribution
function of $\ln \sigma^F_t - \ln \hat{\sigma}_t$. Here one can see
that the SV model yields a better quality of forecasts than does GARCH 
with regard to the under-estimates, and essentially the same quality
with regard to the over-estimates.

\begin{figure}[htb!]
  \centering
    \includegraphics[scale=0.55, clip=true, trim=27 287 0
    154]{../pics/nordea_2012.pdf}
  \caption{\small \it Blue: SV forecasts; Green: GARCH forecasts; Red:
    sample mean forecasts. Left: cumulative distribution function;
    Right: complementary distribution function.}
  \label{fig:nordea_2012}
\end{figure}

Another measure of the forecasts' quality can be the percentage of good
forecasts, where the criterion of ``good'' is defined, respectively,
as deviating no more than 1\%, 5\%, or 10\% from the measured value of
the realized volatility. Table \ref{tab:nordea_2012_good} shows the
respective percentage of the 3 kinds of forecasts. Again in this table
it is seen that the SV model gives more accurate forecasts than does
GARCH.
\begin{table}[htb!]
  \centering
  \begin{tabular}{|c|c|c|c|}
    \hline
    ${\ln \sigma^F_t - \ln \hat{\sigma}_t \over |\ln \hat{\sigma}_t|}$
    & SV & GARCH & sample mean \\
    \hline
    1\% & 0.2173 & 0.1383 & 0.1136 \\
    \hline
    5\% & 0.7457 & 0.7062 & 0.5333 \\
    \hline
    10\% & 0.9728 & 0.9383 & 0.8914 \\
    \hline
  \end{tabular}
  \caption{\small \it The percentage of ``good'' forecasts when the
    criterion of being good is deviating no more than 1\%, 5\% or
    10\%.}
  \label{tab:nordea_2012_good}
\end{table}

\section{Case Study: Volvo in 2013/14}
\label{sec:volvo}
In this section we model the log-volatility of {\it Volvo B} 30-minute
returns during the period 2013/10/10 - 2014/04/04. This series
contains 1884 log-volatilities computed using 2-minute returns in each
30-minute interval. Among them we use the first 1507 for model
estimation and the last 377 for forecast and model verification.

The left plot of figure \ref{fig:volvo_inno_and_lv_acf} shows the
distribution of $(r_t - \E(r_t))/\sigma_t$. We observe in the figure a
nice Gaussian variate, so we can be sure that the sum of squared
2-minute returns makes a good approximation to the variance of
30-minute returns in this particular case.
\begin{figure}[htb!]
  \centering
    \includegraphics[scale=0.5, clip=true, trim=0 256 0
    151]{../pics/volvo_inno_and_lv_acf.pdf}
    \label{fig:volvo_inno_and_lv_acf}
  \caption{\small \it Left: probability plot of $(r_t -
    \E(r_t))/\sigma_t$; Right: auto-correlations of $\ln \sigma_t$}
\end{figure}

Guided by the auto-correlations of $\ln\sigma_t$ shown in the right
plot of figure \ref{fig:volvo_inno_and_lv_acf} we find the following
model:
\begin{eqnarray}
  && (1 - B)(1 - B^s) \ln \sigma_t \nonumber \\
  &=& (1 - \theta_1 B - \theta_2B^2 -
  \theta_3B^3 - \theta_4B^4) (1 - \Theta B^s)
  y_t \label{eq:volvo_lv_model}
\end{eqnarray}
where $s = 16$ is the seasonality and is apparent from the
auto-correlations of $\ln\sigma_t$. Fitting this model to the
measured realized volatilities yields the parameter values listed in
table
\ref{tab:volvo_params}.
\begin{table}[htb!]
  \centering
  \begin{tabular}{|c|c|c|c|c|c|c|}
    \hline
    Parameter & $\theta_1$ & $\theta_2$ & $\theta_3$ & $\theta_4$ &
    $\Theta$ & residual variance \\
    \hline
    Value & 0.7305 & 0.0575 & 0.0574 & 0.0346 & 0.8324 & 0.1340\\
    \hline
  \end{tabular}
  \caption{\small \it Volvo B log-volatility parameters}
  \label{tab:volvo_params}
\end{table}
Forecasting using the estimated model parameters gives the forecast
series $\ln \sigma^{\text{SV}}_t$. To access the quality of the
forecast, we also estimate a GARCH model using the returns. The result
is a GARCH(1, 1) model, whose parameter values are listed in table
\ref{tab:volvo_garch}.
\begin{table}[htb!]
  \centering
  \begin{tabular}{|c|c|c|c|}
    \hline
    Parameter & $\alpha_0$ & $\alpha_1$  & $\beta_1$ \\
    \hline
    Value & $3.125 \times 10^{-7}$ & 0.05 & 0.90 \\
    \hline
  \end{tabular}
  \caption{\small \it GARCH(1, 1) model of Volvo B 30-minute returns}
  \label{tab:volvo_garch}
\end{table}

The forecasts from SV, GARCH, and the sample mean are compared
using the difference $\ln \sigma^F_t - \ln\hat{\sigma}_t$, where $\ln
\sigma^F_t$ stands for the forecast. The distributions of this
difference is plotted in figure \ref{fig:volvo_slv_garch_cmp}; the
mean and the standard deviation of the distributions are listed in
table \ref{tab:volvo_slv_garch_cmp}.
\begin{table}[htb!]
  \centering
  \begin{tabular}{|c|c|c|c|}
    \hline
    & SV & GARCH & Sample mean \\
    \hline
    $\E(\ln \sigma^F_t - \ln \hat{\sigma}_t)$ & -0.0123 &
    0.0242 & -0.1055 \\
    \hline
    $\text{std}(\ln \sigma^F_t - \ln \hat{\sigma}_t)$ & 0.3261 &
    0.4250 & 0.3708 \\
    \hline
  \end{tabular}
  \caption{\small \it Standard deviation of $\ln\sigma^F_t -
    \ln\hat{\sigma}_t$}
  \label{tab:volvo_slv_garch_cmp}
\end{table}

\begin{figure}[htb!]
  \centering
    \includegraphics[scale=0.5, clip=true, trim=10 282 0
    243]{../pics/volvo_slv_garch_cmp.pdf}
    \label{fig:volvo_slv_garch_cmp}
  \caption{\small \it Comparison of the forecast from SV, GARCH,
    and sample mean. Blue: SV forecast; Green: GARCH forecast; Red:
    Sample mean forecast.}
\end{figure}
Figure \ref{fig:volvo_slv_garch_cmp} shows, as in the previous case of
Nordea Bank 15-minute returns, the SV model performs the best, GARCH
the second, and the sample mean the worst. However, when it comes to
over-estimates, the sample mean appears to be the best estimator,
while SV excels over GARCH. But a check of $\E(\ln\sigma_t)$ over
the sample for estimation (1507 data points) and over the sample for
comparison (377 data points) reveals that the first sample has mean
-6.3127 while the second has -6.2072. This increment explains the low
probability of over-estimation when using the first sample mean as
forecast. 

Table \ref{tab:volvo_good_percentage} compares the fraction of
``good'' estimates as measured by ${|\ln\sigma^F_t -
  \ln\hat{\sigma}_t| \over |\ln \hat{\sigma}_t|}$ being less than 1\%,
5\% and 10\%.
\begin{table}[htb!]
  \centering
  \begin{tabular}{|c|c|c|c|}
    \hline
    ${|\ln \sigma^F_t - \ln \hat{\sigma}_t| \over |\ln
      \hat{\sigma}_t|}$ &
    SV & GARCH & Sample Mean \\
    \hline
    1\% & 0.2175 & 0.1220 & 0.1406 \\
    \hline
    5\% & 0.7162 & 0.6154 & 0.6631 \\
    \hline
    10\% & 0.9231 & 0.8806 & 0.8992 \\
    \hline
  \end{tabular}
  \caption{\small \it Fraction of ``good'' forecasts as defined by
    ${|\ln \sigma^F_t - \ln \hat{\sigma}_t| \over |\ln
      \hat{\sigma}_t|}$ being less than 1\%, 5\% and 10\%.}
  \label{tab:volvo_good_percentage}
\end{table}
We see from the table that the SV model consistently excels over
the other two alternatives.

\section{Unconditional Distribution Functions of SV Models}
\label{sec:XieCalc}
In this section we study the unconditional distribution function of
the Stochastic Log-Volatility model specified as equation
\ref{eq:SLV_spec}. As is discussed in \S\ref{sec:SLV_model},
$\ln\sigma_t$ can be described by an ARMA or ARIMA \footnote{ARIMA ---
  Integrated Autoregressive Moving Average} model, possibly with
seasonal components. Here we note that all these models can be
re-written as a moving average model, which is infinite in extent if
autoregressive components are present:
\begin{eqnarray*}
  \ln \sigma_t &=& y_t + \sum_{n=1}^\infty c_n y_{t-n} + \text{Const.}
\end{eqnarray*}
Since the $y_t$ are independent and identically distributed,
\begin{eqnarray*}
  y_t + \sum_{n=1}^\infty c_n y_{t-n}  
\end{eqnarray*}
has Gaussian distribution by the central limit theorem, on condition
that $y_t$ for all $t$ have finite second moment --- this is what we
assume in the rest of this section. It follows from the above equation
that the unconditional distribution of $\ln \sigma_t$ is the same as the
distribution of $\bar{v} + v$ where $v \sim N(0, \sigma)$ and
$\bar{v}$, $\sigma$ are constants. Now we can state that the
unconditional distribution of the returns $r_t$
\begin{eqnarray*}
  r_t &=& \mu + \sigma_t b_t\\
  &=& \mu + \exp\left(
    y_t + \sum_{n=1}^\infty c_n y_{t-n} + \text{Const.}
  \right)
\end{eqnarray*}
 is the same as
\begin{equation}  \label{eq:UnconditionalPdf}
  \begin{aligned}
    r &= \mu + e^{\bar{v} + v} b \\
  \end{aligned}
\end{equation}
where $b \sim N(0, 1)$. For convenience, let $r' = e^v b$

In \S\ref{sec:SLV_Symmetric} we first study the model in the
relatively simple case when $v$ and $b$ are uncorrelated and $\mu
= 0$. If this simplified version proves inadequate, one may resort to
the general model studied in \S\ref{sec:SLV_Asymmetric}.

\subsection{The Simplified model}\label{sec:SLV_Symmetric}
In the following we derive the unconditional PDF of $r'$, denoted
$f_{r'}(x)$. Then the PDF of $r$ is $e^{-\bar{v}}f_{r'}(e^{-\bar{v}}x)$.
\begin{eqnarray*}
  P(r' < x) &=& P(b < xe^{-v}) \\
  f_{r'}(x) &=& f_b(xe^{-v}) e^{-v}
\end{eqnarray*}
Averaging over all $v$, we get
\begin{equation}\label{eq:UncondPDFSymmetric}
  \begin{aligned}
    f_{r'}(x) =& \int_{-\infty}^{\infty} dv (2\pi\sigma^2)^{-1/2}
    e^{-v^2/2\sigma^2}(2\pi)^{-1/2} \exp(-x^2e^{-2v}/2) e^{-v} \\
    =& {1 \over 2\pi\sigma} \int_{-\infty}^{\infty} dv
    \exp\left(
      -{1 \over 2\sigma^2} v^2 - v -{1 \over 2} x^2 e^{-2v}
    \right)
  \end{aligned}
  \end{equation}
The last part of the integrand, $e^{-x^2 e^{-2v} / 2}$, is plotted in
figure \ref{fig:DoubleExp}.
\begin{figure}[htb!]
  \centering
  \includegraphics[scale=0.5, clip=true, trim=85 252 100
  231]{../pics/DoubleExp.pdf}
  \caption{\small \it Plot of $\exp(-{1 \over 2} x^2 e^{-2v})$}
  \label{fig:DoubleExp}
\end{figure}
Therefore we make the following approximation:
\[
\exp\left(-{1 \over 2} x^2 e^{-2v}\right) \approx \left\{
  \begin{array}{lr}
    0 & \text{if } v < \ln|x| -{1 \over 2} \ln(2\ln 2) \\
    1 & \text{otherwise}
  \end{array}
\right.
\]
Here we note that $\exp\left(-{1 \over 2} x^2 e^{-2v}\right) = 1/2$ at
$v = \ln|x| -{1 \over 2} \ln(2\ln 2)$.

With this approximation we have
\begin{eqnarray*}
  f_{r'}(x) &=& {1\over C}{1 \over 2\pi\sigma} \int_{\ln(|x|/\sqrt{\ln
      4})}^{\infty} dv
  \exp\left(-{1 \over 2\sigma^2} v^2 - v\right) \\
  &=& {1\over C}{e^{\sigma^2 / 2} \over \sqrt{8\pi}} \text{erfc} \left(
    {1 \over \sqrt{2}\sigma} \ln{|x| \over \sqrt{\ln 4}} + {\sigma
      \over \sqrt{2}}
  \right)
\end{eqnarray*}
where $1/C$ has been added for the purpose of normalization.

At large $x$, we may use the asymptotic expansion of $\mathrm{erfc}$
to write
\begin{eqnarray*}
  Cf_{r'}(x) &=& {e^{\sigma^2 / 2} \over \sqrt{8\pi}} \mathrm{erfc}(\xi) \\
  &=& {e^{\sigma^2 / 2} \over \sqrt{8 \pi}}
  \frac{e^{-\xi^2}}{\xi\sqrt{\pi}}\left[
    1 +
    \sum_{n=1}^N (-1)^n \frac{(2n-1)!!}{(2\xi^2)^n} \right] +
  O(\xi^{-2N-1} e^{-\xi^2})
\end{eqnarray*}
where $\xi = {1 \over \sqrt{2}\sigma} \ln{|x| \over \sqrt{\ln 4}} +
{\sigma \over \sqrt{2}}$. The slowest-decaying term is
\[
f_{r,0}(x) = {e^{\sigma^2 / 2} \over \sqrt{8\pi}}
\frac{e^{-\xi^2}}{\xi\sqrt{\pi}}
\]
Let $\zeta = {|x| \over \sqrt{\ln 4}}$. With a bit manipulation one
obtains
\begin{equation*}
  f_{r,0}(x) = {1 \over \pi \sqrt{8}}{1 \over
    \left(\ln\zeta/\sigma\sqrt{2} +
      \sigma/\sqrt{2}\right)\zeta\zeta^{\ln \zeta / 2\sigma^2}
  }
\end{equation*}
From the last equation one can see that, at any neighborhood of large
$x$, $f_{r'}(x)$ may be approximated by $C/|x|^\alpha$, i.e. a power law.

In the following we work out the normalization constant C.
\begin{eqnarray*}
  \int_{-\infty}^{\infty} f_{r'}(x) dx &=& {1 \over C}{e^{\sigma^2 / 2} \over
    \sqrt{8\pi}} \int_{-\infty}^{\infty} \text{erfc} \left({1 \over
      \sqrt{2}\sigma} \ln{|x| \over \sqrt{\ln 4}} + {\sigma \over
      \sqrt{2}} \right) dx\\
  &=& {2 \over C}{e^{\sigma^2 / 2} \over
    \sqrt{8\pi}} \int_{0}^{\infty} \text{erfc} \left({1 \over
      \sqrt{2}\sigma} \ln{x \over \sqrt{\ln 4}} + {\sigma \over
      \sqrt{2}} \right) dx\\
\end{eqnarray*}
Let
\begin{eqnarray*}
  a &=& {1 \over \sigma \sqrt 2} \\
  b &=& -{1 \over 2 \sigma \sqrt 2}\ln\ln 4 + {\sigma \over \sqrt 2} \\
  y &=& a\ln|x|+b \\
\end{eqnarray*}
Then
\begin{eqnarray*}
  \int_{-\infty}^{\infty} f_{r'}(x) dx &=& {2 \over C}{e^{\sigma^2 / 2} \over
    \sqrt{8\pi}} \int_{-\infty}^{\infty} dy {e^{(y-b)/a} \over a}
  \text{erfc}(y) \\
  &=& {2 \over C}{e^{\sigma^2 / 2} \over
    \sqrt{8\pi}} \left.e^{(y-b)/a}
    \text{erfc}(y)\right|_{y=-\infty}^{\infty}
  + {2 \over C}{e^{\sigma^2 / 2} \over
    \sqrt{8\pi}} \int_{-\infty}^{\infty} dy {2 \over \sqrt \pi}
  e^{(y-b)/a} e^{-y^2} \\
  &=& {2 \over C}{e^{\sigma^2 / 2} \over \sqrt{8\pi}}  2 e^{\ln\ln 4/2
    - \sigma^2/2}\\
  C &=& \sqrt{2 \ln 4\over \pi} \\
  &\approx& 0.9394
\end{eqnarray*}
Thus we can write in summary:
\begin{eqnarray*}
  f_r(x) &=& e^{-\bar{v}} f_{r'}(e^{-\bar{v}} x) \\
    &=& {e^{-\bar{v}} \over C}{e^{\sigma^2 / 2} \over \sqrt{8\pi}}
    \text{erfc} \left({1 \over \sqrt{2}\sigma} \ln{|e^{-\bar{v}}x| \over \sqrt{\ln
          4}} + {\sigma \over \sqrt{2}}
    \right) \\
\end{eqnarray*}
Using the same technique for integration as for normalization, the
cummulative distribution function of $r$ is found to be $F(x)$,
which is the following:
\begin{enumerate}
\item if $x < 0$
  \begin{eqnarray*}
    F(x) &=& {\sqrt{\ln 4} \over C}{e^{\sigma^2 / 2} \over \sqrt{8\pi}}
    \left[
      {e^{-\bar{v}}x \over \sqrt{\ln 4}}\text{erfc}\left(
        {1 \over \sigma \sqrt 2} \ln{-e^{-\bar{v}}x \over \sqrt{\ln
            4}} + {\sigma \over \sqrt 2}
      \right) \right.\\
      && \left. + e^{-\sigma^2 / 2} \text{erfc}\left(
        {1 \over \sigma \sqrt 2} \ln{-e^{-\bar{v}}x \over \sqrt{\ln 4}}
      \right)
    \right]
  \end{eqnarray*}
\item if $x \geq 0$
  \begin{eqnarray*}
    F(x) &=& \frac{1}{2} + {\sqrt{\ln 4} \over C}{e^{\sigma^2 / 2} \over
      \sqrt{8\pi}} \left[
      {e^{-\bar{v}}x \over \sqrt{\ln 4}} \text{erfc}\left(
        {1 \over \sigma \sqrt 2} \ln{e^{-\bar{v}}x \over \sqrt{\ln
            4}} + {\sigma \over \sqrt 2}
      \right) \right. \\
      && \left. + e^{-\sigma^2/2} \text{erfc}\left(
        -{1 \over \sigma \sqrt 2} \ln{e^{-\bar{v}}x \over \sqrt{\ln 4}}
      \right)
    \right]
  \end{eqnarray*}
\end{enumerate}

To verify the validity of the model, we fit the above probability
density function to the de-meaned 30min returns of Volvo B
\footnote{By ``de-meaned returns'' we mean the quantity $r_t - 
  \mean{r_t}$, where $r_t$ are the measured returns and $\mean{r_t}$
  is the sample mean. The data set covers the transaction records of
  Volvo B on the OMX market (Stockholm) between 2013-10-10 and
  2014-03-12. The returns are computed using 1-minute mean prices.}
by means of maximum likelihood estimate using the MATLAB function
``mle''. Then for the parameters $\sigma$ and $\bar{v}$ we get
\begin{eqnarray*}
  \sigma &=& 0.6355 \\
  \bar{v} &=& -6.0308
\end{eqnarray*}
Then we plot $P(r' > x)$ of the model against its empirical
counterpart on a log-log scale, as shown in figure
\ref{fig:volvo_30min_ret}.
\begin{figure}[htb!]
  % \vspace{-15mm}
  \begin{center}
    \includegraphics[scale=0.4, clip=true, trim=98 231 116
    126]{../pics/volvo_30min_ret.pdf}
  \end{center}
  %\vspace{-5mm}
  \caption{\small \it{$P(r' > x)$ for $x > 0$. Blue: empirical
      probabilities. Red: probabilities predicted by the model.}}
  \label{fig:volvo_30min_ret}
\end{figure}

In general figure \ref{fig:volvo_30min_ret} shows a good fit, but a
closer look reveals that deviations are significant in the regions $r
\in (0, \sigma_r)$ and $r \in (2\sigma_r, 3\sigma_r)$, where
$\sigma_r$ stands for the empirical standard deviation of the
de-meaned returns. These observations are
shown in greater details in figure \ref{fig:volvo_30min_ret2}.
\begin{figure}[htb!]
  \centering
  \subfigure[$\ln(P(r > x))$ with $x \in (0, \sigma_r)$]{
    \includegraphics[scale=0.4, clip=true, trim=93 224 115
    125]{../pics/volvo_30min_ret_0-sigma.pdf}
  }
  \subfigure[$\ln(P(r > x))$ with $x \in (2\sigma_r, 3\sigma_r)$]{
    \includegraphics[scale=0.4, clip=true, trim=93 224 115
    125]{../pics/volvo_30min_ret_2sigma-3sigma.pdf}
  }
  \caption{\small \it Surviving probabilities. Blue: empirical values of $\ln(P(r
    > x))$. Red: Predicted values of $\ln(P(r > x))$.}
  \label{fig:volvo_30min_ret2}
\end{figure}

Moreover, the inefficiency of the model is also manifest in the
skewness of the data. For the Volvo 30min returns, the data has a
skewness of 0.2419, but our model is strictly symmetric, since
$x$ appears in equation \ref{eq:UncondPDFSymmetric} only as $x^2$.
Hence the above model needs to be improved to accommodate the non-zero
skewness as well as to account for the discrepancies shown in figure
\ref{fig:volvo_30min_ret2}. This is the subject of the next section.

\subsection{The General model}\label{sec:SLV_Asymmetric}
It has long been hypothesized in the liturature that skewness is the
result of price-volatility correlation (for example
\cite{Potters2003}). Therefore, the most apparent modification is to
allow $v$ and $b$ to be correlated. For convenience we decompose
$v \sim N(0, \sigma)$ as $v = \sigma a$ and assume
\begin{eqnarray*}
  \begin{pmatrix}
    a \\
    b
  \end{pmatrix} \sim N(0, \Sigma)
\end{eqnarray*}
with a covariance matrix
\begin{eqnarray*}
    \Sigma &=&
  \begin{pmatrix}
    1 & \psi \\
    \psi & 1
  \end{pmatrix}
\end{eqnarray*}
where $|\psi| < 1$. Then we rewrite equation \ref{eq:UnconditionalPdf}
as
\begin{equation}
  \label{eq:r_t}
  \begin{aligned}
    r &= \mu + e^{\bar{v}} r' \\
    r' &= e^{\sigma a} b \\
  \end{aligned}
\end{equation}
Then
\begin{eqnarray*}
  && P(r' < x)\\
  &=& P(b < e^{-a\sigma}x) \\
  &=& {1 \over 2\pi |\det(\Sigma)|^{1/2}}
  \int_{-\infty}^{\infty} da \int_{-\infty}^{e^{-a\sigma}x} db
  \exp\left[
    -{1 \over 2} (a, b) \Sigma^{-1}
    \begin{pmatrix}
      a \\
      b
    \end{pmatrix}
  \right] \\
\end{eqnarray*}
After some manipulations we get
\begin{equation}\label{eq:UncondCDFAsymmetric}
  \begin{aligned}
    & P(r' < x) \\  
    &= {1 \over 2\pi \sqrt{1 - \psi^2}} \int_{-\infty}^{\infty} da
    e^{-a^2/2} \int_{-\infty}^{e^{-a\sigma}x} db
    \exp\left[
      - {(b - a\psi)^2 \over 2(1 - \psi^2)}
    \right] \\
    &= {1 \over 2\sqrt{2\pi}} \int_{-\infty}^{\infty} e^{-a^2/2}
    \text{erfc}{a\psi - e^{-a\sigma} x \over \sqrt{2(1-\psi^2)}} da
  \end{aligned}
\end{equation}
Differentiating with respect to $x$ yields
\begin{eqnarray}
  f_{r'}(x) &=& {1 \over 2\pi \sqrt{1 - \psi^2}} \times \nonumber \\
  && \int_{-\infty}^\infty da \exp
  \left[- {
      a^2 + 2\sigma(1 - \psi^2)a - 2a\psi e^{-a\sigma} x + e^{-2a\sigma} x^2
      \over
      2(1 - \psi^2)
  }
  \right] \label{eq:UncondPDFAsymmetric} \\
  f_r(x) &=&  \td{r'}{r} f_{r'}[e^{-\bar{v}} (x - \mu)] \nonumber \\
  &=& e^{-\bar{v}} f_{r'}[e^{-\bar{v}}
  (x-\mu)] \label{eq:UncondPDFAsymmetric1}
\end{eqnarray}
Unlike \ref{eq:UncondPDFSymmetric} where a relatively simple
approximation can be found and thus leads to an analytic result of the
integral, no such approximation has been found by the
author for the integral \ref{eq:UncondPDFAsymmetric}. However, the
moment generating functions (MGF) of PDF \ref{eq:UncondPDFAsymmetric}
and \ref{eq:UncondPDFAsymmetric1} are easy enough to find and give the
moments about the origin in closed form. Then by matching the analytic
expressions of the moments with their statistical values from the
sample, the parameters $\sigma$, $\psi$ and $\bar{v}$ corresponding to the
sample can be obtained.

The MGF of \ref{eq:UncondPDFAsymmetric1} can be found as follows:
\begin{eqnarray*}
  && M_{r}(t) \\
  &=& \E(e^{tr}) \\
  &=& {1 \over 2\pi \sqrt{1 - \psi^2}} \int_{-\infty}^{\infty} da \exp \left(
    -{a^2 \over 2} - \sigma a
  \right) \int_{-\infty}^{\infty} dx \exp\left[
    t(e^{\bar{v}}x + \mu) - {
      (a\psi - e^{-a \sigma} x)^2 \over 2 (1 - \psi^2)
    }
  \right] \\
  &=& {1 \over \sqrt{2 \pi}} \int_{-\infty}^{\infty} da \exp\left[
    -{a^2 \over 2} + {1 \over 2} e^{2 a \sigma + 2\bar{v}} (1 - \psi^2) t^2 +
    a \psi e^{a \sigma + \bar{v}} t + \mu t
  \right]
\end{eqnarray*}
With the MGF, the first 4 moments of $r$ can be computed:
\begin{equation}
  \label{eq:AsymmetricMoments}
  \begin{aligned}
    \E(r) &= \mu +e^{\bar{v}+\frac{\sigma ^2}{2}} \sigma  \psi\\
    \E(r^2) &= \mu ^2+2 e^{\bar{v}+\frac{\sigma ^2}{2}} \mu  \sigma  \psi
    +e^{2 \left(\bar{v}+\sigma ^2\right)} \left(1+4 \sigma ^2 \psi^2\right)\\
    \E(r^3) &= \mu ^3+3 e^{\bar{v}+\frac{\sigma ^2}{2}} \mu ^2 \sigma \psi
    +9 e^{3 \bar{v}+\frac{9 \sigma ^2}{2}} \sigma \psi \left(1+3 \sigma ^2
      \psi ^2\right)
    +3e^{2 \left(\bar{v}+\sigma ^2\right)} \mu  \left(1+4 \sigma ^2 \psi
      ^2\right) \\
    \E(r^4) &= \mu ^4+4 e^{\bar{v}+\frac{\sigma ^2}{2}} \mu ^3 \sigma  \psi
    +36 e^{3 \bar{v}+\frac{9 \sigma ^2}{2}} \mu  \sigma  \psi  \left(1+3
      \sigma ^2 \psi ^2\right)\\ 
    & +6 e^{2 \left(\bar{v}+\sigma ^2\right)} \mu ^2 \left(1+4 \sigma ^2 \psi
      ^2\right)
    +e^{4 \bar{v}+8 \sigma ^2} \left(3+96 \sigma ^2 \psi ^2+256 \sigma ^4
      \psi ^4\right)
  \end{aligned}
\end{equation}
These equations are rather complicated and directly solving them to
obtain the parameters $\mu$, $\sigma$, $\psi$ and $\bar{v}$ is
infeasible. However, in practice, $\E(r)$ is often very small --- so, to
get a rough estimate of the parameters, we may set $\mu = 0$ in the
above equations and, with a bit of manipulation, find the following
equation for $\psi\sigma$:
\begin{equation}
  \label{eq:moment1}
  {\E(r^3) \E(r^4)^{-3/8} \over \E(r^2)^{3/4}} =
  {9\sigma \psi (1 + 3\sigma^2\psi^2) \over (1 +
    4\sigma^2\psi^2)^{3/4}}
  {1
    \over
    (3 + 96\sigma^2\psi^2 + 256\sigma^4 \psi^4)^{3/8}
  }
\end{equation}
This equation can be solved numerically for a given sample to yield an
estimate for $\sigma\psi$, which in turn can be substituted in
equations \ref{eq:AsymmetricMoments}, where $\mu$ has been set to 0,
to give estimates for all the 4 parameters. These estimates then serve
as initial values in a numerical solution to
\ref{eq:AsymmetricMoments} where $\mu$ is kept as a free
variable. This full solution can now be used as the initial estimate
in an MLE procedure.

It is particularly important in our situation to have a good initial
estimate, because, as has been shown earlier, the PDF and CDF cannot
be obtained in closed forms and consequently have to be evaluated by
numerical integration. This is a rather costly procedure especially in
the context of MLE. So the computation of the moments under the
assumption of $\mu = 0$ is worthwhile.

Following the aforementioned procedure, i.e. computing the initial
estimate by matching the moments and then refining the estimate by
MLE, we obtain the parameter values for a number of return series,
including the Volvo B 30-minute returns described in
\S\ref{sec:SLV_Symmetric}. Table \ref{tab:assets_params} and
\ref{tab:assets_moments} present the obtained parameter values and the
resulting moments, respectively. Figure \ref{fig:Volvo_B_30m_returns},
\ref{fig:Nordea_30m_returns}, and \ref{fig:Ericsson_30m_returns}
compare the empirical distribution functions with the analytic
distribution functions evaluated at these parameters.

Clearly these figures show a fairly good match of the empirical and
the analytic distribution functions. However, we also see that the
skewness of the returns of Volvo B and Ericsson B have rather
different values when computed from the sample and from the
model. This could be the consequence of the limited sample size or
deficiencies in the estimation procedure described above. We leave
these issues to future studies.

%  This indicates the parameter $\psi$, which is the
% correlation between the log-volatility and the return, may be more
% complicated than the simple constant that has been assumed so far. We
% leave this to later studies.

\subsection{Relation to Conditional Distribution Functions}
In the last section we have shown that the unconditional distribution of
the returns are skewed, which implies the variate in the
log-volatility $a$ is correlated to the variate in the return
$b$.

In the context of conditional distributions and forecast, this
correlation translates to the correlation between the residual of the
log-volatility, denoted $y_t$ in \S\ref{chp:nordea_15min} and
\S\ref{sec:volvo}, and $b_t$. Now let us consider the forecast
function of an ARIMA model:
\begin{eqnarray*}
  \ln \sigma_t &=& y_t + \sum_{i=1}^P \phi_i \ln \sigma_{t-i} -
  \sum_{i=1}^Q \theta_i y_{t-i}
\end{eqnarray*}
where $\phi_i \leq 1$ if the model involves integration, and $\phi_i <
1$ otherwise. Comparing this equation with equation \ref{eq:r_t}, one
immediately realizes the PDF of $r_t$
\begin{eqnarray*}
  r_t &=& \mu + \sigma_t b_t
\end{eqnarray*}
is given by equation \ref{eq:UncondPDFAsymmetric1} and its moments
given by the equations \ref{eq:AsymmetricMoments} if one makes the
following substitutions:
\begin{eqnarray*}
  \sigma &\to& \text{std}(y_t) \\
  \bar{v} &\to& \sum_{i=1}^P \phi_i \ln \sigma_{t-i} - \sum_{i=1}^Q
  \theta_i y_{t-i} \\
  \psi &\to& \text{corr}(y_t, b_t)
\end{eqnarray*}

For the Nordea series considered in \S\ref{chp:nordea_15min},
$\text{corr}(y_t, b_t)$ is found to be $3.56 \times 10^{-2}$; while
for the Volvo series considered in \S\ref{sec:volvo},
$\text{corr}(y_t, b_t)$ is found to be $-7.3 \times 10^{-3}$. These
values are comparable to those in table \ref{tab:assets_params}, where
$\psi$ for the Nordea series is $1.7234 \times 10^{-2}$ and $-1.5747
\times 10^{-2}$ for the Volvo series. The similarity in these values
provides some evidence about the validity of the model.

It is certainly impossible to compare a predicted conditional
distribution with observations, but conditional distributions are of
great interest in the context of risk management and derivative
pricing, so we point out how they may be calculated for ARIMA
log-volatility models.
