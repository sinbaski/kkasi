\documentclass{article}
\usepackage{amssymb,amsmath,amsthm,bm}
\usepackage{graphicx}
\usepackage{subfigure}
\usepackage[usenames,dvipsnames]{color}
\usepackage{mathrsfs}
\usepackage{multirow}

\graphicspath{{../pics/InCaseNeeded/}}

\usepackage{amssymb,amsmath,amsthm}
%\newtheorem{mdef}{Definition}
%\newtheorem{theorem}{Theorem}
\newcommand{\eqsplit}[2]{
  \begin{equation}\label{#2}
    \begin{split}
      #1
    \end{split}
  \end{equation}}
\newcommand{\eqnsplit}[1]{
  \begin{eqnarray*}
    #1
  \end{eqnarray*}}
\newcommand{\tran}[1]{
  \tilde{#1}
}
\newcommand{\td}[2]{
  \frac{d #1}{d #2}
}
\newcommand{\pd}[2]{
  \frac{\partial #1}{\partial #2}
}
\newcommand{\ppd}[2]{
  \frac{\partial^2 #1}{\partial #2^2}
}
\newcommand{\pdd}[3]{
  \frac{\partial^2 #1}{\partial #2 \partial #3}
}
\newcommand{\otd}[1]{
  \frac{d}{d #1}
}
\newcommand{\opd}[1]{
  \frac{\partial}{\partial #1}
}
\newcommand{\oppd}[1]{
  \frac{\partial^2}{\partial #1^2}
}
\newcommand{\opdd}[2]{
  \frac{\partial^2}{\partial #1 \partial #2}
}
\newcommand{\ket}[1]{
  |#1\rangle
}
\newcommand{\bra}[1]{
  \langle#1|
}
\newcommand{\inn}[1]{
  \langle#1\rangle
}
\newcommand{\mean}[1]{
  \langle#1\rangle
}
\newcommand{\tr}{
  \text{tr}\,
}
\newcommand{\re}{
  \text{Re}\,
}
\newcommand\im{
  \text{Im}\,
}
\newcommand{\var}{
  \text{var}
}
\newcommand{\arcsinh}{
  \sinh^{-1}
}
\newcommand{\arccosh}{
  \cosh^{-1}
}
\newcommand{\erfc}{
  \text{erfc}
}
\newcommand{\E}{
  \mathbb{E}
}
\renewcommand{\P}{
  \mathbb{P}
}
\newcommand{\I}[1]{
  \mathbf{1}_{\{#1\}}
}
\newcommand{\1}[1]{
  \mathds{1}_{\{#1\}}
}
\newcommand{\diag}{
  \text{diag\,}
}
\newcommand{\M}{
  {\text{max}}
}
\newcommand{\m}{
  {\text{min}}
}
\newcommand{\ph}{
  {\text{arg}\,}
}
\newcommand\erf{
  \text{erf}
}
\renewcommand\vec[1]{
  \mathbf{#1}
}
\newcommand\mtx[1]{
  \mathbf{#1}
}
\newcommand\ed{
  \,{\buildrel d \over =}\,
}




\begin{document}

\begin{figure}[htb!]
  \centering
  \includegraphics[scale=0.6]{eigmax_TW.pdf}
  \caption{Distribution of simulated $\lambda_\M$ (blue) compared to
    Tracy-Widom (red). From the 1st to the 4th row, q = 0.1, 0.2, 0.5,
    1. $\lambda_\M$ has been centered and rescaled as is required in
    the case of Wishart matrices. The Tracy-Widom density is computed
    with variance $e^{2v}$, for $v \le 0.25$. $e^{2v}$ is $\E \sigma^2$,
    where $\sigma \sim \text{Lognormal}(0, v)$.}
  \label{fig:eigmax_TW}
\end{figure}

When $v \ge 0.25$, a power-law is observed. Table
\ref{tab:tail_indices} list the tail indices of the $\lambda_\M$
corresponding to different values of $q$ and $v$.
\begin{table}[htb!]
  \centering
  \begin{tabular}{|c|c|c|c|c|c|c|}
    \hline
    \multicolumn{2}{|c|}{} & \multicolumn{5}{|c|}{$v$} \\
    \hline
    \multicolumn{2}{|c|}{} & 0.25 & 0.27 & 0.33 & 0.50 & 0.75 \\
    \hline
    \multirow{4}{*}{$q$} & 0.1 & 3.75 & 3.86 & 3.46 & 2.99 & 2.42 \\
    & 0.2 & 3.23 & 3.98 & 3.24 & 2.90 & 2.73 \\
    & 0.5 & 3.88 & 4.11 & 3.60 & 2.71 & 2.54 \\
    & 1.0 & 3.41 & 3.66 & 4.02 & 2.90 & 2.55 \\
    \hline
  \end{tabular}
  \caption{Tail indices of the distribution function of
    $\lambda_\M$. By tail index $\alpha$, I mean $\lim_{x \to \infty}
    x^\alpha P(\lambda_\M > x) = \text{const}$}
  \label{tab:tail_indices}
\end{table}

\begin{figure}[htb!]
  \centering
  \includegraphics[scale=0.6, clip=true, trim=98 233 120
  127]{AppleLogvol-QQ.pdf}
  \caption{QQ-plot of $\ln{ S_\M - S_\m \over S_\m}$ against standard
    normal. Here $S_\M$ and $S_\m$ stand for the highest and the
    lowest price of each trading day, respectively. The data are daily
    trading statistics of Apple Inc. from 2005-01-02 to 2015-01-23. The
    lower tail of sample log-vol is a bit lighter and the upper tail a
    bit heavier than their counterparts of standard normal.}
  \label{fig:qqplot_logvol}
\end{figure}

\end{document}
