\documentclass{article}
\usepackage{amsmath}
\usepackage{amsthm}
\usepackage{enumerate}
\usepackage[bookmarks=true]{hyperref}
\usepackage{bookmark}
\usepackage{graphicx}
\usepackage{multirow}
\usepackage{adjustbox}
\usepackage{wrapfig}
\usepackage{subcaption}
\usepackage{color}
\usepackage[dvipsnames]{xcolor}

\usepackage{amssymb,amsmath,amsthm,amsfonts}
\usepackage{mathrsfs}
\usepackage{dsfont}
\usepackage{enumerate}

%\newtheorem{mdef}{Definition}
%\newtheorem{theorem}{Theorem}
\newcommand{\eqsplit}[2]{
  \begin{equation}\label{#2}
    \begin{split}
      #1
    \end{split}
  \end{equation}}
\newcommand{\eqnsplit}[1]{
  \begin{eqnarray*}
    #1
  \end{eqnarray*}}
\newcommand{\tran}[1]{
  \tilde{#1}
}
\newcommand{\td}[2]{
  \frac{d #1}{d #2}
}
\newcommand{\pd}[2]{
  \frac{\partial #1}{\partial #2}
}
\newcommand{\ppd}[2]{
  \frac{\partial^2 #1}{\partial #2^2}
}
\newcommand{\pdd}[3]{
  \frac{\partial^2 #1}{\partial #2 \partial #3}
}
\newcommand{\otd}[1]{
  \frac{d}{d #1}
}
\newcommand{\opd}[1]{
  \frac{\partial}{\partial #1}
}
\newcommand{\oppd}[1]{
  \frac{\partial^2}{\partial #1^2}
}
\newcommand{\opdd}[2]{
  \frac{\partial^2}{\partial #1 \partial #2}
}
\newcommand{\ket}[1]{
  |#1\rangle
}
\newcommand{\bra}[1]{
  \langle#1|
}
\newcommand{\inn}[1]{
  \langle#1\rangle
}
\newcommand{\mean}[1]{
  \langle#1\rangle
}
\newcommand{\tr}{
  \text{tr}\,
}
\newcommand{\re}{
  \text{Re}\,
}
\newcommand\im{
  \text{Im}\,
}
\newcommand{\var}{
  \text{var}
}
\newcommand{\arcsinh}{
  \sinh^{-1}
}
\newcommand{\arccosh}{
  \cosh^{-1}
}
\newcommand{\erfc}{
  \text{erfc}
}
\newcommand{\E}{
  \mathbb{E}
}
\renewcommand{\P}{
  \mathbb{P}
}
\newcommand{\I}[1]{
  \mathbf{1}_{\{#1\}}
}
\newcommand{\1}[1]{
  \mathds{1}_{\{#1\}}
}
\newcommand{\diag}{
  \text{diag\,}
}
\newcommand{\M}{
  {\text{max}}
}
\newcommand{\m}{
  {\text{min}}
}
\newcommand{\ph}{
  {\text{arg}\,}
}
\newcommand\erf{
  \text{erf}
}
\renewcommand\vec[1]{
  \mathbf{#1}
}
\newcommand\mtx[1]{
  \mathbf{#1}
}
\newcommand\ed{
  \,{\buildrel d \over =}\,
}



\title{Potential Projects}
\author{Xie Xiaolei}
\date{\today}
\begin{document}
\maketitle

\begin{enumerate}
\item Eigenvalues and eigenvectors of tail dependence matrix
\item Extend the theory with Casper and Thomas to situations when the
  return distribution of the stock has a Poisson component.
\item With Cheng Dan: The old project with Thomas and Olivier.
\item Cellular automata. Find out whether the theories of random
  matrices and extreme values are useful.
\item {\bf Maybe we can model market crash as a phase transition when
  the probabilistic rule involves a heavy-tail distribution?}
  \begin{itemize}
  \item Critical Behavior of a Probabilistic Local and Nonlocal
    Site-Exchange Cellular Automaton
  \item Boccara, N.; Nasser, J.; Roger, M.
  \item International Journal of Modern Physics C, Volume 5, Issue 03,
    pp. 537-545 (1994). (IJMPC Homepage)
  \item Publication Date:	00/1994
  \item Keywords: Cellular Automata, Critical Behavior
  \item Abstract:
    We study the critical behavior of a probabilistic automata network
    whose local rule consists of two subrules. The first one, applied
    synchronously, is a probabilistic one-dimensional range-one cellular
    automaton rule. The second, applied sequentially, exchanges the values
    of a pair of sites. According to whether the two sites are
    first-neighbors or not, the exchange is said to be local or
    nonlocal. The evolution of the system depends upon two parameters, the
    probability p characterizing the probabilistic cellular automaton, and
    the degree of mixing m resulting from the exchange process. Depending
    upon the values of these parameters, the system exhibits a bifurcation
    similar to a second order phase transition characterized by a
    nonnegative order parameter, whose role is played by the stationary
    density of occupied sites. When m is very large, the correlations
    created by the application of the probabilistic cellular automaton
    rule are destroyed, and, as expected, the behavior of the system is
    then correctly predicted by a mean-field-type approximation. According
    to whether the exchange of the site values is local or nonlocal, the
    critical behavior is qualitatively different as m varies.
  \end{itemize}



\end{enumerate}
\bibliographystyle{unsrt}
\bibliography{../thesis/econophysics}
\end{document}



