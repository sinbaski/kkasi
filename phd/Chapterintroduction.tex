\chapter{Introduction}\label{ch:intr}


\section{Regular Variation}
The concept of regular variation is defined by the following scaling
property: If a function $f(\cdot)$ satisfies
\[
\lim_{x \to \infty} {f(c x) \over f(x)} = c^\alpha
\]
then we say $f(x)$ is regularly varying with index $\alpha$.
Distribution functions, say $F(\cdot)$, whose survival function
$\bar F(x) = 1 - F(x)$ satisfies the above scaling property, are
observed in a variety of phenomena, for example, phase transition of
matter and black body radiation as studied in physics, neuronal
avalanches in biology, claim sizes of insurance mathematics and stock
returns in finance. The last application is indeed the focus of this
thesis.

When expanded to multiple dimensions, the scaling property of regular
variation is better described in terms of weak convergence to a
spectral measure: If a random vector $V$ satisfies
\[
\lim_{x \to \infty}
{
  \P(V/|V| \in \cdot, |V| > c x)
  \over
  \P(|V| > x)
}
\overset{w}{\to} c^{-\alpha} \mu_\alpha(\cdot)
\]
then we say the components of $V$ is jointly regularly varying with
index $\alpha$. Here $\mu_\alpha(\cdot)$ is a probability measure on
the unit sphere \cite{buraczewski:damek:mikosch:2016}. Clearly, if $V$
is jointly regularly varying with index $\alpha$, then each component
of it and each linear combination of its components are regularly
varying with the same index $\alpha$. This follows from Feller
\cite{feller}, p. 278. Cf. also Jessen and Mikosch
\cite{JessenMikosch2006}, lemma 3.1, and Embrechts et al
\cite{embrechts:klueppelberg:mikosch:1997}, lemma 1.3.1.

Thus estimating the tail index $\alpha$ is particularly important for
understanding the behaviour of a heavy-tailed series. The earliest
method proposed for this purpose is due to Hill \cite{hill1975simple}:
\[
\hat \alpha_H = \left[
  {1 \over k} \sum_{i=1}^k \log \left(
  X_{(i)} \over X_{(k+1)}
  \right)
  \right]^{-1}
\]
where $X_1, X_2, \dots$ is the series whose tail index is the subject
of estimation, and $X_{(i)}$ is the $i$-th upper order statistics of
this series. Several authors have contributed to showing the
weak consistency of the estimator $\hat \alpha_H$,
i.e. if $k \to \infty, k/n \to 0$ as $n \to \infty$, then
$\hat \alpha_H \overset{P}{\to} \alpha$.

Mason \cite{mason:1982} first proved that the estimator
was consistent when $X_1, X_2, \dots$ were iid; later Rootz\'en,
Leadbetter and de Haan \cite{rootzen:leadbetter:dehaan1992} and Hsing
\cite{hsing:1991} proved its consistency when $X_1, X_2, \dots$ were
weakly dependent; and Resnick and St{\u a}ric{\u a}
\cite{resnick:starica:1995, resnick:starica:1997} proved its
consistency when $X_t$ was a linear process.

When considered as the multiplicative inverse of the parameter of the
{\em Generalized Extreme Value} distribution, there are other methods
in the literature for estimating the tail index, e.g. Pickand's 
estimator \cite{pickands1975statistical} $\hat \alpha_P$ and the
Deckers-Einmahl-de Haan estimator $\hat \alpha_{\text{DEH}}$
\begin{eqnarray*}
{1 \over \hat \alpha_P}
&=&
{1 \over \log 2}
\log {
  X_{(k)} - X_{(2k)}
  \over
X_{(2k)} - X_{(4k)}  
} \\
{1 \over \hat \alpha_{\text{DEH}}}
&=&
1 + {1 \over \alpha_H} + {1 \over 2} \left(
  {H \over \alpha_H^2} - 1
\right)^{-1}
\end{eqnarray*}
where it is similarly assumed $k \to \infty$ and $k/n \to 0$ as $n \to
\infty$. The Deckers-Einmahl-de Haan estimator makes use of Hill's
estimator $\hat \alpha_H$ and computes
\[
H = \left[
  {1 \over k} \sum_{i=1}^k \log \left(
    { X_{(i)} \over X_{(k+1)} }
  \right)^2
\right]^{-1}
\]
Apparently $1/H$ can be interpreted as the 2nd empirical moment of
$\log(X_{(i)}/X_{(k+1)})$ for $i \leq k$.
A major drawback of Pickand's and Deckers-Einmahl-de Haan's estimators
is that, when applied to estimating the tail index, they discard the
information that the tail index is always positive, hence resulting in
a larger confidence band compared with that obtained for Hill's
estimator. Therefore we stick to Hill's estimator in the empirical
work included in this thesis.

\section{Stochastic Recurrence Equation}
One of the most important dynamical mechanisms that lead to regularly
varying r.v. is stochastic recursion of the following form:
\[
X_t = A_t X_{t-1} + B_t
\]
where $X_t$ is a $d$-dimensional random vector, $A_t$ is a $d\times d$
random matrix and $B_t$ is a $d$-dimensional vector, random or
deterministic. The sequence $\{A_i\}_{i=1,2,\dots}$ and
$\{B_i\}_{i=1,2,\dots}$  are iid and independent of each other.

Kesten \cite{kesten:1973} showed that, when $A_t$ and $B_t$ were almost
surely non-negative, had no row or column of only zeros, and there was
a positive probability that $B_t$ was strictly positive, the strictly
stationary solution to the equation
$V \overset{d}{=} A V + B$ had power-law tails
for its marginal distributions, assuming the following conditions (M)
and (A):
\begin{itemize}
\item Condition (M)
  \begin{enumerate}
  \item The top Lyapunov exponent
    \[
    \gamma = \inf_{n \geq 1} {1 \over n}\E \log \|A_n \cdots A_1\|
    \]
    is negative.
  \item There exists $\xi > 0$ such that
    \[
    1 = \lambda(\xi) = \lim_{n \to \infty} {1 \over n} \log \E \|A_n \cdots A_1\|^\xi
    \]
  \item $\E (\|A_1\|^\xi \log^+\|A_1\|) < \infty$
  \item $\E |B_1|^\xi < \infty$
  \end{enumerate}
\item Condition (A) : The group generated by
  \[
  \{\log\rho(s): s = A_n \cdots A_1 \text{ for some } n \geq 1\}
  \]
  is dense in $\reals$, where $\rho(s)$ denotes the spectral
  radius of matrix $s$.
\end{itemize}
Upon these conditions, Kesten's theorem gives
\begin{equation*}
  u^\xi \P(u^{-1} V \in \cdot) \overset{v}{\to} \mu_\xi(\cdot)
\end{equation*}
where $\mu_\xi$ is a non-null Radon measure on
$\reals^d_+ \setminus \{0\}$ with the property
$\mu_\xi(a A) = a^{-\xi} \mu_\xi(A)$.

Recently, Collamore and Mentemeier \cite{collamore:mentemeier:2016}
extended Kesten's result and gave an explicit expression for $\mu_\xi$:
\begin{equation}
  \label{eq:CollamoreMentemeierIntro}
  \lim_{u \to \infty} u^{\xi} \E \left[
    f(u^{-1} V)
    \right]
  =
  {C \over \lambda'(\xi)}  
  \int_{\sphere^{d-1}_+ \times R} e^{-\xi s} f(e^s x) \ell_\xi(dx) ds
\end{equation}
where $C$ is a constant (cf. eq.(2.15) of Collamore and Mentemeier
\cite{collamore:mentemeier:2016}), $f(\cdot)$ is any bounded
continuous function on $\reals^{d}_+ \setminus \{0\}$  and
$\ell_\xi$ is a probability measure on $\sphere_+^{d-1}$. Its definition
is also found in \eqref{eq:eigenmeasure}.

From \eqref{eq:CollamoreMentemeierIntro} a representation for $\mu_\xi
(\cdot)$ immediately follows  
\[
\mu_\xi (\cdot) = {C \over \lambda'(\xi)} \mathcal L_\xi(\cdot)
\]
Here $\mathcal L_\xi$ is a non-null Radon measure  on
$\reals^d_+ \setminus \{0\}$ that satisfies, for all
bounded continuous function $f(\cdot)$ on
$\reals_+^{d} \setminus \{0\}$:
\[
\int_{\reals_+^d\setminus \{0\}} f(x) \mathcal L_\xi(dx)
=
\int_{\sphere^{d-1}_+ \times R} e^{-\xi s} f(e^s x) \ell_\xi(dx) ds
\]


 
\section{Contribution of this thesis}\label{sec:contr}

In this section we summarize our results from the research papers.

\subsection{Tail parameters of equity return series}
We have established that, in the case of an equity return series with
two-sided, functionally independent Pareto tails, investor
preference functionals are monotone increasing/decreasing with the
tail index/scale parameters. Thus in a market dominated by such
equities, the investors would pursue the largest tail index in the
market, leading to a shared common tail index for all equities.

The empirical results presented in section \ref{sec:1} suggest this
may well be the case for the ``Consumer Staples'' sector of S\&P 500,
given the Hill estimates of tail indices shown in figure \ref{fig:1}
and the largely positive results of tests for equal tail indices shown
in figure \ref{fig:PairTest}.

On the other hand, we have also seen that, when the left and the right
tails have the same indices, investor preference over the equity has
more sophisticated variations in the parameters' space including the
tail parameters of the equity, the interest rate, the investor's risk
apetite as captured by his utility function, and his threshold of
disappointment.

We also acknowledge that our model of the market and the investor is a
simple one, not accoounting for the dependence between equities, nor
the categorization of investors and their interactions. These are
potential topics of future work.

