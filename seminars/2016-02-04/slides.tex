%%%%%%%%%%%%%%%%%%%%%%%%%%%%%%%%%%%%%%%%%
% Beamer Presentation
% LaTeX Template
% Version 1.0 (10/11/12)
%
% This template has been downloaded from:
% http://www.LaTeXTemplates.com
%
% License:
% CC BY-NC-SA 3.0 (http://creativecommons.org/licenses/by-nc-sa/3.0/)
%
%%%%%%%%%%%%%%%%%%%%%%%%%%%%%%%%%%%%%%%%%

%----------------------------------------------------------------------------------------
%	PACKAGES AND THEMES
%----------------------------------------------------------------------------------------

\documentclass{beamer}

\mode<presentation> {

% The Beamer class comes with a number of default slide themes
% which change the colors and layouts of slides. Below this is a list
% of all the themes, uncomment each in turn to see what they look like.

%\usetheme{default}
%\usetheme{AnnArbor}
%\usetheme{Antibes}
%\usetheme{Bergen}
%\usetheme{Berkeley}
%\usetheme{Berlin}
%\usetheme{Boadilla}
%\usetheme{CambridgeUS}
\usetheme{Copenhagen}
%\usetheme{Darmstadt}
%\usetheme{Dresden}
%\usetheme{Frankfurt}
%\usetheme{Goettingen}
%\usetheme{Hannover}
%\usetheme{Ilmenau}
%\usetheme{JuanLesPins}
%\usetheme{Luebeck}
%\usetheme{Madrid}
%\usetheme{Malmoe}
%\usetheme{Marburg}
%\usetheme{Montpellier}
%\usetheme{PaloAlto}
%\usetheme{Pittsburgh}
%\usetheme{Rochester}
%\usetheme{Singapore}
%\usetheme{Szeged}
%\usetheme{Warsaw}

% As well as themes, the Beamer class has a number of color themes
% for any slide theme. Uncomment each of these in turn to see how it
% changes the colors of your current slide theme.

%\usecolortheme{albatross}
%\usecolortheme{beaver}
%\usecolortheme{beetle}
%\usecolortheme{crane}
%\usecolortheme{dolphin}
%\usecolortheme{dove}
%\usecolortheme{fly}
%\usecolortheme{lily}
%\usecolortheme{orchid}
%\usecolortheme{rose}
%\usecolortheme{seagull}
%\usecolortheme{seahorse}
%\usecolortheme{whale}
%\usecolortheme{wolverine}

%\setbeamertemplate{footline} % To remove the footer line in all slides uncomment this line
%\setbeamertemplate{footline}[page number] % To replace the footer line in all slides with a simple slide count uncomment this line

%\setbeamertemplate{navigation symbols}{} % To remove the navigation symbols from the bottom of all slides uncomment this line
}

\usepackage{graphicx} % Allows including images
\usepackage{booktabs} % Allows the use of \toprule, \midrule and \bottomrule
                      % in tables

\usepackage{amssymb,amsmath,amsthm,amsfonts}
\usepackage{mathrsfs}
\usepackage{dsfont}
\usepackage{enumerate}

%\newtheorem{mdef}{Definition}
%\newtheorem{theorem}{Theorem}
\newcommand{\eqsplit}[2]{
  \begin{equation}\label{#2}
    \begin{split}
      #1
    \end{split}
  \end{equation}}
\newcommand{\eqnsplit}[1]{
  \begin{eqnarray*}
    #1
  \end{eqnarray*}}
\newcommand{\tran}[1]{
  \tilde{#1}
}
\newcommand{\td}[2]{
  \frac{d #1}{d #2}
}
\newcommand{\pd}[2]{
  \frac{\partial #1}{\partial #2}
}
\newcommand{\ppd}[2]{
  \frac{\partial^2 #1}{\partial #2^2}
}
\newcommand{\pdd}[3]{
  \frac{\partial^2 #1}{\partial #2 \partial #3}
}
\newcommand{\otd}[1]{
  \frac{d}{d #1}
}
\newcommand{\opd}[1]{
  \frac{\partial}{\partial #1}
}
\newcommand{\oppd}[1]{
  \frac{\partial^2}{\partial #1^2}
}
\newcommand{\opdd}[2]{
  \frac{\partial^2}{\partial #1 \partial #2}
}
\newcommand{\ket}[1]{
  |#1\rangle
}
\newcommand{\bra}[1]{
  \langle#1|
}
\newcommand{\inn}[1]{
  \langle#1\rangle
}
\newcommand{\mean}[1]{
  \langle#1\rangle
}
\newcommand{\tr}{
  \text{tr}\,
}
\newcommand{\re}{
  \text{Re}\,
}
\newcommand\im{
  \text{Im}\,
}
\newcommand{\var}{
  \text{var}
}
\newcommand{\arcsinh}{
  \sinh^{-1}
}
\newcommand{\arccosh}{
  \cosh^{-1}
}
\newcommand{\erfc}{
  \text{erfc}
}
\newcommand{\E}{
  \mathbb{E}
}
\renewcommand{\P}{
  \mathbb{P}
}
\newcommand{\I}[1]{
  \mathbf{1}_{\{#1\}}
}
\newcommand{\1}[1]{
  \mathds{1}_{\{#1\}}
}
\newcommand{\diag}{
  \text{diag\,}
}
\newcommand{\M}{
  {\text{max}}
}
\newcommand{\m}{
  {\text{min}}
}
\newcommand{\ph}{
  {\text{arg}\,}
}
\newcommand\erf{
  \text{erf}
}
\renewcommand\vec[1]{
  \mathbf{#1}
}
\newcommand\mtx[1]{
  \mathbf{#1}
}
\newcommand\ed{
  \,{\buildrel d \over =}\,
}




%----------------------------------------------------------------------------------------
%	TITLE PAGE
%----------------------------------------------------------------------------------------

\title{An Efficient Estimator of 1D MC Exceedance Probability} % The
                                % short title appears at the bottom of
                                % every slide, the full title is only
                                % on the title page

\author{Xie Xiaolei} % Your name
\institute[UCPH] % Your institution as it will appear on the bottom of every slide, may be shorthand to save space
{
Copenhagen University  \\ % Your institution for the title page
\medskip
\textit{hnq365@math.ku.dk} % Your email address
}
\date{\today} % Date, can be changed to a custom date

\begin{document}

\begin{frame}
\titlepage % Print the title page as the first slide
\end{frame}

% \begin{frame}
% \frametitle{Overview}
% \tableofcontents
% \end{frame}

%----------------------------------------------------------------------------------------
%	PRESENTATION SLIDES
%----------------------------------------------------------------------------------------
% \section{Simple dependence models}
%------------------------------------------------
\section{Setup of the Problem}
\begin{frame}
  Consider the recurrence equation 
  \[
  V_n = A_n (V_{n-1} \vee D_n) + B_n
  \]
  where $A, B, D$ satisfy the following conditions:
  \begin{enumerate}
  \item $A$ is positive, absolutely continuous and has a non-trivial continuous density in
    a neighborhood of $\mathbb R$.
  \item Define $\Lambda(\alpha) := \log \E A^\alpha$, $\Lambda(\xi) < \infty$ for some $\xi \in (0,
    \infty)$ and $\Lambda(\cdot)$ is differentiable at $\xi$.
  \item $\E |B|^\xi < \infty$, $\E (A|D|)^\xi < \infty$
  \item $\P(A > 1, B > 0) > 0$ or $P(A > 1, B \geq 0, D > 0) > 0$.
  \end{enumerate}
  Collamore and Vidyashankar showed \cite{Collamore20133378} that
  under these conditions $V_n$ is $\phi$-irreducible and geometrically
  ergodic.
\end{frame}

\begin{frame}
  Aperiodicity and $\phi$-irreducibility imples the MC satisfies a
  minorization condition: There exist a small set $C$ and a measure
  $\nu$ on $(E, \mathcal B(X))$ such that $\nu(C) = 1$
  \[
  P(x, A) > \delta \I{\mathcal C}(x) \nu(A)
  \]
  The minorization condition in turn implies the MC has a regeneration
  structure \cite{Nummelin1978}: There exist random times $1 \leq T_1
  < T_2 < \cdots$ such that
  \begin{enumerate}
  \item $T_0, T_{i+1} - T_{i}$, $i = 1, 2, \cdots$ are finite with
    probability 1 and are independent, identically distributed random
    variables.
  \item The blocks $(V_{T_i}, \dots, V_{T_{i+1} - 1})$ are independent;
  \item $\P(V_{T_i} \in A | V_{T_{i}-1}, \dots, V_0 = \nu(A)$
  \end{enumerate}
\end{frame}

\begin{frame}
  Collamore and Vidyashankar \cite{Collamore20133378} showed that the
  chain $V_n$ is geometrically recurrent, implying it admits an invariant
  measure $\pi$:
  \[
  \pi(A) = \int_A \pi(x) P(x, dy)
  \]
  In other words, $V_n$ converges to a random variable $V$ distributed
  according to $\pi$. We want to have an efficient estimator of the
  probability $\P(V > u)$ as $u \to \infty$. It is understood
  \[
   \lim_{u \to \infty} u^\xi \P(V > u) = C
  \]
  for some constant $C > 0$.
\end{frame}

\begin{frame}
  Collamore and Vidyashankar showed, when $V_0$ is in the set
  $\mathcal C$ and is distributed according to $\gamma$ defined as
  \[
  \gamma(E) = \pi(E)/\pi(\mathcal C)\text{ for all } E \in \mathcal B(\mathcal C)
  \]
  $E_u = N_u |A_{T_u} \cdots A_{1} V_0|^{-\xi}\1{T_u < \tau}$ is an efficient
  estimator w.r.t the dual measure $\mathcal D$. Here by ``efficient'' we mean
  \[
  \lim_{u \to \infty} {
    \var_{\mathcal D}(\mathcal E_u)
    \over
    (\E_{\mathcal D} \mathcal E_u)^2
  } < \infty
  \]
  as defined by Asmusen. 
  \begin{eqnarray*}
    \tau &\overset{d}{=}& T_{i+1} - T_i \\
    T_u &:=& \min\{n \geq 1: V_n > u\} \\
    N_u &:=& \sum_{n=1}^{\tau} \1{V_n > u}
  \end{eqnarray*}
\end{frame}

\begin{frame}
  $\E_D$ and $\var$ denote expectation and variance taken w.r.t to the
  dual measure $\mathcal D$, respectively: If $n < T_u$, i.e. the MC
  has not exceeded the level $u$, $A_{n+1}$ is sampled from the
  $\xi$-shifted distribution $\mu_\xi$; otherwise it is sampled from
  $\mu$.
  \[
  \mu_\xi (S) = {\E A^\xi \I{S}(A)
    \over
    \E A^\xi
    }
  \]
\end{frame}


\section{Deriving the Estimator}
\bibliographystyle{unsrt}
\bibliography{../../thesis/econophysics}
\end{document} 

