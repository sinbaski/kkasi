%%%%%%%%%%%%%%%%%%%%%%%%%%%%%%%%%%%%%%%%%
% Beamer Presentation
% LaTeX Template
% Version 1.0 (10/11/12)
%
% This template has been downloaded from:
% http://www.LaTeXTemplates.com
%
% License:
% CC BY-NC-SA 3.0 (http://creativecommons.org/licenses/by-nc-sa/3.0/)
%
%%%%%%%%%%%%%%%%%%%%%%%%%%%%%%%%%%%%%%%%%

%----------------------------------------------------------------------------------------
%	PACKAGES AND THEMES
%----------------------------------------------------------------------------------------

\documentclass{beamer}

\mode<presentation> {

% The Beamer class comes with a number of default slide themes
% which change the colors and layouts of slides. Below this is a list
% of all the themes, uncomment each in turn to see what they look like.

%\usetheme{default}
%\usetheme{AnnArbor}
%\usetheme{Antibes}
%\usetheme{Bergen}
%\usetheme{Berkeley}
%\usetheme{Berlin}
%\usetheme{Boadilla}
%\usetheme{CambridgeUS}
\usetheme{Copenhagen}
%\usetheme{Darmstadt}
%\usetheme{Dresden}
%\usetheme{Frankfurt}
%\usetheme{Goettingen}
%\usetheme{Hannover}
%\usetheme{Ilmenau}
%\usetheme{JuanLesPins}
%\usetheme{Luebeck}
%\usetheme{Madrid}
%\usetheme{Malmoe}
%\usetheme{Marburg}
%\usetheme{Montpellier}
%\usetheme{PaloAlto}
%\usetheme{Pittsburgh}
%\usetheme{Rochester}
%\usetheme{Singapore}
%\usetheme{Szeged}
%\usetheme{Warsaw}

% As well as themes, the Beamer class has a number of color themes
% for any slide theme. Uncomment each of these in turn to see how it
% changes the colors of your current slide theme.

%\usecolortheme{albatross}
%\usecolortheme{beaver}
%\usecolortheme{beetle}
%\usecolortheme{crane}
%\usecolortheme{dolphin}
%\usecolortheme{dove}
%\usecolortheme{fly}
%\usecolortheme{lily}
%\usecolortheme{orchid}
%\usecolortheme{rose}
%\usecolortheme{seagull}
%\usecolortheme{seahorse}
%\usecolortheme{whale}
%\usecolortheme{wolverine}

%\setbeamertemplate{footline} % To remove the footer line in all slides uncomment this line
%\setbeamertemplate{footline}[page number] % To replace the footer
%line in all slides with a simple slide count uncomment this line

%\setbeamertemplate{navigation symbols}{} % To remove the navigation
%symbols from the bottom of all slides uncomment this line
}

\usepackage{graphicx} % Allows including images
\usepackage{booktabs} % Allows the use of \toprule, \midrule and \bottomrule
                      % in tables

\usepackage{amssymb,amsmath,amsthm}
%\newtheorem{mdef}{Definition}
%\newtheorem{theorem}{Theorem}
\newcommand{\eqsplit}[2]{
  \begin{equation}\label{#2}
    \begin{split}
      #1
    \end{split}
  \end{equation}}
\newcommand{\eqnsplit}[1]{
  \begin{eqnarray*}
    #1
  \end{eqnarray*}}
\newcommand{\tran}[1]{
  \tilde{#1}
}
\newcommand{\td}[2]{
  \frac{d #1}{d #2}
}
\newcommand{\pd}[2]{
  \frac{\partial #1}{\partial #2}
}
\newcommand{\ppd}[2]{
  \frac{\partial^2 #1}{\partial #2^2}
}
\newcommand{\pdd}[3]{
  \frac{\partial^2 #1}{\partial #2 \partial #3}
}
\newcommand{\otd}[1]{
  \frac{d}{d #1}
}
\newcommand{\opd}[1]{
  \frac{\partial}{\partial #1}
}
\newcommand{\oppd}[1]{
  \frac{\partial^2}{\partial #1^2}
}
\newcommand{\opdd}[2]{
  \frac{\partial^2}{\partial #1 \partial #2}
}
\newcommand{\ket}[1]{
  |#1\rangle
}
\newcommand{\bra}[1]{
  \langle#1|
}
\newcommand{\inn}[1]{
  \langle#1\rangle
}
\newcommand{\mean}[1]{
  \langle#1\rangle
}
\newcommand{\tr}{
  \text{tr}\,
}
\newcommand{\re}{
  \text{Re}\,
}
\newcommand\im{
  \text{Im}\,
}
\newcommand{\var}{
  \text{var}
}
\newcommand{\arcsinh}{
  \sinh^{-1}
}
\newcommand{\arccosh}{
  \cosh^{-1}
}
\newcommand{\erfc}{
  \text{erfc}
}
\newcommand{\E}{
  \mathbb{E}
}
\renewcommand{\P}{
  \mathbb{P}
}
\newcommand{\I}[1]{
  \mathbf{1}_{\{#1\}}
}
\newcommand{\1}[1]{
  \mathds{1}_{\{#1\}}
}
\newcommand{\diag}{
  \text{diag\,}
}
\newcommand{\M}{
  {\text{max}}
}
\newcommand{\m}{
  {\text{min}}
}
\newcommand{\ph}{
  {\text{arg}\,}
}
\newcommand\erf{
  \text{erf}
}
\renewcommand\vec[1]{
  \mathbf{#1}
}
\newcommand\mtx[1]{
  \mathbf{#1}
}
\newcommand\ed{
  \,{\buildrel d \over =}\,
}




%----------------------------------------------------------------------------------------
%	TITLE PAGE
%----------------------------------------------------------------------------------------

\title{An Efficient Estimator of 1D MC Exceedance Probability} % The
                                % short title appears at the bottom of
                                % every slide, the full title is only
                                % on the title page

\author{Xie Xiaolei} % Your name
\institute[UCPH] % Your institution as it will appear on the bottom of every slide, may be shorthand to save space
{
Copenhagen University  \\ % Your institution for the title page
\medskip
\textit{hnq365@math.ku.dk} % Your email address
}
\date{\today} % Date, can be changed to a custom date

\begin{document}

\begin{frame}
\titlepage % Print the title page as the first slide
\end{frame}

% \begin{frame}
% \frametitle{Overview}
% \tableofcontents
% \end{frame}

%----------------------------------------------------------------------------------------
%	PRESENTATION SLIDES
%----------------------------------------------------------------------------------------
% \section{Simple dependence models}
%------------------------------------------------
\section{Introduction}
\begin{frame}
  We are interested in the {\it Stochastic Fixed Point Equations}:
  \[
  V \overset{d}{=} f(V)
  \]
  \begin{enumerate}
  \item Forward sequence
    \[
    V_n = f_{Y_n} \circ f_{Y_{n-1}} \circ \cdots \circ f_{Y_1} (V_0)
    \]
    \item Backward sequence
      \[
      Z_n  = f_{Y_1} \circ f_{Y_{2}} \circ \cdots \circ f_{Y_n} (Z_0)
      \]
    \end{enumerate}
    where $Y_i$ are the random variables drving the $i$-th iteration.
    {\it GARCH(p, q)} models are typical examples of forward
    sequences. In the case of GARCH(1,1),
    \[
    \sigma_n^2 = (\beta_1 + \alpha_1 \eta_{n-1}^2)\sigma_{n-1}^2 + \alpha_0
    \]
    with $f_{Y_{n-1}}(w) = (\beta_1 + \alpha_1 \eta_{n-1}^2) w + \alpha_0$ 
\end{frame}

\begin{frame}

  
\end{frame}

\section{Setup of the Problem}
\begin{frame}
  Consider the recurrence equation 
  \[
  V_n = A_n (V_{n-1} \vee D_n) + B_n
  \]
  where $A, B, D$ satisfy the following conditions:
  \begin{enumerate}
  \item $A$ is positive, absolutely continuous and has a non-trivial continuous density in
    a neighborhood of $\mathbb R$.
  \item Define $\Lambda(\alpha) := \log \E A^\alpha$, $\Lambda(\xi) < \infty$ for some $\xi \in (0,
    \infty)$ and $\Lambda(\cdot)$ is differentiable at $\xi$.
  \item $\E |B|^\xi < \infty$, $\E (A|D|)^\xi < \infty$
  \item $\P(A > 1, B > 0) > 0$ or $P(A > 1, B \geq 0, D > 0) > 0$.
  \end{enumerate}
  Collamore and Vidyashankar \cite{Collamore20133378} showed that
  under these conditions $V_n$ is an aperiodic, $\phi$-irreducible and
  geometrically ergodic Markov chain.
\end{frame}

\begin{frame}
  The recurrent property of a geometriclly ergodic MC means imples the
  process the MC has a regeneration structure \cite{Nummelin1978}:
  There exist random times $1 \leq T_1 < T_2 < \cdots$ such that
  \begin{enumerate}
  \item $T_0, T_{i+1} - T_{i}$, $i = 1, 2, \cdots$ are finite with
    probability 1 and are independent, identically distributed random
    variables.
  \item The blocks $(V_{T_i}, \dots, V_{T_{i+1} - 1})$ are independent;
  \item $\P(V_{T_i} \in A | V_{T_{i}-1}, \dots, V_0 = \nu(A)$
  \end{enumerate}
\end{frame}

\begin{frame}
  A recurrent MC is also positive, i.e. it admits an invariant
  measure $\pi$ \cite{Meyn:2009:MCS:1550713}:
  \[
  \pi(A) = \int_E \pi(dx) P(x, A)
  \]
  for all $A \in \mathcal B(E)$.
  By geometric ergodicity, $V_n$ converges to a random variable $V$
  distributed according to $\pi$. We want to have an efficient estimator of the
  probability $\P(V > u)$ as $u \to \infty$. It is understood
  \[
   \lim_{u \to \infty} u^\xi \P(V > u) = C
  \]
  for some constant $C > 0$.
\end{frame}

\begin{frame}
  Collamore and Vidyashankar \cite{collamore2014} showed, when
  $V_0$ is in the set $\mathcal C$ and is distributed according to
  $\gamma$ defined as
  \[
  \gamma(E) = \pi(E)/\pi(\mathcal C)\text{ for all } E \in \mathcal B(\mathcal C)
  \]
  $\mathcal E_u = \pi(\mathcal C) N_u |A_{T_u} \cdots A_{1}
  V_0|^{-\xi}\1{T_u < \tau}$ is an unbiased and efficient estimator
  w.r.t the dual measure $\mathcal D$:
  \[
  \E_{\mathcal D} E_u = \P(V > u)
  \]
  and by ``efficient'' we mean
  \[
  \lim_{u \to \infty} {
    \var_{\mathcal D}(\mathcal E_u)
    \over
    (\E_{\mathcal D} \mathcal E_u)^2
  } < \infty
  \]
  as defined by Asmusen \cite{opac-b1123521}.
  \begin{eqnarray*}
    K &:=& \inf\{n \geq 1: V_n \in \mathcal C\} \\
    T_u &:=& \min\{n \geq 1: V_n > u\} \\
    N_u &:=& \sum_{n=0}^{K-1} \1{V_n > u}
  \end{eqnarray*}
\end{frame}

\begin{frame}
  $\E_D$ and $\var$ denote expectation and variance taken w.r.t to the
  dual measure $\mathcal D$, respectively: If $n < T_u$, i.e. the MC
  has not exceeded the level $u$, $A_{n+1}$ is sampled from the
  $\xi$-shifted distribution $\mu_\xi$; otherwise it is sampled from
  $\mu$. For a set $S \subseteq \text{supp}(A)$
  \begin{eqnarray*}
    \mu_\xi (S) &=& {
      \int_{S} a^\xi \mu(da)
      \over
      \int_{\text{supp}(A)} a^\xi \mu(da)
    } = {\E A^\xi \I{S}(A)
    \over
    \E A^\xi
    }
  \end{eqnarray*}
  Denote
  \begin{eqnarray*}
    \lambda(\alpha) &=& \E A^{\alpha} \\
    \lambda_\beta(\alpha) &=& \E_\beta A^{\alpha} = {
      \E A^{\alpha + \beta}
      \over
      \E A^{\beta}
    }
  \end{eqnarray*}
\end{frame}

\section{Sketch of the Proof of Consistency}
\begin{frame}
  Assert:
  \[
  \lim_{k \to \infty} \lim_{n \to \infty} \hat{\pi}_k \hat{\mathcal
    E}_{u,n} = \P(V > u)
  \]
  What follows is a sketch of the proof by Collamore, Guoqing and
  Vidyashankar \cite{collamore2014}. By the law of large numbers for
  Markov chains
  \[
  \P(V > u) = \lim_{n \to \infty} {1 \over n} \left\{
  \sum_{i=0}^{K_{R_n} - 1} \1{V_i > u}
  +
  \sum_{i=K_{R_n}}^{n} \1{V_i > u}
  \right\}
  \]
  where $K_0 = 0$ and $K_n$ is the time of the $n$-th visit to
  $\mathcal C$. $R_n = \sum_{i=1}^n \1{V_i \in C}$ is the number of
  visits to $\mathcal C$ before time $n$.
\end{frame}

\begin{frame}
  By the Markov renewal theorem and the geometric ergodicity of $V_n$,
  \[
  \lim_{n \to \infty} e^{\epsilon(J(n) - I(n))} = {
    \E \tau e^{\epsilon \tau}
    \over
    \E \tau
  }  < \infty
  \]
  where $\tau$ is a typical regeneration time, $I(n)$ is the last
  regeneration time before time $n$ and $J(n)$ the first regeneration
  time after $n$. Then by a Borel-Cantelli argument
  \[
  {1 \over n}\sum_{i=K_{R_n}}^n \1{V_i > u} \to 0
  \]
\end{frame}

\section{Sketch of the Proof of Efficiency}
\begin{frame}
  To show that the estimator is efficient, we need to show
  \[
  \lim_{u \to \infty} {
    \var_{\mathcal D}(\mathcal E_u)
    \over
    (\E_{\mathcal D} \mathcal E_u)^2
  } < \infty
  \]
  Because
  \[
  \lim_{u \to \infty} u^\xi \P(V > u) = C
  \]
  It suffices to show
  \[
  \lim_{u \to \infty}
    \E_{\mathcal D}(\mathcal E_u^2 u^{2\xi}) < \infty
  \]
\end{frame}

\begin{frame}
  First consider the case $\lambda(-\alpha) < \infty$ for some $\alpha
  > \xi$. Using
  \begin{eqnarray*}
    V_{T_u} &>& u \\
    {V_{T_u}
      \over
      A_{T_u} \cdots A_0
    } &\leq& \sum_{n=0}^{T_u} {
      \tilde B_n
      \over
      A_n \cdots A_0
    }
  \end{eqnarray*}
  where $A_{0} := 1$, $A_i$ for $i =1, 2, \dots$ are iid and $\tilde B_n :=
  A_n |D_n| + |B_n|$. We get
  \[
  \E_{\mathcal D}(\mathcal E_u^2 u^{2\xi}) \leq
  \E_{\mathcal D} \left[
    N_u^2
    \left(
    \sum_{n=0}^{\infty} {
      \tilde B_n
      \over
      |A_n \cdots A_0|
    }
    \1{n \leq T_u < K}
    \right)^{2\xi}
    \right]
  \]
\end{frame}

\begin{frame}
  If $\xi \geq 1/2$, applying Minkowski's inequality, changing to
  the original measure and then using H\"older's inequality gives
  \begin{eqnarray*}
    (u^{2\xi} \E_{\mathcal D} \mathcal{E}^{2\xi}_u)^{1/2\xi} &\leq&
    \sum_{n=0}^\infty
    (\E N_u^{2r} \1{n-1 < T_u \wedge K})^{1/2r\xi} \times \\
    && [\E (A_n^{-1} \tilde B_n^2)^{s\xi}]^{1/2s\xi} \times \\
    && \E[
    (A_{n-1} \cdots A_0)^{-s\xi}
    \1{n-1 < T_u \wedge K}
    ]^{1/2s\xi}
  \end{eqnarray*}
  For the last term on RHS, change to the $-s\xi$ shifted measure to
  obtain
  \[
  \E[
    (A_{n-1} \cdots A_0)^{-s\xi}
    \1{n-1 < T_u \wedge K}
    ]^{1/2s\xi} = \lambda(-s\xi)^{n-1} \P_{-s\xi}(n-1 < T_u \wedge K)
  \]
\end{frame}

\begin{frame}
  Using the drift condition in the $-s\xi$-shifted measure leads to
  \begin{eqnarray*}
    \P_{-s\xi}(n-1 < T_u \wedge K) &\leq& C
    \lambda_{-s\xi}(\alpha)^{n-1} \\
    &=& C \left[
      {
      \lambda(-s\xi + \alpha)
      \over
      \lambda(-s\xi)
    }
    \right]^{n-1}. \\
    \lambda(-s\xi)^{n-1} \P_{-s\xi}(n-1 < T_u \wedge K) &\leq& C
    \lambda(-s\xi + \alpha)^{n-1}
  \end{eqnarray*}
  for some $0 < \alpha < \xi$. By assumption $\lambda(-s\xi) < \infty$
  and hence $\alpha$ can be chosen such that $\lambda(-s\xi + \alpha)
  < 1$.
\end{frame}

\begin{frame}
  As for $\E N_u^{2r}$, first notice
  \[
  N_u = \sum_{i=0}^{K-1} \1{V_i > u} \leq K
  \]
  Hence
  \begin{eqnarray*}
    \E N_u^{2r}\1{n-1 \leq T_u < K} &\leq& \sum_{i=n}^\infty i^{2r}
    \P(K > i-1) \\
    &\leq& c \sum_{i=n}^\infty i^{2r} \lambda(\alpha)^i < \infty
  \end{eqnarray*}
  for some constant $c$ and $\alpha$ chosen such that $\lambda(\alpha)
  < 1$.
\end{frame}

\begin{frame}
  The case $\xi < 1/2$ is proven analogously by exploiting
  subadditivity of the function $(\cdot)^{2\xi}$ in place of
  Minkowski's inequality. The somewhat more complicated case
  $\lambda(-\xi) = \infty$ is proven with an extra assumption
  \[
  \E (A^{-1} \tilde B)^\alpha < \infty
  \]
  for all $\alpha > 0$. First Notice when $n \leq T_u < K$
  \begin{eqnarray*}
    |V_n| &\leq& A_n |V_{n-1}| \left(
      1 + {
        \tilde B_n
        \over
        A_n |V_{n-1}|
      }
    \right)
  \end{eqnarray*}
  and $M \leq V_n < u$. Hence
  \[
  A_n^{-s\xi} \leq \left(
    {
      u \over M
    }
    \right)^{s\xi} 
    \left(
    1 + {
      \tilde B_n
      \over
      A_n M
    }
  \right)^{s\xi}
  \]
\end{frame}

\begin{frame}
  Since by assumption $\E (A_n^{-1} \tilde B_n)^{s\xi} < \infty$, we
  have
  \[
  \E (A_{n} \cdots A_0)^{-s\xi} < \infty
  \]
  for fixed $n \leq T_u$. Then it suffices to show
  \[
  \sum_{n=0}^\infty {
    1 \over
    (A_n \cdots A_0)^{s\xi}
  } < \infty
  \]
  The proof of finiteness of the other terms are identical to that in
  the other case.
\end{frame}

\begin{frame}
  Define a sequence $L_k \to 0$ as $k \to \infty$ and
  \[
  \E {
    1 \over
    (A_k \cdots A_0)^\zeta
  }  \1{F_k^\complement} < {1 \over k^2}
  \]
  where $F_k$ denotes the event $A_1, A_2, \dots, A_{k-1} \geq L_k$.
\end{frame}

\begin{frame}
  Let
  \begin{eqnarray*}
    \bar A_{n,k} &=& A_{n} \1{A_n > L_k} + L_k \1{A_n \leq L_k} \\
    \lambda_k(\alpha) &=& \E \bar A_{1,k}^\alpha
  \end{eqnarray*}
  Observe that $\bar A_{n,k} \to A_n$ and $\lambda_k(\alpha) \to
  \lambda(\alpha)$ as $k \to \infty$. By Minkowski's inequality
  followed by H\"older's inequality, we obtain
  \begin{eqnarray*}
    \E_{-s\xi} \left( |V_{1,k}|^\beta | V_{0,k} = w\right) &\leq& {
    t [\E \bar A_{1,k}^{p(\beta - \zeta)}]^{1/p}
    \over
    \lambda_k(-\zeta)
  } w^\beta \times \\
  && \left\{
    t^{-q} \E \left[
      1 + {
        \tilde B_1
        \over
        w \bar A_{1,k}
      }
    \right]^{q\beta}
  \right\}^{1/q}
  \end{eqnarray*}
  for some $t > 1$. Notice that $p$ may be chosen so small and $\beta$
  chosen appropriately so that
  \[
  \E \bar A_{1,k}^{p(\beta - \zeta)} < \rho < 1 \text{ for all } k > k_0
  \]
\end{frame}

\begin{frame}
  This is because $\lambda_{1,k}(p(\beta - \zeta)) \to \lambda(p(\beta
  - \zeta))$ and
  \[
  \inf_{\alpha > 1} \lambda(\alpha) < 1
  \]
  Choose $M$ sufficiently large so that $\forall w > M$,
  \[
    t^{-q} \E \left[
      1 + {
        \tilde B_1
        \over
        w \bar A_{1,k}
      }
    \right]^{q\beta} < 1
  \]
  So we have obtained an inequality similar to the drift condition in
  the $\lambda(-\xi) < \infty$ case. Iterate on this inequality yields
  \[
  \sum_{k=0}^\infty \E {
    1 \over
    (A_k \cdots A_0)^\zeta \1{F_k}
  } < \infty
  \]

\end{frame}
\bibliographystyle{unsrt}
\bibliography{../../thesis/econophysics}
\end{document} 

