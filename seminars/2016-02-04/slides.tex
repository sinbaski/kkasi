%%%%%%%%%%%%%%%%%%%%%%%%%%%%%%%%%%%%%%%%%
% Beamer Presentation
% LaTeX Template
% Version 1.0 (10/11/12)
%
% This template has been downloaded from:
% http://www.LaTeXTemplates.com
%
% License:
% CC BY-NC-SA 3.0 (http://creativecommons.org/licenses/by-nc-sa/3.0/)
%
%%%%%%%%%%%%%%%%%%%%%%%%%%%%%%%%%%%%%%%%%

%----------------------------------------------------------------------------------------
%	PACKAGES AND THEMES
%----------------------------------------------------------------------------------------

\documentclass{beamer}

\mode<presentation> {

% The Beamer class comes with a number of default slide themes
% which change the colors and layouts of slides. Below this is a list
% of all the themes, uncomment each in turn to see what they look like.

%\usetheme{default}
%\usetheme{AnnArbor}
%\usetheme{Antibes}
%\usetheme{Bergen}
%\usetheme{Berkeley}
%\usetheme{Berlin}
%\usetheme{Boadilla}
%\usetheme{CambridgeUS}
\usetheme{Copenhagen}
%\usetheme{Darmstadt}
%\usetheme{Dresden}
%\usetheme{Frankfurt}
%\usetheme{Goettingen}
%\usetheme{Hannover}
%\usetheme{Ilmenau}
%\usetheme{JuanLesPins}
%\usetheme{Luebeck}
%\usetheme{Madrid}
%\usetheme{Malmoe}
%\usetheme{Marburg}
%\usetheme{Montpellier}
%\usetheme{PaloAlto}
%\usetheme{Pittsburgh}
%\usetheme{Rochester}
%\usetheme{Singapore}
%\usetheme{Szeged}
%\usetheme{Warsaw}

% As well as themes, the Beamer class has a number of color themes
% for any slide theme. Uncomment each of these in turn to see how it
% changes the colors of your current slide theme.

%\usecolortheme{albatross}
%\usecolortheme{beaver}
%\usecolortheme{beetle}
%\usecolortheme{crane}
%\usecolortheme{dolphin}
%\usecolortheme{dove}
%\usecolortheme{fly}
%\usecolortheme{lily}
%\usecolortheme{orchid}
%\usecolortheme{rose}
%\usecolortheme{seagull}
%\usecolortheme{seahorse}
%\usecolortheme{whale}
%\usecolortheme{wolverine}

%\setbeamertemplate{footline} % To remove the footer line in all slides uncomment this line
%\setbeamertemplate{footline}[page number] % To replace the footer line in all slides with a simple slide count uncomment this line

%\setbeamertemplate{navigation symbols}{} % To remove the navigation symbols from the bottom of all slides uncomment this line
}

\usepackage{graphicx} % Allows including images
\usepackage{booktabs} % Allows the use of \toprule, \midrule and \bottomrule
                      % in tables

\usepackage{amssymb,amsmath,amsthm,amsfonts}
\usepackage{mathrsfs}
\usepackage{dsfont}
\usepackage{enumerate}

%\newtheorem{mdef}{Definition}
%\newtheorem{theorem}{Theorem}
\newcommand{\eqsplit}[2]{
  \begin{equation}\label{#2}
    \begin{split}
      #1
    \end{split}
  \end{equation}}
\newcommand{\eqnsplit}[1]{
  \begin{eqnarray*}
    #1
  \end{eqnarray*}}
\newcommand{\tran}[1]{
  \tilde{#1}
}
\newcommand{\td}[2]{
  \frac{d #1}{d #2}
}
\newcommand{\pd}[2]{
  \frac{\partial #1}{\partial #2}
}
\newcommand{\ppd}[2]{
  \frac{\partial^2 #1}{\partial #2^2}
}
\newcommand{\pdd}[3]{
  \frac{\partial^2 #1}{\partial #2 \partial #3}
}
\newcommand{\otd}[1]{
  \frac{d}{d #1}
}
\newcommand{\opd}[1]{
  \frac{\partial}{\partial #1}
}
\newcommand{\oppd}[1]{
  \frac{\partial^2}{\partial #1^2}
}
\newcommand{\opdd}[2]{
  \frac{\partial^2}{\partial #1 \partial #2}
}
\newcommand{\ket}[1]{
  |#1\rangle
}
\newcommand{\bra}[1]{
  \langle#1|
}
\newcommand{\inn}[1]{
  \langle#1\rangle
}
\newcommand{\mean}[1]{
  \langle#1\rangle
}
\newcommand{\tr}{
  \text{tr}\,
}
\newcommand{\re}{
  \text{Re}\,
}
\newcommand\im{
  \text{Im}\,
}
\newcommand{\var}{
  \text{var}
}
\newcommand{\arcsinh}{
  \sinh^{-1}
}
\newcommand{\arccosh}{
  \cosh^{-1}
}
\newcommand{\erfc}{
  \text{erfc}
}
\newcommand{\E}{
  \mathbb{E}
}
\renewcommand{\P}{
  \mathbb{P}
}
\newcommand{\I}[1]{
  \mathbf{1}_{\{#1\}}
}
\newcommand{\1}[1]{
  \mathds{1}_{\{#1\}}
}
\newcommand{\diag}{
  \text{diag\,}
}
\newcommand{\M}{
  {\text{max}}
}
\newcommand{\m}{
  {\text{min}}
}
\newcommand{\ph}{
  {\text{arg}\,}
}
\newcommand\erf{
  \text{erf}
}
\renewcommand\vec[1]{
  \mathbf{#1}
}
\newcommand\mtx[1]{
  \mathbf{#1}
}
\newcommand\ed{
  \,{\buildrel d \over =}\,
}




%----------------------------------------------------------------------------------------
%	TITLE PAGE
%----------------------------------------------------------------------------------------

\title{Importance Sampling Estimator of 1D MC Exceedance Probability} % The
                                % short title appears at the bottom of
                                % every slide, the full title is only
                                % on the title page

\author{Xie Xiaolei} % Your name
\institute[UCPH] % Your institution as it will appear on the bottom of every slide, may be shorthand to save space
{
Copenhagen University  \\ % Your institution for the title page
\medskip
\textit{hnq365@math.ku.dk} % Your email address
}
\date{\today} % Date, can be changed to a custom date

\begin{document}

\begin{frame}
\titlepage % Print the title page as the first slide
\end{frame}

% \begin{frame}
% \frametitle{Overview}
% \tableofcontents
% \end{frame}

%----------------------------------------------------------------------------------------
%	PRESENTATION SLIDES
%----------------------------------------------------------------------------------------
% \section{Simple dependence models}
%------------------------------------------------
\section{Setup of the Problem}
\begin{frame}
  Consider the recurrence equation 
  \[
  V_n = A_n (V_{n-1} \vee D_n) + B_n
  \]
  where $A, B, D$ satisfy the following conditions:
  \begin{enumerate}
  \item $A$ is positive, absolutely continuous and has a non-trivial continuous density in
    a neighborhood of $\mathbb R$.
  \item Define $\Lambda(\alpha) := \log \E A^\alpha$, $\Lambda(\xi) < \infty$ for some $\xi \in (0,
    \infty)$ and $\Lambda(\cdot)$ is differentiable at $\xi$.
  \item $\E |B|^\xi < \infty$, $\E (A|D|)^\xi < \infty$
  \item $\P(A > 1, B > 0) > 0$ or $P(A > 1, B \geq 0, D > 0) > 0$.
  \end{enumerate}
  Collamore and Vidyashankar showed \cite{Collamore20133378} that
  under these conditions $V_n$ is $\phi$-irreducible and geometrically
  ergodic.
\end{frame}

\begin{frame}
  W. Doeblin, J.L.Doob and S.Orey showed $\phi$-irreducibility implies
  the existence of a small set $\mathcal C$, i.e. $\exists J \subset
  E$ such that $|\mathcal C| > 0, |J| > 0$ and $\forall x \in \mathcal
  C, \forall y \in J$,
  \[
  p(x, y) > \beta > 0
  \]
  where $p(\cdot, \cdot)$ is the transition kernel of the MC. Then it
  follows
  \begin{eqnarray*}
    P(x, A) := \P(X_n \in A | X_{n-1}=x) &\geq& \beta |J|
    \I{\mathcal C}(x) {|A \cap J| \over |J|} \\
    &:=& c \I{\mathcal C}(x) \nu(A)
  \end{eqnarray*}
  where $|\cdot|$ denotes the Lebesgue measure. This is called a
  minorization condition.
\end{frame}

\begin{frame}
  The minorization condition imples the MC has a regeneration
  structure: There exist random times $1 \leq T_1 < T_2 < \cdots$ such
  that
  \begin{enumerate}
  \item The blocks $(V_{T_i}, \dots, V_{T_{i+1} - 1}, T_{i+1} -
    T_{i})$ are independent;
  \item $T_{i+1} - T_{i}$ are iid r.v.
  \item $\P(V_{T_i} \in A | V_{T_{i}-1}) = \nu(A)$
  \end{enumerate}
  
\end{frame}

\begin{frame}
  We want to derive an efficient estimator of the probability
  \[
  \lim_{u \to \infty} \P(V > u)
  \]
\end{frame}


\section{Deriving the Estimator}
\bibliographystyle{unsrt}
\bibliography{../../thesis/econophysics}
\end{document} 
