%%%%%%%%%%%%%%%%%%%%%%%%%%%%%%%%%%%%%%%%%
% Beamer Presentation
% LaTeX Template
% Version 1.0 (10/11/12)
%
% This template has been downloaded from:
% http://www.LaTeXTemplates.com
%
% License:
% CC BY-NC-SA 3.0 (http://creativecommons.org/licenses/by-nc-sa/3.0/)
%
%%%%%%%%%%%%%%%%%%%%%%%%%%%%%%%%%%%%%%%%%

%----------------------------------------------------------------------------------------
%	PACKAGES AND THEMES
%----------------------------------------------------------------------------------------

\documentclass{beamer}

\mode<presentation> {

% The Beamer class comes with a number of default slide themes
% which change the colors and layouts of slides. Below this is a list
% of all the themes, uncomment each in turn to see what they look like.

%\usetheme{default}
%\usetheme{AnnArbor}
%\usetheme{Antibes}
%\usetheme{Bergen}
%\usetheme{Berkeley}
%\usetheme{Berlin}
%\usetheme{Boadilla}
%\usetheme{CambridgeUS}
\usetheme{Copenhagen}
%\usetheme{Darmstadt}
%\usetheme{Dresden}
%\usetheme{Frankfurt}
%\usetheme{Goettingen}
%\usetheme{Hannover}
%\usetheme{Ilmenau}
%\usetheme{JuanLesPins}
%\usetheme{Luebeck}
%\usetheme{Madrid}
%\usetheme{Malmoe}
%\usetheme{Marburg}
%\usetheme{Montpellier}
%\usetheme{PaloAlto}
%\usetheme{Pittsburgh}
%\usetheme{Rochester}
%\usetheme{Singapore}
%\usetheme{Szeged}
%\usetheme{Warsaw}

% As well as themes, the Beamer class has a number of color themes
% for any slide theme. Uncomment each of these in turn to see how it
% changes the colors of your current slide theme.

%\usecolortheme{albatross}
%\usecolortheme{beaver}
%\usecolortheme{beetle}
%\usecolortheme{crane}
%\usecolortheme{dolphin}
%\usecolortheme{dove}
%\usecolortheme{fly}
%\usecolortheme{lily}
%\usecolortheme{orchid}
%\usecolortheme{rose}
%\usecolortheme{seagull}
%\usecolortheme{seahorse}
%\usecolortheme{whale}
%\usecolortheme{wolverine}

%\setbeamertemplate{footline} % To remove the footer line in all slides uncomment this line
%\setbeamertemplate{footline}[page number] % To replace the footer line in all slides with a simple slide count uncomment this line

%\setbeamertemplate{navigation symbols}{} % To remove the navigation symbols from the bottom of all slides uncomment this line
}

\usepackage{graphicx} % Allows including images
\usepackage{booktabs} % Allows the use of \toprule, \midrule and \bottomrule
                      % in tables

\usepackage{amssymb,amsmath,amsthm,amsfonts}
\usepackage{mathrsfs}
\usepackage{dsfont}
\usepackage{enumerate}

%\newtheorem{mdef}{Definition}
%\newtheorem{theorem}{Theorem}
\newcommand{\eqsplit}[2]{
  \begin{equation}\label{#2}
    \begin{split}
      #1
    \end{split}
  \end{equation}}
\newcommand{\eqnsplit}[1]{
  \begin{eqnarray*}
    #1
  \end{eqnarray*}}
\newcommand{\tran}[1]{
  \tilde{#1}
}
\newcommand{\td}[2]{
  \frac{d #1}{d #2}
}
\newcommand{\pd}[2]{
  \frac{\partial #1}{\partial #2}
}
\newcommand{\ppd}[2]{
  \frac{\partial^2 #1}{\partial #2^2}
}
\newcommand{\pdd}[3]{
  \frac{\partial^2 #1}{\partial #2 \partial #3}
}
\newcommand{\otd}[1]{
  \frac{d}{d #1}
}
\newcommand{\opd}[1]{
  \frac{\partial}{\partial #1}
}
\newcommand{\oppd}[1]{
  \frac{\partial^2}{\partial #1^2}
}
\newcommand{\opdd}[2]{
  \frac{\partial^2}{\partial #1 \partial #2}
}
\newcommand{\ket}[1]{
  |#1\rangle
}
\newcommand{\bra}[1]{
  \langle#1|
}
\newcommand{\inn}[1]{
  \langle#1\rangle
}
\newcommand{\mean}[1]{
  \langle#1\rangle
}
\newcommand{\tr}{
  \text{tr}\,
}
\newcommand{\re}{
  \text{Re}\,
}
\newcommand\im{
  \text{Im}\,
}
\newcommand{\var}{
  \text{var}
}
\newcommand{\arcsinh}{
  \sinh^{-1}
}
\newcommand{\arccosh}{
  \cosh^{-1}
}
\newcommand{\erfc}{
  \text{erfc}
}
\newcommand{\E}{
  \mathbb{E}
}
\renewcommand{\P}{
  \mathbb{P}
}
\newcommand{\I}[1]{
  \mathbf{1}_{\{#1\}}
}
\newcommand{\1}[1]{
  \mathds{1}_{\{#1\}}
}
\newcommand{\diag}{
  \text{diag\,}
}
\newcommand{\M}{
  {\text{max}}
}
\newcommand{\m}{
  {\text{min}}
}
\newcommand{\ph}{
  {\text{arg}\,}
}
\newcommand\erf{
  \text{erf}
}
\renewcommand\vec[1]{
  \mathbf{#1}
}
\newcommand\mtx[1]{
  \mathbf{#1}
}
\newcommand\ed{
  \,{\buildrel d \over =}\,
}




%----------------------------------------------------------------------------------------
%	TITLE PAGE
%----------------------------------------------------------------------------------------

\title{Power Law and Markov Chains} % The
                                % short title appears at the bottom of
                                % every slide, the full title is only
                                % on the title page

\author{Xie Xiaolei} % Your name
\institute[UCPH] % Your institution as it will appear on the bottom of every slide, may be shorthand to save space
{
Copenhagen University  \\ % Your institution for the title page
\medskip
\textit{hnq365@math.ku.dk} % Your email address
}
\date{\today} % Date, can be changed to a custom date

\begin{document}

\begin{frame}
\titlepage % Print the title page as the first slide
\end{frame}

% \begin{frame}
% \frametitle{Overview}
% \tableofcontents
% \end{frame}

%----------------------------------------------------------------------------------------
%	PRESENTATION SLIDES
%----------------------------------------------------------------------------------------
% \section{Simple dependence models}
%------------------------------------------------
\section{Power Law}
\begin{frame}
  Let's start with a little quiz: At what rate is the integral tending
  to zero as $u \to \infty$, where $\alpha > 1$
  \[
  \int_{u}^\infty x^{-\alpha} (\log^2 x + \log x + 1)^{1/3} dx
  \]
\end{frame}

\begin{frame}
  The integrand is a regularly varying function. A function $f$ is
  regularly varying if
  \[
  \lim_{u \to \infty} {
    f (au) \over f(u)
  } = a^\alpha
  \]
  where $a$ is a constant and $\alpha$ is the index of regular
  variation. The simplest form of a regularly varying function is a
  power-law
  \[
  f(x) = C x^{-\alpha}
  \]
  If, instead $\lim_{u \to \infty} f(au)/f(u) = 1$, $f(\cdot)$ is
  slowly varying.
\end{frame}

\begin{frame}
  Karamata's theorem tells us, for $\alpha > 1$
  \[
  \int_{u}^\infty x^{-\alpha} L(x) dx \sim {L(u) \over (\alpha - 1) u^{\alpha-1}}
  \]
  where $L(x)$ is a slowly varying function.
\end{frame}

\begin{frame}
  A random variable $X$ is said to have a power-law tail if
  \[
  \lim_{u \to \infty} u^{\alpha} \P(X > u) = C 
  \]
  Random variables with a power-law tail appear in many
  areas of science, for example
  \begin{itemize}
  \item magnitude of earthquakes, crators on the moon, sun flares etc.
  \item Number of neighbors to which a neuron transmits signals.
  \item Frequencies of words in most languages.
  \item Distribution of day-to-day asset returns
  \end{itemize}
\end{frame}

\begin{frame}
  Power laws have an interesting feature: Suppose $\alpha < 2$,
  $\P(X_i > u) \sim C u^{-\alpha}$. Let
  \[
  S_n = X_1 + X_2 + \cdots + X_n
  \]
  Consider the distribution of $a_n^{-1} (S_n - n \E X_1)$. It 
  does NOT converge to a normal law; instead it converges to an
  $\alpha$-stable law, whose distribution function does not have a
  simple analytic form.
\end{frame}

\begin{frame}
  In fact, the sum of random variables with a power-law tail behaves
  rather differently from the sum of random variables with a lighter
  tail ($\P(X > u)$ drops much faster). for example
  \begin{eqnarray*}
    \P(S_n > u) \sim n \P(X > u) \sim \P(\max_{1 \leq i \leq n} X_i > u)
  \end{eqnarray*}
\end{frame}

\begin{frame}
  Here are some more interesting properties of regular variation
  $\mathcal R_{\alpha}$:
  \begin{itemize}
    \item Suppose $X \sim \mathcal R_{-\alpha}$, $Y \sim R_{-\beta}$,
      $\alpha < \beta$. Then $X + Y \sim \mathcal R_{-\alpha}$.
    \item Suppose
      \begin{enumerate}
      \item
        \begin{eqnarray*}
          \P(X > u) &\sim& p u^{-\alpha} \\
          \P(X \leq -u) &\sim& q u^{-\alpha} \\
        \end{eqnarray*}
        with $p + q = 1$
      \item $\E X = 0$ if $\E X < \infty$
      \end{enumerate}
      Then a large deviation result holds
      \[
      \lim_{n \to \infty} \sup_{x > c_n}
      \left|
          {\P(S_n > x)
            \over
           n\P(X > x)
          } - p
      \right| = 0
      \]
      where $c_n$ is some sequence that tends to $\infty$.
  \end{itemize}
\end{frame}

\section{Markov Chains}
\begin{frame}
  So where do power-law tails come from? An explanation is the
  stochastic process
  \[
  V_n = A_n \max\{V_{n-1}, D_n\} + B_n
  \]
  $V_n$ is a Markov chain, i.e. the process is forgetful: Its full
  history does not influence its future more than does its most recent
  state. More precisely
  \[
  \E(V_n | V_{n-1}, \dots, V_0) = \E (V_n | V_{n-1})
  \]
\end{frame}

\begin{frame}
  Collamore and Vidyashankar showed that,
  with some mild conditions, the chain $V_n$ is geometrically
  recurrent, implying it admits an invariant
  measure $\pi$:
  \[
  \pi(A) = \int_E \pi(dx) P(x, A)
  \]
  In other words, $V_n$ converges to a random variable $V$ distributed
  according to $\pi$. It is shown
  \[
   \lim_{u \to \infty} u^\xi \P(V > u) = C
  \]
  for some constant $C > 0$.
\end{frame}

\begin{frame}
  A similar Markov chain gives rise to power-law tails in $d$-dimensions
  \[
  V_n = A_n V_{n-1} + B_n
  \]
  where $V_n, B_n \in \mathbb R^d$,
  $A_n \in \mathbb R_+^{d \times d}$.
  Under appropriate conditions,
  $V_n \overset{d}{\to} V$. $V$ satisfies
  \[
  \lim_{u \to \infty} u^\alpha\P(\vec{e} \cdot V > u) = g_\alpha(\vec{e})
  \]
  where $\vec{e} \in \mathbb S^{d-1}$. This result follows from
  Kesten's theorem.
\end{frame}

\begin{frame}
  \centering
  Thank you for your attention!
\end{frame}

\bibliographystyle{unsrt}
% \bibliography{../../thesis/econophysics}
\end{document} 

