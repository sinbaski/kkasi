%%%%%%%%%%%%%%%%%%%%%%%%%%%%%%%%%%%%%%%%%
% Beamer Presentation
% LaTeX Template
% Version 1.0 (10/11/12)
%
% This template has been downloaded from:
% http://www.LaTeXTemplates.com
%
% License:
% CC BY-NC-SA 3.0 (http://creativecommons.org/licenses/by-nc-sa/3.0/)
%
%%%%%%%%%%%%%%%%%%%%%%%%%%%%%%%%%%%%%%%%%

%----------------------------------------------------------------------------------------
%	PACKAGES AND THEMES
%----------------------------------------------------------------------------------------

\documentclass{beamer}

\mode<presentation> {

% The Beamer class comes with a number of default slide themes
% which change the colors and layouts of slides. Below this is a list
% of all the themes, uncomment each in turn to see what they look like.

%\usetheme{default}
%\usetheme{AnnArbor}
%\usetheme{Antibes}
%\usetheme{Bergen}
%\usetheme{Berkeley}
%\usetheme{Berlin}
%\usetheme{Boadilla}
%\usetheme{CambridgeUS}
\usetheme{Copenhagen}
%\usetheme{Darmstadt}
%\usetheme{Dresden}
%\usetheme{Frankfurt}
%\usetheme{Goettingen}
%\usetheme{Hannover}
%\usetheme{Ilmenau}
%\usetheme{JuanLesPins}
%\usetheme{Luebeck}
%\usetheme{Madrid}
%\usetheme{Malmoe}
%\usetheme{Marburg}
%\usetheme{Montpellier}
%\usetheme{PaloAlto}
%\usetheme{Pittsburgh}
%\usetheme{Rochester}
%\usetheme{Singapore}
%\usetheme{Szeged}
%\usetheme{Warsaw}

% As well as themes, the Beamer class has a number of color themes
% for any slide theme. Uncomment each of these in turn to see how it
% changes the colors of your current slide theme.

%\usecolortheme{albatross}
%\usecolortheme{beaver}
%\usecolortheme{beetle}
%\usecolortheme{crane}
%\usecolortheme{dolphin}
%\usecolortheme{dove}
%\usecolortheme{fly}
%\usecolortheme{lily}
%\usecolortheme{orchid}
%\usecolortheme{rose}
%\usecolortheme{seagull}
%\usecolortheme{seahorse}
%\usecolortheme{whale}
%\usecolortheme{wolverine}

%\setbeamertemplate{footline} % To remove the footer line in all slides uncomment this line
%\setbeamertemplate{footline}[page number] % To replace the footer line in all slides with a simple slide count uncomment this line

%\setbeamertemplate{navigation symbols}{} % To remove the navigation symbols from the bottom of all slides uncomment this line
}

\usepackage{graphicx} % Allows including images
\usepackage{booktabs} % Allows the use of \toprule, \midrule and \bottomrule
                      % in tables

\renewcommand{\P}{
\mathbb P
}

%----------------------------------------------------------------------------------------
%	TITLE PAGE
%----------------------------------------------------------------------------------------

\title{Eigenvalues of a fixed-dimensional matrix} % The short title appears at the bottom of every slide, the full title is only on the title page

\author{Xie Xiaolei} % Your name
\institute[UCPH] % Your institution as it will appear on the bottom of every slide, may be shorthand to save space
{
Copenhagen University  \\ % Your institution for the title page
\medskip
\textit{xie.xiaolei@gmail.com} % Your email address
}
\date{\today} % Date, can be changed to a custom date

\begin{document}

\begin{frame}
\titlepage % Print the title page as the first slide
\end{frame}

% \begin{frame}
% \frametitle{Overview}
% \tableofcontents
% \end{frame}

%----------------------------------------------------------------------------------------
%	PRESENTATION SLIDES
%----------------------------------------------------------------------------------------
% \section{Simple dependence models}
%------------------------------------------------
\section{Approximating Sample Covariance Matrices}
\begin{frame}
\frametitle{$X_{it} = \theta Z_{i, t-1} + Z_{i, t}$}
Consider a data matrix $X$ each of whose rows comprises an MA(1) process:
$$
X_{it} = \theta Z_{i, t-1} + Z_{i, t}
$$
where $Z_{i,t} \in \mathcal R_{-\alpha}$ are iid. It is shown
$$
a_n^{-2} |XX' - \text{diag}(XX')| \overset{d}{\to} 0
$$
where $a_n$ is chosen such that $\P(Z > a_n) = 1/n$. By Weyl's perturbation
theorem
\begin{eqnarray*}
&& \max_{j=1,\dots,p}|\lambda_{(j)}(XX') - \lambda_{(j)}(\text{diag}(XX'))|\\
&\leq& a_n^{-2} \|XX' - \text{diag}(XX')\| \\
&\leq& \max_{i=1,\dots,p}\sum_{j \neq i}(XX')_{ij} \overset{d}{\to} 0
\end{eqnarray*}
\end{frame}

\begin{frame}
  \frametitle{$X_{it} = \theta Z_{i, t-1} + Z_{i, t}$}
  $$
  a_n^{-2}\text{diag}(XX') \overset{d}{\to} (1+\theta^2)
  \begin{pmatrix}
    \sum_{t=1}^n Z_{1t}^2 & 0 & \cdots & 0 \\
    0 & \sum_{t=1}^n Z_{2t}^2 & \cdots & 0 \\
    \vdots & \ddots & \vdots & \vdots \\
    0 & 0 & \cdots & \sum_{t=1}^n Z_{pt}^2
  \end{pmatrix}
  $$
\end{frame}
%------------------------------------------------

\begin{frame}
  \frametitle{$X_{it} = \varphi Z_{i-1, t} + Z_{i,t}$}
  Since $Z_1Z_2 \in \mathcal R_{-\alpha}$ while $Z^2 \in \mathcal
  R_{-\alpha/2}$, among the components that add up to $(XX')_{ij}$, only those
  that comprise terms like $Z^2$ actually contribute. Hence it is shown
  $$
  a_n^{-2} (XX' - A) \overset{P}{\to} 0
  $$
  where
  \[
  A =
  \begin{pmatrix}
    \xi_1 + \varphi^2 \xi_0 & \varphi \xi_1 & 0 & \cdots & 0 \\
    \varphi \xi_1 & \xi_2 + \varphi^2 \xi_1 & \varphi \xi_2 & \cdots & 0 \\
    0 & \varphi \xi_2 & \xi_3 + \varphi^2 \xi_2 & \cdots & 0 \\
    \vdots & \vdots & \vdots & \ddots & \vdots \\
    0 & 0 & 0 & \cdots & \xi_p + \varphi^2 \xi_{p-1}
  \end{pmatrix}
  \]
  $\xi_i$ are iid and follow $S_{\alpha/2}$ distribution.
\end{frame}

\begin{frame}
  \frametitle{$X_{it} = \sum_{k=0}^\infty \sum_{l=0}^\infty h_{kl}
    Z_{i-k,t-l}$}
  With this model $a_n^{-2}(XX' - A) \overset{P}{\to} 0$ where
  $A_{ij} = \sum_{k=0}^\infty \left(\sum_{l=0}^\infty h_{k+|j-i|,l}
    h_{kl}\right)\xi_{(i \wedge j) -k}$
  \begin{scriptsize}
    \begin{eqnarray*}
      A &=& \sum_{k=0}^\infty \sum_{l=0}^\infty h_{kl}
      \begin{pmatrix}
        h_{kl} \xi_{1-k} & 
        h_{k+1,l} \xi_{1-k} & 
        h_{k+2,l} \xi_{1-k} & \cdots & h_{k+p-1,l}
        \xi_{1-k}\\
        h_{k+1,l} \xi_{1-k} &
        h_{kl} \xi_{2-k} & 
        h_{k+1,l} \xi_{2-k} & \cdots & h_{k+p-2,l}
        \xi_{2-k}\\
        h_{k+2,l} \xi_{1-k} &
        h_{k+1,l} \xi_{2-k} & 
        h_{kl} \xi_{3-k} & \cdots &
        h_{k+p-3,l} \xi_{3-k}\\
        \vdots & \vdots & \vdots & \ddots & \vdots \\
        h_{k+p-1} \xi_{1-k} & h_{k+p-2} \xi_{2-k} &
        h_{k+p-3} \xi_{3-k} & \cdots & h_{kl} \xi_{p-k}
      \end{pmatrix}
    \end{eqnarray*}
  \end{scriptsize}
  $\xi_i$ are iid and follow $S_{\alpha/2}$ distribution.
\end{frame}
%------------------------------------------------

% \begin{frame}
% \frametitle{Blocks of Highlighted Text}
% \begin{block}{Block 1}
% Lorem ipsum dolor sit amet, consectetur adipiscing elit. Integer lectus nisl, ultricies in feugiat rutrum, porttitor sit amet augue. Aliquam ut tortor mauris. Sed volutpat ante purus, quis accumsan dolor.
% \end{block}

% \begin{block}{Block 2}
% Pellentesque sed tellus purus. Class aptent taciti sociosqu ad litora torquent per conubia nostra, per inceptos himenaeos. Vestibulum quis magna at risus dictum tempor eu vitae velit.
% \end{block}

% \begin{block}{Block 3}
% Suspendisse tincidunt sagittis gravida. Curabitur condimentum, enim sed venenatis rutrum, ipsum neque consectetur orci, sed blandit justo nisi ac lacus.
% \end{block}
% \end{frame}

%------------------------------------------------

% \begin{frame}
% \frametitle{Multiple Columns}
% \begin{columns}[c] % The "c" option specifies centered vertical alignment while the "t" option is used for top vertical alignment

% \column{.45\textwidth} % Left column and width
% \textbf{Heading}
% \begin{enumerate}
% \item Statement
% \item Explanation
% \item Example
% \end{enumerate}

% \column{.5\textwidth} % Right column and width
% Lorem ipsum dolor sit amet, consectetur adipiscing elit. Integer lectus nisl, ultricies in feugiat rutrum, porttitor sit amet augue. Aliquam ut tortor mauris. Sed volutpat ante purus, quis accumsan dolor.

% \end{columns}
% \end{frame}

%------------------------------------------------
\section{Distribution of the Determinant and the Gaps between Eigenvalues}
%------------------------------------------------
\begin{frame}
\frametitle{Distribution of the determinant}
When the data matrix has iid entries, $a_n^{-2} \lambda_{(i)} \overset{d}{\to}
\xi_{i}$, $\xi_{i} \sim S_{\alpha/2}$. According to Samorodnitsky and
Taqqu\cite{SamorodnitskyTaqqu1994} Theorem 4.4.1
$$
\P(\xi_{(i)} > x) \sim x^{-\alpha/2} C_{\alpha/2}
\int_{0}^{\infty} f^{\alpha/2}(x) m(dx)
$$
as $x \to \infty$ and $n \to \infty$. Here $f(x) > 0$ and $m(dx)$ are found in
the stochastic integral representation of $\xi$:
\[
\xi=\int_{0}^\infty f(x) m(dx)
\]
$C_{\alpha}$ is a constant dependent on $\alpha$ only.
\end{frame}

\begin{frame}
  \frametitle{Distribution of the determinant}
  \begin{eqnarray*}
    \lim_{n \to \infty} a_n^{-2p} \det(XX') &=& \lim_{n \to \infty}
    \prod_{i=1}^p a_n^{-2} \lambda_{(i)} \nonumber \\
    &\overset{d}{=}& \prod_{i=1}^p \xi_{i} \label{eq:a}
  \end{eqnarray*}
  Because RHS of \eqref{eq:a} is a product of iid r.v. with Pareto
  tails, we have
  \[
  \P\left(
    \prod_{i=1}^p \xi_{i} > x
  \right) \sim {\alpha^{p-1} c^p \over (p-1)!}  x^{-\alpha/2} \log^{p-1} x 
  \text{, as } x \to \infty
  \]
  Where
  \[
  c =  C_{\alpha/2} \int_{0}^{\infty} f^\alpha(x) m(dx)
  \]
\end{frame}
\bibliographystyle{unsrt}
\bibliography{../../thesis/econophysics}
\end{document} 
