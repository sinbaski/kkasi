%%%%%%%%%%%%%%%%%%%%%%%%%%%%%%%%%%%%%%%%%
% Beamer Presentation
% LaTeX Template
% Version 1.0 (10/11/12)
%
% This template has been downloaded from:
% http://www.LaTeXTemplates.com
%
% License:
% CC BY-NC-SA 3.0 (http://creativecommons.org/licenses/by-nc-sa/3.0/)
%
%%%%%%%%%%%%%%%%%%%%%%%%%%%%%%%%%%%%%%%%%

%----------------------------------------------------------------------------------------
%	PACKAGES AND THEMES
%----------------------------------------------------------------------------------------

\documentclass{beamer}

\mode<presentation> {

% The Beamer class comes with a number of default slide themes
% which change the colors and layouts of slides. Below this is a list
% of all the themes, uncomment each in turn to see what they look like.

% \usetheme{default}
%\usetheme{AnnArbor}
%\usetheme{Antibes}
%\usetheme{Bergen}
% \usetheme{Berkeley}
%\usetheme{Berlin}
%\usetheme{Boadilla}
%\usetheme{CambridgeUS}
%\usetheme{Copenhagen}
%\usetheme{Darmstadt}
%\usetheme{Dresden}
%\usetheme{Frankfurt}
%\usetheme{Goettingen}
%\usetheme{Hannover}
%\usetheme{Ilmenau}
%\usetheme{JuanLesPins}
%\usetheme{Luebeck}
%\usetheme{Madrid}
%\usetheme{Malmoe}
%\usetheme{Marburg}
%\usetheme{Montpellier}
%\usetheme{PaloAlto}
%\usetheme{Pittsburgh}
%\usetheme{Rochester}
%\usetheme{Singapore}
%\usetheme{Szeged}
%\usetheme{Warsaw}

% As well as themes, the Beamer class has a number of color themes
% for any slide theme. Uncomment each of these in turn to see how it
% changes the colors of your current slide theme.

%\usecolortheme{albatross}
%\usecolortheme{beaver}
%\usecolortheme{beetle}
%\usecolortheme{crane}
%\usecolortheme{dolphin}
%\usecolortheme{dove}
%\usecolortheme{fly}
%\usecolortheme{lily}
%\usecolortheme{orchid}
%\usecolortheme{rose}
%\usecolortheme{seagull}
%\usecolortheme{seahorse}
%\usecolortheme{whale}
%\usecolortheme{wolverine}

%\setbeamertemplate{footline} % To remove the footer line in all slides uncomment this line
%\setbeamertemplate{footline}[page number] % To replace the footer
%line in all slides with a simple slide count uncomment this line

%\setbeamertemplate{navigation symbols}{} % To remove the navigation
%symbols from the bottom of all slides uncomment this line
}

\usepackage{graphicx} % Allows including images
\usepackage{booktabs} % Allows the use of \toprule, \midrule and \bottomrule
                      % in tables
\DeclareGraphicsExtensions{.pdf,.png,.jpg}
\usepackage{amssymb,amsmath,amsthm}
%\newtheorem{mdef}{Definition}
%\newtheorem{theorem}{Theorem}
\newcommand{\eqsplit}[2]{
  \begin{equation}\label{#2}
    \begin{split}
      #1
    \end{split}
  \end{equation}}
\newcommand{\eqnsplit}[1]{
  \begin{eqnarray*}
    #1
  \end{eqnarray*}}
\newcommand{\tran}[1]{
  \tilde{#1}
}
\newcommand{\td}[2]{
  \frac{d #1}{d #2}
}
\newcommand{\pd}[2]{
  \frac{\partial #1}{\partial #2}
}
\newcommand{\ppd}[2]{
  \frac{\partial^2 #1}{\partial #2^2}
}
\newcommand{\pdd}[3]{
  \frac{\partial^2 #1}{\partial #2 \partial #3}
}
\newcommand{\otd}[1]{
  \frac{d}{d #1}
}
\newcommand{\opd}[1]{
  \frac{\partial}{\partial #1}
}
\newcommand{\oppd}[1]{
  \frac{\partial^2}{\partial #1^2}
}
\newcommand{\opdd}[2]{
  \frac{\partial^2}{\partial #1 \partial #2}
}
\newcommand{\ket}[1]{
  |#1\rangle
}
\newcommand{\bra}[1]{
  \langle#1|
}
\newcommand{\inn}[1]{
  \langle#1\rangle
}
\newcommand{\mean}[1]{
  \langle#1\rangle
}
\newcommand{\tr}{
  \text{tr}\,
}
\newcommand{\re}{
  \text{Re}\,
}
\newcommand\im{
  \text{Im}\,
}
\newcommand{\var}{
  \text{var}
}
\newcommand{\arcsinh}{
  \sinh^{-1}
}
\newcommand{\arccosh}{
  \cosh^{-1}
}
\newcommand{\erfc}{
  \text{erfc}
}
\newcommand{\E}{
  \mathbb{E}
}
\renewcommand{\P}{
  \mathbb{P}
}
\newcommand{\I}[1]{
  \mathbf{1}_{\{#1\}}
}
\newcommand{\1}[1]{
  \mathds{1}_{\{#1\}}
}
\newcommand{\diag}{
  \text{diag\,}
}
\newcommand{\M}{
  {\text{max}}
}
\newcommand{\m}{
  {\text{min}}
}
\newcommand{\ph}{
  {\text{arg}\,}
}
\newcommand\erf{
  \text{erf}
}
\renewcommand\vec[1]{
  \mathbf{#1}
}
\newcommand\mtx[1]{
  \mathbf{#1}
}
\newcommand\ed{
  \,{\buildrel d \over =}\,
}




%----------------------------------------------------------------------------------------
%	TITLE PAGE
%----------------------------------------------------------------------------------------

\title{Some Empirical Results about the Sample Covariance Matrix of
  Multivariate Financial Time Series}
% The short title appears at the bottom of
% every slide, the full title is only
% on the title page

\author{Xie Xiaolei} % Your name
\institute[UCPH] % Your institution as it will appear on the bottom of every slide, may be shorthand to save space
{
University of Copenhagen \\ % Your institution for the title page
\medskip
\textit{xie@math.ku.dk} % Your email address
}
\date{\today} % Date, can be changed to a custom date

\begin{document}

\begin{frame}
\titlepage % Print the title page as the first slide
\end{frame}

% \begin{frame}
% \frametitle{Overview}
% \tableofcontents
% \end{frame}

%----------------------------------------------------------------------------------------
%	PRESENTATION SLIDES
%----------------------------------------------------------------------------------------
% \section{Simple dependence models}
%------------------------------------------------
\begin{frame}
  \frametitle{Agenda}
  \begin{itemize}
  \item A Multivariate GARCH Model with Correlated Innovations
  \item Stochastic Fixed Point Equations: Monte Carlo estimation of
    probabilities of large exceedances.
  \end{itemize}
\end{frame}

\section{A Multivariate GARCH Model with Correlated Innovations}
\begin{frame}
  \frametitle{Multivariate GARCH models}
  {\bf VEC(1,1)}:
  Given a $p$-variate series $\{X_{i, t}\}$, $i=1,...,p,
  t=1,...,n$. For all $i, j = 1, ..., p$.
  \begin{eqnarray*}
    % \text{vech}(\text{cov}(X_t)) &=& c + A\;\text{vech}(X_{t-1}
    % X_{t-1}) + G\;\text{vech}(\text{cov}(X_{t-1}))
    \vec H_t &=& (\sigma_{1,t}^2, \sigma_{1,2, t}^2, \dots, \sigma_{2,t}^2,
    \sigma_{2,3, t}^2, \dots, \sigma_{p,t}^2)' \\
    \vec V_t &=& (X_{1,t}^2, X_{1, t} X_{2, t}, \dots, X_{2,t}^2,
    X_{2, t} X_{3,t}, \dots, X_{p,t}^2)' \\
    \vec H_t &=& A_t \vec V_{t-1} + G_t \vec H_{t-1} + \vec c
  \end{eqnarray*}
  where $\sigma_i^2 = \var(X_{i, t} | \mathcal F_{t-1})$, $\sigma_{i,j}^2 =
  \cov(X_{i,t}, X_{j,t} | \mathcal F_{t-1})$.
\end{frame}

\begin{frame}
  \frametitle{Stochastic Volatility Models}
  \begin{eqnarray*}
    X_{i,t} &=& \sigma_{i, t} Z_{i, t} \\
    \ln(\sigma_{i,t}) &\bot& (Z_{1, t}, ..., Z_{i, t}, ..., Z_{p,t}) \\
    (\ln(\sigma_{1,t}), \dots, \ln(\sigma_{p,t})) &\sim& \text{VARMA}
    \\
    Z_{i,t} &\sim& N(0,1), t(\nu), ...
  \end{eqnarray*}
\end{frame}

\begin{frame}
  \frametitle{Factor Model -- Orthogonal GARCH}
  \begin{eqnarray*}
    X_{i, t} &=& \sum_{k=1}^N C_{i,k} Y_{k, t} + \epsilon_{i, t} \\
    Y_{k, t} &\sim& \text{GARCH}(1,1) \\
    \epsilon_{i,t} &\sim& N(0, \sigma_i^2)
  \end{eqnarray*}
\end{frame}

\begin{frame}
  We suggest
  \begin{eqnarray*}
    X_{i, t} &=& \sigma_{i, t} Z_{i, t} \\
    \sigma_{i, t}^2 &=& \alpha_{i} X_{i, t}^2 + \beta_{i} \sigma_{i, t-1}^2 + \omega_i \\
    (Z_{1, t}, ..., Z_{p, t}) &\sim& N(0, \Sigma) \\
    \var(Z_{i,t}) &=& 1
  \end{eqnarray*}
  Advantages:
  \begin{enumerate}
  \item significantly fewer parameters to estimate
  \item can model a different tail index for each series.
  \end{enumerate}
\end{frame}

\begin{frame}
  \frametitle{Unnormalized Sample Covariance Matrix $C$}
  Given a $p$-variate series $\{X_{i, t}\}$, $i=1,...,p, t=1,...,n$.
  Assume $\E X_{i, t} = 0$.
  \[
  C_{i, j} = \sum_{t=1}^n X_{i,t} X_{j,t}
  \]
  Decompose the matrix $C$ into eigenvalues $\lambda_1, ...,
  \lambda_p$ and eigenvectors $\vec v_1, ..., \vec v_p$:
  \[
  C (\vec v_1, ..., \vec v_p) = \diag(\lambda_1, ..., \lambda_p) (\vec v_1, ..., \vec v_p)
  \]
  The goodness of fit of the theoretical eigenvalues and eigenvectors
  to their empirical counterparts gives an indication of how well the
  model describes the data.
\end{frame}

\begin{frame}
  \frametitle{Gaussian Series with the same Covariance Matrix as the
    Observations}
  \begin{figure}[htb!]
    \centering
    \includegraphics[scale=0.4]{Gaussian_eigenvalues.pdf}
    \caption{\scriptsize Eigenvalues of real FX series and Simulated Gaussian series}
  \end{figure}
\end{frame}

\begin{frame}
  \frametitle{Gaussian Series with the same Covariance Matrix as the
    Observations}
  \begin{figure}[htb!]
    \centering
    \includegraphics[width=1.0\linewidth]{Gaussian_eigenvectors.pdf}
    \caption{\scriptsize Eigenvectors of real FX series and Simulated Gaussian series}
  \end{figure}
\end{frame}

\subsection{FX Series}
\begin{frame}
  \frametitle{Foreign Exchange Rates}
  Daily return series of 17 exchange rates against Swedish krona (SEK):
  AUD,
  CAD,
  CNY,
  CZK,
  DKK,
  EUR,
  GBP,
  HKD,
  HUF,
  JPY,
  KRW,
  MAD,
  MXN,
  NOK,
  NZD,
  SGD,
  USD
  \begin{figure}[htb!]
    \centering
    \includegraphics[scale=0.35]{FX_eigenvalues.pdf}
    \caption{\scriptsize Eigenvalues of real and Simulated GARCH series}
  \end{figure}
\end{frame}

\begin{frame}
  \frametitle{Foreign Exchange Rates}
  Eigenvectors:
  \begin{figure}[htb!]
    \centering
    \includegraphics[scale=0.2]{FX_eigenvectors.pdf}
    \caption{\scriptsize Eigenvectors of real and Simulated GARCH series}
  \end{figure}
\end{frame}

\subsection{Utilities}
\begin{frame}
  \frametitle{sector of Utilities of S\&P 500}
  \begin{figure}[htb!]
    \centering
    \includegraphics[scale=0.35]{Utilities_eigenvalues.pdf}
    \caption{\scriptsize Eigenvalues of real and Simulated GARCH series}
  \end{figure}
\end{frame}

\begin{frame}
  \frametitle{sector of Utilities of S\&P 500}
  \begin{figure}[htb!]
    \centering
    \includegraphics[width=1.0\linewidth]{Utilities_eigenvectors1.pdf}
    \caption{\scriptsize Eigenvectors of real and Simulated GARCH series}
  \end{figure}
\end{frame}

\subsection{Materials}
\begin{frame}
  \frametitle{sector of Materials of S\&P 500}
  \begin{figure}[htb!]
    \centering
    \includegraphics[scale=0.35]{Materials_eigenvalues.pdf}
    \caption{\scriptsize Eigenvalues of real and Simulated GARCH series}
  \end{figure}
\end{frame}

\begin{frame}
  \frametitle{sector of Materials of S\&P 500}
  \begin{figure}[htb!]
    \centering
    \includegraphics[width=1.0\linewidth]{Materials_eigenvectors1.pdf}
    \caption{\scriptsize Eigenvectors of real and Simulated GARCH series}
  \end{figure}
\end{frame}

\subsection{Emergence of the Observed Spectrum}
\begin{frame}
  \frametitle{Emergence of the Observed Spectrum}
  Conjecture: Joint regular variation of the entries of the covariance
  matrix leads to the observed spectral pattern.
  \begin{itemize}
  \item $(X_{1,t}, ..., X_{p,t})$ jointly regularly varying with
    different indices $\kappa_1, ..., \kappa_p$.
  \item $(X_{1,t}^2, ..., X_{p,t}^2, \{X_{i,t}X_{j,t}\}_{1 \leq i < j
      \leq p})$ jointly regularly varying with different indices
  \item Tail index of $X_{i,t} X_{j,t}$ ($\kappa_{i,j}$) possibly
    smaller than $\min\{\kappa_i, \kappa_j\}$.
  \end{itemize}
  % \[
  % \lim_{u \to \infty} u^{\kappa}
  % \P\left(
  %   \inn{\vec{x}, V} > u
  % \right) = e_{\kappa}(\vec{x}),\; \vec x \in \mathbb S^{p-1}
  % \]
  % where $V=(\E X_{1,t}^2, ..., \E X_{p,t}^2,
  % \{\E (X_{i,t}X_{j,t})\}_{1 \leq i < j \leq p})$
  % This way, we have a covariance matrix where some of the non-diagonal
  % entries are non-zero under proper normalization.
\end{frame}

\begin{frame}
  \frametitle{The Tail index $\kappa_i$}
  Consider the tail indices of the entries of the covariance matrix C:
  For the diagonal entries
    \begin{eqnarray*}
      C_{i, i} &=& \sum_{t=1}^n X_{i, t}^2 \\
               &=& \sum_{t=1}^n Z_{i, t}^2 \sigma_{i, t}^2 \\
    \end{eqnarray*}
    $Z_{i, t}^2$ is light tailed, so $Z_{i, t}^2 \sigma_{i, t}^2$ has
    the same tail index as $\sigma_{i, t}^2$.
    \[
    \sigma_{i, t}^2 = \omega + (\alpha_i Z_{i,t}^2 + \beta_i)\sigma_{i,t-1}
    \]
    So by Goldie-Kesten theorem, $\kappa_i$ is given by
    \[
    \E (\alpha_i Z^2 + \beta_i)^{\kappa_i} = 1
    \]
    where $Z \sim N(0, 1)$.
  \end{frame}

  \begin{frame}
  \frametitle{The Tail index $\kappa_{i,j}$}
    For the non-diagonal entries
    \begin{eqnarray*}
      C_{i, j} &=& \sum_{t=1}^n X_{i, t} X_{j, t} \\
      &=& \sum_{t=1}^n Z_{i, t} Z_{j, t} \sigma_{i, t} \sigma_{j, t}
    \end{eqnarray*}

    \begin{tiny}
      \begin{eqnarray*}
        \begin{pmatrix}
          \sigma_{i,t}^2 \sigma_{j,t}^2 \\
          \sigma_{i,t}^2 \\
          \sigma_{j,t}^2
        \end{pmatrix}
        &=&
        \begin{pmatrix}
          (\alpha_i Z_{i,t}^2 + \beta_i) (\alpha_j Z_{i,t}^2 + \beta_j) & \omega_j & \omega_i \\
          0 & \alpha_i Z_{i,t}^2 + \beta_i & 0 \\
          0 & 0 & \alpha_j Z_{i,t}^2 + \beta_j
        \end{pmatrix}
        \begin{pmatrix}
          \sigma_{i,t-1}^2 \sigma_{j,t-1}^2 \\
          \sigma_{i,t-1}^2 \\
          \sigma_{j,t-1}^2
        \end{pmatrix} +
        \begin{pmatrix}
          \omega_i \omega_j\\
          \omega_i \\
          \omega_j
        \end{pmatrix}
      \end{eqnarray*}
    \end{tiny}
    Can write
    \[
    V_t = A_t V_{t-1} + B_t
    \]
    $A_t$ upper-triangular matrix, Kesten's theorem not applicable.
    % Again, $Z_{i, t} Z_{j, t} \sigma_{i, t} \sigma_{j, t}$ has the
    % same tail index as $\sigma_{i, t} \sigma_{j, t}$, which is given
    % by
    % \[
    % \E [(\alpha_i Z_i + \beta_i)^{\kappa_{i,j}}(\alpha_i Z_i +
    % \beta_i)^{\kappa_{i,j}}] = 1
    % \]
  \end{frame}


  % \begin{itemize}
  % \item For a GARCH(1,1) process
  %   \[
  %   \E (\alpha Z^2 + \beta)^\kappa = 1
  %   \]
  %   gives the index of regular variation $\kappa$ of $\sigma_{i, t}^2$.
  % \item For $\E (X_{i,t} X_{j,t})$:
  %   \begin{eqnarray*}
  %     && \E (X_{i,t} X_{j,t})      \\
  %     &=& \E (Z_{i,t} Z_{j, t}) \E (\sigma_{i,t} \sigma_{j, t})
  %   \end{eqnarray*}
  % \end{itemize}



\subsection{IARCH(1), ARCH(1) and IGARCH(1) processes}
\begin{frame}
  \frametitle{IARCH(1)}
  In the special case of IARCH(1), i.e. $\alpha_{i} = 1$, $\beta_{i} = 0$,
  \[
  \E (Z_i^2)^{\kappa_i} = 1
  \]
  gives $\kappa_i = 1$. Suppose
  \begin{eqnarray*}
    \E (Z_i^2 Z_j^2)^\kappa &=& 1 \\
    (Z_i, Z_j) &\sim& N(0, \Sigma) \\
    \Sigma &=&
    \begin{pmatrix}
      1 & \rho \\
      \rho & 1
    \end{pmatrix}
  \end{eqnarray*}
  $\E (Z_i^2 Z_j^2)^\kappa = 1$ means
  \begin{eqnarray*}
    && {1 \over 2\pi}
    \int \int
    \left(
      \rho^2 x^4 + (1 - \rho^2) x^2 y^2 + 2 \rho \sqrt{1 - \rho^2} x^3 y
    \right)^\kappa \times \\
    &&
    \exp\left(
      -{x^2 + y^2 \over 2}
    \right)
    dx dy = 1
  \end{eqnarray*}
  When $\rho = \pm 1$, it is clear $\kappa = 1/2$.  
\end{frame}

\begin{frame}
  \frametitle{IARCH(1)}
  When $-1 < \rho < 1$, re-writing in polar coordinates gives
  \begin{eqnarray*}
    && {1 \over 2\pi} \int_0^{\infty} r^{4\kappa + 1} e^{-r^2 / 2} dr
    \int_{0}^{2\pi} \cos(\theta)^{2 \kappa} \times \\
    &&
    \left(
      2 \rho^2 \cos(\theta)^2
      + 2 \rho \sqrt{1 - \rho^2} \cos(\theta) \sin(\theta) - \cos(\theta)^2 + 1 - \rho^2
    \right)^\kappa d\theta = 1
  \end{eqnarray*}
  Inegrating the radial part gives
  \begin{eqnarray*}
    && \int_{0}^{2\pi}
    \cos(\theta)^{2\kappa}
    \left(
      2 \rho^2 \cos(\theta)^2 + \rho \sqrt{1 - \rho^2} \sin(2\theta) - \cos(\theta)^2 - \rho^2 + 1
    \right)^\kappa \times \\
    && d\theta
    = {2 \pi
      \over
      2^{2 \kappa} \Gamma(2\kappa + 1)
    }
  \end{eqnarray*}
\end{frame}

\begin{frame}
  \frametitle{IARCH(1)}
  After some manipulations we arrive at
  \begin{equation}
    \int_{-1}^1 (\sin(\pi z) + \rho)^{2\kappa} dz = {2 \over \Gamma(2\kappa + 1)}
  \end{equation}
  Using this result, we immediately obtain for ARCH(1) models
  \begin{equation*}
    \int_{-1}^1 (\sin(\pi z) + \rho)^{2\kappa} dz =
    {2 \over \Gamma(2\kappa + 1)}
    {1 \over (\alpha_{i} \alpha_{j})^\kappa}
  \end{equation*}
  So Non-diagonal entries of the empirical covariance matrix can have
  a tail as heavy as the heaviest of the diagonal entries.
\end{frame}


\begin{frame}
  \frametitle{ARCH(1)}  
  ARCH(1) models fit the data too.
  \begin{figure}[htb!]
    \centering
    \includegraphics[scale=0.35]{FX_ARCH_eigenvalues.pdf}
    \caption{\tiny Eigenvalues of real FX series and Simulated ARCH(1) series}
  \end{figure}
\end{frame}

\begin{frame}
  \begin{figure}[htb!]
    \centering
    \includegraphics[scale=0.2]{FX_ARCH_eigenvectors.pdf}
    \caption{\tiny Eigenvectors of real FX series and Simulated ARCH(1) series}
  \end{figure}
\end{frame}

\begin{frame}
  \frametitle{General IGARCH(1)}
  For general IGARCH(1) models, $\kappa_{i,j}$ as the solution to
  \begin{equation*}
    \E \left[
      (
      \alpha_i Z_1^2 + 1 - \alpha_i
      )^\xi
      (
      \alpha_j Z_2^2 + 1 - \alpha_j
      )^\xi\right] = 1
  \end{equation*}
  can be solved numerically.
  % {r}{0.5\textwidth}
  \begin{figure}
    \begin{center}
      \includegraphics[scale=0.2]{igarch_rho0dot5.pdf}      
    \end{center}
    \label{fig:xi_rho0.5}
    \caption{$\kappa_{i,j}(\alpha_1, \alpha_2)$ for $\text{cov}(Z_1, Z_2) = 1/2$}
  \end{figure}
  $X_{i,t} X_{j,t}$ can have an even heavier tail than does $X_{i,t}^2$ or $X_{j,t}^2$.
\end{frame}

\begin{frame}
  Ongoing work to show joint regular variation of the entries of
  the covariance matrix and explain the observed spectrum.
\end{frame}

% \section{GARCH(p,q) Processes}
% \begin{frame}
%   \frametitle{GARCH(p,q) processes}  
%   \begin{eqnarray*}
%     \sigma_{t}^2 &=& \omega + \sum_{i=1}^p \alpha_i X_{t-i}^2 +
%     \sum_{i=1}^q \beta_i \sigma_{t-i}^2 \\
%     X_{i, t} &=& \sigma_{i, t} Z_{i, t} \\
%     Z_{i, t} &\sim& N(0, 1), t(\nu), ...
%   \end{eqnarray*}
%   In this study we assume $Z_{i, t} \sim N(0,1)$.
% \end{frame}

% \begin{frame}
%   \begin{equation*}
%     \begin{tiny}
%       \begin{pmatrix}
%         \sigma_{t}^2 \\
%         \sigma_{t-1}^2 \\
%         \vdots \\
%         \sigma_{t-q+1}^2 \\
%         X_{t-1}^2 \\
%         X_{t-2}^2 \\
%         \vdots \\
%         X_{t-p+1}^2
%       \end{pmatrix} =
%       \begin{pmatrix}
%         \alpha_1 Z_{t-1}^2 + \beta_1 & \beta_2 & \cdots & \beta_q & \alpha_2 & \alpha_3 & \cdots & \alpha_p \\
%         1 & 0 & \cdots & 0 & 0 & 0 & \cdots & 0 \\
%         \vdots & \vdots & \cdots & \vdots & \vdots & \vdots & \cdots & \vdots \\
%         0 & 0 & \cdots & 0 & 0 & 0 & \cdots & 0 \\
%         Z_{t-1}^2 & 0 & \cdots & 0 & 0 & 0 & \cdots & 0 \\
%         0 & 0 & \cdots & 0 & 1 & 0 & \cdots & 0 \\
%         \vdots & \vdots & \cdots & \vdots & \vdots & \vdots & \cdots & \vdots \\
%         0 & 0 & \cdots & 0 & 0 & 0 & \cdots & 0 \\    
%       \end{pmatrix}
%       \begin{pmatrix}
%         \sigma_{t-1}^2 \\
%         \sigma_{t-2}^2 \\
%         \vdots \\
%         \sigma_{t-q}^2 \\
%         X_{t-2}^2 \\
%         X_{t-3}^2 \\
%         \vdots \\
%         X_{t-p}^2
%       \end{pmatrix} +
%       \begin{pmatrix}
%         \omega \\
%         0 \\
%         \vdots \\
%         0 \\
%         0 \\
%         0 \\
%         \vdots \\
%         0 \\
%       \end{pmatrix}
%       \end{tiny}
%     \end{equation*}
%     Compactly
%     \[
%     V_t = A_t V_{t-1} + B
%     \]
%   \end{frame}

%   \begin{frame}
%     \frametitle{Properties of $A_t$}
%     \begin{itemize}
%     \item $\inf_{n \geq 1} {1 \over n}\E \ln \|A_n \cdots A_1\| < 0$
%     \item 
%     \end{itemize}
%   \end{frame}
%   \begin{frame}
%     \frametitle{GARCH(p,q) processes}
%     By Kesten's  theorem, this matrix recurrence equation has a
%     stationary solution and the stationary distribution has regularly
%     varying tail:
%     \[
%     \lim_{u \to \infty} u^{-\kappa} P(\inn{x, V_t} > u) = C,\; x \in \mathbb S^{d-1}
%     \]
%     where $d = p + q -1$
%   \end{frame}

%   \begin{frame}
%     \frametitle{The tail index $\kappa$}
%     \begin{itemize}
%     \item Needed for risk management
%     \item Needed for Importance sampling of the small probability $P(\inn{x, V_t}
%       > u)$ for large $u$. $\kappa$ is the optimal parameter in
%       changing the measure of $A_t$.
%     \item Difficult to estimate. Defined by
%       \begin{equation*}
%         \lim_{n \to \infty} {1 \over n} \ln \E \|A_n \cdots A_1\|^\kappa
%         = 0
%       \end{equation*}
%       \begin{equation*}
%         \lim_{n \to \infty} \|A_n \cdots A_1 \| = 0
%       \end{equation*}
%     \end{itemize}
%   \end{frame}

%  \begin{frame}
%    \frametitle{How to estimate $\kappa$ -- Resampling}
%     By Kesten(1973),
%     \[
%     \lim_{n \to \infty} {
%       |A_n \cdots A_1 \vec{x}|
%       \over
%       \|A_n \cdots A_1\|
%     } = 1
%     \]
%     for an arbitrary $\vec x \in S^{d-1}$. Introduce
%     \begin{eqnarray*}
%       E_n &=& {
%         A_n \cdots A_1 \vec{x}
%         \over
%         |A_n \cdots A_1 \vec{x}|
%       }
%     \end{eqnarray*}
%     Re-write
%     \begin{eqnarray*}
%       && |A_n \cdots A_1 \vec{x}|^\kappa      \\
%       &=& |A_n E_{n-1}|^\kappa |A_{n-1} \cdots A_1 \vec{x}|^\kappa \\
%       &=& \prod_{i=1}^n |A_i E_{i-1}|^\kappa \\
%     \end{eqnarray*}
%     Apply re-sampling that favors $E_{i-1}$ that is associated with
%     large $| A_{i-1} \cdots A_1 E_0 |^\kappa$.
%  \end{frame}

%  \begin{frame}
%    \frametitle{Algorithm for estimating $\kappa$}
%    First estimate
%    \[
%    \Lambda(\theta) = 
%    \lim_{n \to \infty} {1 \over n}\ln
%    \E \|A_n \cdots A_1\|^\theta
%    \]
%    for any given $\theta > 0$. Then find $\kappa$ by solving
%    $\Lambda(\kappa) = 0$.
%  \end{frame}

%  \begin{frame}
%    \frametitle{Algorithm for estimating $\kappa$}
%    \begin{figure}[htb!]
%      \centering
%      \includegraphics[scale=0.5]{PseudoCode.pdf}
%      % \caption{Pseudo Code for $\kappa$ estimation}
%      \label{fig:PseudoCode}
%    \end{figure}
%  \end{frame}

%  \begin{frame}
%    \frametitle{Algorithm for estimating $\kappa$}
%    \begin{figure}[htb!]
%      \centering
%      \includegraphics[scale=0.5, trim={4.5cm 0cm 4.5cm 3cm}, clip]{AnandsEstimator.pdf}
%      % \caption{Pseudo Code for $\kappa$ estimation}
%      \label{fig:PseudoCode}
%    \end{figure}
%  \end{frame}

%  \begin{frame}
%    \frametitle{Estimation Results}
%    \begin{minipage}{0.5\linewidth}
%    \begin{figure}[htb!]
%      \centering
%      \includegraphics[width=\linewidth]{Lambda.pdf}     
%      \caption{\tiny $\Lambda(\alpha)$ of S\&P 500 modeled as GARCH(2,1)}
%      \label{fig:SP500_Lambda}
%    \end{figure}
%    \end{minipage}\hfill
%    \begin{minipage}{0.5\linewidth}
%      \begin{figure}[htb!]
%        \centering
%        \includegraphics[width=\linewidth]{SP500_var_HillPlot.pdf}
%        \caption{S\&P 500 $\sigma_t^2$ Hill Plot}
%        \label{fig:SP500_var_HillPlot}
%      \end{figure}
%    \end{minipage}

%    \begin{itemize}
%    \item $\kappa$ = 4.4397
%    \item Hill estimator: 4.3372.
%    \item GARCH(1,1): 4.4465.
%    \end{itemize}
%  \end{frame}
\section{Monte Carlo Estimation of prob. of Large Exceedances}

\begin{frame}
  \frametitle{Stochastic Fixed Point Equations}
  A GARCH(p,q) process:
  \begin{tiny}
    \begin{equation*}
      \begin{pmatrix}
        \sigma_{t}^2 \\
        \sigma_{t-1}^2 \\
        \vdots \\
        \sigma_{t-q+1}^2 \\
        X_{t-1}^2 \\
        X_{t-2}^2 \\
        \vdots \\
        X_{t-p+1}^2
      \end{pmatrix} =
      \begin{pmatrix}
        \alpha_1 Z_{t-1}^2 + \beta_1 & \beta_2 & \cdots & \beta_q & \alpha_2 & \alpha_3 & \cdots & \alpha_p \\
        1 & 0 & \cdots & 0 & 0 & 0 & \cdots & 0 \\
        \vdots & \vdots & \cdots & \vdots & \vdots & \vdots & \cdots & \vdots \\
        0 & 0 & \cdots & 0 & 0 & 0 & \cdots & 0 \\
        Z_{t-1}^2 & 0 & \cdots & 0 & 0 & 0 & \cdots & 0 \\
        0 & 0 & \cdots & 0 & 1 & 0 & \cdots & 0 \\
        \vdots & \vdots & \cdots & \vdots & \vdots & \vdots & \cdots & \vdots \\
        0 & 0 & \cdots & 0 & 0 & 0 & \cdots & 0 \\    
      \end{pmatrix}
      \begin{pmatrix}
        \sigma_{t-1}^2 \\
        \sigma_{t-2}^2 \\
        \vdots \\
        \sigma_{t-q}^2 \\
        X_{t-2}^2 \\
        X_{t-3}^2 \\
        \vdots \\
        X_{t-p}^2
      \end{pmatrix} +
      \begin{pmatrix}
        \omega \\
        0 \\
        \vdots \\
        0 \\
        0 \\
        0 \\
        \vdots \\
        0 \\
      \end{pmatrix}
    \end{equation*}
    Compactly
    \[
    V_t = A_t V_{t-1} + B
    \]
    As $t \to \infty$
    \[
    V_t \overset{d}{=} A_t V_{t-1} + B
    \]
    By Kesten-Goldie theorem:
    \[
    \lim_{u \to \infty} u^{-\kappa} P(\inn{x, V_t} > u) = C \quad x \in \mathbb S^{d-1}
    \]
    where $d = p + q -1$
  \end{tiny}
\end{frame}

\begin{frame}
  \frametitle{Stochastic Fixed Point Equations}
  $C$ often difficult to characterise. Want to simulate the rare event prob.
  \[
  P(\inn{x, V_t} > u) \quad u \to \infty  
  \]
  \underline{Importance Sampling!}
\end{frame}

\bibliographystyle{unsrt}
\bibliography{../../thesis/econophysics}
\end{document} 
